\section{Projective bundles; ample sheaves}
\label{section:II.4}


\subsection{Definition of projective bundles}
\label{subsection:II.4.1}

\begin{definition}[4.1.1]
\label{II.4.1.1}
Let $Y$ be a prescheme, $\sh{E}$ a quasi-coherent $\sh{O}_Y$-module, and $\bb{S}_{\sh{O}_Y}(\sh{E})$ the symmetric $\sh{O}_Y$-algebra of $\sh{E}$ \sref{II.1.7.4}, which is quasi-coherent \sref{II.1.7.7}.
We define the \emph{projective bundle on $Y$ defined by $\sh{E}$}, denoted $\bb{P}(\sh{E})$, to be the $Y$-scheme $P=\Proj(\bb{S}_{\sh{O}_Y}(\sh{E}))$.
The $\sh{O}_P$-module $\sh{O}_P(1)$ is called the \emph{fundamental sheaf on $P$}.
\end{definition}

When $Y$ is affine of ring $A$, then we have $\sh{E}=\widetilde{E}$ for some $A$-module $E$, and we then write $\bb{P}(E)$ instead of $\bb{P}(\widetilde{E})$.

When we take $\sh{E}=\sh{O}_Y^n$, we write $\bb{P}_Y^{n-1}$ instead of $\bb{P}(\sh{E})$;
if, further, $Y$ is affine of ring $A$, then we also write $\bb{P}_A^{n-1}$ instead of $\bb{P}_Y^{n-1}$.
Since $\bb{S}_{\sh{O}_Y}(\sh{O}_Y)$ is canonically identified with $\sh{O}_Y[T]$ \sref{II.1.7.4}, $\bb{P}_Y^0$ is canonically identified with $Y$ \sref{II.3.1.7};
Example~\sref{II.2.4.3} is then exactly $\bb{P}_K^1$.

\begin{env}[4.1.2]
\label{II.4.1.2}
Let $\sh{E}$ and $\sh{F}$ be quasi-coherent $\sh{O}_Y$-modules;
let $u:\sh{E}\to\sh{F}$ be an $\sh{O}_Y$-homomorphism;
there is a canonically corresponding homomorphism $\bb{S}(u):\bb{S}_{\sh{O}_Y}(\sh{E})\to\bb{S}_{\sh{O}_Y}(\sh{F})$ of graded $\sh{O}_Y$-algebras \sref{II.1.7.4}.
If $u$ is \emph{surjective}, then so too is $\bb{S}(u)$, and thus \sref{II.3.6.2}[(i)] $\Proj(\bb{S}(u))$ is a \emph{closed immersion} $\bb{P}(\sh{F})\to\bb{P}(\sh{E})$, which we denote by $\bb{P}(u)$.
We can thus say that $\bb{P}(\sh{E})$ is a \emph{contravariant functor} in $\sh{E}$, with the condition that we only consider \emph{surjective} morphisms of quasi-coherent $\sh{O}_Y$-modules.

Still supposing that $u$ is surjective, and letting $P=\bb{P}(\sh{E})$, $Q=\bb{P}(\sh{F})$, and $j=\bb{P}(u)$, we have, up to isomorphism, that
\[
\label{II.4.1.2.1}
  j^*(\sh{O}_P(n)) = \sh{O}_Q(n)
  \qquad\mbox{for all $n\in\bb{Z}$}
  \tag{4.1.2.1}
\]
by \sref{II.3.6.3}.
\end{env}

\begin{env}[4.1.3]
\label{II.4.1.3}
Now let $\psi:Y'\to Y$ be a morphism, and let $\sh{E}'=\psi^*(\sh{E})$;
then $\bb{S}_{\sh{O}_{Y'}}(\sh{E}') = \psi^*(\bb{S}_{\sh{O}_Y}(\sh{E}))$ \sref{II.1.7.5};
thus \sref{II.3.5.3}
\[
\label{II.4.1.3.1}
  \bb{P}(\psi^*(\sh{E})) = \bb{P}(\sh{E})\times_Y Y'
  \tag{4.1.3.1}
\]
up to canonical isomorphism;
furthermore, we clearly have that
\[
  \psi^*((\bb{S}_{\sh{O}_Y}(\sh{E}))(n)) = (\bb{S}_{\sh{O}_{Y'}}(\sh{E}'))(n)
\]
for all $n\in\bb{Z}$, whence, letting $P=\bb{P}(\sh{E})$ and $P'=\bb{P}(\sh{E}')$, we have \sref{II.3.5.4}, up to isomorphism, that
\[
\label{II.4.1.3.2}
  \sh{O}_{P'}(n) = \sh{O}_p(n)\otimes_Y\sh{O}_{Y'}
  \qquad\mbox{for all $n\in\bb{Z}$.}
  \tag{4.1.3.2}
\]
\end{env}

\oldpage[II]{72}
\begin{proposition}[4.1.4]
\label{II.4.1.4}
Let $\sh{L}$ be an invertible $\sh{O}_Y$-module.
For every quasi-coherent $\sh{O}_Y$-module $\sh{E}$, there exists a canonical $Y$-isomorphism $i_\sh{L}:\bb{P}(\sh{E})\xrightarrow{\sim}\bb{P}(\sh{E}\otimes\sh{L})$;
furthermore, if we let $P=\bb{P}(\sh{E})$ and $Q=\bb{P}(\sh{E}\otimes\sh{L})$, then $i_\sh{L}^*(\sh{O}_Q(n))$ is canonically isomorphic to $\sh{O}_P(n)\otimes_Y\sh{L}^{\otimes n}$ for all $n\in\bb{Z}$.
\end{proposition}

\begin{proof}
Note first of all that, if $A$ is a ring, $E$ an $A$-module, and $L$ a \emph{free monogenous} $A$-module, then we can canonically define a homomorphism of $A$-modules
\[
  \bb{S}_n(E\otimes L) \to \bb{S}_n(E)\otimes L^{\otimes n}
\]
by sending $(x_1\otimes y_1)\ldots(x_n\otimes y_n)$ to the element
\[
  (x_1x_2\ldots x_n)\otimes(y_1\otimes y_2\otimes\ldots\otimes y_n)
  \qquad\mbox{($x_i\in E$, $y_i\in L$, for $i\leq i\leq n$);}
\]
we can immediately see (by restricting to the case where $L=A$) that this homomorphism is in fact an isomorphism.
We thus obtain a canonical isomorphism of graded $A$-algebras $\bb{S}_A(E\otimes L)\xrightarrow{\sim}\bigoplus_{n\geq0}\bb{S}_n(E)\otimes L^{\otimes n}$.
By returning to the conditions of \sref{II.4.1.4}, the above remarks allow us to define a canonical isomorphism of graded $\sh{O}_Y$-algebras
\[
\label{II.4.1.4.1}
  \bb{S}_{\sh{O}_Y}(\sh{E}\otimes_{\sh{O}_Y}\sh{L}) \xrightarrow{\sim} \bigoplus_{n\geq0}\bb{S}_n(\sh{E})\otimes_{\sh{O}_Y}\sh{L}^{\otimes n}
  \tag{4.1.4.1}
\]
by defining this isomorphism as an isomorphism of presheaves, and taking into account \sref{II.1.7.4}, \sref[I]{I.1.3.9}, and \sref[I]{I.1.3.12}.
The proposition then follows from \sref{II.3.1.8}[(iii)] and \sref{II.3.2.10}.
\end{proof}

\begin{env}[4.1.5]
\label{II.4.1.5}
With the hypotheses of \sref{II.4.1.1}, let $P=\bb{P}(\sh{E})$, and denote by $p$ the structure morphism $P\to Y$.
Since, by definition, $\sh{E}=(\bb{S}_{\sh{O}_Y}(\sh{E}))_1$, we have a canonical homomorphism $\alpha_1:\sh{E}\to p_*(\sh{O}_P(1))$ \sref{II.3.3.2.2}, and thus \sref[0]{0.4.4.3} also a canonical homomorphism
\[
\label{II.4.1.5.1}
  \alpha_1^\sharp: p^*(\sh{E}) \to \sh{O}_P(1).
  \tag{4.1.5.1}
\]
\end{env}

\begin{proposition}[4.1.6]
\label{II.4.1.6}
The canonical homomorphism \sref{II.4.1.5.1} is surjective.
\end{proposition}

\begin{proof}
We have seen, in \sref{II.3.3.2}, that $\alpha_1^\sharp$ corresponds functorially to the canonical homomorphism $\sh{E}\otimes_{\sh{O}_Y}\bb{S}_{\sh{O}_Y}(\sh{E}) \to (\bb{S}_{\sh{O}_Y}(\sh{E}))(1)$;
since, by definition, $\sh{E}$ generates $\bb{S}_{\sh{O}_Y}(\sh{E})$, this homomorphism is surjective, whence the conclusion, by \sref{II.3.2.4}
\end{proof}


\subsection{Morphisms from a prescheme to a projective bundle}
\label{subsection:II.4.2}

\begin{env}[4.2.1]
\label{II.4.2.1}
Keeping the notation of \sref{II.4.1.5}, let $X$ be a $Y$-prescheme, $q:X\to Y$ the structure morphism, and let $r:X\to P$ be a $Y$-\emph{morphism} such that the following diagram commutes:
\[
  \xymatrix{
    P \ar[d]_p & X \ar[l]_r \ar[dl]^q
  \\Y
  }
\]
\oldpage[II]{73}

Since the functor $r^*$ is right exact \sref[0]{0.4.3.1}, we obtain, from the surjective homomorphism in \sref{II.4.1.5.1}, a surjective homomorphism
\[
  r^*(\alpha_1^\sharp): r^*(p^*(\sh{E})) \to r^*(\sh{O}_P(1)).
\]

But $r^*(p^*(\sh{E}))=q^*(\sh{E})$, and $r^*(\sh{O}_P(1))$ is locally isomorphic to $r^*(\sh{O}_P)=\sh{O}_X$, or, in other words, the latter is an \emph{invertible} sheaf $\sh{L}_r$ on $\sh{O}_X$, and so we have defined, given $r$, a canonical surjective $\sh{O}_X$-homomorphism
\[
\label{II.4.2.1.1}
  \varphi_r:q^*(\sh{E}) \to \sh{L}_r.
  \tag{4.2.1.1}
\]

When $Y=\Spec(A)$ is affine, and $\mathscr{E}=\widetilde{E}$, we can further clarify this homomorphism in the following way:
given $f\in E$, it follows from \sref{II.2.6.3} that
\[
\label{II.4.2.1.2}
  r^{-1}(D_+(f)) = X_{\varphi_r^\flat(f)}.
  \tag{4.2.1.2}
\]

Now let $V$ be an affine open subset of $X$ that is contained inside $r^{-1}(D_+(f))$, and let $B$ be its ring, which is an $A$-algebra;
let $S=\bb{S}_A(E)$;
the restriction of $r$ to $V$ corresponds to an $A$-homomorphism $\omega:\bb{S}_f\to B$, and we have that $q^*(\sh{E})|V = (E\otimes_A B)^\sim$ and $\sh{L}_r|V = \widetilde{L_r}$, whence $L_r = (S(1))_{(f)}\otimes_{S_{(f)}}B_{[\omega]}$ \sref[I]{I.1.6.5}.
The restriction of
\end{env}


% \subsection{The Segre morphism}
% \label{subsection:II.4.3}


% \subsection{Immersions in projective bundles; very ample sheaves}
% \label{subsection:II.4.4}


% \subsection{Ample sheaves}
% \label{subsection:II.4.5}


% \subsection{Relatively ample sheaves}
% \label{subsection:II.4.6}
