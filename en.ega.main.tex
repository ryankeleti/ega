\documentclass[10pt,oneside]{book}
\usepackage{preamble}

\begin{document}
\title{\'El\'ements de g\'eom\'etrie alg\'ebrique\\
(EGA)}
\author{A.~Grothendieck and J.~Dieudonn\'e\\
Publications math\'ematiques de l'I.H.\'E.S}
\date{1960}
\maketitle
\frontmatter
\chapter*{What this is}
\label{chapter-what-this-is}
    This is a community translation of Grothendieck's EGA.
    As it is a work in progress by multiple people, it will probably have a few
    mistakes---if you spot any then please feel free to
    \href{https://github.com/ryankeleti/en.ega/issues}{let us know}! To
    contribute, please visit
    \begin{center}
      \url{https://github.com/ryankeleti/en.ega}.
    \end{center}
    On est d\'esol\'es, Grothendieck.

\noindent
    \textbf{Note.} --- EGA uses \textbf{prescheme} for what is now usually
    called a scheme, and \textbf{scheme} for what is now usually called a
    separated scheme.
    \begin{center}
      \textbf{Contributors}\\
      Tim Hosgood\\
      Ryan Keleti
    \end{center}
{
  \hypersetup{
    linkcolor=[rgb]{0,0,0}
  }
  \tableofcontents{}
}
\clearpage
\mainmatter

%%%%%%%%%%%%%%%%%%
%% Introduction %%
%%%%%%%%%%%%%%%%%%
\documentclass[../main.tex]{subfiles}

\begin{document}

\epigraph{\emph{To Oscar Zariski and Andr\'e Weil.}}

Maybe I'll do this later.

\end{document}



\clearpage

%%%%%%%%%%%%%
%% EGA 0_I %%
%%%%%%%%%%%%%
\setcounter{chapter}{-1}
\chapter{Preliminaries}
\label{0-prelim}

\section{Rings of fractions}
\label{0-prelim-1}

\setcounter{subsection}{-1}
\subsection{Rings and Algebras}
\label{0-prelim-1.0}

\begin{env}{1.0.1}
\label{env-0.1.0.1}
\oldpage{11}
All the rings considered in this treatise will have a \emph{unit element}; all the modules
on such a ring will be assumed to be \emph{unitary}; the ring homomorphisms will always be
assumed to \emph{send the unit element to a unit element}; unless otherwise stated,
a subring of a ring $A$ will be assumed to \emph{contain the unit element of $A$}. We will
consider especially \emph{commutative} rings, and when we speak of a ring without
specification, it will be implied that it is commutative. If $A$ is a ring not necessarily
commutative, by $A$-module we will we mean a left module, unless stated otherwise.
\end{env}

\begin{env}{1.0.2}
\label{env-0.1.0.2}
Let $A$, $B$ be two rings, not necessarily commutative, $\vphi:A\to B$ a homomorphism.
Any left (resp. right) $B$-module $M$ can be provided with a left (resp. right) $A$-module
structure by $a\cdot m=\vphi(a)\cdot m$ (resp. $m\cdot a=m\cdot\vphi(a)$); when it will
be necessary to distinguish $M$ as an $A$-module or a $B$-module, we will denote by
$M_{[\vphi]}$ the left (resp. right) $A$-module as defined. If $L$ is an $A$-module, then
a homomorphism $u:L\to M_{[\vphi]}$ is a homomorphism of commutative groups such that
$u(a\cdot x)=\vphi(a)\cdot u(x)$ for $a\in A$, $x\in L$; we will also say that it is a
$\vphi$-\emph{homomorphism} $L\to M$, and that the pair $(\vphi,u)$ (or, by misuse of
langauge, $u$) is a \emph{di-homomorphism} of $(A,L)$ in $(B,M)$. The pairs $(A,L)$ formed by
a ring $A$ and an $A$-module $L$ thus form a \emph{category} for which the morphisms are
di-homomorphisms.
\end{env}

\begin{env}{1.0.3}
\label{env-0.1.0.3}
Under the hypothesis of \sref{env}{1.0.2}, if $\mathfrak{J}$ is a left (resp. right) ideal of
$A$, we denote by $B\mathfrak{J}$ (resp. $\mathfrak{J}B$) the left (resp. right) ideal
$B\vphi(\mathfrak{J})$ (resp. $\vphi(\mathfrak{J})B$) of $B$ generated by
$\vphi(\mathfrak{J})$; it is also the image of the canonical homomorphism
$B\otimes_A\mathfrak{J}\to B$ (resp. $\mathfrak{J}\otimes_A B\to B$) of left (resp. right)
$B$-modules.
\end{env}

\begin{env}{1.0.4}
\label{env-0.1.0.4}
If $A$ is a (commutative) ring, $B$ a non necessarily commutative ring, the data of
a structure of an \emph{$A$-algebra} on $B$ is equivalent to the data of a ring
homomorphism $\vphi:A\to B$ such that $\vphi(A)$ is contained in the center of $B$.
For all ideals $\mathfrak{J}$ of $A$, $\mathfrak{J}B=B\mathfrak{J}$ is then a two-sided ideal
of $B$, and for every $B$-module $M$, $\mathfrak{J}M$ is then a $B$-module equal to
$(B\mathfrak{J})M$.
\end{env}

\begin{env}{1.0.5}
\label{env-0.1.0.5}
We will not revisit the notions of \emph{module of finite type} and \emph{algebra}
(commutative) \emph{of finite type}; to say that an $A$-module $M$ is of finite type means
that there exists
\oldpage{12}
an exact sequence $A^p\to M\to 0$. We say that an $A$-module $M$ admits a \emph{finite
presentation} if it is isomorphic to the cokernel of a homomorphism $A^p\to A^q$, in other
words, there exists an exact sequence $A^p\to A^q\to M\to 0$. We note that for a
\emph{Noetherian} ring $A$, every $A$-module of finite type admits a finite presentation.

Let us recall that an $A$-algebra $B$ is called \emph{integral} over $A$ if every element
in $B$ is a root in $B$ of a monic polynomial with coefficients in $A$; equivalently, every
element of $B$ is contained in a subalgebra of $B$ which is an $A$-\emph{module of finite
type}. When this is so, and $B$ is commutative, the subalgebra of $B$ generated by a finite
subset of $B$ is an $A$-module of finite type; for a commutative algebra $B$ to be integral
and of finite type over $A$, it is necessary and therefore sufficient that $B$ be an
$A$-module of finite type; we also say that $B$ is an \emph{integral} $A$-\emph{algebra of
finite type} (or simply \emph{finite} if there is no confusion). It will be observed that in
these definitions, it is not assumed that the homomorphism $A\to B$ defining the structure of
an $A$-algebra is injective.
\end{env}

\begin{env}{1.0.6}
\label{env-0.1.0.6}
An \emph{integral} ring (or an \emph{integral domain}) is a ring in which the product of a
finite family of elements $\neq 0$ is $\neq 0$; equivalently, in such a ring we have
$0\neq 1$ and the product of two elements $\neq 0$ is non zero. A \emph{prime} ideal of a
ring $A$ is an ideal $\mathfrak{p}$ such that $A/\mathfrak{p}$ is integral; this therefore
implies that $\mathfrak{p}\neq A$. For a ring $A$ to have at least one prime ideal, it is
necessary and sufficent that $A\neq\{0\}$.
\end{env}

\begin{env}{1.0.7}
\label{env-0.1.0.7}
A \emph{local} ring is a ring $A$ in which there exists a unique maximal ideal, which is then
the complement of the invertible elements and contains all the ideals $\neq A$. If $A$ and $B$
are two local rings, $\mathfrak{m}$ and $\mathfrak{n}$ their respective maximal ideals, we
say that a homomorphism $\vphi:A\to B$ is \emph{local} if
$\vphi(\mathfrak{m})\subset\mathfrak{n}$ (or, equivalently, if
$\vphi^{-1}(\mathfrak{n})=\mathfrak{m}$). By passing to quotients, such a homomorphism then
defines a momomorphism from the residue field $A/\mathfrak{m}$ to the residue field
$B/\mathfrak{n}$. The composition of two local homomorphisms is a local homomorphism.
\end{env}

\subsection{Radical of an ideal. Nilradical and radical of a ring}
\label{0-prelim-1.1}

\begin{env}{1.1.1}
\label{env-0.1.1.1}
Let $\mathfrak{a}$ be an ideal of a ring $A$; the \emph{radical} of $\mathfrak{a}$, denoted
by $\rad(\mathfrak{a})$, is the set of $x\in A$ such that $x^n\in\mathfrak{a}$ for
an integer $n>0$; it is an ideal containing $\mathfrak{a}$. We have
$\rad(\mathfrak{r}(\mathfrak{a}))=\rad(\mathfrak{a})$; the relation
$\mathfrak{a}\subset\mathfrak{b}$ leads to $\rad(\mathfrak{a})\subset\rad(\mathfrak{b})$;
the radical of a finite intersection of ideals is the intersection of their radicals. If
$\vphi$ is a homomorphism of a ring $A'$ into $A$, then we have
$\rad(\vphi^{-1}(\mathfrak{a}))=\vphi^{-1}(\rad(\mathfrak{a}))$ for any ideal
$\mathfrak{a}\subset A$. For an ideal to be the radical of an ideal, it is necessary and
sufficient that it be an intersection of prime ideals. The radical of an ideal $\mathfrak{a}$
is the intersection of the \emph{minimal} prime ideals which contain $\mathfrak{a}$; if $A$
is Noetherian, these minimal prime ideals are finite in number.

The radical of the ideal $(0)$ is also called the \emph{nilradical} of $A$; it is the set
$\nilrad$ of the nilpotent elements of $A$. It is said that the ring $A$ is \emph{reduced} if
$\nilrad=(0)$; for every ring $A$, the quotient $A/\nilrad$ of $A$ by its nilradical is a
reduced ring.
\end{env}

\begin{env}{1.1.2}
\label{env-0.1.1.2}
Recall that the \emph{nilradical} $\nilrad(A)$ of a ring $A$ (not necessarily commutative) is
the intersection of the maximal left ideals of $A$ (and also the intersection of maximal
right ideals). The nilradical of $A/\nilrad(A)$ is $(0)$.
\end{env}

\subsection{Modules and rings of fractions}
\label{0-prelim-1.2}

\begin{env}{1.2.1}
\label{env-0.1.2.1}
\oldpage{13}
We say that a subset $S$ of a ring $A$ is \emph{multiplicative} if $1\in S$ and if the
product of two elements of $S$ is in $S$. The examples which will be the most important for
the following are: 1\textsuperscript{st} the set $S_f$ of powers $f^n$ ($n\geqslant 0$) of an
element $f\in A$; 2\textsuperscript{nd} the complement $A-\mathfrak{p}$ of a \emph{prime}
ideal $\mathfrak{p}$ of $A$.
\end{env}

\begin{env}{1.2.2}
\label{env-0.1.2.2}
Let $S$ be a multiplicative subset of a ring $A$, $M$ an $A$-module; in the set $M\times S$,
the relation between pairs $(m_1,s_1)$, $(m_2,s_2)$:
\begin{center}
   ``there exists an $s\in S$ such that $s(s_1 m_2-s_2 m_1)=0$''
\end{center}
is an equivalence relation. We denote by $S^{-1}M$ the quotient set of $M\times S$ by this
relation, by $m/s$ the canonical image in $S^{-1}M$ of the pair $(m,s)$; we call the
\emph{canonical} map of $M$ to $S^{-1}M$ the map $i_M^S:m\mapsto m/1$ (also denoted $i^S$).
This map is generally neither injective nor surjective; its kernel is the set of $m\in M$
such that there exists an $s\in S$ for which $sm=0$.

On $S^{-1}M$ we define an additive group law by taking
\[
  (m_1/s_1)+(m_2/s_2)=(s_2 m_1+s_1 m_2)/(s_1 s_2)
\]
(we check that it is independent of the expressions of the elements of $S^{-1}M$ considered).
On $S^{-1}A$ we further define a multiplicative law by setting
$(a_1/s_1)(a_2/s_2)=(a_1 a_2)/(s_1 s_2)$, and finally an external law on $S^{-1}M$, having
$S^{-1}A$ as a set of operators, by setting $(a/s)(m/s')=(am)/(ss')$. It is thus verified
that $S^{-1}A$ is provided with a ring structure (called \emph{the ring of fractions of $A$
with denominators in $S$}) and $S^{-1}M$ the structure of an $S^{-1}A$-module (called
\emph{the  module of fractions of $M$ with denominators in $S$}); for all $s\in S$, $s/1$ is
invertible in $S^{-1}A$, its inverse being $1/s$. The canonical map $i_A^S$ (resp. $i_M^S$)
is a homomorphism of rings (resp. a homomorphism of $A$-modules, $S^{-1}M$ being considered
as an $A$-module by means of the homomorphism $i_A^S:A\to S^{-1}A$).
\end{env}

\begin{env}{1.2.3}
\label{env-0.1.2.3}
If $S_f=\{f^n\}_{n\geqslant 0}$ for a $f\in A$, we write $A_f$ and $M_f$ instead of
$S_f^{-1}A$ and $S_f^{-1}M$; when $A_f$ is considered as algebra over $A$, we can write
$A_f=A[1/f]$. $A_f$ is isomorphic to the quotient algebra $A[T]/(fT-1)A[T]$. When $f=1$,
$A_f$ and $M_f$ identify canonically with $A$ and $M$; if $f$ is nilpotent, $A_f$ and $M_f$
are reduced to $0$.

When $S=A-\mathfrak{p}$, where $\mathfrak{p}$ is a prime ideal of $A$, we
write $A_\mathfrak{p}$ and $M_\mathfrak{p}$ instead of $S^{-1}A$ and $S^{-1}M$;
$A_\mathfrak{p}$ is a \emph{local ring} whose maximal ideal $\mathfrak{q}$ is generated by
$i_A^S(\mathfrak{p})$, and we have $(i_A^S)^{-1}(\mathfrak{q})=\mathfrak{p}$; by passing to
quotients, $i_A^S$ gives a monomorphism from the integral ring $A/\mathfrak{p}$ to the
field $A_\mathfrak{p}/\mathfrak{q}$, which identifies with the field of fractions of
$A/\mathfrak{p}$.
\end{env}

\begin{env}{1.2.4}
\label{env-0.1.2.4}
The ring of fractions $S^{-1}A$ and the canonical homomorphism $i_A^S$ are a solution of a \emph{universal mapping problem}:
any homomorphism $u$ of $A$ into a ring $B$ such that $u(S)$ is composed of \emph{invertible} elements in $B$ factorizes uniquely as
\[
  u:A\xrightarrow{i_A^S}S^{-1}A\xrightarrow{u^*}B
\]
\oldpage{14}
where $u^*$ is a ring homomorphism. Under the same hypotheses, let $M$ be an $A$-module, $N$
a $B$-module, $v:M\to N$ a homomorphism of $A$-modules (for the $B$-module structure on $N$
defined by $u:A\to B$); then $v$ is factorizes uniquely as
\[
  v:M\xrightarrow{i_M^S}S^{-1}M\xrightarrow{v^*}N
\]
where $v^*$ is a homomorphism of $S^{-1}A$-modules (for the $S^{-1}A$-module structure on $N$
defined by $u^*$).
\end{env}

\begin{env}{1.2.5}
\label{env-0.1.2.5}
We define a canonical isomorphism $S^{-1}A\otimes_A M\isoto S^{-1}M$ of
$S^{-1}A$-modules, sending the element $(a/s)\otimes m$ to the element $(am)/s$, the
isomorphism reciprocally applying $m/s$ to $(1/s)\otimes m$.
\end{env}

\begin{env}{1.2.6}
\label{env-0.1.2.6}
For each ideal $\mathfrak{a}'$ of $S^{-1}A$, $\mathfrak{a}=(i_A^S)^{-1}(\mathfrak{a}')$ is an
ideal of $A$, and $\mathfrak{a}'$ is the ideal of $S^{-1}A$ generated by
$i_A^S(\mathfrak{a})$, which identifies with $S^{-1}\mathfrak{a}$ \sref{env}{1.3.2}. The map
$\mathfrak{p}'\to(i_A^S)^{-1}(\mathfrak{p}')$ is an isomorphism, for the structure order, of
the set of \emph{prime} ideals of $S^{-1}A$ to the set of prime ideals $\mathfrak{p}$ of $A$
such that $\mathfrak{p}\cap S=\emp$. In addition, the local rings $A_\mathfrak{p}$ and
$(S^{-1}A)_{S^{-1}\mathfrak{p}}$ are canonically isomorphic \sref{env}{1.5.1}.
\end{env}

\begin{env}{1.2.7}
\label{env-0.1.2.7}
When $A$ is an \emph{integral} ring, for which $K$ denotes its field of fractions, the
canonical map $i_A^S:A\to S^{-1}A$ is injective for any multiplicative subset $S$ not
containing $0$, and $S^{-1}A$ then identifies canonically with a subring of $K$ containing
$A$. In particular, for every prime ideal $\mathfrak{p}$ of $A$ , $A_\mathfrak{p}$ is a local
ring containing $A$, with maximal ideal $\mathfrak{p}A_\mathfrak{p}$, and we have
$\mathfrak{p}A_\mathfrak{p}\cap A=\mathfrak{p}$.
\end{env}

\begin{env}{1.2.8}
\label{env-0.1.2.8}
If $A$ is a \emph{reduced} ring \sref{env}{1.1.1}, so is $S^{-1}A$: indeed, if $(x/s)^n=0$ for
$x\in A$, $s\in S$, it means that there exists $s'\in S$ such that $s'x^n=0$, hence
$(s'x)^n=0$, which, by hypothesis, implies $s'x=0$, so $x/s=0$.
\end{env}

\subsection{Functorial properties}
\label{0-prelim-1.3}

\begin{env}{1.3.1}
\label{env-0.1.3.1}
Let $M$, $N$ be two $A$-modules, $u$ an $A$-homomorphism $M\to N$. If $S$ is a multiplicative
subset of $A$, we define a $S^{-1}A$-homomorphism $S^{-1}M\to S^{-1}N$, denoted by $S^{-1}u$,
by putting $S^{-1}u(m/s)=u(m)/s$; if $S^{-1}M$ and $S^{-1}N$ are canonically identified with
$S^{-1}A\otimes_A M$ and $S^{-1}A\otimes_A N$ \sref{env}{1.2.5}, $S^{-1}u$ is considered as
$1\otimes u$. If $P$ is a third $A$-module, $v$ an $A$-homomorphism $N\to P$, we have
$S^{-1}(v\circ u)=(S^{-1}v)\circ(S^{-1}u)$; in other words, $S^{-1}M$ is a \emph{covariant
functor in} $M$, of the category of $A$-modules into that of $S^{-1}A$-modules ($A$ and $S$
being fixed).
\end{env}

\begin{env}{1.3.2}
\label{env-0.1.3.2}
The functor $S^{-1}M$ is \emph{exact}; in other words, if the following
\[
  M\xrightarrow{u}N\xrightarrow{v}P
\]
is exact, so is the following
\[
  S^{-1}M\xrightarrow{S^{-1}u}S^{-1}N\xrightarrow{S^{-1}v}S^{-1}P.
\]
In particular, if $u:M\to N$ is injective (resp. surjective), the same is true for $S^{-1}u$;
\oldpage{15}
if $N$ and $P$ are two submodules of $M$, $S^{-1}N$ and $S^{-1}P$ identify canonically with
submodules of $S^{-1}M$, and we have
\[
  S^{-1}(N+P)=S^{-1}N+S^{-1}P\quad\text{and}\quad S^{-1}(N\cap P)=(S^{-1}N)\cap(S^{-1}P).
\]
\end{env}

\begin{env}{1.3.3}
\label{env-0.1.3.3}
Let $(M_\alpha,\vphi_{\beta\alpha})$ be an inductive system of $A$-modules; then
$(S^{-1}M_\alpha,S^{-1}\vphi_{\beta\alpha})$ is an inductive system of $S^{-1}A$-modules.
Expressing the $S^{-1}M_\alpha$ and $S^{-1}\vphi_{\beta\alpha}$ as tensor products
(\sref{env}{1.2.5} and \sref{env}{1.3.1}), it follows from the permutability of the tensor
product and inductive limit operations that we have a canonical isomorphism
\[
  S^{-1}\varinjlim M_\alpha\isoto\varinjlim S^{-1}M_\alpha
\]
which is further expressed by saying that the functor $S^{-1}M$ (in $M$) \emph{commutes with
inductive limits}.
\end{env}

\begin{env}{1.3.4}
\label{env-0.1.3.4}
Let $M$, $N$ be two $A$-modules; there is a canonical \emph{functorial} isomorphism (in $M$
and $N$)
\[
  (S^{-1}M)\otimes_{S^{-1}A}(S^{-1}N)\isoto S^{-1}(M\otimes_A N)
\]
which sends $(m/s)\otimes(n/t)$ to $(m\otimes n)/st$.
\end{env}

\begin{env}{1.3.5}
\label{env-0.1.3.5}
We also have a \emph{functorial} homomorphism (in $M$ and $N$)
\[
  S^{-1}\Hom_A(M,N)\longrightarrow\Hom_{S^{-1}A}(S^{-1}M,S^{-1}N)
\]
which sends $u/s$ to the homomorphism $m/t\mapsto u(m)/st$. When $M$ has a finite
presentation, the preceding homomorphism is an \emph{isomorphism}: it is immediate when $M$
is of the form $A^r$, and we pass to the general case by starting with the following exact
sequence $A^p\to A^q\to M\to 0$, and using the exactness of the functor $S^{-1}M$ and
the left-exactness of the functor $\Hom_A(M,N)$ in $M$. Note that this case always occurs
when $A$ is \emph{Noetherian} and the $A$-module $M$ is \emph{of finite type}.
\end{env}

\subsection{Change of multiplicative subset}
\label{0-prelim-1.4}

\begin{env}{1.4.1}
\label{env-0.1.4.1}
Let $S$, $T$ be two multiplicative subsets of a ring $A$ such that $S\subset T$; there exists
a canonical homomorphism $\rho_A^{T,S}$ (or simply $\rho^{T,S}$) of $S^{-1}A$ into $T^{-1}A$,
sending the element denoted $a/s$ of $S^{-1}A$ to the element denoted $a/s$ in $T^{-1}A$; we
have ${i_A^T=\rho_A^{T,S}\circ i_A^S}$. For every $A$-module $M$, there exists in the same
way an $S^{-1}A$-linear map from $S^{-1}M$ to $T^{-1}M$ (the latter considered as an
$S^{-1}A$-module thanks to the homomorphism $\rho_A^{T,S}$), which sends the element $m/s$ of
$S^{-1}M$ to the element $m/s$ of $T^{-1}M$; we denote this map by $\rho_M^{T,S}$, or simply
$\rho^{T,S}$, and we still have $i_M^T=\rho_M^{T,S}\circ i_M^S$; in the canonical
identification \sref{env}{1.2.5}, $\rho_M^{T,S}$ identifies with $\rho_A^{T,S}\otimes 1$. The
homomorphism $\rho_M^{T,S}$ is a \emph{functorial morphism} (or natural transformation) from
the functor $S^{-1}M$ to the functor $T^{-1}M$, in other words, the diagram
\[
  \xymatrix{
  S^{-1}M\ar[r]^{S^{-1}u}\ar[d]_{\rho_M^{T,S}} & S^{-1}N\ar[d]^{\rho_N^{T,S}}\\
  T^{-1}M\ar[r]^{T^{-1}u} & T^{-1}N
  }
\]
\oldpage{16}
is commutative, for every homomorphism $u:M\to N$; $T^{-1}u$ is entirely determined by
$S^{-1}u$, since for $m\in M$ and $t\in T$, we have
\[
  (T^{-1}u)(m/t)=(t/1)^{-1}\rho^{T,S}((S^{-1}u)(m/1)).
\]
\end{env}

\begin{env}{1.4.2}
\label{env-0.1.4.2}
With the same notations, for two $A$-modules $M$, $N$, the diagrams (cf. \sref{env}{1.3.4}
and \sref{env}{1.3.5})
\[
  \xymatrix{
    (S^{-1}M)\otimes_{S^{-1}A}(S^{-1}N)\ar[r]^\sim \ar[d] & S^{-1}(M\otimes_A N)\ar[d] & &
    S^{-1}\Hom_A(M,N)\ar[r]\ar[d] & \Hom_{S^{-1}A}(S^{-1}M,S^{-1}N)\ar[d]\\
    (T^{-1}M)\otimes_{T^{-1}A}(T^{-1}N)\ar[r]^\sim & T^{-1}(M\otimes_A N) & &
    T^{-1}\Hom_A(M,N)\ar[r] & \Hom_{T^{-1}A}(T^{-1}M,T^{-1}N)
  }
\]
are commutative.
\end{env}

\begin{env}{1.4.3}
\label{env-0.1.4.3}
There is an important case in which the homomorphism $\rho^{T,S}$ is \emph{bijective},
we know that then every element of $T$ is divisor of an element of $S$; we then identify by
$\rho^{T,S}$ the modules $S^{-1}M$ and $T^{-1}M$. We say that $S$ is \emph{saturated} if
every divisor in $A$ of an element of $S$ is in $S$; by replacing $S$ with  the set $T$ of
all the divisors of the elements of $S$ (a set which is multiplicative and saturated), we see
that we can always, if we wish, be limited to the consideration of modules of fractions
$S^{-1}M$, where $S$ is saturated.
\end{env}

\begin{env}{1.4.4}
\label{env-0.1.4.4}
If $S$, $T$, $U$ are three multiplicative subsets of $A$ such that $S\subset T\subset U$, we
have
\[
  \rho^{U,S}=\rho^{U,T}\circ\rho^{T,S}.
\]
\end{env}

\begin{env}{1.4.5}
\label{env-0.1.4.5}
Consider an \emph{increasing filtered family} $(S_\alpha)$ of multiplicative subsets of $A$
(we write $\alpha\leqslant\beta$ for $S_\alpha\subset S_\beta$), and let $S$ be the
multiplicative subset $\bigcup_\alpha S_\alpha$; let us put
$\rho_{\beta\alpha}=\rho_A^{S_\beta,S_\alpha}$ for $\alpha\leq\beta$; according to
\sref{env}{1.4.4}, the homomorphisms $\rho_{\beta\alpha}$ define a ring $A'$ as the
\emph{inductive limit} of the inductive system of rings
$(S_\alpha^{-1}A,\rho_{\beta\alpha})$. Let $\rho_\alpha$ be the canonical map
$S_\alpha^{-1}A\to A'$, and let $\vphi_\alpha=\rho_A^{S,S_\alpha}$; as
$\vphi_\alpha=\vphi_\beta\circ\rho_{\beta\alpha}$ for $\alpha\leqslant\beta$ according to
\sref{env}{1.4.4}, we can uniquely define a homomorphism $\vphi:A'\to S^{-1}A$ such that the
diagram
\[
  \xymatrix{
    & S_\alpha^{-1}A\ar[ddl]_{\rho_\alpha}
                    \ar[d]^{\rho_{\beta\alpha}}
                    \ar[rdd]^{\vphi_\alpha}\\
    & S_\beta^{-1}A\ar[ld]^{\rho_\beta}
                   \ar[rd]_{\vphi_\beta}
    & & (\alpha\leqslant\beta)\\
    A'\ar[rr]_\vphi
    & & S^{-1}A
  }
\]
is commutative. In fact, $\vphi$ is an \emph{isomorphism}; it is indeed immediate by
construction that $\vphi$ is surjective. On the other hand, if
$\rho_\alpha(a/s_\alpha)\in A'$ is such that ${\vphi(\rho_\alpha(a/s_\alpha))=0}$, this means
that $a/s_\alpha=0$ in $S^{-1}A$, that is to say that there exists $s\in S$ such that
$sa=0$; but there is a $\beta\geqslant\alpha$ such that $s\in S_\beta$, and consequently, as
$\rho_\alpha(a/s_\alpha)=\rho_\beta(sa/ss_\alpha)=0$, we find that $\vphi$ is injective.
The case for an $A$-module $M$ is treated likewise, and thus we have defined canonical
isomorphisms
\[
  \varinjlim S_\alpha^{-1}A\isoto(\varinjlim S_\alpha)^{-1}A,\quad
  \varinjlim S_\alpha^{-1}M\isoto(\varinjlim S_\alpha)^{-1}M,
\]
the second being \emph{functorial} in $M$.
\end{env}

\begin{env}{1.4.6}
\label{env-0.1.4.6}
\oldpage{17}
Let $S_1$, $S_2$ be two multiplicative subsets of $A$; then $S_1 S_2$ is also a
multiplicative subset of $A$. Let us denote by $S_2'$ the canonical image of $S_2$ in the
ring $S_1^{-1}A$, which is a multiplicative subset of this ring. For every $A$-module $M$
there is then a functorial isomorphism
\[
  {S_2'}^{-1}(S_1^{-1}M)\isoto(S_1 S_2)^{-1}M
\]
which sends $(m/s_1)/(s_2/1)$ to the element $m/(s_1 s_2)$.
\end{env}

\subsection{Change of ring}
\label{0-prelim-1.5}

\begin{env}{1.5.1}
\label{env-0.1.5.1}
Let $A$, $A'$ be two rings, $\vphi$ a homomorphism $A'\to A$, $S$ (resp. $S'$)
a multiplicative subset of $A$ (resp. $A'$), such that $\vphi(S')\subset S$; the
composition homomorphism $\smash{A'\xrightarrow{\vphi} A\to S^{-1}A}$ factors as
$\smash{A'\to{S'}^{-1}\xrightarrow{\vphi^{S'}} S^{-1}A}$ by virtue of \sref{env}{1.2.4};
where $\vphi^{S'}(a'/s')=\vphi(a')/\vphi(s')$. If $A=\vphi(A')$ and
$S=\vphi(S')$, $\vphi^{S'}$ is \emph{surjective}. If $A'=A$ and if $\vphi$
is the identity, $\vphi^{S'}$ is none other than the homomorphism $\rho_A^{S,S'}$
defined in \sref{env}{1.4.1}.
\end{env}

\begin{env}{1.5.2}
\label{env-0.1.5.2}
Under the hypothesis of \sref{env}{1.5.1}, let $M$ be an $A$-module. There exists a canonical
functorial morphism
\[
  \sigma:{S'}^{-1}(M_{[\vphi]})\longrightarrow(S^{-1}M)_{[\vphi^{S'}]}
\]
of ${S'}^{-1}A'$-modules, sending each element $m/s'$ of ${S'}^{-1}(M_{[\vphi]})$ to
the element $m/\vphi(s')$ of $(S^{-1}M)_{[\vphi^{S'}]}$; in fact, we verify
immediately that this definition does not depend on the expression $m/s'$ of the element
considered. \emph{When $S=\vphi(S')$}, the homomorphism $\sigma$ is \emph{bijective}.
When $A'=A$ and $\vphi$ is the identity, $\sigma$ is none other than the homomorphism
$\rho_M^{S,S'}$ defined in \sref{env}{1.4.1}.

When $M=A$ is taken in particular, the homomorphism $\vphi$ defines on $A$ an $A'$-algebra
structure; ${S'}^{-1}(A_{[\vphi]})$ is then provided with a ring structure, for which it
identifies with $(\vphi(S'))^{-1}A$, and the homomorphism
${\sigma:{S'}^{-1}(A_{[\vphi]})\to S^{-1}A}$ is a homomorphism of ${S'}^{-1}A'$-algebras.
\end{env}

\begin{env}{1.5.3}
\label{env-0.1.5.3}
Let $M$ and $N$ be two $A$-modules; by composing the homomorphisms defined in
\sref{env}{1.3.4} and \sref{env}{1.5.2}, we obtain a homomorphism
\[
  (S^{-1}M\otimes_{S^{-1}A}S^{-1}N)_{[\vphi^{S'}]}
  \longleftarrow{S'}^{-1}((M\otimes A)_{[\vphi]})
\]
which is an isomorphism when $\vphi(S')=S$. Similarly, by composing the homomorphisms in
\sref{env}{1.3.5} and \sref{env}{1.5.2}, we obtain a homomorphism
\[
  {S'}^{-1}((\Hom_A(M,N))_{[\vphi]})
  \longrightarrow(\Hom_{S^{-1}A}(S^{-1}M,S^{-1}N))_{[\vphi^{S'}]}
\]
which is an isomorphism when $\vphi(S')=S$ and $M$ admits a finite presentation.
\end{env}

\begin{env}{1.5.4}
\label{env-0.1.5.4}
Let us now consider an $A'$-module $N'$, and form the tensor product
$N'\otimes_{A'}A_{[\vphi]}$, which can be considered as an $A$-module by setting
$a\cdot(n'\otimes b)=n'\otimes(ab)$. There is a functorial isomorphism of $S^{-1}A$-modules
\[
  \tau:({S'}^{-1}N')\otimes_{{S'}^{-1}A'}(S^{-1}A)_{[\vphi^{S'}]}
  \isoto S^{-1}(N'\otimes_{A'}A_{[\vphi]})
\]
\oldpage{18}
which sends the element $(n'/s')\otimes(a/s)$ to the element $(n'\otimes a)/(\vphi(s')s)$;
indeed, we verify separately that when we replace $n'/s'$ (resp. $a/s$) by another expression
of the same element, $(n'\otimes a)/(\vphi(s')s)$ does not change; on the other hand, we can
define an inverse homomorphism of $\tau$ by sending $(n'\otimes a)/s$ to the element
$(n'/1)\otimes(a/s)$: we use the fact that $S^{-1}(N'\otimes_{A'}A_{[\vphi]})$ is canonically
isomorphic to $(N'\otimes_{A'}A_{[\vphi]})\otimes_A S^{-1}A$ \sref{env}{1.2.5}, so also to
$N'\otimes_{A'}(S^{-1}A)_{[\psi]}$, denoting by $\psi$ the composite homomorphism
$a'\mapsto\vphi(a')/1$ of $A'$ into $S^{-1}A$.
\end{env}

\begin{env}{1.5.5}
\label{env-0.1.5.5}
If $M'$ and $N'$ are two $A'$-modules, by composing the isomorphisms \sref{env}{1.3.4} and
\sref{env}{1.5.4}, we obtain an isomorphism
\[
  {S'}^{-1}M\otimes_{{S'}^{-1}A'}{S'}^{-1}N'\otimes_{{S'}^{-1}A'}S^{-1}A
  \isoto S^{-1}(M'\otimes_{A'}N'\otimes_{A'}A).
\]
Likewise, if $M'$ admits a finite presentation, we have by \sref{env}{1.3.5} and
\sref{env}{1.5.4} an isomorphism
\[
  \Hom_{{S'}^{-1}A'}({S'}^{-1}M',{S'}^{-1}N')\otimes_{{S'}^{-1}A'}S^{-1}A
  \isoto S^{-1}(\Hom_{A'}(M',N')\otimes_{A'}A).
\]
\end{env}

\begin{env}{1.5.6}
\label{env-0.1.5.6}
Under the hypotheses of \sref{env}{1.5.1}, let $T$ (resp. $T'$) be a second multiplicative
subset of $A$ (resp. $A'$) such that $S\subset T$ (resp. $S'\subset T'$) and
$\vphi(T')\subset T$. Then the diagram
\[
  \xymatrix{
    {S'}^{-1}A'\ar[r]^{\vphi^{S'}}\ar[d]_{\rho^{T',S'}} & S^{-1}A\ar[d]^{\rho^{T,S}}\\
    {T'}^{-1}A'\ar[r]^{\vphi^{T'}} & T^{-1}A
  }
\]
is commutative. If $M$ is an $A$-module, the diagram
\[
  \xymatrix{
    {S'}^{-1}(M_{[\vphi]})\ar[r]^\sigma\ar[d]_{\rho^{T',S'}} &
    (S^{-1}M)_{[\vphi^{S'}]}\ar[d]^{\rho^{T,S}}\\
    {T'}^{-1}(M_{[\vphi]})\ar[r]^\sigma & (T^{-1}M)_{[\vphi^{T'}]}
  }
\]
is commutative. Finally, if $N'$ is an $A'$-module, the diagram
\[
  \xymatrix{
    ({S'}^{-1}N')\otimes_{{S'}^{-1}A'}(S^{-1}A)_{[\vphi^{S'}]}\ar[r]^\sim_\tau\ar[d] &
    S^{-1}(N'\otimes_{A'}A_{[\vphi]})\ar[d]^{\rho^{T,S}}\\
    ({T'}^{-1}N')\otimes_{{T'}^{-1}A'}(T^{-1}A)_{[\vphi^{T'}]}
    \ar[r]^\sim_\tau & T^{-1}(N'\otimes_{A'}A_{[\vphi]})
  }
\]
is commutative, the left vertical arrow obtained by applying
$\rho_{N'}^{T',S'}$ to ${S'}^{-1}N'$ and $\rho_A^{T,S}$ to $S^{-1}A$.
\end{env}

\begin{env}{1.5.7}
\label{env-0.1.5.7}
\oldpage{19}
Let $A''$ be a third ring, $\vphi':A''\to A'$ a ring homomorphism, $S''$ a multiplicative
subset of $A''$ such that $\vphi'(S'')\subset S'$. Set $\vphi''=\vphi\circ\vphi'$; then we
have
\[
  {\vphi''}^{S''}=\vphi^{S'}\circ{\vphi'}^{S''}.
\]
Let $M$ be an $A$-module; evidently we have $M_{[\vphi'']}=(M_{[\vphi]})_{[\vphi']}$;
if $\sigma'$ and $\sigma''$ are the homomorphisms defined by $\vphi'$ and $\vphi''$ as
$\sigma$ is defined in \sref{env}{1.5.2} by $\vphi$, we have the transitivity formula
\[
  \sigma''=\sigma\circ\sigma'.
\]

Finally, let $N''$ be an $A''$-module; the $A$-module $N''\otimes_{A''}A_{[\vphi'']}$
identifies canonically with $(N''\otimes_{A''}{A'}_{[\vphi']})\otimes_{A'}A_{[\vphi]}$,
and likewise the $S^{-1}A$-module
${({S''}^{-1}N'')\otimes_{{S''}^{-1}A''}(S^{-1}A)_{[{\vphi''}^{S''}]}}$ identifies
canonically with
$(({S''}^{-1}N'')\otimes_{{S''}^{-1}A''}({S'}^{-1}A')_{[{\vphi'}^{S''}]})
  \otimes_{{S'}^{-1}A'}(S^{-1}A)_{[\vphi^{S'}]}$. With these identifications, if $\tau'$
and $\tau''$ are the isomorphisms defined by $\vphi'$ and $\vphi''$ as $\tau$ is defined
in \sref{env}{1.5.4} by $\vphi$, we have the transitivity formula
\[
  \tau''=\tau\circ(\tau'\otimes 1).
\]
\end{env}

\begin{env}{1.5.8}
\label{env-0.1.5.8}
Let $A$ be a subring of a ring $B$; for every \emph{minimal} prime ideal $\mathfrak{p}$ of
$A$, there exists a minimal prime ideal $\mathfrak{q}$ of $B$ such that
$\mathfrak{p}=A\cap\mathfrak{q}$. Indeed, $A_\mathfrak{p}$ is a subring of $B_\mathfrak{p}$
\sref{env}{1.3.2} and has \emph{a single} prime ideal $\mathfrak{p}'$ \sref{env}{1.2.6};
since $B_\mathfrak{p}$ is not reduced to $0$, it has at least one prime ideal $\mathfrak{q}'$
and we have necessarily $\mathfrak{q}'\cap A_\mathfrak{p}=\mathfrak{p}'$; the prime ideal
$\mathfrak{q}_1$ of $B$, an inverse image of $\mathfrak{q}'$ is thus such that
$\mathfrak{q}_1\cap A=\mathfrak{p}$, and $\emph{a fortiori}$ we have
$\mathfrak{q}\cap A=\mathfrak{p}$ for every minimal prime ideal $\mathfrak{q}$ of $B$
contained in $\mathfrak{q}_1$.
\end{env}

\subsection{Indentification of the module $M_f$ as an inductive limit}
\label{0-prelim-1.6}

\begin{env}{1.6.1}
\label{env-0.1.6.1}
Let $M$ be an $A$-module, $f$ an element of $A$. Consider a sequence $(M_n)$ of $A$-modules,
all identical to $M$, and for each pair of integers $m\leqslant n$, let $\vphi_{nm}$ be the
homomorphism $z\mapsto f^{n-m}z$ of $M_m$ into $M_n$; it is immediate that
$((M_n),(\vphi_{nm}))$ is an \emph{inductive system} of $A$-modules; let $N=\varinjlim M_n$
be the inductive limit of this system. We define a canonical $A$-isomorphism,
\emph{functorial} of $N$ on $M_f$. For this reason, let us note that, for all $n$,
$\theta_n:z\mapsto z/f^n$ is an $A$-homomorphism of $M=M_n$ into $M_f$, and it follows from
the definitions that we have $\theta_n\circ\vphi_{nm}=\theta_m$ for $m\leqslant n$. There
exists therefore an $A$-homomorphism $\theta:N\to M_f$ such that, if $\vphi_n$ denotes the
canonical homomorphism $M_n\to N$, we have $\theta_n=\theta\circ\vphi_n$ for all $n$. Since,
by hypothesis, every element of $M_f$ is of the form $z/f^n$ for at least $n$, it is clear
that $\theta$ is surjective. On the other hand, if $\theta(\vphi_n(z))=0$, in other words
$z/f^n=0$, there exists an integer $k>0$ such that $f^k z=0$, so $\vphi_{n+k,n}(z)=0$, which
gives $\vphi_n(z)=0$. We can therefore identify $M_f$ and $\varinjlim M_n$ by means of
$\theta$.
\end{env}

\begin{env}{1.6.2}
\label{env-0.1.6.2}
Now write $M_{f,n}$, $\vphi_{nm}^f$ and $\vphi_n^f$ instead of $M_n$, $\vphi_{nm}$ and
$\vphi_n$. Let $g$ be a second element of $A$. As $f^n$ divides $f^n g^n$, we have a
functorial homomorphism
\[
  \rho_{fg,f}:M_f\longrightarrow M_{fg}\quad(\sref{env}{1.4.1}\text{ and }\sref{env}{1.4.3});
\]
\oldpage{20}
if we indentify $M_f$ and $M_{fg}$ with $\varinjlim M_{f,n}$ and $\varinjlim M_{fg,n}$
respectively, $\rho_{fg,f}$ identifies with the \emph{inductive limit} of the maps
$\rho_{fg,f}^n:M_{f,n}\to M_{fg,n}$, defined by $\rho_{fg,f}^n(z)=g^n z$. Indeed, this
follows immediately from the commutivity of the diagram
\[
  \xymatrix{
    M_{f,n}\ar[r]^{\rho_{fg,f}^n}\ar[d]_{\vphi_n^f} & M_{fg,n}\ar[d]^{\vphi_n^{fg}}\\
    M_f\ar[r]^{\rho_{fg,f}} & M_{fg}.
  }
\]
\end{env}

\subsection{Support of a module}
\label{0-prelim-1.7}

\begin{env}{1.7.1}
\label{env-0.1.7.1}
Given an $A$-module $M$, we call the \emph{support} of $M$ and denote by $\Supp(M)$
the set of prime ideals $\mathfrak{p}$ of $A$ such that $M_\mathfrak{p}\neq 0$. For $M=0$, it
is necessary and sufficient that $\Supp(M)=\emp$, because if $M_\mathfrak{p}=0$ for all
$\mathfrak{p}$, the annihilator of an element $x\in M$ cannot be contained in any prime
ideal of $A$, so $A$ is \unsure{total}.
\end{env}

\begin{env}{1.7.2}
\label{env-0.1.7.2}
If $0\to N\to M\to P\to 0$ is an exact sequence of $A$-modules, we have
\[
  \Supp(M)=\Supp(N)\cup\Supp(P)
\]
because for every prime ideal $\mathfrak{p}$ of $A$, the sequence
${0\to N_\mathfrak{p}\to M_\mathfrak{p}\to P_\mathfrak{p}\to 0}$ is exact \sref{env}{1.3.2}
and for $M_\mathfrak{p}=0$, it is necessary and sufficient that
$N_\mathfrak{p}=P_\mathfrak{p}=0$.
\end{env}

\begin{env}{1.7.3}
\label{env-0.1.7.3}
If $M$ is the sum of a family $(M_\lambda)$ of submodules, $M_\mathfrak{p}$ is the sum
of the $(M_\lambda)_\mathfrak{p}$ for every prime ideal $\mathfrak{p}$ of $A$
(\sref{env}{1.3.3} and \sref{env}{1.3.2}), so $\Supp(M)=\bigcup_\lambda\Supp(M_\lambda)$.
\end{env}

\begin{env}{1.7.4}
\label{env-0.1.7.4}
If $M$ is an $A$-module \emph{of finite type}, $\Supp(M)$ is the set of prime
ideals \emph{containing the annihilator of} $M$. Indeed, if $M$ is cyclic and
generated by $x$, we say that $M_\mathfrak{p}=0$ means that there exists
$s\not\in\mathfrak{p}$ such that $s\cdot x=0$, so that $\mathfrak{p}$ does not contain the
annihilator of $x$. If now $M$ admits a finite system $(x_i)_{1\leqslant i\leqslant n}$ of
generators and if $\mathfrak{a}_i$ is the annihilator of $x_i$, it follows from
\sref{env}{1.7.3} that $\Supp(M)$ is the set of $\mathfrak{p}$ containing one of
$\mathfrak{a}_i$, or, equivalently, the set of $\mathfrak{p}$ containing
$\mathfrak{a}=\bigcap_i\mathfrak{a}_i$, which is the annihilator of $M$.
\end{env}

\begin{env}{1.7.5}
\label{env-0.1.7.5}
If $M$ and $N$ are two $A$-modules \emph{of finite type}, we have
\[
  \Supp(M\otimes_A N)=\Supp(M)\cap\Supp(N).
\]
It can be seen that if $\mathfrak{p}$ is a prime ideal of $A$, the condition
$M_\mathfrak{p}\otimes_{A_\mathfrak{p}}N_\mathfrak{p}\neq 0$ is equivalent to
``$M_\mathfrak{p}\neq 0$ and $N_\mathfrak{p}\neq 0$'' (taking into account
\sref{env}{1.3.4}). In other words, it is about seeing that if $P$, $Q$ are two modules of
finite type over a \emph{local} ring $B$, not reduced to $0$, then $P\otimes_B Q\neq 0$. Let
$\mathfrak{m}$ be the maximal ideal of $B$. By virtue of Nakayama's lemma, the vector spaces
$P/\mathfrak{m}P$ and $Q/\mathfrak{m}Q$ are not reduced to $0$, so it is the same with the
tensor product
$(P/\mathfrak{m}P)\otimes_{B/\mathfrak{m}}(Q/\mathfrak{m}Q)
  =(P\otimes_B Q)\otimes_B(B/\mathfrak{m})$,
hence the conclusion.

In particular, if $M$ is an $A$-module of finite type, $\mathfrak{a}$ an ideal of $A$,
$\Supp(M/\mathfrak{a}M)$ is the set of prime ideals containing both $\mathfrak{a}$ and the
annihilator $\mathfrak{n}$ of $M$ \sref{env}{1.7.4}, that is, the set of prime ideals
containing $\mathfrak{a}+\mathfrak{n}$.
\end{env}

\section{Irreducible spaces. Noetherian spaces}
\label{0-prelim-2}

\subsection{Irreducible spaces}
\label{0-prelim-2.1}

\begin{env}{2.1.1}
\label{env-0.2.1.1}
\oldpage{21}
We say that a topological space $X$ is \emph{irreducible} if it is nonempty and if it is not
a union of two distinct closed subspaces of $X$. It is the same to say that $X\neq\emp$ and
that the intersection of two nonempty open sets (and consequently of a finite number of open
sets) of $X$ is nonempty, or that every nonempty open set is everywhere dense, or that any
closed set is \emph{rare}, or finally that all open sets of $X$ are \emph{connected}.
\end{env}

\begin{env}{2.1.2}
\label{env-0.2.1.2}
For a subspace $Y$ of a topological space $X$ to be irreducible, it is necessary and
sufficient that its closure $\overline{Y}$ be irreducible. In particular, any subspace which
is the closure $\overline{\{x\}}$ of a singleton is irreducible; we will express the relation
$y\in\overline{\{x\}}$ (equivalent to $\overline{\{y\}}\subset\overline{\{x\}}$) by saying
that there is a \emph{specialization of} $x$ or that there is a \emph{generalization of} $y$.
When there exists in an irreducible space $X$ a point $x$ such that $X=\overline{\{x\}}$, we
will say that $x$ is a \emph{generic point} of $X$. Any nonempty open subset of $X$ then
contains $x$, and any subspace containing $x$ admits $x$ for a generic point.
\end{env}

\begin{env}{2.1.3}
\label{env-0.2.1.3}
Recall that a \emph{Kolmogoroff space} is a topological space $X$ satisfying the axiom of
separation:

$(T_0)$ If $x\neq y$ are any two points of $X$, there is an open set containing one of the
points $x$, $y$ and not the other.

If an irreducible Kolmogoroff space admits a generic point, it admits \emph{only} one since a
nonempty open set contains any generic point.

Recall that a topological space $X$ is said to be \emph{quasi-compact} if, from any
collection of open sets of $X$, one can extract a finite cover of $X$ (or, equivalently, if
any decreasing filtered family of nonempty closed sets has a nonempty intersection). If $X$
is a quasi-compact space, then any nonempty closed subset $A$ of $X$ contains a
\emph{minimal} nonempty closed set $M$, because the set of nonempty closed subsets of $A$ is
inductive for the relation $\supset$; if in addition $X$ is a Kolmogoroff space, $M$ is
necessarily reduced to a single point (or, as we say by abuse of language, is a \emph{closed
point}).
\end{env}

\begin{env}{2.1.4}
\label{env-0.2.1.4}
In an irreducible space $X$, every nonempty open subspace $U$ is irreducible, and if $X$
admits a generic point $x$, $x$ is also a generic point of $U$.

Let $(U_\alpha)$ be a cover (whose set of indices is nonempty) of a topological space $X$,
consisting of nonempty open sets; if $X$ is irreducible, it is necessary and sufficient that
$U_\alpha$ is irreducible for all $\alpha$, and that $U_\alpha\cap U_\beta\neq\emp$ for any
$\alpha$, $\beta$. The condition is clearly necessary; to the that it is sufficient, it
suffices to prove that if $V$ is a nonempty open subset of $X$, then $V\cap U_\alpha$ is
nonempty for all $\alpha$, since then $V\cap U_\alpha$ is dense in $U_\alpha$ for all
$\alpha$, and consequently $V$ is dense in $X$. Now there is at least one index $\gamma$ such
that $V\cap U_\gamma\neq\emp$, so $V\cap U_\gamma$ is dense in $U_\gamma$, and as for all
$\alpha$, $U_\alpha\cap V_\alpha\neq\emp$, we also have
$V\cap U_\alpha\cap U_\gamma\neq\emp$.
\end{env}

\begin{env}{2.1.5}
\label{env-0.2.1.5}
\oldpage{22}
Let $X$ be an irreducible space, $f$ a continuous map of $X$ into a topological space $Y$.
Then $f(X)$ is irreducible, and if $x$ is a generic point of $X$, $f(x)$ is a generic point
of $f(X)$ and hence also of $\overline{f(X)}$. In particular, if in addition $Y$ is
irreducible and with a single generic point $y$, for $f(X)$ to be everywhere dense, it is
necessary and sufficient that $f(x)=y$.
\end{env}

\begin{env}{2.1.6}
\label{env-0.2.1.6}
Any irreducible subspace of a topological space $X$ is contained in a maximal irreducible
subspace, which is necessarily closed. Maximal irreducible subspaces of $X$ are called the
\emph{irreducible components} of $X$. If $Z_1$, $Z_2$ are two irreducible components distinct
from the space $X$, $Z_1\cap Z_2$ is a closed \emph{rare} set in each of the subspaces $Z_1$,
$Z_2$; in particular, if an irreducible component of $X$ admits a generic point
\sref{env}{2.1.2} such a point can not belong to any other irreducible component. If $X$ has
only a \emph{finite} number of irreducible components $Z_i$ ($1\leqslant i\leqslant n$), and
if, for each $i$, we put $U_i=Z_i\cap\mathrm{C}(\bigcup_{j\neq i}Z_j)$, the $U_i$ are open,
irreducible, disjoint, and their union is dense in $X$. Let $U$ be an open subset of a
topological space $X$. If $Z$ is an irreducible subset of $X$ that intersects with $U$,
$Z\cap U$ is open and dense in $Z$, thus irreducible; conversely, for any irreducible closed
subset $Y$ of $U$, the closure $\overline{Y}$ of $Y$ in $X$ is irreducible and
$\overline{Y}\cap U=Y$. We conclude that there is a \emph{bijective correspondence} between
the irreducible components of $U$ and the irreducible components of $X$ which intersect $U$.
\end{env}

\begin{env}{2.1.7}
\label{env-0.2.1.7}
If a topological space $X$ is a union of a \emph{finite} number of irreducible closed
subspaces $Y_i$, the irreducible components of $X$ are the maximal elements of the set of
$Y_i$, because if $Z$ is an irreducible closed subset of $X$, $Z$ is the union of the
$Z\cap Y_i$, from which one sees that $Z$ must be contained in one of the $Y_i$. Let $Y$ be a
subspace of a topological space $X$, and suppose that $Y$ has only a finite number of
irreducible components $Y_i$, ($1\leqslant i\leqslant n$); then the closures $\overline{Y_i}$
in $X$ are the irreducible components of $Y$.
\end{env}

\begin{env}{2.1.8}
\label{env-0.2.1.8}
Let $Y$ be an irreducible space admitting a single generic point $y$. Let $X$ be a
topological space, $f$ a continuous map from $X$ to $Y$. Then, for any irreducible
component $Z$ of $X$ intersecting $f^{-1}(y)$, $f(Z)$ is dense in $Y$. The converse is not
necessarily true; however, if $Z$ has a generic point $z$, and if $f(Z)$ is dense in $Y$, we
must have $f(z)=y$ \sref{env}{2.1.5}; in addition, $Z\cap f^{-1}(y)$ is then the closure of
$\{z\}$ in $f^{-1}(y)$ and is therefore irreducible, and like any irreducible subset of
$f^{-1}(y)$ containing $z$ is necessarily contained in $Z$ \sref{env}{2.1.6}, $z$ is a
generic point of $Z\cap f^{-1}(y)$. As any irreducible component of $f^{-1}(y)$ is contained
in an irreducible component of $X$, we see that if any irreducible component $Z$ of $X$
intersecting $f^{-1}(y)$ admits a generic point, then there is a \emph{bijective
correspondence} between all these components and all the irreducible components
$Z\cap f^{-1}(y)$ of $f^{-1}(y)$, the generic points of $Z$ being identical to those of
$Z\cap f^{-1}(y)$.
\end{env}

\subsection{Noetherian spaces}
\label{0-prelim-2.2}

\begin{env}{2.2.1}
\label{env-0.2.2.1}
\oldpage{23}
We say that a topological space $X$ is \emph{Noetherian} if the set of open subsets of $X$
satisfies the \emph{maximal} condition, or, equivalently, if the set of closed subsets of $X$
satisfies the \emph{minimal} condition. We say that $X$ is \emph{locally Noetherian} if all
$x\in X$ admit a neighborhood which is a Noetherian subspace.
\end{env}

\begin{env}{2.2.2}
\label{env-0.2.2.2}
Let $E$ be an ordered set satisfying the \emph{minimal} condition, and let $\mathbf{P}$ be a
property of the elements of $E$ subject to the following condition: if $a\in E$ is such that
for any $x<a$, $\mathbf{P}(x)$ is true, then $\mathbf{P}(a)$ is true. Under these conditions,
$\mathbf{P}(x)$ \emph{is true for all} $x\in E$ (``principle of Noetherian recurrence'').
Indeed, let $F$ be the set of $x\in E$ for which $\mathbf{P}(x)$ is false; if $F$ were not
empty, it would have a minimal element $a$, and as then $\mathbf{P}(x)$ is true for all
$x<a$, $\mathbf{P}(a)$ would be true, which is a contradiction.

We will apply this principle in particular when $E$ is a \emph{set of closed subsets of a
Noetherian space}.
\end{env}

\begin{env}{2.2.3}
\label{env-0.2.2.3}
Any subspace of a Noetherian space is Noetherian. Conversely, any topological space that is a
finite union of Noetherian subspaces is Noetherian.
\end{env}

\begin{env}{2.2.4}
\label{env-0.2.2.4}
Any Noetherian space is quasi-compact; conversely, any  topological space in which all open
sets are quasi-compact is Noetherian.
\end{env}

\begin{env}{2.2.5}
\label{env-0.2.2.5}
A Noetherian space has only a \emph{finite} number of irreducible components, as we see by
Noetherian recurrence.
\end{env}

\section{Supplement on sheaves}
\label{0-prelim-3}

\subsection{Sheaves with values in a category}
\label{0-prelim-3.1}

\begin{env}{3.1.1}
\label{env-0.3.1.1}
Let $\K$ be a category, $(A_\alpha)_{\alpha\in I}$,
$(A_{\alpha\beta})_{(\alpha,\beta)\in I\times I}$ two families of objects of $\K$ such
that $A_{\beta\alpha}=A_{\alpha\beta}$, and
$(\rho_{\alpha\beta})_{(\alpha,\beta)\in I\times I}$ is a family of morphisms
$\rho_{\alpha\beta}:A_\alpha\to A_{\alpha\beta}$. We say that a pair formed by an object $A$
of $\K$ and a family of morphisms $\rho_\alpha:A\to A_\alpha$ is a \emph{solution to the
universal problem} defined by the data of the families $(A_\alpha)$, $(A_{\alpha\beta})$,
and $(\rho_{\alpha\beta})$ if, for every object $B$ of $\K$, the map which sends
$f\in\Hom(B,A)$ to the family
$(\rho_\alpha\circ f)\in\Pi_\alpha\Hom(B,A_\alpha)$ is a \emph{bijection} of $\Hom(B,A)$ to
the set of all $(f_\alpha)$ such that
$\rho_{\alpha\beta}\circ f_\alpha=\rho_{\beta\alpha}\circ f_\beta$ for any pair of indices
$(\alpha,\beta)$. If such a solution exists, it is unique up to an isomorphism.
\end{env}

\begin{env}{3.1.2}
\label{env-0.3.1.2}
We will not recall the defintion of a \emph{presheaf} $U\mapsto\sh{F}(U)$ on a topological
space $X$ with values in a category $\K$ (G, I, 1.9); we say that such a presheaf is a
\emph{sheaf with values in} $\K$ if it satifies the following axiom:\\

(F) \emph{For any covering $(U_\alpha)$ of an open set $U$ of $X$ by open sets $U_\alpha$
    contained in $U$, if we denote by $\rho_\alpha$ (resp. $\rho_{\alpha\beta}$) the
    restriction morphism}
    \[
      \sh{F}(U)\to\sh{F}(U_\alpha)
      \quad(\text{\emph{resp. }}\sh{F}(U_\alpha)\to\sh{F}(U_\alpha\cap U_\beta)),
    \]
\oldpage{24}
    \emph{the pair formed by $\sh{F}(U)$ and the family $(\rho_\alpha)$ are a solution to
    the universal problem for $(\sh{F}(U_\alpha))$, $(\sh{F}(U_\alpha\cap U_\beta))$, and
    $(\rho_{\alpha\beta})$ in} \sref{env}{3.1.1}\footnote{This is a special case of the more
    general notion of a \emph{projective limit} (non-filtered) (\emph{see} (T, I, 1.8) and
    the book in preparation announced in the introduction).}.\\

Equivalently, we can say that, for each object $T$ of $\K$, the family
$U\mapsto\Hom(T,\sh{F}(U))$ is a \emph{sheaf of sets}.
\end{env}

\begin{env}{3.1.3}
\label{env-0.3.1.3}
Assume that $\K$ is the category defined by a ``type of structure with morphisms''
$\Sigma$,
the objects of $\K$ being the sets with structures of type $\Sigma$ and morphisms those of
$\Sigma$. Suppose that the category $\K$ also satisfies the following condition:\\

(E) If $(A,(\rho_\alpha))$ is a solution of a universal mapping problem \emph{in the
    category $\K$} for families $(A_\alpha)$, $(A_{\alpha\beta})$, $(\rho_{\alpha\beta})$,
    then it is also a solution of the universal mapping problem for the same families
    \emph{in the category of sets} (that is, when we consider $A$, $A_\alpha$, and
    $A_{\alpha\beta}$ as sets, $\rho_\alpha$ and $\rho_{\alpha\beta}$ as
    functions)\footnote{It can be proved that it also means that the canonical functor
    $\K\to(\mathrm{Set})$ \emph{commutes with projective limits} (not necessarily
    filtered).}.\\

Under these conditions, the condition (F) gives that, when considered as a presheaf \emph{of
sets}, $U\mapsto\sh{F}(U)$ is a \emph{sheaf}. In addition, for a map $u:T\to\sh{F}(U)$ to be
a morphism of $\K$, it is necessary and sufficient, under (F), that each map
$\rho_\alpha\circ u$ is a morphism $T\to\sh{F}(U_\alpha)$, which means that the structure of
type $\Sigma$ on $\sh{F}(U)$ is the \emph{initial structure} for the morphisms $\rho_\alpha$.
Conversely, suppose a presheaf $U\mapsto\sh{F}(U)$ on $X$, with values in $\K$, is a
\emph{sheaf of sets} and satisfies the previous condition; it is then clear that it satisfies
(F), so it is a \emph{sheaf with values in} $\K$.
\end{env}

\begin{env}{3.1.4}
\label{env-0.3.1.4}
When $\Sigma$ is a type of a group or ring structure, the fact that the presheaf
$U\mapsto\sh{F}(U)$ with values in $\K$ is a sheaf of \emph{sets} leads \emph{ipso facto}
that it is a sheaf with values in $\K$ (in other words, a sheaf of groups or rings within the
meaning of (G))\footnote{This is because in the category $\K$, any morphism that is a
\emph{bijection} (as a map of sets) is an \emph{isomorphism}. This is no longer true when
$\K$ is the category of topological spaces, for example.}. But it is not the same when, for
example, $\K$ is the category of \emph{topological rings} (with morphisms as continuous
homomorphisms): a sheaf with values in $\K$ is a sheaf of rings $U\mapsto\sh{F}(U)$ such that
for any open $U$ and any covering of $U$ by open sets $U_\alpha\subset U$, the topology of
the ring $\sh{F}(U)$ is to be \emph{the least fine}, making the homomorpisms
$\sh{F}(U)\to\sh{F}(U_\alpha)$ continuous. We will say in this case that $U\mapsto\sh{F}(U)$,
considered as a sheaf of rings (without a topology), is \emph{underlying} the sheaf of
topological rings $U\mapsto\sh{F}(U)$. Morphisms $u_V:\sh{F}(V)\to\sh{G}(V)$ ($V$ an
arbitrary open subset of $X$) of sheaves of topological rings are therefore homomorphisms of
the underlying sheaves of rings, such that $u_V$ be \emph{continuous} for all open
$V\subset X$; to distinguish them from any homomorphisms of the sheaves of the underlying
rings, we will call them continuous homomorphisms of sheaves of topological rings. We have
similar definitions and conventions for sheaves of topological spaces or topological groups.
\end{env}

\begin{env}{3.1.5}
\label{env-0.3.1.5}
\oldpage{25}
It is clear that for any category $\K$, if there is a presheaf (respectively a sheaf)
$\sh{F}$ on $X$ with values in $\K$ and $U$ is an open set of $X$, the $\sh{F}(V)$ for open
$V\subset U$ constitute a presheaf (or a sheaf) with values in $\K$, which we call the
presheaf (or sheaf) \emph{induced} by $\sh{F}$ on $U$ and denote it by $\sh{F}|U$.

For any morphism $u:\sh{F}\to\sh{G}$ of presheaves on $X$ with values in $\K$, we denote by
$u|U$ the morphism $\sh{F}|U\to\sh{G}|U$ formed by the $u_V$ for $V\subset U$.
\end{env}

\begin{env}{3.1.6}
\label{env-0.3.1.6}
Suppose now that the category $\K$ admits \emph{inductive limits} (T, 1.8); then, for any
presheaf (and in particular any sheaf) $\sh{F}$ on $X$ with values in $\K$ and all $x\in X$,
we can define the \emph{stalk} $\sh{F}_x$ as the object of $\K$ defined by the inductive
limit of the $\sh{F}(U)$ with respect to the filtering set (for $\supset$) of the open
neighborhoods $U$ of $x$ in $X$, and for the morphisms $\rho_U^V:\sh{F}(V)\to\sh{F}(U)$. If
$u:\sh{F}\to\sh{G}$ is a morphism of presheaves with values in $\K$, we define for all
$x\in X$ the morphism $u_x:\sh{F}_x\to\sh{G}_x$ as the inductive limit of
$u_U:\sh{F}(U)\to\sh{G}(U)$ with respect to all open neighborhoods of $x$; we thus define
$\sh{F}_x$ as a covariant functor in $\sh{F}$, with values in $\K$, for all $x\in X$.

When $\K$ is further defined by a kind of structure with morphisms $\Sigma$, we call
\emph{sections over $U$} of a \emph{sheaf} $\sh{F}$ with values in $\K$ the elements of
$\sh{F}(U)$, and we write $\Gamma(U,\sh{F})$ instead of $\sh{F}(U)$; for
$s\in\Gamma(U,\sh{F})$, $V$ an open set contained in $U$, we write $s|V$ instead of
$\rho_V^U(s)$; for all $x\in U$, the canonical image of $s$ in $\sh{F}_x$ is the \emph{germ}
of $s$ at the point $x$, denoted by $s_x$ (\emph{we will never replace the notation $s(x)$ in
this sense,} this notation being reserved for another notion relating to sheaves which will
be considered in this treatise \sref{env}{5.5.1}).

If then $u:\sh{F}\to\sh{G}$ is a morphism of sheaves with values in $\K$, we will write
$u(s)$ instead of $u_V(s)$ for all $s\in\Gamma(U,\sh{F})$.

If $\sh{F}$ is a sheaf of commutative groups, or rings, or modules, we say that the set of
$x\in X$ such that $\sh{F}_x\neq\{0\}$ is the \emph{support} of $\sh{F}$, denoted
$\Supp(\sh{F})$; this set is not necessarily closed in $X$.

When $\K$ is defined by a type of structure with morphisms, \emph{we systematically refrain
from using the point of view of ``\'etal\'e spaces''} in terms of relating to sheaves with
values in $\K$; in other words, we will never consider a sheaf as a topological space (nor
even as the whole union of its fibers), and we will not consider also a morphism
$u:\sh{F}\to\sh{G}$ of such sheaves on $X$ as a continuous map of topological spaces.
\end{env}

\subsection{Presheaves on an open basis}
\label{0-prelim-3.2}

\begin{env}{3.2.1}
\label{env-0.3.2.1}
We will restrict to the following categories $\K$ admitting \emph{projective limits}
(generalized, that is, corresponding to not necessarily filtered preordered sets,
cf. (T, 1.8)). Let $X$ be a topological space, $\mathfrak{B}$ an open basis for the topology
of $X$. We will call a \emph{presheaf on $\mathfrak{B}$, with values in $\K$,} to be a family
of objects $\sh{F}(U)\in\K$, corresponding to each $U\in\mathfrak{B}$, and a family of
morphisms $\rho_U^V:\sh{F}(V)\to\sh{F}(U)$ defined for any pair $(U,V)$ of elements of
$\mathfrak{B}$ such that $U\subset V$,
\oldpage{26}
with the conditions $\rho_U^U=$ identity and $\rho_U^W=\rho_U^V\circ\rho_V^W$ if $U$, $V$,
$W$ in $\mathfrak{B}$ are such that $U\subset V\subset W$. We can associate a \emph{presheaf
with values in} $\K$: $U\mapsto\sh{F}(U)$ in the ordinary sense, taking for all open $U$,
$\sh{F}'(U)=\varprojlim\sh{F}(V)$, where $V$ runs through the ordered set (for $\subset$,
\emph{not filtered} in general) of $V\in\mathfrak{B}$ sets such that $V\subset U$, since the
$\sh(V)$ form a projective system for the $\rho_V^W$ ($V\subset W\subset U$,
$V\in\mathfrak{B}$, $W\in\mathfrak{B}$). Indeed, if $U$, $U'$ are two open sets of $X$ such
that $U\subset U'$, we define ${\rho'}_U^{U'}$ as the projective limit (for $V\subset U$) of
the canonical morphisms $\sh{F}'(U')\to\sh{F}(V)$, in other words the unique morphism
$\sh{F}'(U')\to\sh{F}'(U)$, which, when composed with the canonical morphisms
$\sh{F}'(U)\to\sh{F}(V)$, gives the canonical morphisms $\sh{F}'(U')\to\sh{F}(V)$; the
verification of the transitivity of ${\rho'}_U^{U'}$ is then immediate. Moreover, if
$U\in\mathfrak{B}$, the canonical morphism $\sh{F}'(U)\to\sh{F}(U)$ is an isomorphism,
allowing to identify these two objects\footnote{If $X$ is a \emph{Noetherian} space, we can
still define $\sh{F}'(U)$ and show that it is a presheaf (in the ordinary sense) when one
supposes only that $\K$ admits projective limits for \emph{finite} projective systems.
Indeed, if $U$ is any open set of $X$, there is a \emph{finite} covering $(V_i)$ of $U$
formed by sets of $\mathfrak{B}$; for every couple $(i,j)$ of indices, let $(V_{ijk})$ be a
finite covering of $V_i\cap V_j$ formed by sets of $\mathfrak{B}$. Let $I$ be the set of $i$
and triples $(i,j,k)$, ordered only by the relations $i>(i,j,k)$, $j>(i,j,k)$; we then take
$\sh{F}'(U)$ to be the projective limit of the system of $\sh{F}(V_i)$ and $\sh{F}(V_{ijk})$;
it is easy to verify that this does not depend on the coverings $(V_i)$ and $(V_{ijk})$ and
that $U\mapsto\sh{F}'(U)$ is a presheaf.}.
\end{env}

\begin{env}{3.2.2}
\label{env-0.3.2.2}
For the presheaf $\sh{F}'$ thus defined to be a \emph{sheaf}, it is necessary and sufficient
that the presheaf $\sh{F}$ on $\mathfrak{B}$ satisfies the condition:\\

(F$_0$) \emph{For any covering $(U_\alpha)$ of $U\in\mathfrak{B}$ by sets
        $U_\alpha\in\mathfrak{B}$ contained in $U$, and for any object $T\in\K$, the map
        which takes $f\in\Hom(T,\sh{F}(U))$ to the family
        $(\rho_{U_\alpha}^U\circ f)\in\Pi_\alpha\Hom(T,\sh{F}(U_\alpha))$ is a bijection of
        $\Hom(T,\sh{F}(U))$ on the set of all $(f_\alpha)$ such that
        $\rho_V^{U_\alpha}\circ f_\alpha=\rho_V^{U_\beta}\circ f_\beta$ for any pair of
        indices $(\alpha,\beta)$ and any $V\in\mathfrak{B}$ such that
        $V\subset U_\alpha\cap U_\beta$\footnote{It also means that the pair formed by
        $\sh{F}(U)$ and the $\rho_\alpha=\rho_{U_\alpha}^U$ is a \emph{solution to the
        universal problem} defined in \sref{env}{3.1.1} by the data of
        $A_\alpha=\sh{F}(U_\alpha)$, $A_{\alpha\beta}=\Pi\sh{F}(V)$ (for $V\in\mathfrak{B}$
        such that $V\subset U_\alpha\cap U_\beta$) and
        $\rho_{\alpha\beta}=(\rho_V''):\sh{F}(U_\alpha)\to\Pi\sh{F}(V)$ defined by the
        condition that for $V\in\mathfrak{B}$, $V'\in\mathfrak{B}$, $W\in\mathfrak{B}$,
        $V\cup V'\subset U_\alpha\cap U_\beta$, $W\subset V\cap V'$,
        $\rho_W^V\circ\rho_V''=\rho_W^{V'}\circ\rho_{V'}''$.}.}\\

The condition is obviously necessary. To show that it is sufficient, consider first a second
basis $\mathfrak{B}'$ of the topology of $X$, \emph{contained in} $\mathfrak{B}$, and show
that if $\sh{F}''$ denotes the presheaf induced by the subfamily
$(\sh{F}(V))_{V\in\mathfrak{B}'}$, $\sh{F}''$ is \emph{canonically isomorphic} to $\sh{F}'$.
Indeed, firstly the projective limit (for $V\in\mathfrak{B}'$, $V\subset U$) canonical
morphisms ${\sh{F}'(U)\to\sh{F}(V)}$ is a morphism $\sh{F}'(U)\to\sh{F}''(U)$ for all open
$U$. If $U\in\mathfrak{B}$, this morphism is an isomorphism, because by hypothesis the
canonical morphisms $\sh{F}''(U)\to\sh{F}(V)$ for $V\in\mathfrak{B}'$, $V\subset U$,
factorize into $\sh{F}''(U)\to\sh{F}(U)\to\sh{F}(V)$, and it is immediate to see that the
composition of morphisms $\sh{F}(U)\to\sh{F}''(U)$ and $\sh{F}''(U)\to\sh{F}(U)$ thus defined
are the identities. This being so, for all open $U$, the morphisms
$\sh{F}''(U)\to\sh{F}''(W)=\sh{F}(W)$ for $W\in\mathfrak{B}$ and $W\subset U$ satisfy the
conditions characterizing the projective limit of $\sh{F}(W)$ ($W\in\mathfrak{B}$,
$W\subset U$), which proves our assertion given the uniqueness of a projective limit up
to isomorphism.

This posed, let $U$ be any open set of $X$, $(U_\alpha)$ a covering of $U$ by the open sets
contained in $U$, and $\mathfrak{B}'$ the subfamily of $\mathfrak{B}$ formed by the sets
\oldpage{27}
of $\mathfrak{B}$ contained in at least $U_\alpha$; it is clear that $\mathfrak{B}'$ is still
a basis of the topology of $X$, so $\sh{F}'(U)$ (resp. $\sh{F}''(U_\alpha)$) is the
projective limit of $\sh{F}(V)$ for $V\in\mathfrak{B}'$ and $V\subset U$
(resp., $V\subset U_\alpha$), the axiom (F) is then immediately verified by virtue of the
definition of the projective limit.

When (F$_0$) is satisfied, we will say by abuse of language that the presheaf $\sh{F}$ on the
basis $\mathfrak{B}$ is a sheaf.
\end{env}

\begin{env}{3.2.3}
\label{env-0.3.2.3}
Let $\sh{F}$, $\sh{G}$ be two presheaves on a basis $\mathfrak{B}$, with values in $\K$; we
define a \emph{morphism} $u:\sh{F}\to\sh{G}$ as a family $(u_V)_{V\in\mathfrak{B}}$ of
morphisms $u_V:\sh{F}(V)\to\sh{G}(V)$ satisfying the usual compatibility conditions with the
restriction morphisms $\rho_V^W$. With the notation of \sref{env}{3.2.1}, we have a morphism
$u':\sh{F}'\to\sh{G}'$ of (ordinary) presheaves by taking for $u_U'$ the projective limit of
the $u_V$ for $V\in\mathfrak{B}$ and $V\subset U$; the verification of the compatibility
conditions with the ${\rho'}_U^{U'}$ follows from the functorial properties of the projective
limit.
\end{env}

\begin{env}{3.2.4}
\label{env-0.3.2.4}
If the category $\K$ admits inductive limits, and if $\sh{F}$ is a presheaf on the basis
$\mathfrak{B}$, with values in $\K$, for each $x\in X$ the neighborhoods of $x$ belonging to
$\mathfrak{B}$ form a cofinal set (for $\supset$) in the set of neighborhoods of $x$,
therefore, if $\sh{F}'$ is the (ordinary) presheaf corresponding to $\sh{F}$, the stalk
$\sh{F}_x'$ is equal to $\varinjlim_{\mathfrak{B}}\sh{F}(V)$ over the set of
$V\in\mathfrak{B}$ containing $x$. If $u:\sh{F}\to\sh{G}$ is morphism of presheaves on
$\mathfrak{B}$ with values in $\K$, $u':\sh{F}'\to\sh{G}'$ the corresponding morphism of
ordinary presheaves, $u_x'$ is likewise the inductive limit of the morphisms
$u_V:\sh{F}(V)\to\sh{G}(V)$ for $V\in\mathfrak{B}$, $x\in V$.
\end{env}

\begin{env}{3.2.5}
\label{env-0.3.2.5}
We return to the general conditions of \sref{env}{3.2.1}. If $\sh{F}$ is an ordinary
\emph{sheaf} with values in $\K$, $\sh{F}_1$ the sheaf \emph{on} $\mathfrak{B}$ obtained
by the restriction of $\sh{F}$ to $\mathfrak{B}$, the ordinary sheaf $\sh{F}_1'$ obtained
from $\sh{F}_1$ by the procedure of \sref{env}{3.2.1} is canonically isomorphic to $\sh{F}$,
by virtue of the condition (F) and the uniqueness properties of the projective limit. We
identify the ordinary sheaf $\sh{F}$ with $\sh{F}_1'$.

If $\sh{G}$ is a second (ordinary) sheaf on $X$ with values in $\K$, and $u:\sh{F}\to\sh{G}$
a morphism, the preceding remark shows that the data of the $u_V:\sh{F}(V)\to\sh{G}(V)$
\emph{for only the $V\in\mathfrak{B}$} completely determines $u$; conversely, it is
sufficient, the $u_V$ being given for $V\in\mathfrak{B}$, to verify the commutative
diagram with the restriction morphisms $\rho_V^W$ for $V\in\mathfrak{B}$, $W\in\mathfrak{B}$,
and $V\subset W$, for there to exist a morphism $u'$ and a unique $\sh{F}$ in $\sh{G}$ such
that $u_V'=u_V$ for each $V\in\mathfrak{B}$ \sref{env}{3.2.3}.
\end{env}

\begin{env}{3.2.6}
\label{env-0.3.2.6}
Suppose that $\K$ admits projective limits. Then the category of \emph{sheaves on $X$
with values in $\K$} admits \emph{projective limits}; if $(\sh{F}_\lambda)$ is a
projective system of sheaves on $X$ with values in $\K$, the
$\sh{F}(U)=\varprojlim_\lambda\sh{F}_\lambda(U)$ indeed define a presheaf with values in
$\K$, and the verification of the axiom (F) follows from the transitivity of projective
limits; the fact that $\sh{F}$ is then the projective limit of the $\sh{F}_\lambda$ is
immediate.

When $\K$ is the category of sets, for each projective system $(\mathfrak{H}_\lambda)$ such
\oldpage{28}
that $\mathfrak{H}_\lambda$ is a \emph{subsheaf} of $\sh{F}_\lambda$ for each $\lambda$,
$\varprojlim_\lambda\mathfrak{H}_\lambda$ canonically identifies with a \emph{subsheaf} of
$\varprojlim_\lambda\sh{F}_\lambda$. If $\K$ is the category of abelian groups, the covariant
functor $\varprojlim_\lambda\sh{F}_\lambda$ is \emph{additive} and \emph{left exact}.
\end{env}

\subsection{Gluing of sheaves}
\label{0-prelim-3.3}

\begin{env}{3.3.1}
\label{env-0.3.3.1}
Suppose still that the category $\K$ admits (generalized) projective limits. Let $X$ be
a topological space, $\mathfrak{U}=(U_\lambda)_{\lambda\in L}$ an open cover of $X$, and for
each $\lambda\in L$, let $\sh{F}_\lambda$ be a sheaf on $U_\lambda$, with values in $\K$; for
each pair of indices $(\lambda,\mu)$, suppose that we are given an \emph{isomorphism}
$\theta_{\lambda\mu}:\sh{F}_\mu|(U_\lambda\cap U_\mu)\isoto\sh{F}|(U_\lambda\cap U_\mu)$; in
addition, suppose that for each triple $(\lambda,\mu,\nu)$, if we denote by
$\theta_{\lambda\mu}'$, $\theta_{\mu\nu}'$, $\theta_{\lambda\nu}'$ the restrictions of
$\theta_{\lambda\mu}$, $\theta_{\mu\nu}$, $\theta_{\lambda\nu}$ to
$U_\lambda\cap U_\mu\cap U_\nu$, then we have
$\theta_{\lambda\nu}'=\theta_{\lambda\mu}'\circ\theta_{\mu\nu}'$ (\emph{gluing condition} for
the $\theta_{\lambda\mu}$). Then, there exists a sheaf $\sh{F}$ on $X$, with values in $\K$,
and for each $\lambda$ an isomorphism
$\eta_\lambda:\sh{F}|U_\lambda\isoto\sh{F}_\lambda$ such that, for each pair
$(\lambda,\mu)$, if we denote by $\eta_\lambda'$ and $\eta_\mu'$ the restrictions of
$\eta_\lambda$ and $\eta_\mu$ to $U_\lambda\cap U_\mu$, then we have
$\theta_{\lambda\mu}=\eta_\lambda'\circ{\eta_\mu'}^{-1}$; in addition, $\sh{F}$ and the
$\eta_\lambda$ are determined up to unique isomorphism by these conditions. The uniqueness
indeed follows immediately from \sref{env}{3.2.5}. To establish the existence of $\sh{F}$,
denote by $\mathfrak{B}$ the open basis consisting of the open sets contained in at least one
$U_\lambda$, and for each $U\in\mathfrak{B}$, choose (by the Hilbert function $\tau$) one of
the $\sh{F}_\lambda(U)$ for one of the $\lambda$ such that $U\subset U_\lambda$; if we denote
this object by $\sh{F}(U)$, the $\rho_U^V$ for $U\subset V$, $U\in\mathfrak{B}$,
$V\in\mathfrak{B}$ are defined in an evident way (by means of the $\theta_{\lambda\mu}$), and
the transitivity conditions is a consequence of the gluing condition; in addition, the
verification of (F$_0$) is immediate, so the presheaf on $\mathfrak{B}$ thus clearly defines
a sheaf, and we deduce by the general procedure \sref{env}{3.2.1} an (ordinary) sheaf still
denoted $\sh{F}$ and which answers the question. We say that $\sh{F}$ is obtained by
\emph{gluing the $\sh{F}_\lambda$ by means of the $\theta_{\lambda\mu}$} and we usually
indentify the $\sh{F}_\lambda$ and $\sh{F}|U_\lambda$ by means of the $\eta_\lambda$.

It is clear that each sheaf $\sh{F}$ on $X$ with values in $\K$ can be considered as being
obtained by the gluing of the sheaves $\sh{F}_\lambda=\sh{F}|U_\lambda$ (where $(U_\lambda)$
is an arbitrary open cover of $X$), by means of the isomorphisms $\theta_{\lambda\mu}$
reduced to the identity.
\end{env}

\begin{env}{3.3.2}
\label{env-0.3.3.2}
With the same notations, let $\sh{G}_\lambda$ be a second sheaf on $U_\lambda$ (for each
$\lambda\in L$) with values in $\K$, and for each pair $(\lambda,\mu)$ let us be given an
isomorphism
$\omega_{\lambda\mu}:\sh{G}_\mu|(U_\lambda\cap U_\mu)
\isoto\sh{G}_\lambda|(U_\lambda\cap U_\mu)$, these isomorphisms satisfying the
gluing condition. Finally, suppose that we are given for each $\lambda$ a morphism
$u_\lambda:\sh{F}_\lambda\to\sh{G}_\lambda$, and that the diagrams
\[
  \xymatrix{
    \sh{F}_\mu|(U_\lambda\cap U_\mu)\ar[r]^{u_\mu}\ar[d] &
    \sh{G}_\mu|(U_\lambda\cap U_\mu)\ar[d]\\
    \sh{F}_\lambda|(U_\lambda\cap U_\mu)\ar[r]^{u_\lambda} &
    \sh{G}_\lambda|(U_\lambda\cap U_\mu)
  }
  \tag{3.3.2.1}
\]
are commutative. Then, if $\sh{G}$ is obtained by gluing the $\sh{G}_\lambda$ by means of the
$\omega_{\lambda\mu}$, there exists a unique morphism $u:\sh{F}\to\sh{G}$ such that the
diagrams
\oldpage{29}
\[
  \xymatrix{
    \sh{F}|U_\lambda\ar[r]^{u|U_\lambda}\ar[d] &
    \sh{G}|U_\lambda\ar[d]\\
    \sh{F}_\lambda\ar[r]^{u_\lambda} &
    \sh{G}_\lambda
  }
\]
are commutative; this follows immediately from \sref{env}{3.2.3}. The correspondence between
the family $(u_\lambda)$ and $u$ is in a functorial bijection with the subset of
$\Pi_\lambda\Hom(\sh{F}_\lambda,\sh{G}_\lambda)$ satisfying the conditions (3.3.2.1) on
$\Hom(\sh{F},\sh{G})$.
\end{env}

\begin{env}{3.3.3}
\label{env-0.3.3.3}
With the notations of \sref{env}{3.3.1}, let $V$ be an open set of $X$; it is immediate that
the restrictions to $V\cap U_\lambda\cap U_\mu$ of the $\theta_{\lambda\mu}$ satisfy the
gluing condition for the induced sheaves $\sh{F}_\lambda|(V\cap U_\lambda)$ and that the
sheaves on $V$ obtained by gluing the latter identifies canonically with $\sh{F}|V$.
\end{env}

\subsection{Direct images of presheaves}
\label{0-prelim-3.4}

\begin{env}{3.4.1}
\label{env-0.3.4.1}
Let $X$, $Y$ be two topological spaces, $\psi:X\to Y$ a continuous map. Let $\sh{F}$ be a
presheaf on $X$ with values in a category $\K$; for each open $U\subset Y$, let
$\sh{G}(U)=\sh{F}(\psi^{-1}(U))$, and if $U$, $V$ are two open subsets of $Y$ such that
$U\subset V$, let $\rho_U^V$ be the morphism $\sh{F}(\psi^{-1}(V))\to\sh{F}(\psi^{-1}(U))$;
it is immediate that the $\sh{G}(U)$ and the $\rho_U^V$ define a \emph{presheaf} on $Y$ with
values in $\K$, that we call the \emph{direct image of $\sh{F}$ by $\psi$} and we denote it
by $\psi_*(\sh{F})$. If $\sh{F}$ is a sheaf, we immediately verify the axiom (F) for the
presheaf $\sh{G}$, so $\psi_*(\sh{F})$ is a sheaf.
\end{env}

\begin{env}{3.4.2}
\label{env-0.3.4.2}
Let $\sh{F}_1$, $\sh{F}_2$ be two presheaves of $X$ with values in $\K$, and let
$u:\sh{F}_1\to\sh{F}_2$ be a morphism. When $U$ varies over the set of open subsets of $Y$,
the family of morphisms $u_{\psi^{-1}(U)}:\sh{F}_1(\psi^{-1}(U))\to\sh{F}_2(\psi^{-1}(U))$
satisfies the compatibility conditions with the restriction morphisms, and as a result
defines a a morphism $\psi_*(u):\psi_*(\sh{F}_1)\to\psi_*(\sh{F}_2)$. If
$v:\sh{F}_2\to\sh{F}_3$ is a morphism from $\sh{F}_2$ to a third preshead on $X$ with values
in $\K$, we have $\psi_*(v\circ u)=\psi_*(v)\circ\psi_*(u)$; in other words, $\psi_*(\sh{F})$
is a \emph{covariant functor} in $\sh{F}$, from the category of presheaves (resp. sheaves) on
$X$ with values in $\K$, to that of presheaves (resp. sheaves) on $Y$ with values in $\K$.
\end{env}

\begin{env}{3.4.3}
\label{env-0.3.4.3}
Let $Z$ be a third topological space, $\psi':Y\to Z$ a continuous map, and let
$\psi''=\psi'\circ\psi$. It is clear that we have $\psi_*''(\sh{F})=\psi_*'(\psi_*(\sh{F}))$
for each presheaf $\sh{F}$ on $X$ with values in $\K$; in addition, for each morphism
$u:\sh{F}\to\sh{G}$ of such presheaves, we have $\psi_*''(u)=\psi_*'(\psi_*(u))$. In other
words, $\psi_*''$ is the \emph{composition} of the functors $\psi_*'$ and $\psi_*$, and this
can be written as
\[
  (\psi'\circ\psi)_*=\psi_*'\circ\psi_*.
\]

In addition, for each open set $U$ of $Y$, the image under the restriction
$\psi|\psi^{-1}(U)$ of the induced presheaf $\sh{F}|\psi^{-1}(U)$ is none other than the
induced presheaf $\psi_*(\sh{F})|U$.
\end{env}

\begin{env}{3.4.4}
\label{env-0.3.4.4}
Suppose that the category $\K$ admits inductive limits, and let $\sh{F}$ be a presheaf on $X$
with values in $\K$; for all $x\in X$, the morphisms $\Gamma(\psi^{-1}(U),\sh{F})\to\sh{F}_x$
($U$ an open neighborhood of $\psi(x)$ in $Y$) form an inductive limit, which gives by
passing
\oldpage{30}
to the limit a morphism $\psi_x:(\psi_*(\sh{F}))_{\psi(x)}\to\sh{F}_x$ of the stalks; in
general, these morphisms are \emph{neither injective or surjective}. It is functorial;
indeed, if $u:\sh{F}_1\to\sh{F}_2$ is a morphism of presheaves on $X$ with values in $\K$,
the diagram
\[
  \xymatrix{
    (\psi_*(\sh{F}_1))_{\psi(x)}\ar[r]^{\psi_x}\ar[d]_{(\psi_*(u))_{\psi(x)}} &
    (\sh{F}_1)_x\ar[d]^{u_x}\\
    (\psi_*(\sh{F}_2))_{\psi(x)}\ar[r]^{\psi_x} &
    (\sh{F}_2)_x
  }
\]
is commutative. If $Z$ is a third topological space, $\psi':Y\to Z$ a continuous map, and
$\psi''=\psi'\circ\psi$, we have $\psi_x''=\psi_x\circ\psi_{\psi(x)}'$ for $x\in X$.
\end{env}

\begin{env}{3.4.5}
\label{env-0.3.4.5}
Under the hypotheses of \sref{env}{3.4.4}, suppose in addition that $\psi$ is a
\emph{homeomorphism} from $X$ to the subspace $\psi(X)$ of $Y$. Then, for each $x\in X$,
$\psi_x$ is an \emph{isomorphism}. This applies in particular to the canonical injection $j$
of a subset $X$ of $Y$ into $Y$.
\end{env}

\begin{env}{3.4.6}
\label{env-0.3.4.6}
Suppose that $\K$ be the category of groups, or of rings, etc. If $\sh{F}$ is a sheaf on $X$
with a values in $\K$, of support $S$, and if $y\not\in\overline{\psi(S)}$, it follows from
the definition of $\psi_*(\sh{F})$ that $(\psi_*(\sh{F}))_y=\{0\}$, or in other words the
support of $\psi_*(\sh{F})$ is contained in $\overline{\psi(S)}$; but it is not necessarily
contined in $\psi(S)$. Under the same hypotheses, if $j$ is the canonical injection of a
subset $X$ of $Y$ into $Y$, the sheaf $j_*(\sh{F})$ induces $\sh{F}$ on $X$; if moreover
$X$ is \emph{closed} in $Y$, $j_*(\sh{F})$ is the sheaf on $Y$ which induces $\sh{F}$ on $X$
and $0$ on $Y-X$ (G, II, 2.9.2), but it is in general distinct from the latter when we
suppose that $X$ is locally closed but not closed.
\end{env}

\subsection{Inverse images of presheaves}
\label{0-prelim-3.5}

\begin{env}{3.5.1}
\label{env-0.3.5.1}
Under the hypotheses of \sref{env}{3.4.1}, if $\sh{F}$ (resp. $\sh{G}$) is a presheaf on $X$
(resp. $Y$) with values in $\K$, each morphism $u:\sh{G}\to\psi_*(\sh{F})$ of presheaves on
$Y$ is then called a \emph{$\phi$-morphism} from $\sh{G}$ to $\sh{F}$, and we denote it also
by $\sh{G}\to\sh{F}$. We denote also by $\Hom_\phi(\sh{G},\sh{F})$ the set of
$\Hom_Y(\sh{G},\psi_*(\sh{F}))$ the $\psi$-morphisms from $\sh{G}$ to $\sh{F}$. For each
pair $(U,V)$, where $U$ is an open set of $X$, $V$ an open set of $Y$ such that
$\psi(U)\subset V$, we have a morphism $u_{U,V}:\sh{G}(U)\to\sh{F}(U)$ by composing the
restriction morphism $\sh{F}(\psi^{-1}(V))\to\sh{F}(U)$ and the morphism
$u_V:\sh{G}(V)\to\psi_*(\sh{F})(V)=\sh{F}(\psi^{-1}(V))$; it is immediate that these
morphisms renders commutative the diagrams
\[
  \xymatrix{
    \sh{G}(V)\ar[r]^{u_{U,V}}\ar[d] &
    \sh{F}(U)\ar[d]\\
    \sh{G}(V')\ar[r]^{u_{U',V'}} &
    \sh{F}(U')
  }
  \tag{3.5.1.1}
\]
for $U'\subset U$, $V'\subset V$, $\psi(U')\subset V'$. Conversely, the data of a family
$(u_{U,V})$ of morphisms rendering commutative the diagrams (3.5.1.1) define a
$\psi$-morphism $u$, since it suffices to take $u_V=u_{\psi^{-1}(V),V}$.

\oldpage{31}
If the category $\K$ admits (generalized) projective limits, and if $\mathfrak{B}$,
$\mathfrak{B}'$ are the bases of the topologies of $X$ and $Y$ respectively, to define a
$\psi$-morphism $u$ of \emph{sheaves}, we can restrict to giving the $u_{U,V}$ for
$U\in\mathfrak{B}$, $V\in\mathfrak{B}'$, and $\psi(U)\subset V$, satisfying the compatibility
conditions of (3.5.1.1) for $U$, $U'$ in $\mathfrak{B}$ and $V$, $V'$ in $\mathfrak{B}'$; it
indeed suffices to define $u_W$, for each open $W\subset Y$, as the projective limit of the
$u_{U,V}$ for $V\in\mathfrak{B}'$ and $V\subset W$, $U\in\mathfrak{B}$ and
$\psi(U)\subset V$.

When the category $\K$ admits inductive limits, we have, for each $x\in X$, a morphism
$\sh{G}(V)\to\sh{F}(\psi^{-1}(V))\to\sh{F}_x$, for each open neighborhood $V$ of $\psi(x)$ in
$Y$, and these morphisms form an inductive system which gives by passing to the limit a
morphism $\sh{G}_{\psi(x)}\to\sh{F}_x$.
\end{env}

\begin{env}{3.5.2}
\label{env-0.3.5.2}
Under the hypotheses of \sref{env}{3.4.3}, let $\sh{F}$, $\sh{G}$, $\sh{H}$ be presheaves
with values in $\K$ on $X$, $Y$, $Z$ respectively, and let $u:\sh{G}\to\psi_*(\sh{F})$,
$v:\sh{H}\to\psi_*'(\sh{G})$ a $\psi$-morphism and a $\psi'$-morphism respectively. We obtain
a $\psi''$-morphism
$w:\sh{H}\xrightarrow{v}\psi_*'(\sh{G})\xrightarrow{\psi_*'(u)}
\psi_*'(\psi_*(\sh{F}))=\psi_*''(\sh{F})$, that we call, by definition, the
\emph{composition} of $u$ and $v$. We can therefore consider the pairs $(X,\sh{F})$
consisting of a topological space $X$ and a presheaf $\sh{F}$ on $X$ (with values in $\K$) as
forming a \emph{category}, the morphisms being the pairs
$(\psi,\theta):(X,\sh{F})\to(Y,\sh{G})$ consisting of a continuous map $\psi:X\to Y$ and
of a $\psi$-morphism $\theta:\sh{G}\to\sh{F}$.
\end{env}

\begin{env}{3.5.3}
\label{env-0.3.5.3}
Let $\psi:X\to Y$ be a continuous map, $\sh{G}$ a \emph{presheaf} on $Y$ with values in $\K$.
We call the \emph{inverse image of $\sh{G}$ under $\psi$} the pair $(\sh{G}',\rho)$, where
$\sh{G}'$ is a \emph{sheaf} on $X$ with values in $\K$, and $\rho:\sh{G}\to\sh{G}'$ a
$\psi$-morphism (in other words a homomorphism $\sh{G}\to\psi_*(\sh{G}')$) such that, for
each \emph{sheaf} $\sh{F}$ on $X$ with values in $\K$, the map
\[
  \Hom_X(\sh{G}',\sh{F})\longrightarrow\Hom_\psi(\sh{G},\sh{F})
  \longrightarrow\Hom_Y(\sh{G},\psi_*(\sh{F}))
  \tag{3.5.3.1}
\]
sending $v$ to $\psi_*(v)\circ\rho$, is a \emph{bijection}; this map, is functorial in
$\sh{F}$, defines then an isomorphism of functors in $\sh{F}$. The pair $(\sh{G}',\rho)$
is the solution of a universal problem, and we say it is \emph{determined up to unique
isomorphism} when it exists. We then write $\sh{G}'=\psi^*(\sh{G})$, $\rho=\rho_\sh{G}$, and
by abuse of language, we say that $\psi^*(\sh{G})$ is \emph{the inverse image sheaf} of
$\sh{G}$ under $\psi$, and we agree that $\psi^*(\sh{G})$ is considered as equipped with a
\emph{canonical $\psi$-morphism $\rho_\sh{G}:\sh{G}\to\psi^*(\sh{G})$}, that is to say the
\emph{canonical homomorphism} of presheaves on $Y$:
\[
  \rho_\sh{G}:\sh{G}\longrightarrow\psi_*(\psi^*(\sh{G})).
  \tag{3.5.3.2}
\]

For each homomorphism $v:\psi^*(\sh{G})\to\sh{F}$ (where $\sh{F}$ is a sheaf on $X$ with
values in $\K$), we put $v^\flat=\psi_*(v)\circ\rho_\sh{G}:\sh{G}\to\psi_*(\sh{F})$. By
definition, \emph{each} morphism of presheaves $u:\sh{G}\to\psi_*(\sh{F})$ is of the form
$v^\flat$ for a unique $v$, which we will denote $u^\sharp$. In other words, each morphism
$u:\sh{G}\to\psi_*(\sh{F})$ of presheaves factorizes in a unique way as
\[
  u:\sh{G}\xrightarrow{\rho_\sh{G}}\psi_*(\psi^*(\sh{G}))
  \xrightarrow{\psi_*(u^\sharp)}\psi_*(\sh{F}).
  \tag{3.5.3.3}
\]
\end{env}

\begin{env}{3.5.4}
\label{env-0.3.5.4}
\oldpage{32}
Suppose now that the category $\K$ be such\footnote{In the book mentioned in the
introduction, we will give very general conditions on the category $\K$ ensuring the
existence of inverse images of presheaves with values in $\K$.} that \emph{each} presheaf
$\sh{F}$ on $Y$ with values in $\K$ admits an inverse image under $\psi$, and we denote it
by $\psi^*(\sh{G})$.

We will see that we can define $\psi^*(\sh{G})$ as a \emph{covariant functor} in $\sh{G}$,
from the category of presheaves on $Y$ with values in $\K$, to that of sheaves on $X$ with
values in $\K$, in such a way that the isomorphism $v\mapsto v^\flat$ is an \emph{isomorphism
of bifunctors}
\[
  \Hom_X(\psi^*(\sh{G}),\sh{F})\isoto\Hom_Y(\sh{G},\psi_*(\sh{F}))
  \tag{3.5.4.1}
\]
in $\sh{G}$ and $\sh{F}$.

Indeed, for each morphism $w:\sh{G}_1\to\sh{G}_2$ of presheaves on $Y$ with values in $\K$,
consider the composite morphism
$\sh{G}_1\xrightarrow{w}\sh{G}_2\xrightarrow{\rho_{\sh{G}_2}}\psi_*(\psi^*(\sh{G}_2))$; to it
corresponds a morphism $(\rho_{\sh{G}_2}\circ w)^\sharp:\psi^*(\sh{G}_1)\to\psi^*(\sh{G}_2)$,
that we denote by $\psi^*(w)$. We therefore have, according to (3.5.3.3),
\[
  \psi_*(\psi^*(w))\circ\rho_{\sh{G}_1}=\rho_{\sh{G}_2}\circ w.
  \tag{3.5.4.2}
\]
For each morphism $u:\sh{G}_2\to\psi_*(\sh{F})$, where $\sh{F}$ is a sheaf on $X$ with values
in $\K$, we have, according to (3.5.3.3), (3.5.4.2), and the definition of $u^\flat$
\[
  (u^\sharp\circ\psi^*(w))^\flat=\psi_*(u^\sharp)\circ\psi_*(\psi^*(w))\circ\rho_{\sh{G}_1}
  =\psi_*(u^\sharp)\circ\rho_{\sh{G}_2}\circ w=u\circ w
\]
where again
\[
  (u\circ w)^\sharp=u^\sharp\circ\psi^*(w).
  \tag{3.5.4.3}
\]

If we take in particular for $u$ a morphism
$\sh{G}_2\xrightarrow{w'}\sh{G}_3\xrightarrow{\rho_{\sh{G}_3}}\psi_*(\psi^*(\sh{G}_3))$, it
becomes
$\psi^*(w'\circ w)=(\rho_{\sh{G}_3}\circ w'\circ w)^\sharp
=(\rho_{\sh{G}_3}\circ w')^\sharp\circ\psi^*(w)=\psi^*(w')\circ\psi^*(w)$, hence our
assertion.

FInally, for each sheaf $\sh{F}$ on $X$ with values in $\K$, let $i_\sh{F}$ be the identity
morphism of $\psi_*(\sh{F})$ and denote by
\[
  \sigma_\sh{F}:\psi^*(\psi_*(\sh{F})\longrightarrow\sh{F}
\]
the morphism $(i_\sh{F})^\sharp$; the formula (3.5.4.3) gives in particular the factorization
\[
  u^\sharp:\psi^*(\sh{G})\xrightarrow{\psi^*(u)}\psi^*(\psi_*(\sh{F}))
  \xrightarrow{\sigma_\sh{F}}\sh{F}
  \tag{3.5.4.4}
\]
for each morphism $u:\sh{G}\to\psi_*(\sh{F})$. We say that the morphism $\sigma_\sh{F}$ is
\emph{canonical}.
\end{env}

\begin{env}{3.5.5}
\label{env-0.3.5.5}
Let $\psi':Y\to Z$ be a continuous map, and suppose that each presheaf $\sh{H}$ on $Z$ with
values in $\K$ admits an inverse image ${\psi'}^*(\sh{H})$ under $\psi'$. Then (with the
hypotheses of \sref{env}{3.5.4}) each presheaf $\sh{H}$ on $Z$ with values in $\K$ admits
an inverse image under $\psi''=\psi'\circ\psi$ and we have a canonical functorial isomorphism
\[
  {\psi''}^*(\sh{H})\isoto\psi^*({\psi'}^*(\sh{H})).
  \tag{3.5.5.1}
\]
\oldpage{33}
This indeed follows immediately from the definitions, taking into account that
$\psi_*''=\psi_*'\circ\psi_*$. In addition, if $u:\sh{G}\to\psi_*(\sh{F})$ is a
$\psi$-morphism, $v:\sh{H}\to\psi_*'(\sh{G})$ a $\psi'$-morphism, and $w=\psi_*'(u)\circ v$
their composition \sref{env}{3.5.2}, we have immediately that $w^\sharp$ is the composite
morphism
\[
  w^\sharp:\psi^*({\psi'}^*(\sh{H}))\xrightarrow{\psi^*(v^\sharp)}\psi^*(\sh{G})
  \xrightarrow{u^\sharp}\sh{F}.
\]
\end{env}

\begin{env}{3.5.6}
\label{env-0.3.5.6}
We take in particular for $\psi$ the identity map $1_X:X\to X$. Then if the inverse image
under $\psi$ of a presheaf $\sh{F}$ on $X$ with values in $\K$ exists, we say that this
inverse image is the \emph{sheaf associated to the presheaf $\sh{F}$}. Each morphism
$u:\sh{F}\to\sh{F}'$ from $\sh{F}$ to a \emph{sheaf} $\sh{F}'$ with values in $\K$ factorizes
in a unique way as
$\sh{F}\xrightarrow{\rho_\sh{F}}1_X^*(\sh{F})\xrightarrow{u^\sharp}\sh{F}'$.
\end{env}

\subsection{Simple sheaves and locally simple sheaves}
\label{0-prelim-3.6}

\begin{env}{3.6.1}
\label{env-0.3.6.1}
We say that a \emph{presheaf} $\sh{F}$ on $X$, with values in $\K$, is \emph{constant} if
the canonical morphisms $\sh{F}(X)\to\sh{F}(U)$ are \emph{isomorphisms} for each nonempty
open $U\subset X$; we note that $\sh{F}$ is not necessarily a sheaf. We say that a
\emph{sheaf} is \emph{simple} if it is the associated sheaf \sref{env}{3.5.6} of a constant
presheaf. We say that a sheaf $\sh{F}$ is \emph{locally simple} if each $x\in X$ admits an
open neighborhood $U$ such that $\sh{F}|U$ is simple.
\end{env}

\begin{env}{3.6.2}
\label{env-0.3.6.2}
Suppose that $X$ is \emph{irreducible} \sref{env}{2.1.1}; then the following properties are
equivalent:
\begin{enumerate}[label=(\alph*)]
  \item \emph{$\sh{F}$ is a constant presheaf on $X$};
  \item \emph{$\sh{F}$ is a simple sheaf on $X$};
  \item \emph{$\sh{F}$ is a locally simple sheaf on $X$}.
\end{enumerate}
\end{env}

Indeed, let $\sh{F}$ be a constant presheaf on $X$; if $U$, $V$ are two nonempty open sets in
$X$, $U\cap V$ is nonempty, so $\sh{F}(X)\to\sh{F}(U)\to\sh{F}(U\cap V)$ and
$\sh{F}(X)\to\sh{F}(U)$ are isomorphisms, and similarly both
$\sh{F}(U)\to\sh{F}(U\cap V)$ and $\sh{F}(V)\to\sh{F}(U\cap V)$ are isomorphisms. We
therefore conclude immediatelt that the axiom (F) of \sref{env}{3.1.2} is clearly satisfied,
$\sh{F}$ is isomorphic to its associated sheaf, and as a result (a) implies (b).

Now let $(U_\alpha)$ be an open cover of $X$ by nonempty open sets and let $\sh{F}$ be a
sheaf on $X$ such that $\sh{F}|U_\alpha$ is simple for each $\alpha$; as $U_\alpha$ is
irreducible, $\sh{F}|U_\alpha$ is a constant presheaf according to the above. As
$U_\alpha\cap U_\beta$ is not empty, $\sh{F}(U_\alpha)\to\sh{F}(U_\alpha\cap U_\beta)$ and
$\sh{F}(U_\beta)\to\sh{F}(U_\alpha\cap U_\beta)$ are isomorphisms, hence we have a canonical
isomorphism $\theta_{\alpha\beta}:\sh{F}(U_\alpha)\to\sh{F}(U_\beta)$ for each pair of
indices. But them if we apply the condition (F) for $U=X$, we see that for each index
$\alpha_0$, $\sh{F}(U_{\alpha_0})$ and the $\theta_{\alpha_0\alpha}$ are solutions to the
universal problem, which (according to the uniqueness) implies that
$\sh{F}(X)\to\sh{F}(U_{\alpha_0})$ is an isomorphism, and hence proves that (c) implies (a).

\subsection{Inverse images of presheaves of groups or rings}
\label{0-prelim-3.7}

\begin{env}{3.7.1}
\label{env-0.3.7.1}
\oldpage{34}
We will show that when we take $\K$ to be the category of sets, the inverse image under
$\psi$ for each presheaf $\sh{G}$ with values in $\K$ \emph{always exists} (the notations
and hypotheses on $X$, $Y$, $\psi$ being that of \sref{env}{3.5.3}). Indeed, for each open
$U\subset X$, define $\sh{G}'(U)$ as follows: an element $s'$ of $\sh{G}'(U)$ is a family
$(s_x')_{x\in U}$, where $s_x'\in\sh{G}_{\psi(x)}$ for each $x\in U$, and where, for each
$x\in U$, the following condition is satisfied: there exists an open neighborhood $V$ of
$\psi(x)$ in $Y$, a neighborhood $W\subset\psi^{-1}(V)\cap U$ of $x$, and an element
$s\in\sh{G}(V)$ such that $s_z'=s_{\psi(x)}$ for all $z\in W$. We verify immediately that
$U\mapsto\sh{G}'(U)$ clearly satisfies the axioms of a \emph{sheaf}.

Now let $\sh{F}$ be a sheaf of sets on $X$, and let $u:\sh{G}\to\psi_*(\sh{F})$,
$v:\sh{G}'\to\sh{F}$ be morphisms. We define $u^\sharp$ and $v^\flat$ in the following
manner: if $s'$ is a section of $\sh{G}'$ over a neighborhood $U$ of $x\in X$ and if $V$ is
an open neighborhood of $\psi(x)$ and $s\in\sh{G}(V)$ such that we have $s_z'=s_{\psi(x)}$
for $z$ in a neighborhood of $x$ contained in $\psi^{-1}(V)\cap U$, we take
$u_x^\sharp(s_x')=u_{\psi(x)}(s_{\psi(x)})$. Similarly, if $s\in\sh{G}(V)$ ($V$ open in $Y$),
$v^\flat(s)$ is the section of $\sh{F}$ over $\psi^{-1}(V)$, the image under $v$ of the
sectin $s'$ of $\sh{G}'$ such that $s_x'=s_{\psi(x)}$ for all $x\in\psi^{-1}(V)$. In
addition, the canonical homomorphism \sref{env}{3.5.3} $\rho:\sh{G}\to\psi_*(\psi^*(\sh{G}))$
is defined in the following manner: for each open $V\subset Y$ and each section
$s\in\Gamma(V,\sh{G})$, $\rho(s)$ is the section $(s_{\psi(x)})_{x\in\psi^{-1}(V)}$ of
$\psi^*(\sh{G})$ over $\psi^{-1}(V)$. The verification of the relations $(u^\sharp)^\flat=u$,
$(v^\flat)^\sharp=v$, and $v^\flat=\psi_*(v)\circ\rho$ is immediate, and proves our
assertion.

We check that, if $w:\sh{G}_1\to\sh{G}_2$ is a homomorphism of sheaves of sets on $Y$,
$\psi^*(w)$ is expressed in the following manner: if $s'=(s_x')_{x\in U}$ is a section of
$\psi^*(\sh{G}_1)$ over an open set $U$ of $X$, $(\psi^*(w))(s')$ is the family
$(w_{\psi(x)}(s_x'))_{x\in U}$. Finally, it is immediate that for each open set $V$ of $Y$,
the inverse image of $\sh{G}|V$ under the restriction of $\psi$ to $\psi^{-1}(V)$ is
identical to the induced sheaf $\psi^*(\sh{G})|\psi^{-1}(V)$.

When $\psi$ is the identity $1_X$, we recover the definition of a sheaf of sets associated
to a presheaf (G, II, 1.2). The above considerations apply without change when $\K$ is the
category of groups or of rings (not necessarily commutative).

When $X$ is any subset of a topological space $Y$, and $j$ the canonical injection $X\to Y$,
for each sheaf $\sh{G}$ on $Y$ with values in a category $\K$, we call the \emph{induced}
sheaf of $X$ by $\sh{G}$ the inverse image $j^*(\sh{G})$ (whenever it exists); for the
sheaves of sets (or of groups, or of rings) we recover the usual definition (G, II, 1.5).
\end{env}

\begin{env}{3.7.2}
\label{env-0.3.7.2}
Keeping the notations and hypotheses of \sref{env}{3.5.3}, suppose that $\sh{G}$ is a
\emph{sheaf} of groups (resp. of rings) on $Y$. The definition of sections of
$\psi^*(\sh{G})$ \sref{env}{3.7.1} shows (considering \sref{env}{3.4.4}) that the
homomorphism of stalks $\psi_x\circ\rho_{\psi(x)}:\sh{G}_{\psi(x)}\to(\psi^*(\sh{G}))_x$
is a \emph{functorial isomorphism in $\sh{G}$}, that identifies the two stalks; with this
identification, $u_x^\sharp$ is identical to the homomorphism defined in \sref{env}{3.5.1},
and in particular, we have $\Supp(\psi^*(\sh{G}))=\psi^{-1}(\Supp(\sh{G}))$.

An immediate consequence of this result is that \emph{the functor $\psi^*(\sh{G})$ is exact
in $\sh{G}$} in the abelian category of sheaves of abelian groups.
\end{env}

\subsection{Sheaves on pseudo-discrete spaces}
\label{0-prelim-3.8}

\begin{env}{3.8.1}
\label{env-0.3.8.1}
\oldpage{35}
Let $X$ be a topological space whose topology admits a basis $\mathfrak{B}$ consisting of
open \emph{quasi-compact} subsets. Let $\sh{F}$ be a \emph{sheaf of sets} on $X$; if we equip
each of the $\sh{F}(U)$ with the \emph{discrete} topology, $U\mapsto\sh{F}(U)$ is a
\emph{presheaf of topological spaces}. We will see that there exists a \emph{sheaf of
topological spaces $\sh{F}'$ associated to $\sh{F}$} \sref{env}{3.5.6} such that
$\Gamma(U,\sh{F}')$ is the discrete space $\sh{F}(U)$ for each open \emph{quasi-compact}
subsets $U$. It will suffice to show that the presheaf $U\mapsto\sh{F}(U)$ of discrete
topological spaces \emph{on} $\mathfrak{B}$ satisfy the condition (F$_0$) of
\sref{env}{3.2.2}, and more generally that if $U$ is an open quasi-compact subset and if
$(U_\alpha)$ is a cover of $U$ by sets of $\mathfrak{B}$, then the least fine topology
$\mathscr{T}$ on $\Gamma(U,\sh{F})$ renders continuous the maps
$\Gamma(U,\sh{F})\to\Gamma(U_\alpha,\sh{F})$ is the \emph{discrete} topology. There exists
a finite number of indices $\alpha_i$ such that $U=\bigcup_i U_{\alpha_i}$.
Let $s\in\Gamma(U,\sh{F})$ and let $s_i$ be its image in $\Gamma(U_{\alpha_i},\sh{F})$;
the intersection of the inverse images of the sets $\{s_i\}$ is by definition a neighborhood
of $s$ for $\mathscr{T}$; but since $\sh{F}$ is a sheaf of sets and the $U_{\alpha_i}$ cover
$U$, this intersection is reduced to $s$, hence our assertion.

We note that if $U$ is an open non quasi-compact subset of $X$, the topological space
$\Gamma(U,\sh{F}')$ still has $\Gamma(U,\sh{F})$ as the underlying set, but the topology is
not discrete in general: it is the least fine rendering commutative the maps
$\Gamma(U,\sh{F})\to\Gamma(V,\sh{F})$, for $V\in\mathfrak{B}$ and $V\subset U$ (the
$\Gamma(V,\sh{F})$ being discrete).

The above considerations apply without modification to sheaves of groups or of rings (not
necessarily commutative), and associate to them sheaves of \emph{topological
groups} or \emph{topological rings}, respectively. To summarize, we say that the sheaf
$\sh{F}'$ is the \emph{pseudo-discrete} sheaf of \emph{spaces} (resp. \emph{groups},
\emph{rings}) associated to a sheaf of sets (resp. groups, rings) $\sh{F}$.
\end{env}

\begin{env}{3.8.2}
\label{env-0.3.8.2}
Let $\sh{F}$, $\sh{G}$ be two sheaves of sets (resp. groups, rings) on $X$,
$u:\sh{F}\to\sh{G}$ a homomorphism. Then $u$ is thus a \emph{continuous} homomorphism
$\sh{F}'\to\sh{G}'$, if we denote by $\sh{F}'$ and $\sh{G}'$ the pseudo-discrete sheaves
associated to $\sh{F}$ and $\sh{G}$; this follows in effect from \sref{env}{3.2.5}.
\end{env}

\begin{env}{3.8.3}
\label{env-0.3.8.3}
Let $\sh{F}$ be a sheaf of sets, $\sh{H}$ a subsheaf of $\sh{F}$, $\sh{F}'$ and $\sh{H}'$ the
pseudo-discrete sheaves associated to $\sh{F}$ and $\sh{H}$ respectively. Then, for each open
$U\subset X$, $\Gamma(U,\sh{H}')$ is \emph{closed} in $\Gamma(U,\sh{F}')$: indeed, it is the
intersection of the inverse images of the $\Gamma(V,\sh{H})$ (for $V\in\mathfrak{B}$,
$V\subset U$) under the continuous maps $\Gamma(U,\sh{F})\to\Gamma(V,\sh{F})$, and
$\Gamma(V,\sh{H})$ is closed in the discrete space $\Gamma(V,\sh{F})$.
\end{env}

\section{Ringed spaces}
\label{0-prelim-4}

\subsection{Ringed spaces, sheaves of $\sh{A}$-modules, $\sh{A}$-algebras}
\label{0-prelim-4.1}

\begin{env}{4.1.1}
\label{env-0.4.1.1}
A \emph{ringed space} (resp. topologically ringed space) is a couple $(X,\sh{A})$
consisting of a topological space $X$ and a sheaf of rings (not necessarily commutative)
(resp. of a sheaf of topological rings) $\sh{A}$ on $X$; we say that $X$ is the
\emph{underlying} topological space of the ringed space $(X,\sh{A})$, and $\sh{A}$
the \emph{structure sheaf}. The latter is denoted $\OO_X$, and its stalk at a point
$x\in X$ is denotes $\OO_{X,x}$ or simply $\OO_x$ when there is no chance of confusion.

We denote by $1$ or $e$ the \emph{unit section} of $\OO_X$ over $X$ (the unit element
of $\Gamma(X,\OO_X)$).

As in this treatise we will have to consider in particular sheaves of \emph{commutative}
rings, it will be understood, when we speak of a ringed space $(X,\sh{A})$ without
specification, that $\sh{A}$ is a sheaf of commutative rings.

The ringed spaces with with not necessarily commutative structure sheaves
(resp. the topologically ringed spaces) form a \emph{category}, where we define
a \emph{morphism} $(X,\sh{A})\to(Y,\sh{B})$ as a couple $(\psi,\theta)=\Psi$
consisting of a continuous map $\psi:X\to Y$ and a \emph{$\psi$-morphism}
$\theta:\sh{G}\to\sh{F}$ (3.5.1) of sheaves of rings (resp. of sheaves of
topological rings); the \emph{composition} of a second morphism
$\Psi'=(\psi',\theta'):(Y,\sh{B})\to(Z,\sh{C})$ and of $\Psi$, denoted
$\Psi''=\Psi'\circ\Psi$, is the morphism $(\psi'',\theta'')$ where $\psi''=\psi'\circ\psi$,
and $\theta''$ is the composition of $\theta$ and $\theta'$ (equal to
$\psi_*'(\theta)\circ\theta'$, cf. \sref{env}{3.5.2}). For ringed spaces, remember that we
then have ${\theta''}^\sharp=\theta^\sharp\circ\psi^*({\theta'}^\sharp)$ \sref{env}{3.5.5};
therefore if ${\theta'}^\sharp$ and $\theta^\sharp$ are \emph{injective} (resp.
\emph{surjective}), the same is true of ${\theta''}^\sharp$, taking into account that
$\psi_x\circ\rho_{\psi(x)}$ is an isomorphism for all $x\in X$ \sref{env}{3.7.2}. We verify
immediately, thanks to the above, that when $\psi$ is an \emph{injective} continuous map and
$\theta^\sharp$ is a \emph{surjective} homomophism of sheaves of rings, the morphism
$(\psi,\theta)$ is a \emph{momomorphism} (T, 1.1) in the category of ringed spaces.

By abuse of language, we will often replace $\psi$ by $\Psi$ in notation, for
example in writing $\Psi^{-1}(U)$ in place of $\psi^{-1}(U)$ for a subset $U$ of $Y$,
when the is no risk of confusion.
\end{env}

\begin{env}{4.1.2}
\label{env-0.4.1.2}
For each subset $M$ of $X$, the pair $(M,\sh{A}|M)$ is evidently a ringed space, said
to be \emph{induced} on $M$ by the ringed space $(X,\sh{A})$ (and is still called
the \emph{restriction} of $(X,\sh{A})$ to $M$). If $j$ is the canonical injection
$M\to X$ and $\omega$ is the identity map of $\sh{A}|M$, $(j,\omega^\flat)$ is a
monomorphism $(M,\sh{A}|M)\to(X,\sh{A})$ of ringed spaces, called the
\emph{canonical injection}. The composition of a morphism
$\Psi:(X,\sh{A})\to(Y,\sh{B})$ and this injection is called the \emph{restriction}
of $\Psi$ to $M$.
\end{env}

\begin{env}{4.1.3}
\label{env-0.4.1.3}
We will not revisit the defintions of \emph{$\sh{A}$-modules} or
\emph{algebraic sheaves} on a ringed space $(X,\sh{A})$ (G, II, 2.2);
when $\sh{A}$ is a sheaf of not necessarily commutative rings, by $\sh{A}$-module
it will always mean ``left $\sh{A}$-module'' unless expressly stated otherwise. The
$\sh{A}$-submodules of $\sh{A}$ will be called \emph{sheaves of ideals} (left,
right, or two-sided) in $\sh{A}$ or \emph{$\sh{A}$-ideals}.

When $\sh{A}$ is a sheaf of commutative rings, and in the definition of
$\sh{A}$-modules, we replace everywhere the \emph{module} structure by that of
an \emph{algebra}, we obtain the definition of an \emph{$\sh{A}$-algebra} on $X$.
It is the same to say that an $\sh{A}$-algebra (not necessarily commutative) is
a $\sh{A}$-module $\sh{C}$, given with a homomorphism of $\sh{A}$-modules
$\vphi:\sh{C}\otimes_\sh{A}\sh{C}\to\sh{C}$ and a section $e$ over $X$,
such that: 1\textsuperscript{st} the diagram
\[
  \xymatrix{
    \sh{C}\otimes_\sh{A}\sh{C}\otimes_\sh{A}\sh{C}
    \ar[r]^{\vphi\otimes 1}\ar[d]_{1\otimes\vphi} &
    \sh{C}\otimes_\sh{A}\sh{C}\ar[d]^\vphi\\
    \sh{C}\otimes_\sh{A}\sh{C}\ar[r]^\vphi & \sh{C}
  }
\]
is commutative; 2\textsuperscript{nd} for each open $U\subset X$ and each section
$s\in\Gamma(U,\sh{C})$, we have $\vphi((e|U)\otimes s)=\vphi(s\otimes(e|U))=s$.
We say that $\sh{C}$ is a commutative $\sh{A}$-algebra if the diagram
\[
  \xymatrix{
    \sh{C}\otimes_\sh{A}\sh{C}\ar[rr]^\sigma\ar[rd]_\vphi
    & & \sh{C}\otimes_\sh{A}\sh{C}\ar[ld]^\vphi\\
    & \sh{C}
  }
\]
is commutative, $\sigma$ denoting the canonical symmetry (twist) map of the tensor product
$\sh{C}\otimes_\sh{A}\sh{C}$.

The homomorphisms of $\sh{A}$-algebras are also defined as the homomorphisms of
$\sh{A}$-modules in (G, II, 2.2), but naturally no longer form an abelian group.

If $\sh{M}$ is an $\sh{A}$-submodule of an $\sh{A}$-algebra $\sh{C}$, the
\emph{$\sh{A}$-subalgebra of $\sh{C}$ generated by $\sh{M}$} is the sum of the images of the
homomorphisms $\bigotimes^n\sh{M}\to\sh{C}$ (for each $n\geqslant 0$). This is also the sheaf
associated to the presheaf $U\mapsto\sh{B}(U)$ of algebras, $\sh{B}(U)$ being the subalgebra
of $\Gamma(U,\sh{C})$ generated by the submodule $\Gamma(U,\sh{M})$.
\end{env}

\begin{env}{4.1.4}
\label{env-0.4.1.4}
We say that a sheaf of rings $\sh{A}$ on a topological space $X$ is \emph{reduced at a point
$x$ in $X$} if the stalk $\sh{A}_x$ is a \emph{reduced} ring \sref{env}{1.1.1}; we say that
$\sh{A}$ is \emph{reduced} if it is reduced at all points of $X$. Recall that a ring $A$ is
called \emph{regular} if each of the local rings $A_\mathfrak{p}$ (where $\mathfrak{p}$ runs
through the set of prime ideals of $A$) is a regular local ring; we will say that a sheaf of
rings $\sh{A}$ on $X$ is \emph{regular at a point $x$} (resp. \emph{regular}) if the stalk
$\sh{A}_x$ is a regular ring (resp. if $\sh{A}$ is regular at each point). Finally, we will
say that a sheaf of rings $\sh{A}$ on $X$ is \emph{normal at a point $x$} (resp.
\emph{normal}) if the stalk $\sh{A}_x$ is an \emph{integral and integrally closed} ring
(resp. if $\sh{A}$ is normal at each point). We will say that a ringed space $(X,\sh{A})$ has
any of these preceeding properties if the sheaf of rings $\sh{A}$ has that property.

A \emph{graded} sheaf of rings $\sh{A}$ is by definition a sheaf of rings that is the direct
sum (G, II, 2,7) of a family $(\sh{A}_n)_{n\in\bb{Z}}$ of sheaves of abelian groups with the
conditions $\sh{A}_m\sh{A}_n\subset\sh{A}_{m+n}$; a \emph{graded $\sh{A}$-module} is an
$\sh{A}$-module $\sh{F}$ that is the direct sum of a family $(\sh{F}_n)_{n\in\bb{Z}}$ of
sheaves of abelian groups, satisfying the conditions $\sh{A}_m\sh{F}_n\subset\sh{F}_{m+n}$.
It is equivalent to say that $(\sh{A}_m)_x(\sh{A}_n)_x\subset(\sh{A}_{m+n})_x$ (resp.
$(\sh{A}_m)_x(\sh{F}_n)_x\subset(\sh{F}_{m+n})_x$) for each point $x$.
\end{env}

\begin{env}{4.1.5}
\label{env-0.4.1.5}
Given a ringed space $(X,\sh{A})$ (not necessarily commutative), we will not recall here the
definitions of the bifunctors $\sh{F}\otimes_\sh{A}\sh{G}$,
$\shHom_\sh{A}(\sh{F},\sh{F})$, and $\Hom_\sh{A}(\sh{F},\sh{G})$ (G, II, 2.8 and 2.2) in
the categories of left or right (depending on the case) $\sh{A}$-modules, with values in the
category of sheaves of abelian groups (or more generally of $\sh{C}$-modules, if $\sh{C}$ is
the center of $\sh{A}$). The stalk $(\sh{F}\otimes_\sh{A}\sh{G})_x$ for each point $x\in X$
identifies canonically with $\sh{F}_x\otimes_{\sh{A}_x}\sh{G}_x$ and we define a canonical
and functorial homomorphism
$(\shHom_\sh{A}(\sh{F},\sh{G}))_x\to\Hom_{\sh{A}_x}(\sh{F}_x,\sh{G}_x)$ which is in general
not injective or surjective. The bifunctors considered above are additive and in particular,
commute with finite direct limits; $\sh{F}\otimes_\sh{A}\sh{G}$ is right exact in $\sh{F}$
and in $\sh{G}$, commutes with inductive limits, and $\sh{A}\otimes_\sh{A}\sh{G}$ (resp.
$\sh{F}\otimes_\sh{A}\sh{A}$) identifies canonically with $\sh{G}$ (resp. $\sh{F}$). The
functors $\shHom_\sh{A}(\sh{F},\sh{G})$ and $\Hom_\sh{A}(\sh{F},\sh{G})$ are \emph{left
exact} in $\sh{F}$ and $\sh{G}$; more precisely, if we have an exact sequence of the form
$0\to\sh{G}'\to\sh{G}\to\sh{G}''$, the sequence
\[
  0\longrightarrow\shHom_\sh{A}(\sh{F},\sh{G}')\longrightarrow
  \shHom_\sh{A}(\sh{F},\sh{G})\longrightarrow\shHom_\sh{A}(\sh{F},\sh{G}'')
\]
is exact, and if we have an exact sequence of the form $\sh{F}'\to\sh{F}\to\sh{F}''\to 0$,
the sequence
\[
  0\longrightarrow\shHom_\sh{A}(\sh{F}'',\sh{G})\longrightarrow
  \shHom_\sh{A}(\sh{F},\sh{G})\longrightarrow\shHom_\sh{A}(\sh{F}',\sh{G})
\]
is exact, with the analagous properties for the functor $\Hom$. In addiiton,
$\shHom_\sh{A}(\sh{A},\sh{G})$ identifies canonically with $\sh{G}$; finally, for each open
$U\subset X$, we have
\[
  \Gamma(U,\shHom_\sh{A}(\sh{F},\sh{G})=\Hom_{\sh{A}|U}(\sh{F}|U,\sh{G}|U).
\]

For each left (resp. right) $\sh{A}$-module, we define the \emph{dual} of $\sh{F}$ and
denote it by $\dual{\sh{F}}$ the right (resp. left) $\sh{A}$-module
$\shHom_\sh{A}(\sh{F},\sh{A})$.

Finally, if $\sh{A}$ is a sheaf of commutative rings, $\sh{F}$ an $\sh{A}$-module,
$U\mapsto\wedge^p\Gamma(U,\sh{F})$ is a presheaf whose associated sheaf is an $\sh{A}$-module
denoted $\wedge^p\sh{F}$ and is called the \emph{$p$-th exterior power of $\sh{F}$}; we
verify easily that the canonical map of the presheaf $U\mapsto\wedge^p\Gamma(U,\sh{F})$ to
the associated sheaf $\wedge^p\sh{F}$ is \emph{injective}, and for each $x\in X$,
$(\wedge^p\sh{F})_x=\wedge^p(\sh{F}_x)$. It is clear that $\wedge^p\sh{F}$ is a covariant
functor in $\sh{F}$.
\end{env}

\begin{env}{4.1.6}
\label{env-0.4.1.6}
Suppose that $\sh{A}$ is a sheaf of non necessarily commutative rings, $\sh{J}$ a left sheaf
of ideals of $\sh{A}$, $\sh{F}$ an left $\sh{A}$-module; we then denote by $\sh{J}\sh{F}$ the
$\sh{A}$-submodule of $\sh{F}$, the image of $\sh{J}\otimes_\bb{Z}\sh{F}$ (where $\bb{Z}$ is
the sheaf associated to the constant presheaf $U\mapsto\bb{Z}$) under the canonical map
$\sh{J}\otimes_\bb{Z}\sh{F}\to\sh{F}$; it is clear that for each $x\in X$, we have
$(\sh{J}\sh{F})_x=\sh{J}_x\sh{F}_x$. When $\sh{A}$ is commutative, $\sh{J}\sh{F}$ is also
the canonical image of $\sh{J}\otimes_\sh{A}\sh{F}\to\sh{F}$. It is immediate that
$\sh{J}\sh{F}$ is also the $\sh{A}$-module associated to the presheaf
$U\mapsto\Gamma(U,\sh{J})\Gamma(U,\sh{F})$. If $\sh{J}_1$, $\sh{J}_2$ are two left sheaves
of ideals of $\sh{A}$, we have $\sh{J}_1(\sh{J}_2\sh{F})=(\sh{J}_1\sh{J}_2)\sh{F}$.
\end{env}

\begin{env}{4.1.7}
\label{env-0.4.1.7}
Let $(X_\lambda,\sh{A}_\lambda)_{\lambda\in L}$ be a family of ringed spaces; for each couple
$(\lambda,\mu)$, suppose we are given an open subset $V_{\lambda\mu}$ of $X_\lambda$, and an
isomorphism of ringed spaces
$\vphi_{\lambda\mu}:(V_{\mu\lambda},\sh{A}_\mu|V_{\lambda\mu})\isoto
(V_{\lambda\mu},\sh{A}_\lambda|V_{\lambda\mu})$, with $V_{\lambda\lambda}=X_\lambda$,
$\vphi_{\lambda\lambda}$ being the identity. Furthermore, suppose that, for each triple
$(\lambda,\mu,\nu)$, if we denote by $\vphi_{\mu\lambda}'$ the restriction of
$\vphi_{\mu\lambda}$ to $V_{\lambda\mu}\cap V_{\lambda\nu}$, $\vphi_{\mu\lambda}'$ is an
isomorphism from
$(V_{\lambda\mu}\cap V_{\lambda\nu},\sh{A}_\lambda|(V_{\lambda\mu}\cap V_{\lambda\nu}))$ to
$(V_{\mu\nu}\cap V_{\mu\lambda},\sh{A}_\mu|(V_{\mu\nu}\cap V_{\mu\lambda}))$ and that we have
$\vphi_{\lambda\nu}'=\vphi_{\lambda\mu}'\circ\vphi_{\mu\nu}'$ (\emph{gluing condition} for
the $\vphi_{\lambda\mu}$). We can first consider the topological space obtained by gluing
(by means of the $\vphi_{\lambda\mu}$) of the $X_\lambda$
\oldpage{39}
along the $V_{\lambda\mu}$; if we identify $X_\lambda$ with the corresponding open subset
$X_\lambda'$ in $X$, the hypotheses imply that the three sets
$V_{\lambda\mu}\cap V_{\lambda\nu}$, $V_{\mu\nu}\cap V_{\mu\lambda}$,
$V_{\nu\lambda}\cap V_{\nu\mu}$ identify with $X_\lambda'\cap X_\mu'\cap X_\nu'$. We can
also transport to $X_\lambda'$ the ringed space structure of $X_\lambda$, and if
$\sh{A}_\lambda'$ are the transported sheaves of rings corresponding to the $\sh{A}_\lambda$,
the $\sh{A}_\lambda'$ verify the gluing condition \sref{env}{3.3.1} and therefore define a
sheaf of rings $\sh{A}$ on $X$; we say tat $(X,\sh{A})$ is the ringed space obtained by
\emph{gluing the $(X_\lambda,\sh{A}_\lambda)$ along the $V_{\lambda\mu}$}, by means of the
$\vphi_{\lambda\mu}$.
\end{env}

\subsection{Direct image of an $\sh{A}$-module}
\label{0-prelim-4.2}

\begin{env}{4.2.1}
\label{env-0.4.2.1}
Let $(X,\sh{A})$, $(Y,\sh{B})$ be two ringed spaces, $\Psi=(\psi,\theta)$ a morphism
$(X,\sh{A})\to(Y,\sh{B})$; $\psi_*(\sh{A})$ is then a sheaf of rings on $Y$, and $\theta$ a
homomorphism $\sh{B}\to\psi_*(\sh{A})$ of sheaves of rings. Then let $\sh{F}$ be an
$\sh{A}$-module; the direct image $\psi_*(\sh{F})$ is a sheaf of abelian groups on $Y$. In
addition, for each open $U\subset Y$,
\[
  \Gamma(U,\psi_*(\sh{F}))=\Gamma(\psi^{-1}(U),\sh{F})
\]
is equipped with the structure of a module over the ring
$\Gamma(U,\psi_*(\sh{A}))=\Gamma(\psi^{-1}(U),\sh{A})$; the bilinear maps which define these
structures are compatible with the restriction operations, defining on $\psi_*(\sh{F})$ the
structure of a $\psi_*(\sh{A})$-module. The homomorphism $\theta:\sh{B}\to\psi_*(\sh{A})$
then defines also on $\psi_*(\sh{F})$ a \emph{$\sh{B}$-module} structure; we say that this
$\sh{B}$-module is the \emph{direct image of $\sh{F}$ under the morphism $\Psi$}, and we
denote it $\Psi_*(\sh{F})$. If $\sh{F}_1$, $\sh{F}_2$ are two $\sh{A}$-modules over $X$ and
$u$ an $\sh{A}$-homomorphism $\sh{F}_1\to\sh{F}_2$, it is immediate (by considering the
sections over the open subsets of $Y$) that $\psi_*(u)$ is a $\psi_*(\sh{A})$-homomorphism
$\psi_*(\sh{F}_1)\to\psi_*(\sh{F}_2)$, and \emph{a fortiori} a $\sh{B}$-homomorphism
$\Psi_*(\sh{F}_1)\to\Psi_*(\sh{F}_2)$; as a $\sh{B}$-homomorphism, we denote it by
$\Psi_*(u)$. So we see that $\Psi_*$ is a \emph{covariant functor} from the category of
$\sh{A}$-modules to that of $\sh{B}$-modules. In addition, it is immediate that this functor
is \emph{left exact} (G, II, 2.12).

On $\psi_*(\sh{A})$, the structure of a $\sh{B}$-module and the structure of a sheaf of rings
define a $\sh{B}$-algebra structure; we denote by $\Psi_*(\sh{A})$ this $\sh{B}$-algebra.
\end{env}

\begin{env}{4.2.2}
\label{env-0.4.2.2}
Let $\sh{M}$, $\sh{N}$ be two $\sh{A}$-modules. For each open set $U$ of $Y$, we have a
canonical map
\[
  \Gamma(\psi^{-1}(U),\sh{M})\times\Gamma(\psi^{-1}(U),\sh{N})
  \longrightarrow\Gamma(\psi^{-1}(U),\sh{M}\otimes_\sh{A}\sh{N})
\]
which is bilinear over the ring $\Gamma(\psi^{-1}(U),\sh{A})=\Gamma(U,\psi_*(\sh{A}))$, and
\emph{a fortiori} over $\Gamma(U,\sh{B})$; it therefore defines a homomorphism
\[
  \Gamma(U,\Psi_*(\sh{M}))\otimes_{\Gamma(U,\sh{B})}\Gamma(U,\Psi_*(\sh{N}))
  \longrightarrow\Gamma(U,\Psi_*(\sh{M}\otimes_\sh{A}\sh{N}))
\]
and as we check immediately that these homomorphisms are compatible with the restriction
operations, they give a canonical functorial homomorphism of $\sh{B}$-modules
\[
  \Psi_*(\sh{M})\otimes_\sh{B}\Psi_*(\sh{N})
  \longrightarrow\Psi_*(\sh{M}\otimes_\sh{A}\sh{N})
  \tag{4.2.2.1}
\]
\oldpage{40}
which is in general neither injective or surjective. If $\sh{P}$ is a third $\sh{A}$-module,
we check immediately that the diagram
\[
  \xymatrix{
    \Psi_*(\sh{M})\otimes_\sh{B}\Psi_*(\sh{N})\otimes_\sh{B}\Psi_*(\sh{P})\ar[r]\ar[d]
    & \Psi_*(\sh{M}\otimes_\sh{A}\sh{N})\otimes_\sh{B}\Psi_*(\sh{P})\ar[d]\\
    \Psi_*(\sh{M})\otimes_\sh{B}\Psi_*(\sh{N}\otimes_\sh{A}\sh{P})\ar[r]
    & \Psi_*(\sh{M}\otimes_\sh{A}\sh{N}\otimes_\sh{A}\sh{P})
  }
  \tag{4.2.2.2}
\]
is commutative.
\end{env}

\begin{env}{4.2.3}
\label{env-0.4.2.3}
Let $\sh{M}$, $\sh{N}$ be two $\sh{A}$-modules. For each open $U\subset Y$, we have by
definition that
$\Gamma(\psi^{-1}(U),\shHom_\sh{A}(\sh{M},\sh{N}))=\Hom_{\sh{A}|V}(\sh{M}|V,\sh{N}|V)$, where
we put $V=\psi^{-1}(U)$; the map $u\mapsto\Psi_*(u)$ is a homomorphism
\[
  \Hom_{\sh{A}|V}(\sh{M}|V,\sh{N}|V)
  \longrightarrow\Hom_{\sh{B}|U}(\Psi_*(\sh{M})|U,\Psi_*(\sh{N})|U)
\]
on the $\Gamma(U,\sh{B})$-module structures; these homomorphisms are compatible with the
restriction operations, hence they define a canonical functorial homomorphism of
$\sh{B}$-modules
\[
  \Psi_*(\shHom_\sh{A}(\sh{M},\sh{N}))
  \longrightarrow\shHom_\sh{B}(\Psi_*(\sh{M}),\Psi_*(\sh{N})).
  \tag{4.2.3.1}
\]
\end{env}

\begin{env}{4.2.4}
\label{env-0.4.2.4}
If $\sh{C}$ is an $\sh{A}$-algebra, the composite homomorphism
\[
  \Psi_*(\sh{C})\otimes_\sh{B}\Psi_*(\sh{C})
  \longrightarrow\Psi_*(\sh{C}\otimes_\sh{A}\sh{C})
  \longrightarrow\Psi_*(\sh{C})
\]
defines on $\Psi_*(\sh{C})$ the structure of a \emph{$\sh{B}$-algebra}, as a result of
(4.2.2.2). We see similarly that if $\sh{M}$ is a $\sh{C}$-module, $\Psi_*(\sh{M})$ is
canonically equipped with the structure of a $\Psi_*(\sh{C})$-module.
\end{env}

\begin{env}{4.2.5}
\label{env-0.4.2.5}
Consider in particular the case where $X$ is a \emph{closed} subspace of $Y$ and where $\psi$
is the canonical injection $j:X\to Y$. If $\sh{B}'=\sh{B}|X=j^*(\sh{B})$ is the restriction
of the sheaf of rings $\sh{B}$ to $X$, an $\sh{A}$-module $\sh{M}$ can be considered as a
$\sh{B}'$-module by means of the homomorphism $\theta^\sharp:\sh{B}'\to\sh{A}$; then
$\Psi_*(\sh{M})$ is the $\sh{B}$-module which induces $\sh{M}$ on $X$ and $0$ elsewhere. If
$\sh{N}$ is a second $\sh{A}$-module, $\Psi_*(\sh{M})\otimes_\sh{B}\Psi_*(\sh{N})$
canonically identifies with $\Psi_*(\sh{M}\otimes_{\sh{B}'}\sh{N})$ and
$\shHom_\sh{B}(\Psi_*(\sh{M}),\Psi_*(\sh{N}))$ with
$\Psi_*(\shHom_{\sh{B}'}(\sh{M},\sh{N}))$.
\end{env}

\begin{env}{4.2.6}
\label{env-0.4.2.6}
Let $(Z,\sh{C})$ be a third ringed space, $\Psi'=(\psi',\theta')$ a morphism
$(Y,\sh{B})\to(Z,\sh{C})$; if $\Psi''$ is the composite morphism $\Psi'\circ\Psi$, it is
clear that we have $\Psi_*''=\Psi_*'\circ\Psi_*$.
\end{env}

\subsection{Inverse image of an $\sh{A}$-module}
\label{0-prelim-4.3}

\begin{env}{4.3.1}
\label{env-0.4.3.1}
The hypotheses and notation being the same as \sref{env}{4.2.1}, let $\sh{G}$ be a
$\sh{B}$-module and $\psi^*(\sh{G})$ the inverse image \sref{env}{3.7.1} which is therefore
a sheaf of abelian groups on $X$. The deinition of sections of $\psi^*(\sh{G})$ and of
$\psi^*(\sh{B})$ \sref{env}{3.7.1} shows that $\psi^*(\sh{G})$ is canonically equipped with
a $\psi^*(\sh{B})$-module structure. On the other hand, the homomorphism
$\theta^\sharp:\psi^*(\sh{B})\to\sh{A}$ endows $\sh{A}$ with the a $\psi^*(\sh{B})$-module
structure, which we denote by $\sh{A}_{[\theta]}$ when necessary to avoid confusion; the
tensor product $\psi^*(\sh{G})\otimes_{\psi^*(\sh{B})}\sh{A}_{[\theta]}$ is then equipped
with an $\sh{A}$-module structure. We say that this $\sh{A}$-module is \emph{the inverse
image of $\sh{G}$ under the morphism $\Psi$}
\oldpage{41}
and we denote it by $\Psi^*(\sh{G})$. If $\sh{G}_1$, $\sh{G}_2$ are two $\sh{B}$-modules over
$Y$, $v$ a $\sh{B}$-homomorphism $\sh{G}_1\to\sh{G}_2$, $\psi^*(v)$, as we check immediately,
is a $\psi^*(\sh{B})$-homomorphism from $\psi^*(\sh{G}_1)$ to $\psi^*(\sh{G}_2)$; as a result
$\psi^*(v)\otimes 1$ is an $\sh{A}$-homomorphism $\Psi^*(\sh{G}_1)\to\Psi^*(\sh{G}_2)$, which
we denote by $\Psi^*(v)$. So we define $\Psi^*$ as a \emph{covariant functor} from the
category of $\sh{B}$-modules to that of $\sh{A}$-modules. Here, this functor (contrary to
$\psi^*$) is no longer exact in general, but only \emph{right exact}, the tensorization by
$\sh{A}$ being a right exact functor to the category of $\psi^*(\sh{B})$-modules.

For each $x\in X$, we have
$(\Psi^*(\sh{G}))_x=\sh{G}_{\psi(x)}\otimes_{\sh{B}_{\psi(x)}}\sh{A}_x$, according to
\sref{env}{3.7.2}. The support of $\Psi^*(\sh{G})$ is thus contained in
$\psi^{-1}(\Supp(\sh{G}))$.
\end{env}

\begin{env}{4.3.2}
\label{env-0.4.3.2}
Let $(\sh{G}_\lambda)$ be an inductive system of $\sh{B}$-modules, and let
$\sh{G}=\varinjlim\sh{G}_\lambda$ be its inductive limit. The canonical homomorphisms
$\sh{G}_\lambda\to\sh{G}$ define the $\psi^*(\sh{B})$-homomorphisms
$\psi^*(\sh{G}_\lambda)\to\psi^*(\sh{G})$, which give a canonical homomorphism
$\varinjlim\psi^*(\sh{G}_\lambda)\to\psi^*(\sh{G})$. As the stalk at a point of an
inductive limit of sheaves is the inductive limit of the stalks at the same point
(G, II, 1.11), the preceding canonical homomorphism is \emph{bijective} \sref{env}{3.7.2}.
In addition, the tensor product commutes with inductive limits of sheaves, and we thus have
a \emph{canonical functorial isomorphism}
$\varinjlim\Psi^*(\sh{G}_\lambda)\isoto\Psi^*(\varinjlim\sh{G}_\lambda)$ of $\sh{A}$-modules.

On the other hand, for a finite direct sum $\bigoplus_i\sh{G}_i$ of $\sh{B}$-modules, it is
clear that $\psi^*(\bigoplus_i\sh{G}_i)=\bigoplus_i\psi^*(\sh{G}_i)$, therefore, by
tensoring with $\sh{A}_{[\theta]}$,
\[
  \Psi^*\big(\bigoplus_i\sh{G}_i\big)=\bigoplus_i\Psi^*(\sh{G}_i).
  \tag{4.3.2.1}
\]
By passing to the inductive limit, we deduce, in light of the above, that the above
equality is still true for \emph{any} direct sum.
\end{env}

\begin{env}{4.3.3}
\label{env-0.4.3.3}
Let $\sh{G}_1$, $\sh{G}_2$ be two $\sh{B}$-modules; from the definition of the inverse images
of sheaves of abelian groups \sref{env}{3.7.1}, we obtain immediately a canonical
homomorphism
$\psi^*(\sh{G}_1)\otimes_{\psi^*(\sh{B})}\psi^*(\sh{G}_2)
\to\psi^*(\sh{G}_1\otimes_\sh{B}\sh{G}_2)$ of $\psi^*(\sh{B})$-modules, and the stalk at a
point of a tensor product of sheaves being the tensor product of the stalks at this point
(G, II, 2.8), we deduce from \sref{env}{3.7.2} that the above homomorphism is in fact a
\emph{isomorphism}. By tensoring with $\sh{A}$, we obtain a \emph{canonical functorial
isomorphism}
\[
  \Psi^*(\sh{G}_1)\otimes_\sh{A}\Psi^*(\sh{G}_2)\isoto\Psi^*(\sh{G}_1\otimes_\sh{B}\sh{G}_2).
  \tag{4.3.3.1}
\]
\end{env}

\begin{env}{4.3.4}
\label{env-0.4.3.4}
Let $\sh{C}$ be a $\sh{B}$-algebra; the data of the algebra structure on $\sh{C}$ is the same
as the data of a $\sh{B}$-homomorphism $\sh{C}\otimes_\sh{B}\sh{C}\to\sh{C}$ satifying the
associativity and commutativity conditions (conditions which are checked stalk-wise); the
above isomorphism allows us to consider this homomorphism as a homomorphism of
$\sh{A}$-modules $\Psi^*(\sh{C})\otimes_\sh{A}\Psi^*(\sh{C})\to\Psi^*(\sh{C})$ satisfying the
same conditions, so $\Psi^*(\sh{C})$ is thus equipped with an $\sh{A}$-algebra structure. In
particular, it follows immediately from the definitions that the $\sh{A}$-algebra
$\Psi^*(\sh{B})$ is \emph{equal to $\sh{A}$} (up to a canonical isomorphism).

Similarly, if $\sh{M}$ is a $\sh{C}$-module, the data of this module structure is the same
\oldpage{42}
as that of a $\sh{B}$-homomorphism $\sh{C}\otimes_\sh{B}\sh{M}\to\sh{M}$ satisying the
associativity condition; hence we give a $\Psi^*(\sh{C})$-module structure on
$\Psi^*(\sh{M})$.
\end{env}

\begin{env}{4.3.5}
\label{env-0.4.3.5}
Let $\sh{J}$ be a sheaf of ideals of $\sh{B}$; as the functor $\psi^*$ is exact, the
$\psi^*(\sh{B})$-module $\psi^*(\sh{J})$ identifies canonically with a shead of ideals of
$\psi^*(\sh{B})$; the canonical injection $\psi^*(\sh{J})\to\psi^*(\sh{B})$ then gives a
homomorphism of $\sh{A}$-modules
$\Psi^*(\sh{J})=\psi^*(\sh{J})\otimes_{\psi^*(\sh{B})}\sh{A}_{[\theta]}\to\sh{A}$; we denote
by $\Psi^*(\sh{J})\sh{A}$, or $\sh{J}\sh{A}$ if there is no fear of confusion, the image of
$\Psi^*(\sh{J})$ under this homomorphism. So we have by definition
$\sh{J}\sh{A}=\theta^\sharp(\psi^*(\sh{J}))\sh{A}$ and in particular, for each $x\in X$,
$(\sh{J}\sh{A})_x=\theta_x^\sharp(\sh{J}_{\psi(x)})\sh{A}_x$, taking into account the
canonical identification between the stalks of $\psi^*(\sh{J})$ and those of $\sh{J}$
\sref{env}{3.7.2}. If $\sh{J}_1$, $\sh{J}_2$ are two sheaves of ideals of $\sh{B}$, we have
$(\sh{J}_1\sh{J}_2)\sh{A}=\sh{J}_1(\sh{J}_2\sh{A})=(\sh{J}_1\sh{A})(\sh{J}_2\sh{A})$.

If $\sh{F}$ is an $\sh{A}$-module, we put $\sh{J}\sh{F}=(\sh{J}\sh{A})\sh{F}$.
\end{env}

\begin{env}{4.3.6}
\label{env-0.4.3.6}
Let $(Z,\sh{C})$ be a third ringed space, $\Psi'=(\psi',\theta')$ a morphism
$(Y,\sh{B})\to(Z,\sh{C})$; if $\Psi''$ is the composite morphism $\Psi'\circ\Psi$, it follows
from the definition \sref{env}{4.3.1} and from (4.3.3.1) that we have
${\Psi''}^*=\Psi^*\circ{\Psi'}^*$.
\end{env}

\subsection{Relation between direct and inverse images}
\label{0-prelim-4.4}

\begin{env}{4.4.1}
\label{env-0.4.4.1}
The hypotheses and notations being the same as in \sref{env}{4.2.1}, let $\sh{G}$ be a
$\sh{B}$-module. By definition, a homomorphism $u:\sh{G}\to\Psi_*(\sh{F})$ of
$\sh{B}$-modules is still called a \emph{$\Psi$-morphisms from $\sh{G}$ to $\sh{F}$}, or
simply a \emph{homomorphism from $\sh{G}$ to $\sh{F}$} and we write it as $u:\sh{G}\to\sh{F}$
when no confusion will occur. To give such a homomorphism is the same as giving, for each
pair $(U,V)$ where $U$ is an open set of $X$, $V$ an open set of $Y$ such that
$\psi(U)\subset V$, a \emph{homomorphism $u_{U,V}:\Gamma(V,\sh{G})\to\Gamma(U,\sh{F})$ of
$\Gamma(V,\sh{B})$-modules}, $\Gamma(U,\sh{F})$ being considered as a
$\Gamma(V,\sh{B})$-module by means of the ring homomorphism
$\theta_{U,V}:\Gamma(V,\sh{B})\to\Gamma(U,\sh{A})$; the $u_{U,V}$ must in addition render
commutative the diagrams (3.5.1.1). It suffices, moreover, to define $u$ by the data of the
$u_{U,V}$ when $U$ (resp. $V$) varies over a basis $\mathfrak{B}$ (resp. $\mathfrak{B}'$) for
the topology of $X$ (resp. $Y$) and to check the commutativity of (3.5.1.1) for these
restrictions.
\end{env}

\begin{env}{4.4.2}
\label{env-0.4.4.2}
Under the hypotheses of \sref{env}{4.2.1} and \sref{env}{4.2.6}, let $\sh{H}$ be a
$\sh{C}$-module, $v:\sh{H}\to\Psi_*'(\sh{G})$ a $\Psi'$-morphism; then
$w:\sh{H}\xrightarrow{v}\Psi_*'(\sh{G})\xrightarrow{\Psi_*'(u)}\Psi_*'(\Psi_*(\sh{F}))$ is
a $\Psi''$-morphism which we call the \emph{composition} of $u$ and $v$.
\end{env}

\begin{env}{4.4.3}
\label{env-0.4.4.3}
We will now see that we can define a canonical \emph{isomorphism} of \emph{bifunctors} in
$\sh{F}$ and $\sh{G}$
\[
  \Hom_\sh{A}(\Psi^*(\sh{G}),\sh{F})\isoto\Hom_\sh{B}(\sh{G},\Psi_*(\sh{F}))
  \tag{4.4.3.1}
\]
which we denote by $v\mapsto v_\theta^\flat$ (or simply $v\mapsto v^\flat$ if there is no
chance of confusion); we denote by $u\mapsto u_\theta^\sharp$, or $u\mapsto u^\sharp$, the
inverse isomorphism. This definition is the following: by composing
$v:\Psi^*(\sh{G})\to\sh{F}$ with the canonical map $\psi^*(\sh{G})\to\Psi^*(\sh{G})$, we
obtain a homomorphism of sheaves of groups $v':\psi^*(\sh{G})\to\sh{F}$, which is also a
homomorphism of $\psi^*(\sh{B})$-modules. We obtain \sref{env}{3.7.1} a homomorphism
${v'}^\flat:\sh{G}\to\psi_*(\sh{F})=\Psi_*(\sh{F})$, which is also a homomorphism of
$\sh{B}$-modules as we
\oldpage{43}
check easily; we take $v_\theta^\flat={v'}^\flat$. Similarly, for
$u:\sh{G}\to\Psi_*(\sh{F})$, which is a homomorphism of $\sh{B}$-modules, we obtain
\sref{env}{3.7.1} a homomorphism $u^\sharp:\psi^*(\sh{G})\to\sh{F}$ of
$\psi^*(\sh{B})$-modules, hence by tensoring with $\sh{A}$ we have a homomorphism of
$\sh{A}$-modules $\Psi^*(\sh{G})\to\sh{F}$, which we denote by $u_\theta^\sharp$. It is
immediate to check that $(u_\theta^\sharp)_\theta^\flat=u$ and
$(v_\theta^\flat)_\theta^\sharp=v$, so we have established the functorial nature in $\sh{F}$
of the isomorphism $v\mapsto v_\theta^\flat$. The functorial nature in $\sh{G}$ of
$u\mapsto u_\theta^\sharp$ is then formally shown as in \sref{env}{3.5.4} (reasoning that
would also prove the functorial nature of $\Psi^*$ established in \sref{env}{4.3.1}
directly).

If we take for $v$ the identity homomorphism of $\Psi^*(\sh{B})$, $v_\theta^\flat$ is a
homomorphism
\[
  \rho_\sh{G}:\sh{G}\longrightarrow\Psi_*(\Psi^*(\sh{G}));
  \tag{4.4.3.2}
\]
if we take for $u$ the identity homomorphism of $\Psi_*(\sh{F})$, $u_\theta^\sharp$ is a
homomorphism
\[
  \sigma_\sh{F}:\Psi^*(\Psi_*(\sh{F}))\longrightarrow\sh{F};
  \tag{4.4.3.3}
\]
these homomorphisms will be called \emph{canonical}. They are in general neither injective or
surjective. We have canonical factorizations analogous to (3.5.3.3) and (3.5.4.4).

We note that if $s$ is a section of $\sh{G}$ over an open set $V$ of $Y$, $\rho_\sh{G}(s)$ is
the section $s'\otimes 1$ of $\Psi^*)(\sh{G})$ over $\psi^{-1}(V)$, $s'$ being such that
$s_x'=s_{\psi(x)}$ for all $x\in\psi^{-1}(V)$. We also note that if
$u:\sh{G}\to\psi_*(\sh{F})$ is a homomorphism, it defines for all $x\in X$ a homomorphism
$u_x:\sh{G}_{\psi(x)}\to\sh{F}_x$ on the stalks, obtained by composing
$(u^\sharp)_x:(\Psi^*(\sh{G}))_x\to\sh{F}_x$ and the canonical homomorphism
$s_x\mapsto s_x\otimes 1$ from $\sh{G}_{\psi(x)}$ to
$(\Psi^*(\sh{G}))_x=\sh{G}_{\psi(x)}\otimes_{\sh{B}_{\psi(x)}}\sh{A}_x$. The homomorphism
$u_x$ is obtained also by passing to the inductive limit relative to the homomorphisms
$\Gamma(V,\sh{G})\xrightarrow{u}\Gamma(\psi^{-1}(V),\sh{F})\to\sh{F}_x$, where $V$ varies
over the neighborhoods of $\psi(x)$.
\end{env}

\begin{env}{4.4.4}
\label{env-0.4.4.4}
Let $\sh{F}_1$, $\sh{F}_2$ be $\sh{A}$-modules, $\sh{G}_1$, $\sh{G}_2$ be $\sh{B}$-modules,
$u_i$ ($i=1,2$) a homomorphism from $\sh{G}_i$ to $\sh{F}_i$. We denote by $u_1\otimes u_2$
the homomorphism $u:\sh{G}_1\otimes_\sh{B}\sh{G}_2\to\sh{F}_1\otimes_\sh{A}\sh{F}_2$ such
that $u^\sharp=(u_1)^\sharp\otimes(u_2)^\sharp$ (taking into account (4.3.3.1)); we check
that $u$ is also the composition
$\sh{G}_1\otimes_\sh{B}\sh{G}_2\to\Psi_*(\sh{F}_1)\otimes_\sh{B}\Psi_*(\sh{F}_2)
\to\Psi_*(\sh{F}_1\otimes_\sh{A}\sh{F}_2)$, where the first arrow is the ordinary tensor
product $u_1\otimes_\sh{B}u_2$ and the second is the canonical homomorphism (4.2.2.1).
\end{env}

\begin{env}{4.4.5}
\label{env-0.4.4.5}
Let $(\sh{G}_\lambda)_{\lambda\in L}$ be an inductive system of $\sh{B}$-modules, and, for
each $\lambda\in L$, let $u_\lambda$ be a homomorphisme $\sh{G}_\lambda\to\Psi_*(\sh{F})$,
form an inductive limit; we put $\sh{G}=\varinjlim\sh{G}_\lambda$ and
$u=\varinjlim u_\lambda$; then the $(u_\lambda)^\sharp$ form an inductive system of
homomorphisms $\Psi^*(\sh{G}_\lambda)\to\sh{F}$, and the inductive limit of this system is
non other than $u^\sharp$.
\end{env}

\begin{env}{4.4.6}
\label{env-0.4.6.6}
Let $\sh{M}$, $\sh{N}$ be two $\sh{B}$-modules, $V$ an open set of $Y$, $U=\psi^{-1}(V)$; the
map $v\mapsto\Psi^*(v)$ is a homomorphism
\[
  \Hom_{\sh{B}|V}(\sh{M}|V,\sh{N}|V)
  \longrightarrow\Hom_{\sh{A}|U}(\Psi^*(\sh{M})|U,\Psi^*(\sh{N})|U)
\]
for the $\Gamma(V,\sh{B})$-module structures
($\Hom_{\sh{A}|U}(\Psi^*(\sh{M})|U,\Psi^*(\sh{N})|U)$ is normaly equipped with the a
$\Gamma(U,\psi^*(\sh{B}))$-module structure, and thanks to the canonical homomorphism
\oldpage{44}
\sref{env}{3.7.2} $\Gamma(V,\sh{B})\to\Gamma(U,\psi^*(\sh{B}))$, it is also a
$\Gamma(V,\sh{B})$-module). We see immediately that these homomorphisms are compatible with
the restriction morphisms, and as a result define a canonical functorial homomorphism
\[
  \gamma:\shHom_\sh{B}(\sh{M},\sh{N})
  \longrightarrow\Psi_*(\shHom_\sh{A}(\Psi^*(\sh{M}),\Psi^*(\sh{N}));
\]
it also corresponds to this homomorphism the homomorphism
\[
  \gamma^\sharp:\Psi^*(\shHom_\sh{B}(\sh{M},\sh{N}))
  \longrightarrow\shHom_\sh{A}(\Psi^*(\sh{M}),\Psi^*(\sh{N}))
\]
and these canonical morphisms are functorial in $\sh{M}$ and $\sh{N}$.
\end{env}

\begin{env}{4.4.7}
\label{env-0.4.4.7}
Suppose that $\sh{F}$ (resp. $\sh{G}$) is an $\sh{A}$-algebra (resp. a $\sh{B}$-algebra). If
$u:\sh{G}\to\Psi_*(\sh{F})$ is a homomorphism of $\sb{B}$-algebras, $u^\sharp$ is a
homomorphism $\Psi^*(\sh{G})\to\sh{F}$ of $\sh{A}$-algebras; this follows from the
commutativity of the diagram
\[
  \xymatrix{
    \sh{G}\otimes_\sh{B}\sh{G}\ar[r]\ar[d] &
    \sh{G}\ar[d]^u\\
    \Psi_*(\sh{F}\otimes_\sh{A}\sh{F})\ar[r] &
    \Psi_*(\sh{F})
  }
\]
and from \sref{env}{4.4.4}. Similarly, if $v:\Psi^*(\sh{G})\to\sh{F}$ is a homomorphism of
$\sh{A}$-algebras, $v^\flat:\sh{G}\to\Psi_*(\sh{F})$ is a homomorphism of $\sh{B}$-algebras.
\end{env}

\begin{env}{4.4.8}
\label{env-0.4.4.8}
Let $(Z,\sh{C})$ be a third ringed space, $\Psi'=(\psi',\theta')$ a morphism
$(Y,\sh{B})\to(Z,\sh{C})$, and $\Psi'':(X,\sh{A})\to(Z,\sh{C})$ the composite morphism
$\Psi'\circ\Psi$. Let $\sh{H}$ be a $\sh{C}$-module, $u'$ a homomorphsim from $\sh{H}$ to
$\sh{G}$; the composition $v''=v\circ v'$ is by definition the homomorphism from $\sh{H}$ to
$\sh{F}$ defined by
$\sh{H}\xrightarrow{v'}\Psi_*'(\sh{G})\xrightarrow{\Psi_*'(v)}\Psi_*'(\Psi_*(\sh{F}))$; we
check that ${v''}^\sharp$ is the homomorphism
\[
  \Psi^*({\Psi'}^*(\sh{H}))\xrightarrow{\Psi^*({v'}^\sharp)}\Psi^*(\sh{G})
  \xrightarrow{v^\sharp}\sh{F}.
\]
\end{env}

\section{Quasi-coherent and coherent sheaves}
\label{0-prelim-5}

\subsection{Quasi-coherent sheaves}
\label{0-prelim-5.1}

\begin{env}{5.1.1}
\label{env-0.5.1.1}
Let $(X,\OO_X)$ be a ringed space, $\sh{F}$ an $\OO_X$-module. The data of a homomorphism
$u:\OO_X\to\sh{F}$ of $\OO_X$-modules is equivalent to that of the section
$s=u(1)\in\Gamma(X,\sh{F})$. Indeed, when $s$ is given, for each section
$t\in\Gamma(U,\OO_X)$, we necessarily have $u(t)=t\cdot(s|U)$; we say that $u$ is
\emph{defined by the section $s$}. If now $I$ is any set of indices, consider the direct
sum sheaf $\OO_X^{(I)}$, and for each $i\in I$, let $h_i$ be the canonical injection of
the $i$-th factor into $\OO_X^{(I)}$; we know that $u\mapsto(u\circ h_i)$ is an isomorphism
from $\Hom_{\OO_X}(\OO_X^{(I)},\sh{F})$ to the product $(\Hom_{\OO_X}(\OO_X,\sh{F}))^I$. So
there is a canonical one-to-one correspondence between the homomorphisms
$u:\OO_X^{(I)}\to\sh{F}$ and the \emph{families of sections $(s_i)_{i\in I}$ of $\sh{F}$ over
$X$}. The homomorphism $u$ corresponding to $(s_i)$ sends an element
$(a_i)\in(\Gamma(U,\OO_X))^{(I)}$ to $\sum_{i\in I}a_i\cdot(s_i|U)$.

We say that $\sh{F}$ is \emph{generated by the family $(s_i)$} if the homomorphism
$\OO_X^{(I)}\to\sh{F}$ defined
\oldpage{45}
for each family is \emph{surjective} (in other words, if, for each $x\in X$, $\sh{F}_x$ is an
$\OO_x$-module generated by the $(s_i)_x$). We say that $\sh{F}$ is \emph{generated by its
sections over $X$} if it is generated by the family of all these sections (or by a
subfamily), in other words, if there exists a surjective homomorphism $\OO_X^{(I)}\to\sh{F}$
for a suitable $I$.

We note that a $\OO_X$-module $\sh{F}$ can be such that there exists a point $x_0\in X$ for
which $\sh{F}|U$ is not generated by its sections over $U$, \emph{regardless of the choice
of neighborhood $U$ of $x_0$}: it suffices to take $X=\bb{R}$, for $\OO_X$ the simple sheaf
$\bb{Z}$, for $\sh{F}$ the algebraic subsheaf of $\OO_X$ such that $\sh{F}_0=\{0\}$,
$\sh{F}_x=\bb{Z}$ for $x\neq 0$, and finally $x_0=0$: the only section of $\sh{F}|U$ over $U$ 
is $0$ for a neighborhood $U$ of $0$.
\end{env}

\begin{env}{5.1.2}
\label{env-0.5.1.2}
Let $f:X\to Y$ be a morphism of ringed spaces. If $\sh{F}$ is a $\OO_X$-module generated by
its sections over $X$, then the canonical homomorphism $f^*(f_*(\sh{F}))\to\sh{F}$ (4.4.3.3)
is \emph{surjective}; indeed, with the notations of \sref{env}{5.1.1}, $s_i\otimes 1$ is a
section of $f^*(f_*(\sh{F}))$ over $X$, and its image in $\sh{F}$ is $s_i$. The example of
\sref{env}{5.1.1} where $f$ is the identity shows that the inverse of this proposition is
false in general.
\end{env}

\begin{env}{5.1.3}
\label{env-0.5.1.3}
We say that an $\OO_X$-module $\sh{F}$ is \emph{quasi-coherent} if, for each $x\in X$, there
is an open neighborhood $U$ of $x$ wuch that $\sh{F}|U$ is isomorphic to the \emph{cokernel}
of a homomorphism of the form $\OO_X^{(I)}|U\to\OO_X^{(J)}|U$, where $I$ and $J$ are sets of
arbitrary indices. It is clear that $\OO_X$ is itself a quasi-coherent $\OO_X$-module, and
that any direct sum of quasi-coherent $\OO_X$-modules is again a quasi-coherent
$\OO_X$-module. We say that an \emph{$\OO_X$-algebra $\sh{A}$} is \emph{quasi-coherent} if
it is quasi-coherent as an $\OO_X$-module.
\end{env}

\begin{env}{5.1.4}
\label{env-0.5.1.4}
Let $f:X\to Y$ be a morphism of ringed spaces. If $\sh{G}$ is a quasi-coherent
$\OO_Y$-module, then $f^*(\sh{G})$ is a quasi-coherent $\OO_X$-module. Indeed, for each
$x\in X$, there is an open neighborhood $V$ of $f(x)$ in $Y$ such that $\sh{G}|V$ is the
cokernel of a homomorphism $\OO_Y^{(I)}|V\to\OO_Y^{(J)}|V$. If $U=f^{-1}(V)$, and if $f_U$ is
the restriction of $f$ to $U$, we have $f^*(\sh{G})|U=f_U^*(\sh{G}|V)$; as $f_U^*$ is right
exact and commutes with direct sums, $f_U^*(\sh{G}|V)$ is the cokernel of a homomorphism
$\OO_X^{(I)}|U\to\OO_X^{(J)}|U$.
\end{env}

\subsection{Sheaves of finite type}
\label{0-prelim-5.2}

\begin{env}{5.2.1}
\label{env-0.5.2.1}
We say that an $\OO_X$-module $\sh{F}$ is \emph{of finite type} if, for each $x\in X$, there
exists an open neighborhood $U$ of $x$ such that $\sh{F}|U$ is generated by a \emph{finite}
family of sections over $U$, or if it is isomorphic to a sheaf quotient of a sheaf of the
form $(\OO_X|U)^p$ where $p$ is finite. Each sheaf quotient of a sheaf of finite type is
again a sheaf of finite type, as well as each finite direct sum and each finite tensor
product of sheaves of finite type. An $\OO_X$-module of finite type is not necessarily
quasi-coherent, as we can see for the $\OO_X$-module $\OO_X/\sh{F}$, where $\sh{F}$ is the
example in \sref{env}{5.1.1}. If $\sh{F}$ is of finite type, $\sh{F}_x$ is a $\OO_x$-module
of finite type for each $x\in X$, but the example in \sref{env}{5.1.1} shows that this
condition is necessary but not sufficient in general.
\end{env}

\begin{env}{5.2.2}
\label{env-0.5.2.2}
Let $\sh{F}$ be an $\OO_X$-module \emph{of finite type}. If $s_i$ ($1\leqslant i\leqslant n$)
are the sections of $\sh{F}$ over an open neighborhood $U$ of a point $x\in X$ and the
$(s_i)_x$ generate $\sh{F}_x$, there exists an open neighborhood $V\subset U$ of $x$ such
that the $(s_i)_y$ generate $\sh{F}_y$ for all $y\in Y$ (FAC, I, 2, 12, prop.~1). In
particular, we conclude that the support of $\sh{F}$ is \emph{closed}.

\oldpage{46}
Similarly, if $u:\sh{F}\to\sh{G}$ is a homomorphism such that $u_x=0$, then there exists a
neighborhood $U$ of $x$ suc that $u_y=0$ for all $y\in U$.
\end{env}

\begin{env}{5.2.3}
\label{env-0.5.2.3}
Suppose that $X$ is \emph{quasi-compact}, and let $\sh{F}$, $\sh{G}$ be two $\OO_X$-modules
such that $\sh{G}$ is \emph{of finite type}, $u:\sh{F}\to\sh{G}$ a \emph{surjective}
homomorphism. In addition, suppose that $\sh{F}$ is the inductive limit of an inductive
system $(\sh{F}_\lambda)$ of $\OO_X$-modules. Then there exists an index $\mu$ such that the
homomorphism $\sh{F}_\mu\to\sh{G}$ is \emph{surjective}. Indeed, for each $x\in X$, there
exists a finite system of sections $s_i$ of $\sh{G}$ over an open neighborhood $U(x)$ of $x$
such that the $(s_i)_y$ generate $\sh{G}_y$ for all $y\in U(x)$; there is then an open
neighborhood $V(x)\subset U(x)$ of $x$ and $n$ sections $t_i$ of $\sh{F}$ over $V(x)$ such
that $s_i|V(s)=u(t_i)$ for all $i$; we can also suppose that the $t_i$ are the canonical
images of sections of a similar sheaf $\sh{F}_{\lambda(x)}$ over $V(x)$. We then cover $X$
with a finite number of neighborhoods $V(x_k)$, and let $\mu$ be the maximal index of the
$\lambda(x_k)$; it is clear that this index gives the answer.

Suppose still that $X$ is quasi-compact, and let $\sh{F}$ be an $\OO_X$-module of finite type
generated by its sections over $X$ \sref{env}{5.1.1}; then $\sh{F}$ is generated by a
\emph{finite} subfamily of these sections: indeed, it suffices to cover $X$ by a finite
number of open neighborhoods $U_k$ such that, for each $k$, there is a finite number of
sections $s_{ik}$ of $\sh{F}$ over $X$ whose restrictions to $U_k$ generate $\sh{F}|U_k$; it
is clear that the $s_{ik}$ then generate $\sh{F}$.
\end{env}

\begin{env}{5.2.4}
\label{env-0.5.2.4}
Let $f:X\to Y$ be a morphism of ringed spaces. If $\sh{G}$ is an $\OO_Y$-module of finite
type, then $f^*(\sh{G})$ is an $\OO_X$-module of finite type. Indeed, for each $x\in X$,
there is an open neighborhood $V$ of $f(x)$ in $Y$ and a surjective homomorphism
$v:\OO_Y^p|V\to\sh{G}|V$. If $U=f^{-1}(V)$ and if $f_U$ is the restriction of $f$ to $U$, we
have $f^*(\sh{G})|U=f_U^*(\sh{G}|V)$; as $f_U$ is right exact \sref{env}{4.3.1} and commutes
with direct sums \sref{env}{4.3.2}, $f_U^*(v)$ is a surjective homomorphism
$\OO_X^p|U\to f^*(\sh{G})|U$.
\end{env}

\begin{env}{5.2.5}
\label{env-0.5.2.5}
We say that an $\OO_X$-module $\sh{F}$ \emph{admits a finite presentation} if, for each
$x\in X$, there exists an open neighborhood $U$ of $x$ such that $\sh{F}|U$ is isomorphic
to a \emph{cokernel of a $(\OO_X|U)$-homomorphism $\OO_X^p|U\to\OO_X^q|U$}, $p$ and $q$ being
two integers $>0$. Such an $\OO_X$-module is therefore of finite type and quasi-coherent. If
$f:X\to Y$ is a morphism of ringed spaces, and if $\sh{G}$ is an $\OO_Y$-module admitting a
finite presentation, $f^*(\sh{G})$ admits a finite presentation, as shown in the argument of
\sref{env}{5.1.4}.
\end{env}

\begin{env}{5.2.6}
\label{env-0.5.2.6}
Let $\sh{F}$ be an $\OO_X$-module admitting a finite presentation \sref{env}{5.2.5}; then,
for each $\OO_X$-module $\sh{H}$, the canonical functorial homomorphism
\[
  (\shHom_{\OO_X}(\sh{F},\sh{H}))_x\longrightarrow\Hom_{\OO_x}(\sh{F}_x,\sh{H}_x)
\]
is \emph{bijective} (T, 4.1.1).
\end{env}

\begin{env}{5.2.7}
\label{env-0.5.2.7}
Let $\sh{F}$, $\sh{G}$ be two $\OO_X$-modules admitting a finite presentation. If, for an
$x\in X$, $\sh{F}_x$ ad $\sh{G}_x$ are \emph{isomorphic} as $\OO_x$-modules, then there
exists an open neighborhood $U$ of $x$ such that $\sh{F}|U$ and $\sh{G}|U$ are
\emph{isomorphic}. Indeed, if $\vphi:\sh{F}_x\to\sh{G}_x$ and $\psi:\sh{G}_x\to\sh{F}_x$ are
an isomorphism and its inverse isomorphism, then there exists, according to
\sref{env}{5.2.6}, an open neighborhood $V$ of $x$ and a section $u$ (resp. $v$) of
$\shHom_{\OO_X}(\sh{F},\sh{G})$ (resp. $\shHom_{\OO_X}(\sh{G},\sh{F})$) over $V$ such
\oldpage{47}
that $u_x=\vphi$ (resp. $v_x=\psi$). As $(u\circ v)_x$ and $(v\circ u)_x$ are the identity
automorphisms, there exists an open neighborhood $U\subset V$ of $x$ such that $(u\circ v)|U$
and $(v\circ u)|U$ are the identity automorphisms, hence the proposition.
\end{env}

\subsection{Coherent sheaves}
\label{0-prelim-5.3}

\begin{env}{5.3.1}
\label{env-0.5.3.1}
We say that an $\OO_X$-module $\sh{F}$ is \emph{coherent} if it satisfies the two following
conditions:
\begin{enumerate}[label=(\alph*)]
  \item $\sh{F}$ is of finite type.
  \item for each open $U\subset X$, integer $n>0$, and homomorphism $u:\OO_X^n|U\to\sh{F}|U$,
        the kernel of $u$ is of finite type.
\end{enumerate}
We note that these two conditions are of a \emph{local} nature.

For most of the proofs of the properties of coherent sheaves in what follows,
cf. (FAC, I, 2).
\end{env}

\begin{env}{5.3.2}
\label{env-0.5.3.2}
Each coherent $\OO_X$-module admits a finite presentation \sref{env}{5.2.5}; the inverse is
not necessarily true, since $\OO_X$ itself is not necessarily a coherent $\OO_X$-module.

Each $\OO_X$-submodule \emph{of finite type} of a coherent $\OO_X$-module is coherent; each
\emph{finite} direct sum of coherent $\OO_X$-modules is a coherent $\OO_X$-module.
\end{env}

\begin{env}{5.3.3}
\label{env-0.5.3.3}
If $0\to\sh{F}\to\sh{G}\to\sh{H}\to 0$ is an exact sequence  of $\OO_X$-modules and if two of
these $\OO_X$-modules are coherent, so is the third.
\end{env}

\begin{env}{5.3.4}
\label{env-0.5.3.4}
If $\sh{F}$ and $\sh{G}$ are two coherent $\OO_X$-modules, $u:\sh{F}\to\sh{G}$ a
homomorphism, then $\Im(u)$, $\Ker(u)$, and $\Coker(u)$ are coherent $\OO_X$-modules. In
particular, if $\sh{F}$ and $\sh{G}$ are $\OO_X$-submodules of a coherent $\OO_X$-module,
then $\sh{F}+\sh{G}$ and $\sh{F}\cap\sh{G}$ are coherent.
\end{env}

\begin{env}{5.3.5}
\label{env-0.5.3.5}
If $\sh{F}$ and $\sh{G}$ are two coherent $\OO_X$-modules, then so are
$\sh{F}\otimes_{\OO_X}\sh{G}$ are $\shHom_{\OO_X}(\sh{F},\sh{G})$.
\end{env}

\begin{env}{5.3.6}
\label{env-0.5.3.6}
Let $\sh{F}$ be a coherent $\OO_X$-module, $\sh{J}$ a coherent sheaf of ideals of $\OO_X$.
Then the $\OO_X$-module $\sh{J}\sh{F}$ is coherent, as the image of
$\sh{J}\otimes_{\OO_X}\sh{F}$ under the canonical homomorphism
$\sh{J}\otimes_{\OO_X}\sh{F}\to\sh{F}$ (\sref{env}{5.3.4} and \sref{env}{5.3.5}).
\end{env}

\begin{env}{5.3.7}
\label{env-0.5.3.7}
We say that an $\OO_X$-algebra $\sh{A}$ is \emph{coherent} if it is coherent as an
$\OO_X$-module. In particular, $\OO_X$ is a \emph{coherent sheaf of rings} if, and only if,
for each open $U\subset X$ and each homomorphism of the form $u:\OO_X^p|U\to\OO_X|U$, the
kernel of $u$ is an $(\OO_X|U)$-module of finite type.

If $\OO_X$ is a coherent sheaf of rings, each $\OO_X$-module $\sh{F}$ admitting a finite
presentation \sref{env}{5.2.5} is coherent, according to \sref{env}{5.3.4}.

The \emph{annihilator} of an $\OO_X$-module $\sh{F}$ is the kernel $\sh{J}$ of the canonical
homomorphism $\OO_X\to\shHom_{\OO_X}(\sh{F},\sh{F})$ which sends each section
$s\in\Gamma(U,\OO_X)$ to the multiplication by $s$ map in $\Hom(\sh{F}|U,\sh{F}|U)$; if
$\OO_X$ is coherent and if $\sh{F}$ is a coherent $\OO_X$-module, then $\sh{J}$ is coherent
(\sref{env}{5.3.4} and \sref{env}{5.3.5}) and for each $x\in X$, $\sh{J}_x$ is the
annihilator of $\sh{F}_x$ \sref{env}{5.2.6}.
\end{env}

\begin{env}{5.3.8}
\label{env-0.5.3.8}
\oldpage{48}
Suppose that $\OO_X$ is coherent; let $\sh{F}$ be a coherent $\OO_X$-module, $x$ a point of
$X$, $M$ a submodule of finite type of $\sh{F}_x$; then there exists an open neighborhood $U$
of $x$ and a coherent $(\OO_X|U)$-submodule $\sh{G}$ of $\sh{F}|U$ such that $\sh{G}_x=M$
(T, 4.1, Lemma~1).

This result, along with the properties of the $\OO_X$-submodules of a coherent
$\OO_X$-module, impose the necessary conditions on the rings $\OO_x$ such that $\OO_X$ is
coherent. For example \sref{env}{5.3.4}, the intersection of two ideals of finite type of
$\OO_x$ must still be an ideal of finite type.
\end{env}

\begin{env}{5.3.9}
\label{env-0.5.3.9}
Supppose that $\OO_X$ is coherent, and let $M$ be an $\OO_x$-module admitting a finite
presentation, therefore isomorphic to a cokernel of a homomorphism $\vphi:\OO_x^p\to\OO_x^q$;
then there exists an open neighborhood $U$ of $X$ and a coherent $(\OO_X|U)$-module $\sh{F}$
such that $\sh{F}_x$ is isomorphic to $M$. Indeed, according to \sref{env}{5.2.6}, there
exists a section $u$ of $\shHom_{\OO_X}(\OO_X^p,\OO_X^q)$ such that $u_x=\vphi$; the cokernel
$\sh{F}$ of the homomorphism $u:\OO_X^p|U\to\OO_X^q|U$ gives the answer \sref{env}{5.3.4}.
\end{env}

\begin{env}{5.3.10}
\label{env-0.5.3.10}
Suppose that $\OO_X$ is coherent, and let $\sh{J}$ be a coherent sheaf of ideals of $\OO_X$.
For a $(\OO_X/\sh{J})$-module $\sh{F}$ to be coherent, it is necessary and sufficient that it
is coherent as a $\OO_X$-module. In particular $\OO_X/\sh{J}$ is a coherent sheaf of rings.
\end{env}

\begin{env}{5.3.11}
\label{env-0.5.3.11}
Let $f:X\to Y$ be a morphism of ringed spaces, and suppose that $\OO_X$ is coherent; then,
for each coherent $\OO_Y$-module $\sh{G}$, $f^*(\sh{G})$ is a coherent $\OO_X$-module.
Indeed, with the notations of \sref{env}{5.2.4}, we can assume that $\sh{G}|V$ is the
cokernel of a homomorphism $v:\OO_Y^q|V\to\OO_Y^p|V$; as $f_U^*$ is right exact,
$f^*(\sh{G})|U=f_U^*(\sh{G}|V)$ is the cokernel of the homomorphism
$f_U^*(v):\OO_X^q|U\to\OO_X^p|U$, hence our assertion.
\end{env}

\begin{env}{5.3.12}
\label{env-0.5.3.12}
Let $Y$ be a closed subset of $X$, $j:Y\to X$ the canonical injection, $\OO_Y$ a sheaf of
rings on $Y$, and put $\OO_X=j_*(\OO_Y)$. For a $\OO_Y$-module $\sh{G}$ to be of finite type
(resp. quasi-coherent, coherent), it is necessary and sufficient that $j_*(\sh{G})$ is an
$\OO_X$-module of finite type (resp. quasi-coherent, coherent).
\end{env}

\subsection{Locally free sheaves}
\label{0-prelim-5.4}

\begin{env}{5.4.1}
\label{env-0.5.4.1}
Let $X$ be a ringed space. We say that an $\OO_X$-module $\sh{F}$ is \emph{locally free}, if,
for each $x\in X$, there exists an open neighborhood $U$ of $x$ such that $\sh{F}|U$ is
isomorphic to a $(\OO_X|U)$-module of the form $\OO_X^{(I)}|U$, where $I$ can depend on $U$.
If for each $U$, $I$ is finite, we say that $\sh{F}$ is \emph{of finite rank}; if for each
$U$, $I$ has the same finite number of elements $n$, we say that $\sh{F}$ is \emph{of rank
$n$}. A locally free $\OO_X$-module of rank $1$ is called \emph{invertible}
(cf. \sref{env}{5.4.3}). If $\sh{F}$ is a locally free $\OO_X$-module of finite rank, for
each $x\in X$, $\sh{F}_x$ is a free $\OO_x$-module of finite rank $n(x)$, and there exists a
neighborhood $U$ of $x$ such that $\sh{F}|U$, is of rank $n(x)$; if $X$ is connected, then
$n(x)$ is \emph{constant}.

It is clear that each locally free sheaf is quasi-coherent, and if $\OO_X$ is a coherent
sheaf of rings, each locally free $\OO_X$-module of finite rank is coherent.

If $\sh{L}$ is locally free, $\sh{L}\otimes_{\OO_X}\sh{F}$ is an \emph{exact} functor in
$\sh{F}$ to the category of $\OO_X$-modules.

We will mostly consider locally free $\OO_X$-modules of finite rank,
\oldpage{49}
and when we speak of locally free sheaves without specifying, it will be understood that
they are of \emph{finite rank}.
\end{env}

\begin{env}{5.4.2}
\label{env-0.5.4.2}
If $\sh{L}$, $\sh{F}$ are two $\OO_X$-modules, we have a canonical functorial homomorphism
\[
  \dual{\sh{L}}\otimes_{\OO_X}\sh{F}=\shHom_{\OO_X}(\sh{L},\OO_X)\otimes_{\OO_X}\sh{F}
  \longrightarrow\shHom_{\OO_X}(\sh{L},\sh{F})
  \tag{5.4.2.1}
\]
defined in the following way: for each open set $U$, send any pair $(u,t)$, where
$u\in\Gamma(U,\shHom_{\OO_X}(\sh{L},\OO_X))=\Hom(\sh{L}|U,\OO_X|U)$ and
$t\in\Gamma(U,\sh{F})$, to the element of $\Hom(\sh{L}|U,\sh{F}|U)$ which, for each $x\in U$,
sends $s_x\in\sh{L}_x$ to the element $u_x(s_x)t_x$ of $\sh{F}_x$. If $\sh{L}$ is
\emph{locally free of finite rank}, this homomorphism is \emph{bijective}; the property being
local, we can in fact reduce to the case where $\sh{L}=\OO_X^n$; as for each $\OO_X$-module
$\sh{G}$, $\shHom_{\OO_X}(\OO_X^n,\sh{G})$ is canonically isomorphic to $\sh{G}^n$, we have
reduced to the case $\sh{L}=\sh{O}_X$, which is immediate.
\end{env}

\begin{env}{5.4.3}
\label{env-0.5.4.3}
If $\sh{L}$ is invertible, so is its dual $\dual{\sh{L}}=\shHom_{\OO_X}(\sh{L},\OO_X)$, since
we can immediately reduce (as the question is local) to the case $\sh{L}=\OO_X$. In addition,
we have a canonical isomorphism
\[
  \shHom_{\OO_X}(\sh{L},\OO_X)\otimes_{\OO_X}\sh{L}\isoto\OO_X
  \tag{5.4.3.1}
\]
as, according to \sref{env}{5.3.2}, it suffices to define a canonical isomorphism
$\shHom_{\OO_X}(\sh{L},\sh{L})\isoto\OO_X$. For \emph{each} $\OO_X$-module $\sh{F}$, we have
a canonical homomorphism $\OO_X\isoto\shHom_{\OO_X}(\sh{F},\sh{F})$ \sref{env}{5.3.7}. It
remains to prove that if $\sh{F}=\sh{L}$ is invertible, this homomorphism is bijective, and
as the question is local, it reduces to the case $\sh{L}=\OO_X$, which is immediate.

Due to the above, we put $\sh{L}^{-1}=\shHom_{\OO_X}(\sh{L},\OO_X)$, and we say that
$\sh{L}^{-1}$ is the \emph{inverse} of $\sh{L}$. The terminology ``invertible sheaf'' can be
justified in the following way when $X$ is reduced to a point and $\OO_X$ is a \emph{local}
ring $A$ with maximal ideal $\mathfrak{m}$; if $M$ and $M'$ are two $A$-modules ($M$ being of
finite type) such that $M\otimes_A M'$ is isomorphic to $A$, as
$(A/\mathfrak{m})\otimes_A(M\otimes_A M')$ identifies with
$(M/\mathfrak{m}M)\otimes_{A/\mathfrak{m}}(M'/\mathfrak{m}M')$, this latter tensor product
of vector spaces over the field $A/\mathfrak{m}$ is isomorphic to $A/\mathfrak{m}$, which
requires $M/\mathfrak{m}M$ and $M'/\mathfrak{m}M'$ to be of dimension $1$. For each element
$z\in M$ not in $\mathfrak{m}M$, we have $M=Az+\mathfrak{m}M$, which implies that $M=Az$
according to Nakayama's lemma, $M$ being of finite type. Moreover, as the annihilator of $z$
kills $M\otimes_A M'$, which is isomorphic to $A$, this annihilator is $\{0\}$, and as a
result $M$ is \emph{isomorphic to $A$}. In the general case, this shows that $\sh{L}$ is an
$\OO_X$-module of finite type, such that there exists an $\OO_X$-module $\sh{F}$ for which
$\sh{L}\otimes_{\OO_X}\sh{F}$ is isomorphic to $\OO_X$, and if in addition the rings $\OO_x$
are local rings, then the $\sh{L}_x$ is an $\OO_x$-module isomorphic to $\OO_x$ for each
$x\in X$. If $\OO_X$ and $\sh{L}$ are assumed to be \emph{coherent}, we then conclude that
$\sh{L}$ is invertible according to \sref{env}{5.2.7}.
\end{env}

\begin{env}{5.4.4}
\label{env-0.5.4.4}
If $\sh{L}$ and $\sh{L}'$ are two invertible $\OO_X$-modules, then so is
$\sh{L}\otimes_{\OO_X}\sh{L}'$, since the question is local, we can assume that
$\sh{L}=\OO_X$, and the result is then trivial. For each integer $n\geqslant 1$, we denote by
$\sh{L}^{\otimes n}$ the tensor product of $n$ copies of the sheaf
\oldpage{50}
$\sh{L}$; we put by convention $\sh{L}^{\otimes 0}=\OO_X$, and for $n\geqslant 1$,
$\sh{L}^{\otimes(-n)}=(\sh{L}^{-1})^{\otimes n}$. With these notations, there is then a
\emph{canonical functorial isomorphism}
\[
  \sh{L}^{\otimes m}\otimes_{\OO_X}\sh{L}^{\otimes n}\isoto\sh{L}^{\otimes(n+m)}
  \tag{5.4.4.1}
\]
for any rational integers $m$, $n$: indeed, by definition, we immediately reduce to the
case where $m=-1$, $n=1$, and the isomorphism in question is then that defined in
\sref{env}{5.4.3}.
\end{env}

\begin{env}{5.4.5}
\label{env-0.5.4.5}
Let $f:Y\to X$ be a morphism of ringed spaces. If $\sh{L}$ is a locally free
(resp. invertible) $\OO_X$-module, $f^*(\sh{L})$ is a locally free (resp. invertible)
$\OO_Y$-module: this follows immediately from that the inverse images of the two locally
isomorphic $\OO_X$-modules are locally isomorphic, that $f^*$ commutes with finite direct
sums, and that $f^*(\OO_X)=\OO_Y$ \sref{env}{4.3.4}. In addition, we know that we have a
canonical functorial homomorphism $f^*(\dual{\sh{L}})\to\dual{(f^*(\sh{L}))}$
\sref{env}{4.4.6}, and when $\sh{L}$ is locally free, this homomorphism is \emph{bijective}:
indeed, we again reduce to the case where $\sh{L}=\OO_X$ which is trivial. We conclude that
if $\sh{L}$ is invertible, $f^*(\sh{L}^{\otimes n})$ canonically identifies with
$(f^*(\sh{L}))^{\otimes n}$ for each rational integer $n$.
\end{env}

\begin{env}{5.4.6}
\label{env-0.5.4.6}
Let $\sh{L}$ be an invertible $\OO_X$-module; we denote by $\Gamma_*(X,\sh{L})$ or simply
$\Gamma_*(\sh{L})$ the abelian group direct sum
$\bigoplus_{n\in\bb{Z}}\Gamma(X,\sh{L}^{\otimes n})$; we equip it with the structure of a
\emph{graded ring}, by corresponding to a pair $(s_n,s_m)$, where
$s_n\in\Gamma(X,\sh{L}^{\otimes n})$, $s_m\in\Gamma(X,\sh{L}^{\otimes m})$, the section of
$\sh{L}^{\otimes(n+m)}$ over $X$ which corresponds canonically (5.4.4.1) to the section
$s_n\otimes s_m$ of $\sh{L}^{\otimes n}\otimes_{\OO_X}\sh{L}^{\otimes m}$; the associativity
of this multiplication is verified in an immediate way. It is clear that $\Gamma_*(X,\sh{L})$
is a covariant functor in $\sh{L}$, with values in the category of graded rings.

If now $\sh{F}$ is any $\OO_X$-module, we put
\[
  \Gamma_*(\sh{L},\sh{F})
  =\bigoplus_{n\in\bb{Z}}\Gamma(X,\sh{F}\otimes_{\OO_X}\sh{L}^{\otimes n}).
\]
We equip this abelian group with the structure of a \emph{graded module} over the graded
ring $\Gamma_*(\sh{L})$ in the following way: to a pair $(s_n,u_m)$, where
$s_n\in\Gamma(X,\sh{L}^{\otimes n})$ and
$u_m\in\Gamma(X,\sh{F}\otimes_{\OO_X}\sh{L}^{\otimes m})$, we associate the section of
$\sh{F}\otimes_{\OO_X}\sh{L}^{\otimes(m+n)}$ which canonically corresponds (5.4.4.1) to
$s_n\otimes u_m$; the verification of the module axioms are immediate. For $X$ and $\sh{L}$
fixed, $\Gamma_*(\sh{L},\sh{F})$ is a covariant functor in $\sh{F}$ with values in the
category of graded $\Gamma_*(\sh{L})$-modules; for $X$ and $\sh{F}$ fixed, it is a covariant
functor in $\sh{L}$ with values in the category of abelian groups.

If $f:Y\to X$ is a morphism of ringed spaces, the canonical homomorphism (4.4.3.2)
$\rho:\sh{L}^{\otimes n}\to f_*(f^*(\sh{L}^{\otimes n}))$ defines a homomorphism of
abelian groups $\Gamma(X,\sh{L}^{\otimes n})\to\Gamma(Y,f^*(\sh{L}^{\otimes n}))$, and as
$f^*(\sh{L}^{\otimes n})=(f^*(\sh{L}))^{\otimes n})$, it follows from the definitions of the
canonical homomorphisms (4.4.3.2) and (5.4.4.1) that the above homomorphism define a
\emph{functorial homomorphism of graded rings $\Gamma_*(\sh{L})\to\Gamma_*(f^*(\sh{L}))$}.
The same canonical homomorphism \sref{env}{4.4.3} similarly defines a homomorphism of abelian
groups
$\Gamma(X,\sh{F}\otimes_{\OO_X}\sh{L}^{\otimes n})
\to\Gamma(Y,f^*(\sh{F}\otimes_{\OO_X}\sh{L}^{\otimes n}))$, and as
\[
  f^*(\sh{F}\otimes_{\OO_X}\sh{L}^{\otimes n})
  =f^*(\sh{F})\otimes_{\OO_Y}(f^*(\sh{L}))^{\otimes n}
  \quad(4.3.3.1),
\]
\oldpage{51}
these homomorphism (for $n$ variable) define a \emph{di-homomorphism of graded modules
$\Gamma_*(\sh{L},\sh{F})\to\Gamma_*(f^*(\sh{L}),f^*(\sh{F}))$}.
\end{env}

\begin{env}{5.4.7}
\label{env-0.5.4.7}
One can show that there exists a \emph{set} $\mathfrak{M}$ (also denoted $\mathfrak{M}(X)$)
of invertible $\OO_X$-modules such that each invertible $\OO_X$-module is isomorphic to an
element of $\mathfrak{M}$ only one;\footnote{See the book in preparation cited in the
introduction.} we define on $\mathfrak{M}$ a composition law by sendins two elements
$\sh{L}$, $\sh{L}$' of $\mathfrak{M}$ to the unique element of $\mathfrak{M}$ isomorphic to
$\sh{L}\otimes_{\OO_X}\sh{L}'$. With this composition law, \emph{$\mathfrak{M}$ is a group
isomorphic to the cohomology group $\HH^1(X,\OO_X^*)$}, where $\OO_X^*$ is the subsheaf of
$\OO_X$ such that $\Gamma(U,\OO_X^*)$ is the group of invertible elements of the ring
$\Gamma(U,\OO_X)$ for each open $U\subset X$ ($\OO_X^*$ is therefore a sheaf of
\emph{multiplicative} abelian groups).

We will note that for all open $U\subset X$, the group of sections $\Gamma(U,\OO_X^*)$
canonically identifies with the \emph{automorphism group} of the $(\OO_X|U)$-module
$\OO_X|U$, the identification sending a section $\varepsilon$ of $\OO_X^*$ over $U$ to the
automorphism $u$ of $\OO_X|U$ such that $u_x(s_x)=\varepsilon_x s_x$ for all $x\in X$ and
all $s_x\in\OO_x$. Then let $\mathfrak{U}=(U_\lambda)$ be an open cover of $X$; the data, for
each pair of indices $(\lambda,\mu)$, of an automorphism $\theta_{\lambda\mu}$ of
$\OO_X|(U_\lambda\cap U_\mu)$ is the same as giving a \emph{$1$-cochain} of the cover
$\mathfrak{U}$, with values in $\OO_X^*$, and say that the $\theta_{\lambda\mu}$ satisfy the
gluing condition \sref{env}{3.3.1}, meaning that the corresponding cochain is a
\emph{cocycle}. Similarly, the data, for each $\lambda$, of an automorphism $\omega_\lambda$
of $\OO_X|U_\lambda$ is the same as the data of a $0$-cochain of the cover $\mathfrak{U}$,
with values in $\OO_X^*$, and its \emph{coboundary} corresponds to the family of
automorphisms
$(\omega_\lambda|(U_\lambda\cap U_\mu)\circ(\omega_\mu|(U_\lambda\cap U_\mu)^{-1}$. We can
send each $1$-cocycle of $\mathfrak{U}$ with values in $\OO_X^*$ to the element of
$\mathfrak{M}$ isomorphic to an invertible $\OO_X$-module obtained by gluing with respect to
the family of automorphisms $(\theta_{\lambda\mu})$ corresponding to this cocycle, and to
two cohomologous coycles correspond two equal elements of $\mathfrak{M}$ \sref{env}{3.3.2};
in other words, we so define a map
$\vphi_\mathfrak{U}:\HH^1(\mathfrak{U},\OO_X^*)\to\mathfrak{M}$. In addition, if
$\mathfrak{B}$ is a second open cover of $X$, finer than $\mathfrak{U}$, the diagram
\[
  \xymatrix{
    \HH^1(\mathfrak{U},\OO_X^*)\ar[rd]^{\vphi_\mathfrak{U}}\ar[dd]\\
    & \mathfrak{M}\\
    \HH^1(\mathfrak{B},\OO_X^*)\ar[ur]_{\vphi_\mathfrak{B}}
  }
\]
where the vertical arrow is the canonical homomorphism (G, II, 5.7), is commutative, as a
result of \sref{env}{3.3.3}. By passing to the inductive limit, we therefore obtain a map
$\HH^1(X,\OO_X^*)\to\mathfrak{M}$, the \v Cech cohomology group $\check{\HH}^1(X,\OO_X^*)$
identifying as with know with the first homomology group $\HH^1(X,\OO_X^*)$
(G, II, 5.9, Cor. of Thm.~5.9.1). This map is \emph{surjective}: indeed, by definition, for
each invertible $\OO_X$-module $\sh{L}$, there is an open cover $(U_\lambda)$ of $X$ such
that $\sh{L}$ is obtained by gluing the sheaves $\OO_X|U_\lambda$ \sref{env}{3.3.1}. It is
also \emph{injective}, since it suffices to prove for the maps
$\HH^1(\mathfrak{U},\OO_X)\to\mathfrak{M}$, and this follows from \sref{env}{3.3.2}. It
remains to show that
\oldpage{52}
the bijection thus defined is a group homomorphism. Given two invertible $\OO_X$-modules
$\sh{L}$, $\sh{L}'$, there is an open cover $(U_\lambda)$ such that $\sh{L}|U_\lambda$ and
$\sh{L}'|U_\lambda$ are isomorphic to $\OO_X|U_\lambda$ for each $\lambda$; so there is for
each index $\lambda$ an element $a_\lambda$ (resp. $a_\lambda'$) of
$\Gamma(U_\lambda,\sh{L})$ (resp. $\Gamma(U_\lambda,\sh{L}')$) such that the elements of
$\Gamma(U_\lambda,\sh{L})$ (resp. $\Gamma(U_\lambda,\sh{L}')$) are the
$s_\lambda\cdot a_\lambda$ (resp. $s_\lambda\cdot a_\lambda'$), where $s_\lambda$ varies over
$\Gamma(U_\lambda,\OO_X)$. The corresponding cocycles $(\varepsilon_{\lambda\mu})$,
$(\varepsilon_{\lambda\mu}')$ are such that $s_\lambda\cdot a_\lambda=s_\mu\cdot a_\mu$
(resp. $s_\lambda\cdot a_\lambda'=s_\mu\cdot a_\mu'$) over $U_\lambda\cap U_\mu$ is
equivalent to $s_\lambda=\varepsilon_{\lambda\mu}s_\mu$
(resp. $s_\lambda=\varepsilon_{\lambda\mu}'s_\mu$) over $U_\lambda\cap U_\mu$. As the
sections of $\sh{L}\otimes_{\OO_X}\sh{L}'$ over $U_\lambda$ are the finite sums of the
$s_\lambda s_\lambda'\cdot(a_\lambda\otimes a_\lambda')$ where $s_\lambda$ and $s_\lambda'$
vary over $\Gamma(U_\lambda,\OO_X)$, it is clear that the cocycle
$(\varepsilon_{\lambda\mu},\varepsilon_{\lambda\mu}')$ corresponds to
$\sh{L}\otimes_{\OO_X}\sh{L}'$, which finishes the proof.\footnote{For a general form of this
result, see the book cited in the note of p.~51.}
\end{env}

\begin{env}{5.4.8}
\label{env-0.5.4.8}
Let $f=(\psi,\omega)$ be a morphism $Y\to X$ of ringed spaces. The functor $f^*(\sh{L})$ to
the category of free $\OO_X$-modules defines a map (which we still denote $f^*$ by abuse of
language) of the set $\mathfrak{M}(X)$ to the set $\mathfrak{M}(Y)$. Second, we have a
canonical homomorphism (T, 3.2.2)
\[
  \HH^1(X,\OO_X^*)\longrightarrow\HH^1(Y,\OO_Y^*).
  \tag{5.4.8.1}
\]
When we canonically identify \sref{env}{5.4.7} $\mathfrak{M}(X)$ and $\HH^1(X,\OO_X^*)$
(resp. $\mathfrak{M}(Y)$ and $\HH^1(Y,\OO_Y^*)$), the homomorphism (5.4.8.1) \emph{identifies
with the map $f^*$}. Indeed, if $\sh{L}$ comes from a cocycle $(\varepsilon_{\lambda\mu})$
corresponding to an open cover $(U_\lambda)$ of $X$, it suffices to show that $f^*(\sh{L})$
comes from a cocycle whose cohomology class is the image under (5.4.8.1) of
$(\varepsilon_{\lambda\mu})$. If $\theta_{\lambda\mu}$ is the automorphism of
$\OO_X|(U_\lambda\cap U_\mu)$ which corresponds to $\varepsilon_{\lambda\mu}$, it is clear
that $f^*(\sh{L})$ is obtained by gluing the $\OO_Y|\psi^{-1}(U_\lambda)$ by means of the
automorphisms $f^*(\theta_{\lambda\mu})$, and it then suffices to check that these latter
automorphisms corresponds to the cocycle $(\omega^\sharp(\varepsilon_{\lambda\mu}))$, which
follows immediately from te definitions (we can identify $\varepsilon_{\lambda\mu}$ with its
canonical image under $\rho$ \sref{env}{3.7.2}, a section of $\psi^*(\OO_X^*)$ over
$\psi^{-1}(U_\lambda\cap U_\mu)$).
\end{env}

\begin{env}{5.4.9}
\label{env-0.5.4.9}
Let $\sh{E}$, $\sh{F}$ be two $\OO_X$-modules, $\sh{F}$ assumed to be \emph{locally free},
and let $\sh{G}$ be an \emph{$\OO_X$-module extension of $\sh{F}$ by $\sh{E}$}, in other
words there exists an exact sequence
$0\to\sh{E}\xrightarrow{i}\sh{G}\xrightarrow{p}\sh{F}\to 0$. Then, for each $x\in X$, there
exists an open neighborhood $U$ of $x$ such that $\sh{G}|U$ is isomorphic to the \emph{direct
sum $\sh{E}|U\oplus\sh{F}|U$}. In fact, we can reduce to the case where $\sh{F}=\OO_X^n$; let
$e_i$ ($1\leqslant i\leqslant n$) be the canonical sections \sref{env}{5.5.5} of $\OO_X^n$;
there then exists an open neighborhood $U$ of $x$ and $n$ sections $s_i$ of $\sh{G}$ over $U$
such that $p(s_i|U)=e_i|U$ for $1\leqslant i\leqslant n$. That being so, let $f$ be the
homomorphism $\sh{F}|U\to\sh{G}|U$ defined by the sections $s_i|U$ \sref{env}{5.1.1}. It is
immediate that for each open $V\subset U$, and each section $s\in\Gamma(V,\sh{G})$ we have
$s-f(p(s))\in\Gamma(V,\sh{E})$, hence our assertion.
\end{env}

\begin{env}{5.4.10}
\label{env-0.5.4.10}
Let $f:X\to Y$ be a morphism of ringed spaces, $\sh{F}$ an $\OO_X$-module, $\sh{L}$ a locally
free $\OO_Y$-module of finite rank. Then there exists a canonical isomorphism
\[
  f_*(\sh{F})\otimes_{\OO_Y}\sh{L}\isoto f_*(\sh{F}\otimes_{\OO_X}f^*(\sh{L}))
  \tag{5.4.10.1}
\]

\oldpage{53}
Indeed, for each $\OO_Y$-module $\sh{L}$, we have a canonical homomorphism
\[
  f_*(\sh{F})\otimes_{\OO_Y}\sh{L}
  \xrightarrow{1\otimes\rho}f_*(\sh{F})\otimes_{\OO_Y}f_*(f^*(\sh{L}))
  \xrightarrow{\alpha}f_*(\sh{F}\otimes_{\OO_X}f^*(\sh{L})),
\]
$\rho$ the homomorphism (4.4.3.2) and $\alpha$ the homomorphism (4.2.2.1). To show that when
$\sh{L}$ is locally free, this homomorphism is bijective, it suffices, the question being
local, to consider the case where $\sh{L}=\OO_X^n$; in addition, $f_*$ and $f^*$ commute with
finite direct sums, so we can assume $n=1$, and in this case the proposition follows
immediately from the definitions and from the relation $f^*(\OO_Y)=\OO_X$.
\end{env}

\subsection{Sheaves on a locally ringed space}
\label{0-prelim-5.5}

\begin{env}{5.5.1}
\label{env-0.5.5.1}
We say that a ringed space $(X,\OO_X)$ is a \emph{locally ringed space} if, for each
$x\in X$, $\OO_x$ is a local ring; these ringed spaces will be by far the most frequent
ringed spaces that we will consider in this work. We then denote by $\mathfrak{m}_x$ the
\emph{maximal ideal} of $\OO_x$, by $\kres(x)$ the \emph{residue field
$\OO_x/\mathfrak{m}_x$}; for each $\OO_X$-module $\sh{F}$, each open set $U$ of $X$, each
point $x\in U$, and each section $f\in\Gamma(U,\sh{F})$, we denote by $f(x)$ the \emph{class}
of the germ $f_x\in\sh{F}_x$ mod $\mathfrak{m}_x\sh{F}_x$, and we say that this is the
\emph{value} of $f$ at the point $x$. The relation $f(x)=0$ then means that
$f_x\in\mathfrak{m}_x\sh{F}_x$; when this is so, we say (by abuse of language) that
\emph{$f$ is zero at $x$}. We will take care not to confuse this relation with $f_x=0$.
\end{env}

\begin{env}{5.5.2}
\label{env-0.5.5.2}
Let $X$ be a locally ringed space, $\sh{L}$ an invertible $\OO_X$-module, $f$ a section of
$\sh{L}$ over $X$. There is then an \emph{equivalence} between the three following properties
for a point $x\in X$:
\begin{enumerate}[label=(\alph*)]
  \item \emph{$f_x$ is a generator of $\sh{L}_x$};
  \item \emph{$f_x\not\in\mathfrak{m}_x\sh{L}_x$} (in other words, $f(x)\neq 0$);
  \item \emph{there exists a section $g$ of $\sh{L}^{-1}$ over an open neighborhood $V$ of
        $x$ such that the canonical image of $f\otimes g$ in $\Gamma(V,\OO_X)$
        \sref{env}{5.4.3} is the unit section}.
\end{enumerate}

Indeed, the question being local, we can reduce to the case where $\sh{L}=\OO_X$; the
equivalence of (a) and (b) are then evident, and it is clear that (c) implies (b).
Conversely, if $f_x\not\in\mathfrak{m}_x$, $f_x$ is invertible in $\OO_x$, say $f_x g_x=1_x$.
By definition of germs of sections, this means that there exists a neighborhood $V$ of $x$
and a section $g$ of $\OO_X$ over $V$ such that $fg=1$ in $V$, hence (c).

It follows immediately from the condition (c) that the set $X_f$ of $x$ satisfying the
equivalent conditions (a), (b), (c) is \emph{open} in $X$; following the terminology
introduced in \sref{env}{5.5.1}, this is the set of the $x$ for which \emph{$f$ does not
vanish}.
\end{env}

\begin{env}{5.5.3}
\label{env-0.5.5.3}
Under the hypotheses of \sref{env}{5.5.2}, let $\sh{L}'$ be a second invertible
$\OO_X$-module; then, if $f\in\Gamma(X,\sh{L})$, $g\in\Gamma(X,\sh{L}')$, we have
\[
  X_f\cap X_g=X_{f\otimes g}.
\]

We can in fact reduce immediately to the case where $\sh{L}=\sh{L}'=\OO_X$ (the question
being local); as $f\otimes g$ then canonically identifies with the product $fg$, the
proposition is evident.
\end{env}

\begin{env}{5.5.4}
\label{env-0.5.5.4}
\oldpage{54}
Let $\sh{F}$ be a locally free $\OO_X$ of rank $n$; it is immediate that $\wedge^p\sh{F}$ is
a locally free $\OO_X$-module of rank $\binom{n}{p}$ if $p\leqslant n$, reduced to $0$ if
$p>n$, since the question is local and we can reduce to the case where $\sh{F}=\OO_X^n$;
in addition, for each $x\in X$, $(\wedge^p\sh{F})_x/\mathfrak{m}_x(\wedge^p\sh{F})_x$ is a
vector space of dimension $\binom{n}{p}$ over $\kres(x)$, which canonically identifies with
$\wedge^p(\sh{F}_x/\mathfrak{m}_x\sh{F}_x)$. Let $s_1,\dots,s_p$ be the sections of $\sh{F}$
over an open set $U$ of $X$, and let $s=s_1\wedge\cdots\wedge s_p$, which is a section of
$\wedge^p\sh{F}$ over $U$ \sref{env}{4.1.5}; we have $s(x)=s_1(x)\wedge\cdots\wedge s_p(x)$,
and as a result, we say that the $s_1(x),\dots,s_p(x)$ are \emph{linearly dependent} means
that $s(x)=0$. We conclude that the \emph{set of the $x\in X$ such that $s_1(x),\dots,s_p(x)$
are linearly independent is open in $X$}: it suffices in fact, by reducing to the case where
$\sh{F}=\OO_X^n$, to apply \sref{env}{5.5.2} to the section image of $s$ under one of the
projections of $\wedge^p\sh{F}=\OO_X^{\binom{n}{p}}$ to the $\binom{n}{p}$ factors.

In particular, if $s_1,\dots,s_n$ are $n$ sections of $\sh{F}$ over $U$ such that
$s_1(x),\dots,s_n(x)$ are linealy independent for each point $x\in U$, the homomorphism
$u:\OO_X^n|U\to\sh{F}|U$ defined by the $s_i$ \sref{env}{5.1.1} is an \emph{isomorphism}:
indeed, we can restrict to the case where $\sh{F}=\OO_X^n$ and where we canonically identify
$\wedge^n\sh{F}$ and $\OO_X$; $s=s_1\wedge\cdots\wedge s_n$ is then an \emph{invertible}
section of $\OO_X$ over $U$, and we define an inverse homomorphism for $u$ by means of the
Cramer formulas.
\end{env}

\begin{env}{5.5.5}
\label{env-0.5.5.5}
Let $\sh{E}$, $\sh{F}$ be two locally free $\OO_X$-modules (of finite rank), and let
$u:\sh{E}\to\sh{F}$ be a homomorphism. For there to exist a neighborhood $U$ of $x\in X$ such
that $u|U$ is \emph{injective} and that $\sh{F}|U$ is \emph{the direct sum of the
$u(\sh{E}|U$ and of a locally free $(\OO_X|U)$-submodule $\sh{G}$}, it is necessary and
sufficient that $u_x:\sh{E}_x\to\sh{F}_x$ gives, by passing to quotients, an \emph{injective}
homomorphism of vector spaces
$\sh{E}_x/\mathfrak{m}_x\sh{E}_x\to\sh{F}_x/\mathfrak{m}_x\sh{F}_x$. The condition is indeed
\emph{necessary}, since $\sh{F}_x$ is then the direct sum of the free $\OO_x$-modules
$u_x(\sh{E}_x)$ and $\sh{G}_x$, so $\sh{F}_x/\mathfrak{m}_x\sh{F}_x$ is the direct sum of
$u_x(\sh{E}_x)/\mathfrak{m}_x u_x(\sh{E}_x)$ and of $\sh{G}_x/\mathfrak{m}_x\sh{G}_x$. The
condition is \emph{sufficient}, since we can reduce to the case where $\sh{E}=\OO_X^m$; let
$s_1,\dots,s_m$ be the images under $u$ of the sections $e_i$ of $\OO_X^m$ such that
$(e_i)_y$ is equal to the $i$-th element of the canonical basis of $\OO_y^m$ for each
$y\in Y$ (\emph{canonical sections} of $\OO_X^m$); by hypothesis $s_1(x),\dots,s_m(x)$ are
linearly independent, so, if $\sh{F}$ is of rank $n$, there exists $n-m$ sections
$s_{m+1},\dots,s_n$ of $\sh{F}$ over a neighborhood $V$ of $x$ such that the $s_i(x)$
($1\leqslant i\leqslant n$) form a basis for $\sh{F}_x/\mathfrak{m}_x\sh{F}_x$. There then
exists \sref{env}{5.5.4} a neighborhood $U\subset V$ of $x$ such that the $s_i(y)$
($1\leqslant i\leqslant n$) form a basis for $\sh{F}_y/\mathfrak{m}_y\sh{F}_y$ for each
$y\in V$, and we conclude \sref{env}{5.5.4} that there is an isomorphism from $\sh{F}|U$ to
$\OO_X^n|U$, sending the $s_i|U$ ($1\leqslant i\leqslant m$) to the $e_i|U$, which finishes
the proof.
\end{env}

\section{Flatness}
\label{0-prelim-6}

\begin{env}{6.0}
\label{env-0.6.0.0}
The notion of flatness is due to J.-P.~Serre \cite{16}; in the following, we omit the
proofs of the results which are presented in the \emph{Alg\`ebre commutative} of N.~Bourbaki,
to which we refer the reader. We assume that all rings are commutative.\footnote{See the
expos\'e cited of N.~Bourbaki for the generalization from most of the results to the
noncommutative case.}

\oldpage{55}
If $M$, $N$ are two $A$-modules, $M'$ (resp. $N'$) a submodule of $M$ (resp. $N$), we denote
by $\Im(M'\otimes_A N')$ the submodule of $M\otimes_A N$, the image under the canonical map
$M'\otimes_A N'\to M\otimes_A N$.
\end{env}

\subsection{Flat modules}
\label{0-prelim-6.1}

\subsection{Change of ring}
\label{0-prelim-6.2}

\subsection{Local nature of flatness}
\label{0-prelim-6.3}

\subsection{Faithfully flat modules}
\label{0-prelim-6.4}

\subsection{Restriction of scalars}
\label{0-prelim-6.5}

\subsection{Faithfully flat rings}
\label{0-prelim-6.6}

\subsection{Flat morphisms of ringed spaces}
\label{0-prelim-6.7}

\section{Adic rings}
\label{0-prelim-7}

\subsection{Admissible rings}
\label{0-prelim-7.1}

\subsection{Adic rings and projective limits}
\label{0-prelim-7.2}

\subsection{Pre-adic Noetherian rings}
\label{0-prelim-7.3}

\subsection{Quasi-finite modules over local rings}
\label{0-prelim-7.4}

\subsection{Rings of restricted formal series}
\label{0-prelim-7.5}

\subsection{Completed rings of fractions}
\label{0-prelim-7.6}

\subsection{Completed tensor products}
\label{0-prelim-7.7}

\subsection{Topologies on modules of homomorphisms}
\label{0-prelim-7.8}



\clearpage

%%%%%%%%%%%
%% EGA I %%
%%%%%%%%%%%
\chapter{The language of schemes}

\section*{Summary}
\label{1-schemes.summary}

\begin{tabular}{ll}
  \textsection1. & Affine schemes.\\
  \textsection2. & Preschemes and morphisms of preschemes.\\
  \textsection3. & Products of preschemes.\\
  \textsection4. & Subpreschemes and immersion morphisms.\\
  \textsection5. & Reduced preschemes; separation condition.\\
  \textsection6. & Finiteness conditions.\\
  \textsection7. & Rational maps.\\
  \textsection8. & Chevalley schemes.\\
  \textsection9. & Supplement on quasi-coherent sheaves.\\
  \textsection10. & Formal schemes.
\end{tabular}\\

\oldpage{79}
The \textsection\textsection1--8 do little more than develop a language, which will be used
in the following. It should be noted, however, that in accordance with the general spirit of
this treatise, \textsection\textsection7--8 will be used less than the others, and in a less
essential way; we have moreover spoken of Chevalley's schemes only to make the link with the
language of Chevalley \cite{1} and Nagata \cite{9}. The \textsection9 gives definitions and
results on quasi-coherent sheaves, some of which are no longer limited to a translation into
a ``geometric'' language of known notions of commutative algebra, but are already of a global
nature; they will be indispensable, in the following chapters, for the global study of
morphisms. Finally, \textsection10 introduces a generalization of the notion of schemes,
which will be used as an intermediary in Chapter~III to formulate and prove in a
convenient way the fundamental results of the cohomological study of the proper morphisms;
moreover, it should be noted that the notion of formal schemes seems indispensable to express
certain facts of the ``theory of modules'' (classification problems of algebraic varieties).
The results of \textsection10 will not be used before \textsection3 of Chapter~III and it is
recommended to omit reading until then.
\bigskip

\setcounter{section}{0}
\section{Affine schemes}
\label{1-schemes-1}

\setcounter{subsection}{0}
\subsection{The prime spectrum of a ring}
\label{1-schemes-1.1}

\begin{env}{1.1.1}
\label{env-1.1.1.1}
\oldpage{80}
\emph{Notation}. Let $A$ be a (commutative) ring, $M$ an $A$-module. In
this chapter and the following, we will constantly use the following notations:
\begin{enumerate}[label=--]
  \item $\Spec(A)=$ \emph{set of prime ideals} of $A$, also called the
        \emph{prime spectrum} of $A$; for an $x\in X=\Spec(A)$, it will often be
        convenient to write $\mathfrak{j}_x$ instead of $x$. When $\Spec(A)$ is
        \emph{empty}, it is necessary and sufficient that the ring $A$ is
        reduced to $0$.
  \item $A_x=A_{\mathfrak{j}_x}=$ \emph{(local) ring of fractions $S^{-1}A$},
        where $S=A-\mathfrak{j}_x$.
  \item $\mathfrak{m}_x=\mathfrak{j}_x A_{\mathfrak{j}_x}=$ \emph{maximal ideal of $A$}.
  \item $\kres(x)=A_x/\mathfrak{m}_x=$ \emph{residue field of $A_x$},
        canonically isomorphic to the field of fractions
        of the integral ring $A/\mathfrak{j}_x$, to which it is identified.
  \item $f(x)=$ \emph{class of $f$} mod. $\mathfrak{j}_x$ in
        $A/\mathfrak{j}_x\subset\kres(x)$,
        for $f\in A$ and $x\in X$. We still say that $f(x)$ is the \emph{value}
        of $f$ at a point $x\in\Spec(A)$; the relations $f(x)=0$ and $f\in\mathfrak{j}_x$
        are \emph{equivalent}.
  \item $M_x=M\otimes_A A_x=$ \emph{module of denominators of fractions in
        $A-\mathfrak{j}_x$}.
  \item $\rad(E)=$ \emph{radical of the ideal of $A$ generated by a subset $E$ of $A$}.
  \item $V(E)=$ \emph{set of $x\in X$ such that $E\subset\mathfrak{j}_x$} (or the set of
        $x\in X$ such that $f(x)=0$ for all $f\in E$), for $E\subset A$. So we have
        \[
          \rad(E)=\bigcap_{x\in V(E)}\mathfrak{j}_x.
          \tag{1.1.1.1}
        \]
  \item $V(f)=V(\{f\})$ for $f\in A$.
  \item $D(f)=X-V(f)=$ \emph{set of $x\in X$ where $f(x)\neq 0$}.
\end{enumerate}
\end{env}

\begin{envs}[Proposition]{1.1.2}
\label{env-1.1.1.2}
We have the following properties:
\begin{enumerate}[label=\rm{(\roman*)}]
  \item $V(0)=X$, $V(1)=\emp$.
  \item The relation $E\subset E'$ implies $V(E)\supset V(E')$.
  \item For each family $(E_\lambda)$ of subsets of $A$,
        $V(\bigcup_\lambda E_\lambda)=V(\sum_\lambda E_\lambda)
          =\bigcap_\lambda V(E_\lambda)$.
  \item $V(EE')=V(E)\cup V(E')$.
  \item $V(E)=V(\mathfrak{r}(E))$.
\end{enumerate}
\end{envs}
The properties (i), (ii), (iii) are trivial, and (v) follows from (ii) and from the
formula (1.1.1.1). It is evident that $V(EE')\supset V(E)\cap V(E')$; conversely, if
$x\not\in V(E)$ and $x\not\in V(E')$, there exists $f\in E$ and $f'\in E'$ such that
$f(x)\neq 0$ and $f'(x)\neq 0$ in $\k(x)$, hence $f(x)f'(x)\neq 0$, i.e., $x\not\in V(EE')$,
which proves (iv).

Proposition \sref{env}{1.1.2} shows, among other things, that sets of the form $V(E)$
(where $E$ runs through all the subsets of $A$) are the \emph{closed sets} of a topology on
$X$, which we will call the \emph{spectral topology}\footnote{The introduction of this
topology in algebraic geometry is due to Zariski. So this topology is usually called
the ``Zariski topology'' of $X$.}; unless expressely stated otherwise, always assume
$X=\Spec(A)$ with the spectral topology.

\begin{env}{1.1.3}
\label{env-1.1.1.3}
\oldpage{81}
For each subset $Y$ of $X$, we denote by $\mathfrak{j}(Y)$ the set of $f\in A$ such that
$f(y)=0$ for all $y\in Y$; equivalently, $\mathfrak{j}(Y)$ is the intersection of the prime
ideals $\mathfrak{j}_y$ for $y\in Y$. It is clear that the relation $Y\subset Y'$ implies
that $\mathfrak{j}(Y)\supset\mathfrak{j}(Y')$ and that we have
\[
  \mathfrak{j}\bigg(\bigcup_\lambda Y_\lambda\bigg)=\bigcap_\lambda\mathfrak{j}(Y_\lambda)
  \tag{1.1.3.1}
\]
for each family $(Y_\lambda)$ of subsets of $X$. Finally we have
\[
  \mathfrak{j}(\{x\})=\mathfrak{j}_x.
  \tag{1.1.3.2}
\]
\end{env}

\begin{envs}[Proposition]{1.1.4}
\label{prop-1.1.1.4}
\begin{enumerate}[label=\rm{(\roman*)}]
  \item For each subset $E$ of $A$, we have $\mathfrak{j}(V(E))=\rad(E)$.
  \item For each subset $Y$ of $X$, $V(\mathfrak{j}(Y))=\overline{Y}$, the closure of $Y$
        in $X$.
\end{enumerate}
\end{envs}
(i) is an immeidate consequence of the definitions and (1.1.1.1); on the other hand,
$V(\mathfrak{j}(Y))$ is closed and contains $Y$; conversely, if $Y\subset V(E)$, we have
$f(y)=0$ for $f\in E$ and all $y\in Y$, so $E\subset\mathfrak{j}(Y)$,
$V(E)\supset V(\mathfrak{j}(Y))$, which proves (ii).

\begin{envs}[Corollary]{1.1.5}
\label{cor-1.1.1.5}
The closed subsets of $X=\Spec(A)$ and the ideals of $A$ equal to their radicals (otherwise
the intersection of prime ideals) correspond bijectively by the \unsure{descent} maps
$Y\mapsto\mathfrak{j}(Y)$, $\mathfrak{a}\mapsto V(\mathfrak{a})$; the union $Y_1\cup Y_2$ of
two closed subsets corresponds to $\mathfrak{j}(Y_1)\cap\mathfrak{j}(Y_2)$, and the
intersection of any family $(Y_\lambda)$ of closed subsets corresponds to the radical of the
sum of the $\mathfrak{j}(Y_\lambda)$.
\end{envs}

\begin{envs}[Corollary]{1.1.6}
\label{cor-1.1.1.6}
If $A$ is a Noetherian ring, $X=\Spec(A)$ is a Noetherian space.
\end{envs}

Note that the converse of this corollary is false, as shown
in the example of a non-Noetherian integral ring with a single prime ideal $\neq\{0\}$, for
example a non-discrete valuation ring of rank $1$.

As an example of ring $A$ whose spectrum is not a Noetherian space, one can consider the ring
$\sh{C}(Y)$ of continuous real functions on an infinite compact space $Y$; we know that as a
whole, $Y$ corresponds with the set of maximal ideals of $A$, and it is easy to see that the
topology induced on $Y$ by that of $X=\Spec(A)$ is the initial topology of $Y$. Since $Y$ is
not a Noetherian space, the same is true for $X$.

\begin{envs}[Corollary]{1.1.7}
\label{cor-1.1.1.7}
For each $x\in X$, the closure of $\{x\}$ is the set of $y\in X$ such that
$\mathfrak{j}_x\subset\mathfrak{j}_y$. For $\{x\}$ to be closed, it is necessary and
sufficient that $\mathfrak{j}_x$ is maximal.
\end{envs}

\begin{envs}[Corollary]{1.1.8}
\label{cor-1.1.1.8}
The space $X=\Spec(A)$ is a Kolmogoroff space.
\end{envs}

If $x$, $y$ are two distinct points of $X$, we have either
$\mathfrak{j}_x\not\subset\mathfrak{j}_y$ or $\mathfrak{j}_y\not\subset\mathfrak{j}_x$, so
one of the points $x$, $y$ does not belong to the closure of the other.

\begin{env}{1.1.9}
\label{env-1.1.1.9}
According to Proposition (\sref{env}{1.1.2}, (iv)), for two elements $f$, $g$ of $A$, we have
\[
  D(fg)=D(f)\cap D(g).
  \tag{1.1.9.1}
\]
Note also that the relation $D(f)=D(g)$ means, according to Proposition
(\sref{prop}{1.1.4}, (i)) and Proposition (\sref{env}{1.1.2}, (v)) that $\rad(f)=\rad(g)$, or
that the minimal prime ideals containing $(f)$ and $(g)$ are the same; in particular, when
$f=ug$, where $u$ is invertible.
\end{env}

\begin{envs}[Proposition]{1.1.10}
\label{prop-1.1.1.10}
\oldpage{82}
\begin{enumerate}[label=\rm{(\roman*)}]
  \item When $f$ ranges over $A$, the sets $D(f)$ forms a basis for the topology of $X$.
  \item For each $f\in A$, $D(f)$ is quasi-compact. In particular $X=D(1)$ is quasi-compact.
\end{enumerate}
\end{envs}

\begin{enumerate}[label=(\roman*)]
  \item Let $U$ be an open set in $X$; by definition, we have $U=X-V(E)$ where $E$ is a subset of
$A$, and $V(E)=\bigcap_{f\in E}V(f)$, hence $U=\bigcup_{f\in E}D(f)$.

  \item According to (i), it is sufficient to prove that if $(f_\lambda)_{\lambda\in L}$ is a
family of elements of $A$ such that $D(f)\subset\bigcup_{\lambda\in L}D(f_\lambda)$, there
exists a finite subset $J$ of $L$ such that $D(f)\subset\bigcup_{\lambda\in J}D(f_\lambda)$.
Let $\mathfrak{a}$ be the ideal of $A$ generated by the $f_\lambda$; we have by hypothesis
that $V(f)\supset V(\mathfrak{a})$, so $\rad(f)\subset\rad(\mathfrak{a})$; as $f\in\rad(f)$,
there exists an integer $n\geqslant 0$ such that $f^n\in\mathfrak{a}$. But then $f^n$ belongs
to the ideal $\mathfrak{b}$ generated by the finite subfamily $(f_\lambda)_{\lambda\in J}$,
and we have $V(f)=V(f^n)\supset V(\mathfrak{b})=\bigcap_{\lambda\in J}V(f_\lambda)$, that is
to say, $D(f)\supset\bigcup_{\lambda\in J}D(f_\lambda)$.
\end{enumerate}

\begin{envs}[Proposition]{1.1.11}
\label{prop-1.1.1.11}
For each ideal $\mathfrak{a}$ of $A$, $\Spec(A/\mathfrak{a})$ canonically identifies with the
closed subspace $V(\mathfrak{a})$ of $\Spec(A)$.
\end{envs}

Indeed, we know there is a canonical bijective correspondence, respecting the inclusion order
structure, between ideals (resp. prime ideals) of $A/\mathfrak{a}$ and ideals (resp. prime
ideals) of $A$ containing $\mathfrak{a}$.

Recall that the set $\nilrad$ of nilpotent elements of $A$ (the \emph{nilradical} of $A$) is
an ideal equal to $\rad(0)$, the intersection of all the prime ideals of $A$
\pref{env}{1.1.1}.

\begin{envs}[Corollary]{1.1.12}
\label{cor-1.1.1.12}
The topological spaces $\Spec(A)$ and $\Spec(A/\nilrad)$ are canonically homeomorphic.
\end{envs}

\begin{envs}[Proposition]{1.1.13}
\label{prop-1.1.1.13}
For $X=\Spec(A)$ to be irreducible \pref{env}{2.1.1}, it is necessary and sufficient that the
ring $A/\nilrad$ is integral (or, equivalently, that the ideal $\nilrad$ is prime).
\end{envs}

By virtue of Corollary \sref{cor}{1.1.12}, we can restrict to the case $\nilrad=0$. If $X$ is
reducible, there exist two distinct closed subsets $Y_1$, $Y_2$ of $X$ such that
$X=Y_1\cup Y_2$, so $\mathfrak{j}(X)=\mathfrak{j}(Y_1)\cap\mathfrak{j}(Y_2)=0$, the ideals
$\mathfrak{j}(Y_1)$ and $\mathfrak{j}(Y_2)$ being distinct from $(0)$ \sref{cor}{1.1.5}; so
$A$ is not an integral. Conversely, if in $A$ there are elements $f\neq 0$, $g\neq 0$ such
that $fg=0$, we have $V(f)\neq X$, $V(g)\neq X$ (since the intersection of the prime ideals
of $A$ is $(0)$), and $X=V(fg)=V(f)\cup V(g)$.

\begin{envs}[Corollary]{1.1.14}
\label{cor-1.1.1.14}
\begin{enumerate}[label=\rm{(\roman*)}]
  \item In the bijective correspondence between closed subsets of $X=\Spec(A)$
        and ideals of $A$ equal to their roots, the irreducible closed subsets
        of $X$ correspond to the prime ideals of $A$. In particular, the irreducible
        components of $X$ correspond to the minimal prime ideals of $A$.
  \item The map $x\mapsto\overline{\{x\}}$ establishes a bijective correspondence
        between $X$ and the set of closed irreducible subsets of $X$
        (\emph{said otherwise,} all closed irreducible subsets of $X$ containing
        only one generic point).
\end{enumerate}
\end{envs}

(i) follows immediately from \sref{prop}{1.1.13} and \sref{prop}{1.1.11}; and for
proving (ii), we can, by virtue of \sref{prop}{1.1.11}, we restrict to the case where
$X$ is irreducible; then, according to Proposition \sref{prop}{1.1.13}, there exists
in $A$ a smaller prime ideal $\nilrad$, which corresponds to the generic point
\oldpage{83}
of $X$; in addition, $X$ does not admit only one generic point since it is a Kolmogoroff
space (\sref{cor}{1.1.8} and \pref{env}{2.1.3}).

\begin{envs}[Proposition]{1.1.15}
\label{prop-1.1.1.15}
If $\mathfrak{J}$ is an ideal in $A$ containing the radical $\nilrad(A)$, the only
neighborhood of $V(\mathfrak{J})$ in $X=\Spec(A)$ is the whole space $X$.
\end{envs}

Indeed, each maximal ideal of $A$ belongs by definition of $V(\mathfrak{J})$.
As each ideal $\mathfrak{a}$ of $A$ is contained in a maximal ideal, we have
$V(\mathfrak{a})\cap V(\mathfrak{J})\neq 0$, hence the proposition.

\subsection{Functorial properties of prime spectra of rings}
\label{1-schemes-1.2}       

\begin{env}{1.2.1}
\label{env-1.1.2.1}
Let $A$, $A'$ be two rings,
\[
  \vphi:A'\longrightarrow A
\]
a homomorphism of rings. For each prime ideal $x=\mathfrak{j}_x\in\Spec(A)=X$, the
ring $A'/\vphi^{-1}(\mathfrak{j}_x)$ is canonically isomorphic to a subring of
$A/\mathfrak{j}_x$, so it is integral, otherwise we say
$\vphi^{-1}(\mathfrak{j}_x)$ is a prime ideal of $A'$; we denote it by
${}^a\vphi(x)$, and we have also defined a map
\[
  {}^a\vphi:X=\Spec(A)\longrightarrow X'=\Spec(A')
\]
(also denoted $\Spec(\vphi)$) we call this the map \emph{associated} to the
homomorphism $\vphi$. We denote by $\vphi^x$ the injective homomorphism of
$A'/\vphi^{-1}(\mathfrak{j}_x)$ to $A/\mathfrak{j}_x$ induced by $\vphi$ by
passing to quotients, so the canonical extention is a monomorphism of fields
\[
  \vphi^x:\kres({}^a\vphi(x))\longrightarrow\kres(x);
\]
for each $f'\in A'$, we therefore have by definition
\[
  \vphi^x(f'({}^a\vphi(x)))=(\vphi(f'))(x)\quad(x\in X).
  \tag{1.2.1.1}
\]
\end{env}

\begin{envs}[Proposition]{1.2.2}
\label{prop-1.1.2.2}
\begin{enumerate}[label=\rm{(\roman*)}]
  \item For each subset $E'$ of $A'$, we have
        \[
          {}^a\vphi^{-1}(V(E'))=V(\vphi(E')),
          \tag{1.2.2.1}
        \]
        and in particular, for each $f'\in A'$,
        \[
          {}^a\vphi^{-1}(D(f'))=D(\vphi(f')).
          \tag{1.2.2.2}
        \]
  \item For each ideal $\mathfrak{a}$ of $A$, we have
        \[
          \overline{{}^a\vphi(V(\mathfrak{a}))}=V(\vphi^{-1}(\mathfrak{a})).
          \tag{1.2.2.3}
        \]
\end{enumerate}
\end{envs}

Indeed, the relation ${}^a\vphi(x)\in V(E')$ is by definition equivalent to
$E'\subset\vphi^{-1}(\mathfrak{j}_x)$, so $\vphi(E')\subset\mathfrak{j}_x$, and
finally $x\in V(\vphi(E'))$, hence (i). To prove (ii), we can suppose that
$\mathfrak{a}$ is equal to its radical, since $V(\rad(\mathfrak{a}))=V(\mathfrak{a})$
(\sref{prop}{1.1.2}, (v)) and
$\vphi^{-1}(\rad(\mathfrak{a}))=\rad(\vphi^{-1}(\mathfrak{a}))$; the relation
$f'\in\mathfrak{a}'$ is by definition equivalent to $f'(x')=0$ for each
$x\in{{}^a\vphi(Y)}$, so, by virtue of the formula (1.2.1.1), it is equivalent as well
to $\vphi(f')(x)=0$ for each $x\in Y$, or $\vphi(f')\in\mathfrak{j}(Y)=\mathfrak{a}$,
since $\mathfrak{a}$ is equal to its radical; hence (ii).

\begin{envs}[Corollary]{1.2.3}
\label{cor-1.1.2.3}
The map ${}^a\vphi$ is continuous.
\end{envs}

We remark that if $A''$ is a third ring, $\vphi'$ a homomorphism $A''\to A'$, we have
${}^a(\vphi'\circ\vphi)={}^a\vphi\circ{}^a\vphi'$; this result and Corollary
\sref{cor}{1.2.3} gives that $\Spec(A)$ is a \emph{contravariant functor} in $A$, from the
category of rings to that of topological spaces.

\begin{envs}[Corollary]{1.2.4}
\label{cor-1.1.2.4}
\oldpage{84}
Suppose that $\vphi$ is such that for each $f\in A$ written as $f=h\vphi(f')$, where $h$ is
invertible in $A$ (\emph{which is in particular the case when $\vphi$ is} surjective). Then
${}^a\vphi$ is a homeomorphism from $X$ to ${}^a\vphi(X)$.
\end{envs}

We show that for each subset $E\subset A$, there exists a subset $E'$ of $A'$ such that
$V(E)=V(\vphi(E'))$; according to the axiom ($T_0$) \sref{cor}{1.1.8} and the formula
(1.2.2.1), this implies first that ${}^a\vphi$ is injective, then, according to (1.2.2.1),
that ${}^a\vphi$ is a homeomorphism. Or, it suffices for each $f\in E$ to have a $f'\in A'$
such that $h\vphi(f')=f$ with $h$ invertible in $A$; the set $E'$ of these elements $f'$
provides the answer.

\begin{env}{1.2.5}
\label{env-1.1.2.5}
In particular, when $\vphi$ is the canonical homomorphism of $A$ to a ring quotient
$A/\mathfrak{a}$, we get \sref{cor}{1.1.12}, and ${}^a\vphi$ is the \emph{canonical
injection} of $V(\mathfrak{a})$, identifed with $\Spec(A/\mathfrak{a})$, in $X=\Spec(A)$.
\end{env}

Another particular case of \sref{cor}{1.2.4}:
\begin{envs}[Corollary]{1.2.6}
\label{cor-1.1.2.6}
If $S$ is a multiplicative subset of $A$, the spectrum $\Spec(S^{-1}A)$ identifies
canonically (with its topology) with the subspace of $X=\Spec(A)$ consisting of the $x$
such that $\mathfrak{j}_x\cap S=\emp$.
\end{envs}

We know by \pref{env}{1.2.6} that the prime ideals of $S^{-1}A$ are the ideals
$S^{-1}\mathfrak{j}_x$ such that $\mathfrak{j}_x\cap S=\emp$, and that we have
$\mathfrak{j}_x=(i_A^S)^{-1}(S^{-1}\mathfrak{j}_x)$. It suffices to apply the $i_A^S$
with Corollary \sref{cor}{1.2.4}.

\begin{envs}[Corollary]{1.2.7}
\label{cor-1.1.2.7}
For ${}^a\vphi(X)$ to be dense in $X'$, it is necessary and sufficient that each element
of the kernel $\Ker\vphi$ is nilpotent.
\end{envs}

Indeed, applying the formula (1.2.2.3) to the ideal $\mathfrak{a}=(0)$, we have
$\widetilde{{}^a\vphi(X)}=V(\Ker\vphi)$, and for $V(\Ker\vphi)=X$ to hold, it is necessary
and sufficient that $\Ker\vphi$ is contained in all the prime ideals of $A'$, that is to say
in the nilradical $\rad'$ of $A'$.

\subsection{Sheaf associated to a module}
\label{1-schemes-1.3}

\begin{env}{1.3.1}
\label{env-1.1.3.1}
Let $A$ be a commutative ring, $M$ an $A$-module, $f$ an element of $A$, $S_f$ the
multiplicative set of the $f^n$, where $n\geqslant 0$. Recall that we put $A_f=S_f^{-1}A$,
$M_f=S_f^{-1}M$. If $S_f'$ is the saturated multiplicative subset of $A$ consisting of the
$g\in A$ which divide an element of $S_f$, we know that $A_f$ and $M_f$ canonically identify 
with ${S_f'}^{-1}A$ and ${S_f'}^{-1}M$ \pref{env}{1.4.3}.
\end{env}

\begin{envs}[Lemma]{1.3.2}
\label{lem-1.1.3.2}
The following conditions are equivalent:
\begin{center}
\rm{(a)} $g\in S_f'$;
\rm{(b)} $S_g'\subset S_f'$;
\rm{(c)} $f\in\rad(g)$;
\rm{(d)} $\rad(f)\subset\rad(g)$;
\rm{(e)} $V(g)\subset V(f)$;
\rm{(f)} $D(f)\subset D(g)$.
\end{center}
\end{envs}
This follows immediately from the definitions and from \sref{cor}{1.1.5}.

\begin{env}{1.3.3}
\label{env-1.1.3.3}
If $D(f)=D(g)$, then Lemma (\sref{lem}{1.3.2}, (b)) shows that $M_f=M_g$. More generally, if
$D(f)\supset D(g)$, then $S_f'\subset S_g'$, and we know \pref{env}{1.4.1} that there exists
a canonical functorial homomorphism
\[
  \rho_{g,f}:M_f\longrightarrow M_g,
\]
and if $D(f)\supset D(g)\supset D(h)$, we have \pref{env}{1.4.4}
\[
  \rho_{h,g}\circ\rho_{g,f}=\rho_{h,f}.
  \tag{1.3.3.1}
\]
\end{env}

\oldpage{85}
When $f$ runs over the elements of $A-\mathfrak{j}_x$ (for a given $x$ in $X=\Spec(A)$), the
sets $S_f'$ constitute an increasing filtered set of subsets of $A-\mathfrak{j}_x$, since for
two elements $f$, $g$ of $A-\mathfrak{j}_x$, $S_f'$ and $S_g'$ are contained in $S_{fg}'$; as
the union of the $S_f'$ for $f\in A-\mathfrak{j}_x$ is $A-\mathfrak{j}_x$, we conclude
\pref{env}{1.4.5} that the $A_x$-module $M_x$ canonically identifies with the \emph{inductive
limit} $\varinjlim M_f$, relative to the family of homomorphisms $(\rho_{g,f})$. We denote by
\[
  \rho_x^f:M_f\longrightarrow M_x
\]
the canonical homomorphism for $f\in A-\mathfrak{j}_x$ (or, equivalently, $x\in D(f)$).

\begin{envs}[Definition]{1.3.4}
\label{defn-1.1.3.4}
We define the structure sheaf of the prime spectrum $X=\Spec(A)$ (resp. sheaf associated to
an $A$-module $M$) and denote it by $\widetilde{A}$ or $\OO_X$ (resp. $\widetilde{M}$) as the
sheaf of rings (resp. the $\widetilde{A}$-module) associated to the presheaf
$D(f)\mapsto A_f$ (resp. $D(f)\mapsto M_f$) over the basis $\mathfrak{B}$ of $X$ consisting
of the $D(f)$ for $f\in A$ (\sref{prop}{1.1.10}, \pref{env}{3.2.1}, and \pref{env}{3.5.6}).
\end{envs}

We saw \pref{env}{3.2.4} that the stalk $\widetilde{A}_x$ (resp. $\widetilde{M}_x$)
\emph{identifies with the ring $A_x$} (resp. \emph{the $A_x$-module $M_x$}); we denote by
\[
  \theta_f:A_f\longrightarrow\Gamma(D(f),\widetilde{A})
\]
\[
  (\text{resp. }\theta_f:M_f\longrightarrow\Gamma(D(f),\widetilde{M})),
\]
the canonical map, so that for each $x\in D(f)$ and each $\xi\in M_f$, we have
\[
  (\theta_f(\xi))_x=\rho_x^f(\xi).
  \tag{1.3.4.1}
\]

\begin{envs}[Proposition]{1.3.5}
\label{prop-1.1.3.5}
$\widetilde{M}$ is an exact covariant functor in $M$, from the category of $A$-modules to the
category of $\widetilde{A}$-modules.
\end{envs}

Indeed, let $M$, $N$ be two $A$-modules, $u$ a homomorphism $M\to N$; for each $f\in A$,
it canonically assigns to $u$ a homomorphism $u_f$ of the $A_f$-module $M_f$ to the
$A_f$-module $N_f$, and the diagram (for $D(g)\subset D(f)$)
\[
  \xymatrix{
    M_f\ar[r]^{u_f}\ar[d]_{\rho_{g,f}} & N_f\ar[d]^{\rho_{g,f}}\\
    M_g\ar[r]^{u_g} & N_g
  }
\]
is commutative \pref{thm}{1.4.1}; these homomorphisms then define a homomorphism of
$\widetilde{A}$-modules $\widetilde{u}:\widetilde{M}\to\widetilde{N}$ \pref{env}{3.2.3}. In
addition, for each $x\in X$, $\widetilde{u}_x$ is the inductive limit of the $u_f$ for
$x\in D(f)$ ($f\in A$), and as a result \pref{env}{1.4.5} if we canonically identify
$\widetilde{M}_x$ and $\widetilde{N}_x$ with $M_x$ and $N_x$ respectively, $\widetilde{u}_x$
identifies with the homomorphism $u_x$ canonically induced by $u$. If $P$ is a third
$A$-module, $v$ a homomorphism $N\to P$ and $w=v\circ u$, it is immediate that
$w_x=v_x\circ u_x$, so $\widetilde{w}=\widetilde{v}\circ\widetilde{u}$. We have therefore
clearly defined a \emph{covariant functor} $\widetilde{M}$ in $M$, from the category of
$A$-modules to that of $\widetilde{A}$-modules. \emph{This functor is exact}, since for each
$x\in X$, $M_x$ is an exact functor in $M$ \pref{env}{1.3.2}; in addition, we have
$\Supp(M)=\Supp(\widetilde{M})$ by the definitions \pref{env}{1.7.1} and \pref{env}{3.1.6}.

\oldpage{86}
\begin{envs}[Proposition]{1.3.6}
\label{prop-1.1.3.6}
For each $f\in A$, the open subset $D(f)\subset X$ canonically identifies with the prime
spectrum $\Spec(A_f)$, and the sheaf $\widetilde{M_f}$ associated to the $A_f$-module
$M_f$ canonically identifies with the restriction $\widetilde{M}|D(f)$.
\end{envs}

The first assertion is a particular case of \sref{cor}{1.2.6}. In addition, for $g\in A$ is
such that $D(g)\subset D(f)$, $M_g$ canonically identifies with the module of fractions of
$M_f$ whose denominators are the powers of the canonical image of $g$ in $A_f$
\pref{env}{1.4.6}. The canonical identification of $\widetilde{M_f}$ with
$\widetilde{M}|D(f)$ then follows from the definitions.

\begin{envs}[Theorem]{1.3.7}
\label{thm-1.1.3.7}
For each $A$-module $M$ and each $f\in A$, the homomorphism
\[
  \theta_f:M_f\longrightarrow\Gamma(D(f),\widetilde{M})
\]
is bijective (\emph{in other words, the presheaf $D(f)\mapsto M_f$ is a} sheaf). In
particular, $M$ identifies by $\theta_1$ with $\Gamma(X,\widetilde{M})$.
\end{envs}

We note that, if $M=A$, $\theta_f$ is a homomorphism of structure rings; Theorem
\sref{thm}{1.3.7} implies then that, if we identify the rings $A_f$ and
$\Gamma(D(f),\widetilde{A})$ by means of the $\theta_f$, the homomorphism
$\theta_f:M_f\to\Gamma(D(f),\widetilde{M})$ is an isomorphism of \emph{modules}.

We show first that $\theta_f$ is \emph{injective}. Indeed, if $\xi\in M_f$ is such that
$\theta_f(\xi)=0$, then this means that for each prime ideal $\mathfrak{p}$ of $A_f$, there
exists $h\not\in\mathfrak{p}$ such that $h\xi=0$; as the annihilator of $\xi$ is not
contained in any prime ideal of $A_f$, each $A_f$ integral, so $\xi=0$.

It remains to show that $\theta_f$ is \emph{surjective}; we can reduce to the case where
$f=1$, the general case deduced by ``localizing'' using \sref{prop}{1.3.6}. Now let $s$ be a
section of $\widetilde{M}$ over $X$; according to \sref{defn}{1.3.4} and
(\sref{prop}{1.1.10}, (ii)), there exists a \emph{finite} cover $(D(f_i))_{i\in I}$ of $X$
($f_i\in A$) such that, for each $i\in I$, the restriction $s_i=s|D(f_i)$ is of the form
$\theta_{f_i}(\xi_i)$, where $\xi_i\in M_{f_i}$. If $i$, $j$ are two indices of $I$, if
the restrictions of $s_i$ and $s_j$ to $D(f_i)\cap D(f_j)=D(f_i f_j)$ are equal, then
it follows by definition of $M$ that
\[
  \rho_{f_i f_j,f_i}(\xi_i)=\rho_{f_i f_j,f_j}(\xi_j).
  \tag{1.3.7.1}
\]
By definition, we can write, for each $i\in I$, $\xi_i=z_i/f_i^{n_i}$, where $z_i\in M$, and
as $I$ is finite, by multiplying each $z_i$ by a power of $f_i$, we can suppose that all the
$n_i$ are equal to the same $n$. Then, by definition, (1.3.7.1) implies that there exists an
integer $m_{ij}\geqslant 0$ such that $(f_i f_j)^{m_{ij}}(f_j^n z_i-f_i^n z_j)=0$, and we can
moreover suppose that the $m_{ij}$ are equal to the same integer $m$; replacing then the
$z_i$ by $f_i^m z_i$, it remains to prove for the case where $m=0$, in other words, the case
where we have
\[
  f_j^n z_i=f_i^n z_j
  \tag{1.3.7.2}
\]
for any $i$, $j$. We have $D(f_i^n)=D(f_i)$, and as the $D(f_i)$ form a cover of $X$,
the ideal generated by the $f_i^n$ is $A$; in other words, there exist elements $g_i\in A$
such that $\sum_i g_i f_i^n=1$. Then consider the element $z=\sum_i g_i z_i$ of $M$; in
(1.3.7.2), we have $f_i^n z=\sum_j g_j f_i^n z_j=(\sum_j g_j f_j^n)z_i=z_i$, where by
definition $\xi_i=z/1$ in $M_{f_i}$. We conclude
\oldpage{87}
that the $s_i$ are the restrictions to $D(f_i)$ of $\theta_1(z)$, which proves that
$s=\theta_1(z)$ and finishes the proof.

\begin{envs}[Corollary]{1.3.8}
\label{cor-1.1.3.8}
Let $M$, $N$ be two $A$-modules; the canonical homomorphism $u\mapsto\widetilde{u}$ from
$\Hom_A(M,N)$ to $\Hom_{\widetilde{A}}(\widetilde{M},\widetilde{N})$ is bijective. In
particular, the relations $M=0$ and $\widetilde{M}=0$ are equivalent.
\end{envs}

Consider the canonical homomorphism $v\mapsto\Gamma(v)$ from
$\Hom_{\widetilde{A}}(\widetilde{M},\widetilde{N})$ to
$\Hom_{\Gamma(\widetilde{A})}(\Gamma(\widetilde{M}),\Gamma(\widetilde{N}))$; the latter
module canonically identifies with $\Hom_A(M,N)$ according to Theorem \sref{thm}{1.3.7}.
It remains to show that $u\mapsto\widetilde{u}$ and $v\mapsto\Gamma(v)$ are inverses of each
other; it is evident that $\Gamma(\widetilde{u})=u$ by definition of $\widetilde{u}$; on the
other hand, if we put $u=\Gamma(v)$ for
$v\in\Hom_{\widetilde{A}}(\widetilde{M},\widetilde{N})$, the map
$w:\Gamma(D(f),\widetilde{M})\to\Gamma(D(f),\widetilde{N})$ canonically induced from $v$
is such that the diagram
\[
  \xymatrix{
    M\ar[r]^u\ar[d]_{\rho_{f,1}} & N\ar[d]^{\rho_{f,1}}\\
    M_f\ar[r]^w & N_f
  }
\]
is commutative; so we have necessarily that $w=u_f$ for all $f\in A$ \pref{env}{1.2.4}, which
shows that $\widetilde{\Gamma(v)}=v$.

\begin{envs}[Corollary]{1.3.9}
\label{cor-1.1.3.9}
\begin{enumerate}[label=\rm{(\roman*)}]
  \item Let $u$ be a homomorphism from an $A$-module $M$ to an $A$-module $N$; then the
        sheaves associated to $\Ker u$, $\Im u$, $\Coker u$, are respectively
        $\Ker\widetilde{u}$, $\Im\widetilde{u}$, $\Coker\widetilde{u}$. In particular, for
        $\widetilde{u}$ to be injective (resp. surjective, bijective), it is necessary and
        sufficient that $u$ is.
  \item If $M$ is an inductive limit (resp. direct sum) of a family of $A$-modules
        $(M_\lambda)$, $\widetilde{M}$ is the inductive limit (resp. direct sum) of the
        family $(\widetilde{M_\lambda})$, via a canonical isomorphism.
\end{enumerate}
\end{envs}
\begin{enumerate}[label=(\roman*)]
  \item If suffices to apply the fact that $\widetilde{M}$ is an exact functor in $M$
        \sref{prop}{1.3.5} to the two exact sequenes of $A$-modules
        \[
          0\longrightarrow\Ker u\longrightarrow M\longrightarrow\Im u\longrightarrow 0,
        \]
        \[
          0\longrightarrow\Im u\longrightarrow N\longrightarrow\Coker u\longrightarrow 0.
        \]
        The second assertion then follows from Theorem \sref{thm}{1.3.7}.
  \item Let $(M_\lambda,g_{\mu\lambda})$ be an inductive system of $A$-modules, with
        inductive limit $M$, and let $g_\lambda$ be the canonical homomorphism
        $M_\lambda\to M$. As we have
        $\widetilde{g_{\nu\mu}}\circ\widetilde{g_{\mu\lambda}}=\widetilde{g_{\nu\lambda}}$
        and $\widetilde{g_\lambda}=\widetilde{g_\mu}\circ\widetilde{g_{\mu\lambda}}$ for
        $\lambda\leqslant\mu\leqslant\nu$,
        $(\widetilde{M_\lambda},\widetilde{g_{\mu\lambda}})$ is an inductive system of
        sheaves on $X$, and if we denote by $h_\lambda$ the canonical homomorphism
        $\widetilde{M_\lambda}\to\varinjlim\widetilde{M_\lambda}$, there is a unique
        homomorphism $v:\varinjlim\widetilde{M_\lambda}\to\widetilde{M}$ such that
        $v\circ h_\lambda=\widetilde{g_\lambda}$. To see that $v$ is bijective, it suffices
        to check, for each $x\in X$, that $v_x$ is a bijection from
        $(\varinjlim\widetilde{M_\lambda})_x$ to $\widetilde{M}_x$; but
        $\widetilde{M}_x=M_x$, and
        \[
          (\varinjlim\widetilde{M_\lambda})_x=\varinjlim(\widetilde{M_\lambda})_x
          =\varinjlim(M_\lambda)_x=M_x\quad\pref{env}{1.3.3}.
        \]
        Conversely, it follows from the definitions that $(\widetilde{g_\lambda})_x$ and
        $(h_\lambda)$ are all equal to the canonical map from $(M_\lambda)_x$ to $M_x$; as
        $(\widetilde{g_\lambda})_x=v_x\circ(h_\lambda)_x$, $v_x$ is the identity.

\oldpage{88}
        Finally, if $M$ is the direct sum of two $A$-modules $N$, $P$, it is immediate that
        $\widetilde{M}=\widetilde{N}\oplus\widetilde{P}$; each direct sum being the inductive
        limit of finite direct sums, the assertions of (ii) are proved.
\end{enumerate}

We note that \sref{cor}{1.3.8} proves that the sheaves isomorphic to the associated sheaves
of $A$-modules forms an \emph{abelian category} (T, I, 1.4).

We also note that it follows from \sref{cor}{1.3.9} that if $M$ is an $A$-module \emph{of
finite type}, that is to say there exists a surjective homomorphism $A^n\to M$, then there
exists a surjective homomorphism $\widetilde{A^n}\to\widetilde{M}$, in other words, the
$\widetilde{A}$-module $\widetilde{M}$ is \emph{generated by a finite family of sections over
$X$} \pref{env}{5.1.1}, and conversely.

\begin{env}{1.3.10}
\label{env-1.1.3.10}
If $N$ is a submodule of an $A$-module $M$, the canonical injection $j:N\to M$ gives by
\sref{cor}{1.3.9} an injective homomorphism $\widetilde{N}\to\widetilde{M}$, which allows us
to canonically identify $\widetilde{N}$ with a \emph{$\widetilde{A}$-submodule} of
$\widetilde{M}$; we will always assume we have made this identification. If $N$ and $P$ are
two submodules of $M$, we then have
\[
  (N+P)^\sim=\widetilde{N}+\widetilde{P},
  \tag{1.3.10.1}
\]
\[
  (N\cap P)^\sim=\widetilde{N}\cap\widetilde{P},
  \tag{1.3.10.2}
\]
since $N+P$ and $N\cap P$ are respectively the images of the canonical homomorphism
$N\oplus P\to M$, and the kernel of the canonical homomorphism $M\to(M/N)\oplus(M/P)$, and
it suffices to apply \sref{cor}{1.3.9}.

We conclude from (1.3.10.1) and (1.3.10.2) that if $\widetilde{N}=\widetilde{P}$, we have
$N=P$.
\end{env}

\begin{envs}[Corollary]{1.3.11}
\label{cor-1.1.3.11}
On the category of sheaves isomorphic to the associated sheaves of $A$-modules, the functor
$\Gamma$ is exact.
\end{envs}

Indeed, let
$\widetilde{M}\xrightarrow{\widetilde{u}}\widetilde{N}
\xrightarrow{\widetilde{v}}\widetilde{P}$ be an exact sequence corresponding to two
homomorphisms $u:M\to N$, $v:N\to P$ of $A$-modules. If $Q=\Im u$ and $R=\Ker v$, we have
$\widetilde{Q}=\Im\widetilde{u}=\Ker\widetilde{v}=\widetilde{R}$ (Corollary \sref{cor}{1.3.9}),
hence $Q=R$.

\begin{envs}[Corollary]{1.3.12}
\label{cor-1.1.3.12}
Let $M$, $N$ be two $A$-modules.
\begin{enumerate}[label=\rm{(\roman*)}]
  \item The sheaf associated to $M\otimes_A N$ canonically identifies with
        $\widetilde{M}\otimes_{\widetilde{A}}\widetilde{N}$.
  \item If in addition $M$ admits a finite presentation, the sheaf associated to
        $\Hom_A(M,N)$ canonically identifies with
        $\shHom_{\widetilde{A}}(\widetilde{M},\widetilde{N})$.
\end{enumerate}
\end{envs}
\begin{enumerate}[label=(\roman*)]
  \item The sheaf $\sh{F}=\widetilde{M}\otimes_{\widetilde{A}}\widetilde{N}$ is associated to
        the presheaf
        \[
          U\longmapsto\sh{F}(U)
          =\Gamma(U,\widetilde{M})\otimes_{\Gamma(U,\widetilde{A})}\Gamma(U,\widetilde{N}),
        \]
        $U$ varying over the basis (\sref{prop}{1.1.10}, (i)) of $X$ consisting of the
        $D(f)$, where $f\in A$. We have that $\sh{F}(D(f))$ canonically identifies with
        $M_f\otimes_{A_f}N_f$ according to \sref{thm}{1.3.7} and \sref{prop}{1.3.6}.
        Moreover, we have that the $A_f$-module $M_f\otimes_{A_f}N_f$ is canonically
        isomorphic to $(M\otimes_A N)_f$ \pref{env}{1.3.4}, which itself is canonically
        isomorphic to $\Gamma(D(f),(M\otimes_A N)^\sim)$ (\sref{thm}{1.3.7} and
        \sref{prop}{1.3.6}). In addition, we check immediately that the canonical
        isomorphisms
        \[
          \sh{F}(D(f))\isoto\Gamma(D(f),(M\otimes_A N)^\sim)
        \]
\oldpage{89}
        thus obtained satisfy the compatibility conditions with respect to the restriction
        operations \pref{env}{1.4.2}, so they define canonical functorial isomorphism
        \[
          \widetilde{M}\otimes_{\widetilde{A}}\widetilde{N}\isoto(M\otimes_A N)^\sim.
        \]
  \item The sheaf $\sh{G}=\shHom_{\widetilde{A}}(\widetilde{M},\widetilde{N})$ is associated
        to the presheaf
        \[
          U\longmapsto\sh{G}(U)=\Hom_{\widetilde{A}|U}(\widetilde{M}|U,\widetilde{N}|U),
        \]
        $U$ varying over the basis of $X$ consisting of the $D(f)$. We have that
        $\sh{G}(D(f))$ canonically identifies with $\Hom_{A_f}(M_f,N_f)$ (\sref{prop}{1.3.6}
        and \sref{cor}{1.3.8}), which, according to the hypotheses on $M$, canonically
        identifies with $(\Hom_A(M,N))_f$ \pref{env}{1.3.5}. Finally, $(\Hom_A(M,N))_f$
        canonically identifies with $\Gamma(D(f),(\Hom_A(M,N))^\sim)$ (\sref{prop}{1.3.6} and
        \sref{thm}{1.3.7}), and the canonical isomorphisms
        $\sh{G}(D(f))\isoto\Gamma(D(f),(\Hom_A(M,N))^\sim)$ thus obtained are compatible with
        the restriction operations \pref{env}{1.4.2}; they thus define a canonical
        isomorphism
        $\shHom_{\widetilde{A}}(\widetilde{M},\widetilde{N})\isoto(\Hom_A(M,N))^\sim$.
\end{enumerate}

\begin{env}{1.3.13}
\label{env-1.1.3.13}
Now let $B$ be a (commutative) $A$-algebra; this can be interpreted by saying that $B$ is an
$A$-module such that we are given an element $e\in B$ and an $A$-homomorphism
$\vphi:B\otimes_A B\to B$, so that: 1\textsuperscript{st} the diagrams
\[
  \xymatrix{
    B\otimes_A B\otimes_A B\ar[r]^{\vphi\otimes 1}\ar[d]_{1\otimes\vphi} &
    B\otimes_A B\ar[d]^\vphi & &
    B\otimes_A B\ar[rr]^\sigma\ar[rd]_\vphi & &
    B\otimes_A B\ar[dl]^\vphi\\
    B\otimes_A B\ar[r]^\vphi &
    B & & & 
    B
  }
\]
($\sigma$ the canonical symmetry map) are commutative; 2\textsuperscript{nd}
$\vphi(e\otimes x)=\vphi(x\otimes e)=x$. According to \sref{cor}{1.3.12}, the homomorphism
$\widetilde{\vphi}:\widetilde{B}\otimes_{\widetilde{A}}\widetilde{B}\to\widetilde{B}$ of
$\widetilde{A}$-modules satisfies the analogous conditions, thus defines an
\emph{$\widetilde{A}$-algebra} structure on $\widetilde{B}$. In a similar way, the data of
a $B$-module $N$ is the same as the data of an $A$-module $N$ and an $A$-homomorphism
$\psi:B\otimes_A N\to N$ such that the diagram
\[
  \xymatrix{
    B\otimes_A B\otimes_A B\ar[r]^{\vphi\otimes 1}\ar[d]_{1\otimes\psi} &
    B\otimes_A N\ar[d]^\psi\\
    B\otimes_A N\ar[r]^\psi &
    N
  }
\]
is commutative and $\psi(e\otimes n)=n$; the homomorphism
$\widetilde{\psi}:\widetilde{B}\otimes_{\widetilde{A}}\widetilde{N}\to\widetilde{N}$
satisfies the analogous condition, and so defines a \emph{$\widetilde{B}$-module} structure
on $\widetilde{N}$.

In a similar way, we see that if $u:B\to B'$ (resp. $v:N\to N'$) is a homomorphism of
$A$-algebras (resp. of $B$-modules), $\widetilde{u}$ (resp. $\widetilde{v}$) is a
homomorphism of $\widetilde{A}$-algebras (resp. of $\widetilde{B}$-modules),
$\Ker\widetilde{u}$ is a $\widetilde{B}$-ideal (resp. $\Ker\widetilde{v}$,
$\Coker\widetilde{v}$, and $\Im\widetilde{v}$ are $\widetilde{B}$-modules). If $N$ is a
$B$-module, $\widetilde{N}$ is a $\widetilde{B}$-module of finite type if and only if $N$
is a $B$-module of finite type \pref{env}{5.2.3}.

\oldpage{90}
If $M$, $N$ are two $B$-modules, the $\widetilde{B}$-module
$\widetilde{M}\otimes_{\widetilde{B}}\widetilde{N}$ canonically identifies with
$(M\otimes_B N)^\sim$; similarly $\shHom_{\widetilde{B}}(\widetilde{M},\widetilde{N})$
canonically identifies with $(\Hom_B(M,N))^\sim$ when $M$ admits a finite presentation; the
proofs are similar to those in \sref{cor}{1.3.12}

If $\mathfrak{J}$ is an ideal of $B$, $N$ a $B$-module, then we have
$(\mathfrak{J}N)^\sim=\widetilde{\mathfrak{J}}\cdot\widetilde{N}$.

Finally, if $B$ is an $A$-algebra \emph{graded} by the $A$-submodules $B_n$ ($n\in\bb{Z}$),
the $\widetilde{A}$-algebra $\widetilde{B}$, the direct sum of the $\widetilde{A}$-modules
$\widetilde{B_n}$ \sref{cor}{1.3.9}, is graded by these $\widetilde{A}$-submodules, the axiom
of graded algebras giving that the image of the homomorphism $B_m\otimes B_n\to B$ is
contained in $B_{m+n}$. Similarly, if $M$ is a $B$-module graded by the submodules $M_n$,
then $\widetilde{M}$ is a $\widetilde{B}$-module graded by the $\widetilde{M_n}$.
\end{env}

\begin{env}{1.3.14}
\label{env-1.1.3.14}
If $B$ is an $A$-algebra, $M$ a submodule of $B$, then the $\widetilde{A}$-subalgebra of
$\widetilde{B}$ generated by $\widetilde{M}$ \pref{env}{4.1.3} is the
$\widetilde{A}$-subalgebra $\widetilde{C}$, where we denote by $C$ the subalgebra of $B$
generated by $M$. Indeed, $C$ is the direct sum of the submodules of $B$ which are the images
of the homomorphisms $\bigotimes^n M\to B$ ($n\geqslant 0$), and it suffices to apply
\sref{cor}{1.3.9} and \sref{cor}{1.3.12}.
\end{env}

\subsection{Quasi-coherent sheaves over a prime spectrum}
\label{1-schemes-1.4}

\begin{envs}[Theorem]{1.4.1}
\label{thm-1.1.4.1}
Let $X$ be the prime spectrum of a ring $A$, $V$ a quasi-compact open subset of $X$, and
$\sh{F}$ an $(\OO_X|V)$-module. The four following conditions are equivalent:
\begin{enumerate}[label=\rm{(\alph*)}]
  \item There exists an $A$-module $M$ such that $\sh{F}$ is isomorphic to
        $\widetilde{M}|V$.
  \item There exists a finite open cover $(V_i)$ of $V$ by sets of the form $D(f_i)$
        ($f_i\in A$) contained in $V$, such that, for each $i$, $\sh{F}|V_i$ is isomorphic to
        a sheaf of the form $\widetilde{M_i}$, where $M_i$ is an $A_{f_i}$-module.
  \item The sheaf $\sh{F}$ is quasi-coherent \pref{env}{5.1.3}.
  \item The two following properties are satisfied:
        \begin{enumerate}[label=\rm{(d\arabic*)}]
          \item For each $f\in A$ such that $D(f)\subset V$ and for each section
                $s\in\Gamma(D(f),\sh{F})$, there exists an integer $n\geqslant 0$ such that
                $f^n s$ extends to a section of $\sh{F}$ over $V$.
          \item For each $f\in A$ such that $D(f)\subset V$ and for each section
                $t\in\Gamma(V,\sh{F})$ such that the restriction of $t$ to $D(f)$ is $0$,
                there exists an integer $n\geqslant 0$ such that $f^n t=0$.
        \end{enumerate}
\end{enumerate}
\end{envs}
(In the statement of the conditions (d1) and (d2), we have tacitly identified $A$ and
$\Gamma(\widetilde{A})$ according to \sref{thm}{1.3.7}).

The fact that (a) implies (b) is an immediate consequence of \sref{prop}{1.3.6} and the fact
that the $D(f_i)$ form a basis for the topology of $X$ \sref{prop}{1.1.10}. As each
$A$-module is isomorphic to the cokernel of a homomorphism of the form $A^{(I)}\to A^{(J)}$,
\sref{prop}{1.3.6} proves that each sheaf associated to an $A$-module is quasi-coherent; so
(b) implies (c). Conversely, if $\sh{F}$ is quasi-coherent, each $x\in V$ has a neighborhood
of the form $D(f)\subset V$ such that $\sh{F}|D(f)$ is isomorphic to the cokernel of a
homomorphism $\widetilde{A_f}^{(I)}\to\widetilde{A_f}^{(J)}$, so a sheaf $\widetilde{N}$
associated to the module $N$, the cokernel of the corresponding homomorphism
$A_f^{(I)}\to A_f^{(J)}$ (\sref{cor}{1.3.8} and \sref{cor}{1.3.9}); as $V$ is quasi-compact,
it is clear that (c) implies (b).

\oldpage{91}
To prove that (b) implies (d1) and (d2), we first assume that $V=D(g)$ for a $g\in A$, and
that $\sh{F}$ is isomorphic to the sheaf $\widetilde{N}$ associated to an $A_g$-module $N$;
by replacing $X$ with $V$ and $A$ with $A_g$ \sref{cor}{1.3.6}, we can reduce to the case
where $g=1$. Then $\Gamma(D(f),\widetilde{N})$ and $N_f$ are canonically identified
(\sref{cor}{1.3.6} and \sref{thm}{1.3.7}), so a section $s\in\Gamma(D(f),\widetilde{N})$
identifies with an element of the form $z/f^n$, where $z\in N$; the section $f^n s$ identifies
with the element $z/1$ of $N_f$ and as a result the restriction to $D(f)$ of a section of
$\widetilde{N}$ over $X$ identifies with the element $z\in N$; hence (d1) in this case.
Similarly, $t\in\Gamma(X,\widetilde{N})$ is identified with an element $z'\in N$, the
restriction of $t$ to $D(f)$ is identified with the image $z'/1$ of $z'$ in $N_f$, and we say
that this image is zero means that there exists an $n\geqslant 0$ such that $f^n z'=0$ in
$N$, or, equivalently, $f^n t=0$.

To finish the proof that (b) implies (d1) and (d2), it suffices to establish the following
lemma:
\begin{envs}[Lemma]{1.4.1.1}
\label{lem-1.1.4.1.1}
Suppose that $V$ is the finite union of sets of the form $D(g_i)$, and that each of the
sheaves $\sh{F}|D(g_i)$, $\sh{F}|(D(g_i)\cap D(g_j))=\sh{F}|D(g_i g_j)$ satify \emph{(d1)}
and \emph{(d2)}; then $\sh{F}$ has the following two properties:
\begin{enumerate}[label=\rm{(d$'$\arabic*)}]
  \item For each $f\in A$ and for each section $s\in\Gamma(D(f)\cap V,\sh{F})$, there exists
        an integer $n\geqslant 0$ such that $f^n s$ extends to a section of $\sh{F}$ over
        $V$.
  \item For each $f\in A$ and for each section $t\in\Gamma(V,\sh{F})$ such that the
        restriction of $t$ to $D(f)\cap V$ is $0$, there exists an integer $n\geqslant 0$
        such that $f^n t=0$.
\end{enumerate}
\end{envs}

We first prove (d$'$2): as $D(f)\cap D(g_i)=D(fg_i)$, there exists for each $i$ an integer
$n_i$ such that the restriction of $(fg_i)^{n_i}t$ to $D(g_i)$ is zero: as the image of $g_i$
in $A_{g_i}$ is invertible, the restriction of $f^{n_i}t$ to $D(g_i)$ is also zero; taking
for $n$ the largest of the $n_i$, we have proved (d$'$2).

To show (d$'$1), we apply (d1) to the sheaf $\sh{F}|D(g_i)$: there exists an integer
$n_i\geqslant 0$ and a section $s_i'$ of $\sh{F}$ over $D(g_i)$ extending the restriction of
$(fg_i)^{n_i}s$ to $D(fg_i)$; as the image of $g_i$ in $A_{g_i}$ is invertible, there is a
section $s_i$ of $\sh{F}$ over $D(g_i)$ such that $s_i'=g_i^{n_i}s_i$, and $s_i$ extends the
restriction of $f^{n_i}s$ to $D(fg_i)$; in addition we can suppose that all the $n_i$ are
equal to the same integer $n$. By construction, the restriction of $s_i-s_j$ to
$D(f)\cap D(g_i)\cap D(g_j)=D(fg_i g_j)$ is zero; according to (d2) applied to the sheaf
$\sh{F}|D(g_i g_j)$, there exists an integer $m_{ij}\geqslant 0$ such that the restriction to
$D(g_i g_j)$ of $(fg_i g_j)^{m_{ij}}(s_i-s_j)$ is zero; as the image of $g_i g_j$ in
$A_{g_i g_j}$ is invertible, the restriction of $f^{m_{ij}}(s_i-s_j)$ to $D(g_i g_j)$ is
zero. We can then assume that all the $m_{ij}$ are equal to the same integer $m$, and so
there exists a section $s'\in\Gamma(V,\sh{F})$ extending the $f^m s_i$; as a result, this
section extends $f^{n+m}s$, hence we have proved (d$'$1).

It remains to prove that (d1) and (d2) imply (a). We show first that (d1) and (d2) imply that
these conditions are satisfied for each sheaf $\sh{F}|D(g)$, where $g\in A$ is such that
$D(g)\subset V$. It is evident for (d1); on the other hand, if $t\in\Gamma(D(g),\sh{F})$ is
such that its restriction to $D(f)\subset D(g)$ is zero, there exists by (d1) an integer
$m\geqslant 0$ such that $g^m t$
\oldpage{92}
extends to a section $s$ of $\sh{F}$ over $V$; applying (d2), we see that there exists an
integer $n\geqslant 0$ such that $f^n g^m t=0$, and as the image of $g$ in $A_g$ is
invertible, $f^n t=0$.

That being so, as $V$ is quasi-compact, Lemma \sref{lem}{1.4.1.1} proves that the
conditions (d$'$1) and (d$'$2) are satisfied. Consider then the $A$-module
$M=\Gamma(V,\sh{F})$, and define a homomorphism of $\widetilde{A}$-modules
$u:\widetilde{M}\to j_*(\sh{F})$, where $j$ is the canonical injection $V\to X$. As the
$D(f)$ form a basis for the topology of $X$, it suffices, for each $f\in A$, to define a
homomorphism $u_f:M_f\to\Gamma(D(f),j_*(\sh{F}))=\Gamma(D(f)\cap V,\sh{F})$, with the usual
compatibility conditions \pref{env}{3.2.5}. As the canonical image of $f$ in $A_f$ is
invertible, the restriction homomorphism $M=\Gamma(V,\sh{F})\to\Gamma(D(f)\cap V,\sh{F})$
factorizes as $M\to M_f\xrightarrow{u_f}\Gamma(D(f)\cap V,\sh{F})$ \pref{env}{1.2.4}, and the
verfication of these compatibility conditions for $D(g)\subset D(f)$ is immediate. This being
so, we show that the condition (d$'$1) (resp. (d$'$2)) implies that each of the $u_f$ are
surjective (resp. injective), which proves that $u$ is \emph{bijective}, and as a result that
$\sh{F}$ is the restriction to $V$ of an $\widetilde{A}$-module isomorphic to
$\widetilde{M}$. If $s\in\Gamma(D(f)\cap V,\sh{F})$, there exists according to (d$'$1) an
integer $n\geqslant 0$ such that $f^n s$ extends to a section $z\in M$; we then have
$u_f(z/f^n)=s$, so $u_f$ is surjective. Similarly, if $z\in M$ is such that $u_f(z/1)=0$,
this means that the restriction to $D(f)\cap V$ of the section $z$ is zero; according to
(d$'$2), there exists an integer $n\geqslant 0$ such that $f^n z=0$, hence $z/1=0$ in $M_f$,
and therefore $u_f$ is injective.
\begin{flushright}
Q.E.D.
\end{flushright}

\begin{envs}[Corollary]{1.4.2}
\label{cor-1.1.4.2}
Each quasi-coherent sheaf over a quasi-compact open subset of $X$ is induced by a
quasi-coherent sheaf on $X$.
\end{envs}

\begin{envs}[Corollary]{1.4.3}
\label{cor-1.1.4.3}
Each quasi-coherent $\OO_X$-algebra over $X=\Spec(A)$ is isomorphic to an $\OO_X$-algebra of
the form $\widetilde{B}$, where $B$ is an algebra over $A$; each quasi-coherent
$\widetilde{B}$-module is isomorphic to a $\widetilde{B}$-module of the form $\widetilde{N}$,
where $N$ is a $B$-module.
\end{envs}

Indeed, a quasi-coherent $\OO_X$-algebra is a quasi-coherent $\OO_X$-module, therefore of the
form $\widetilde{B}$, where $B$ is an $A$-module; the fact that $B$ is an $A$-algebra
follows from the characterization of the structure of an $\OO_X$-algebra using the
homomorphism $\widetilde{B}\otimes_{\widetilde{A}}\widetilde{B}\to\widetilde{B}$ of
$\widetilde{A}$-modules, as well as \sref{cor}{1.3.12}. If $\sh{G}$ is a quasi-coherent
$\widetilde{B}$-module, it suffices to show, in a similar way,
that it is also a quasi-coherent $\widetilde{A}$-module to conclude the proof; as the
question is local, we can, by restricting to an open subset of $X$ of the form $D(f)$, assume
that $\sh{G}$ is the cokernel of a homomorphism $\widetilde{B}^{(I)}\to\widetilde{B}^{(J)}$
of $\widetilde{B}$-modules (and \emph{a fortiori} of $\widetilde{A}$-modules); the
proposition then follows from \sref{cor}{1.3.8} and \sref{cor}{1.3.9}.

\subsection{Coherent sheaves over a prime spectrum}
\label{1-schemes-1.5}

\begin{envs}[Theorem]{1.5.1}
\label{thm-1.1.5.1}
Let $A$ be a \emph{Noetherian} ring, $X=\Spec(A)$ its prime spectrum, $V$ an open subset of
$X$, and $\sh{F}$ an $(\OO_X|V)$-module. The following conditions are equivalent:
\begin{enumerate}[label=\rm{(\alph*)}]
  \item $\sh{F}$ is coherent.
  \item $\sh{F}$ is of finite type and quasi-coherent.
  \item There exists an $A$-module $M$ of finite type such that $\sh{F}$ is isomorphic to
        the sheaf $\widetilde{M}|V$.
\end{enumerate}
\end{envs}

\oldpage{93}
(a) trivially implies (b). To see the (b) implies (c), we have previously seen, since $V$
is quasi-compact \pref{env}{2.2.3}, that $\sh{F}$ is isomorphic to a sheaf $\widetilde{N}|V$,
where $N$ is an $A$-module \sref{thm}{1.4.1}. We have $N=\varinjlim M_\lambda$, where
$M_\lambda$ vary over the set of $A$-submodules of $N$ of finite type, hence
\sref{cor}{1.3.9} $\sh{F}=\widetilde{N}|V=\varinjlim\widetilde{M_\lambda}|V$; but as $\sh{F}$
is of finite type, and $V$ is quasi-compact, there exists an index $\lambda$ such that
$\sh{F}=\widetilde{M_\lambda}|V$ \pref{env}{5.2.3}.

Finally, we show that (c) implies (a). It is clear that $\sh{F}$ is then of finite type
(\sref{prop}{1.3.6} and \sref{cor}{1.3.9}); in addition, the question being local, we can
reduce to the case where $V=D(f)$, $f\in A$. As $A_f$ is Noetherian, we see finally that it
reduces to proving that the kernel of a homomorphism $\widetilde{A^n}\to\widetilde{M}$, where
$M$ is an $A$-module, is of finite type. Such a homomorphism is of the form $\widetilde{u}$,
where $u$ is a homomorphism $A^n\to M$ \sref{cor}{1.3.8}, and if $P=\Ker u$, we have
$\widetilde{P}=\Ker\widetilde{u}$ \sref{cor}{1.3.9}. As $A$ is Noetherian, $P$ is of finite
type, which finishes the proof.

\begin{envs}[Corollary]{1.5.2}
\label{cor-1.1.5.2}
Under the hypotheses of \sref{thm}{1.5.1}, the sheaf $\OO_X$ is a quasi-coherent sheaf of
rings.
\end{envs}

\begin{envs}[Corollary]{1.5.3}
\label{cor-1.1.5.3}
Under the hypotheses of \sref{thm}{1.5.1}, each coherent sheaf over an open subset of $X$ is
induced by a coherent sheaf on $X$.
\end{envs}

\begin{envs}[Corollary]{1.5.4}
\label{cor-1.1.5.4}
Under the hypotheses of \sref{thm}{1.5.1}, each quasi-coherent $\OO_X$-module $\sh{F}$ is
the inductive limit of the coherent $\OO_X$-submodules of $\sh{F}$.
\end{envs}

Indeed, $\sh{F}=\widetilde{M}$ where $M$ is an $A$-module, and $M$ is the inductive limit of
its submodules of finite type; we conclude the proof by \sref{cor}{1.3.9} and
\sref{thm}{1.5.1}.

\subsection{Functorial properties of quasi-coherent sheaves over a prime spectrum}
\label{1-schemes-1.6}

\begin{env}{1.6.1}
\label{env-1.1.6.1}
Let $A$, $A'$ be two rings,
\[
  \vphi:A'\to A
\]
a homomorphism,
\[
  {}^a\vphi:X=\Spec(A)\longrightarrow X'=\Spec(A')
\]
the continuous map associated to $\vphi$ \sref{env}{1.2.1}. We will define a \emph{canonical
homomorphism}
\[
  \widetilde{\vphi}:\OO_{X'}\longrightarrow{}^a\vphi_*(\OO_X)
\]
of sheaves of rings. For each $f'\in A'$, we put $f=\vphi(f')$; we have
${}^a\vphi^{-1}(D(f'))=D(f)$ (1.2.2.2). The rings $\Gamma(D(f'),\widetilde{A'})$ and
$\Gamma(D(f),\widetilde{A})$ identify respectively with $A_{f'}'$ and $A_f$
(\sref{prop}{1.3.6} and \sref{thm}{1.3.7}). The homomorphism $\vphi$ canonically defines a
homomorphism $\vphi_{f'}:A_{f'}'\to A_f$ \pref{env}{1.5.1}, in other words, we have a
homomorphism of rings
\[
  \Gamma(D(f),\widetilde{A'})\longrightarrow\Gamma({}^a\vphi^{-1}(D(f')),\widetilde{A})
  =\Gamma(D(f'),{}^a\vphi_*(\widetilde{A}))
\]
\oldpage{94}
In addition, these homomorphism satisfy the usual compatibility conditions: for
$D(f')\supset D(g')$, the diagram
\[
  \xymatrix{
    \Gamma(D(f'),\widetilde{A'})\ar[r]\ar[d] &
    \Gamma(D(f'),{}^a\vphi_*(\widetilde{A}))\ar[d]\\
    \Gamma(D(g'),\widetilde{A'})\ar[r] &
    \Gamma(D(g'),{}^a\vphi_*(\widetilde{A})
  }
\]
is commutative \pref{env}{1.5.1}; we have thus defined a homomorphism of $\OO_{X'}$-algebras,
as the $D(f')$ form a basis for the topology og $X'$ \pref{env}{3.2.3}. The pair
$\Phi=({}^a\vphi,\widetilde{\vphi})$ is thus a \emph{morphism} of ringed spaces
\[
  \Phi:(X,\OO_X)\longrightarrow(X',\OO_{X'}),
\]
\pref{env}{4.1.1}.

We note further that, if we put $x'={}^a\vphi(x)$, then the homomorphism
$\widetilde{\vphi}_x^\sharp$ \pref{env}{3.7.1} is none other than the homomorphism
\[
  \vphi_x:A_{x'}'\longrightarrow A_x
\]
canonically induced by $\vphi:A'\to A$ \pref{env}{1.5.1}. Indeed, each $z'\in A_{x'}'$
can be written as $g'/f'$, where $f'$, $g'$ are in $A'$ and $f'\not\in\mathfrak{j}_{x'}$;
$D(f')$ is then a neighborhood of $x'$ in $X'$, and the homomorphism
$\Gamma(D(f'),\widetilde{A'})\to\Gamma({}^a\vphi^{-1}(D(f')),\widetilde{A})$ induced by
$\widetilde{\vphi}$ is none other than $\vphi_{f'}$; by considering the section
$s'\in\Gamma(D(f'),\widetilde{A'})$ corresponding to $g'/f'\in A_{f'}'$, we obtain
$\widetilde{\vphi}_x^\sharp(z')=\vphi(g')/\vphi(f')$ in $A_x$.
\end{env}

\begin{env}[Example]{1.6.2}
\label{exm-1.1.6.2}
Let $S$ be a multiplicative subset of $A$, $\vphi$ the canonical homomorphism $A\to S^{-1}A$;
we have already seen \sref{cor}{1.2.6} that ${}^a\vphi$ is a \emph{homeomorphism} from
$Y=\Spec(S^{-1}A)$ to the subspace of $X=\Spec(A)$ consisting of the $x$ such that
$\mathfrak{j}_x\cap S=\emp$. In addition, for each $x$ in this subspace, thus of the form
${}^a\vphi(y)$ with $y\in Y$, the homomorphism $\widetilde{\vphi}_y^\sharp:\OO_x\to\OO_y$ is
\emph{bijective} \pref{env}{1.2.6}; in other words, $\OO_Y$ identifies with the sheaf on $Y$
induced by $\OO_X$.
\end{env}

\begin{envs}[Proposition]{1.6.3}
\label{prop-1.1.6.3}
For each $A$-module $M$, there exists a canonical functorial isomorphism from the
$\OO_{X'}$-module $(M_{[\vphi]})^\sim$ to the direct image $\Phi_*(\widetilde{M})$.
\end{envs}

For purposes of abbreviation, we put $M'=M_{[\vphi]}$, and for each $f'\in A'$, we put
$f=\vphi(f')$. The modules of sections $\Gamma(D(f'),\widetilde{M'})$ and
$\Gamma(D(f),\widetilde{M})$ identify respectively with the modules $M_{f'}'$ and $M_f$
(over $A_{f'}'$ and $A_f$, respectively); in addition, the $A_{f'}'$-module
$(M_f)_{[\vphi_{f'}]}$ is canonically isomorphic to $M_{f'}'$ \pref{env}{1.5.2}. We thus have
a functorial isomorphism of $\Gamma(D(f'),\widetilde{A'})$-modules:
$\Gamma(D(f'),\widetilde{M'})
\isoto\Gamma({}^a\vphi^{-1}(D(f')),\widetilde{M})_{[\vphi_{f'}]}$
and these isomorphisms satisfy the usual compatibility conditions with the restrictions
\pref{env}{1.5.6}, thus defining the desired functorial isomorphism. We note that, in a
precise way, if $u:M_1\to M_2$ is a homomorphism of $A$-modules, it can be considered as a
homomorphism $(M_1)_{[\vphi]}\to(M_2)_{[\vphi]}$ of $A'$-modules; if we denote by
$u_{[\vphi]}$ this homomorphism, $\Phi_*(\widetilde{u})$ identifies with
$(u_{[\vphi]})^\sim$.

This proof also proves that for each \emph{$A$-algebra $B$}, the canonical functorial
isomorphism
\oldpage{95}
$(B_{[\vphi]})^\sim\isoto\Phi_*(\widetilde{B})$ is an isomorphism of
\emph{$\OO_{X'}$-algebras}; if $M$ is a $B$-module, the canonical functorial isomorphism
$(M_{[\vphi]})^\sim\isoto\Phi_*(\widetilde{M})$ is an isomorphism of
$\Phi_*(\widetilde{B})$-modules.

\begin{envs}[Corollary]{1.6.4}
\label{cor-1.1.6.4}
The direct image functor $\Phi_*$ is exact on the category of quasi-coherent $\OO_X$-modules.
\end{envs}

Indeed, it is clear that $M_{[\vphi]}$ is an exact functor in $M$ and $\widetilde{M'}$ is an
exact functor in $M'$ \sref{prop}{1.3.5}.

\begin{envs}[Proposition]{1.6.5}
\label{prop-1.1.6.5}
Let $N'$ be an $A'$-module, $N$ the $A$-module $N'\otimes_{A'}A_{[\vphi]}$; there exists a
canonical functorial isomorphism from the $\OO_X$-module $\Phi^*(\widetilde{N'})$ to
$\widetilde{N}$.
\end{envs}

We first remark that $j:z'\mapsto z'\otimes 1$ is an $A'$-homomorphism from $N'$ to
$N_{[\vphi]}$: indeed, by definition, for $f'\in A'$, we have
$(f'z')\otimes 1=z'\otimes\vphi(f')=\vphi(f')(z'\otimes 1)$. We have \sref{cor}{1.3.8} a
homomorphism $\widetilde{j}:\widetilde{N'}\to(N_{[\vphi]})^\sim$ of $\OO_{X'}$-modules, and
according to \sref{prop}{1.6.3}, we can consider that $\widetilde{j}$ maps $\widetilde{N'}$
to $\Phi_*(\widetilde{N})$. There canonically corresponds to this homomorphism
$\widetilde{j}$ a homomorphism $h=\widetilde{j}^\sharp$ from $\Phi^*(\widetilde{N'})$ to
$\widetilde{N}$ \pref{env}{4.4.3}; we will see that for each stalk, $h_x$ is \emph{bijective}.
Put $x'={}^a\vphi(x)$ and let $f'\in A'$ be such that $x'\in D(f')$; let $f=\vphi(f')$. The
ring $\Gamma(D(f),\widetilde{A})$ identifies with $A_f$, the modules
$\Gamma(D(f),\widetilde{N})$ and $\Gamma(D(f'),\widetilde{N'})$ with $N_f$ and $N_{f'}'$
respectively; let $s\in\Gamma(D(f'),\widetilde{N'})$, identified with $n'/{f'}^p$
($n'\in N'$), $s$ its image under $\widetilde{j}$ in $\Gamma(D(f),\widetilde{N})$; $s$
identifies with $(n'\otimes 1)/f^p$. On the other hand, let $t\in\Gamma(D(f),\widetilde{A})$,
identified with $g/f^q$ ($g\in A$); then, by definition, we have
$h_x(s_x'\otimes t_x)=t_x\cdot s_x$ \pref{env}{4.4.3}. But we can canonically identify $N_f$
with $N_{f'}'\otimes_{A_{f'}'}(A_f)_{[\vphi_{f'}]}$ \pref{env}{1.5.4}; $s$ then corresponds
to the element $(n'/{f'}^p)\otimes 1$, and the section $y\mapsto t_y\cdot s_y$ with
$(n'/{f'}^p)\otimes(g/f^q)$. The compatibility diagram of \pref{env}{1.5.6} show that $h_x$
is none other than the canonical isomorphism
\[
  N_{x'}'\otimes_{A_{x'}'}(A_x)_{[\vphi_{x'}]}\isoto N_x=(N'\otimes_{A'}A_{[\vphi]})_x.
  \tag{1.6.5.1}
\]

In addition, let $v:N_1'\to N_2'$ be a homomorphism of $A'$-modules; as
$\widetilde{v}_{x'}=v_{x'}$ for each $x'\in X'$, it follows immediately from the above that
$\Phi^*(\widetilde{v})$ canonically identifies with $(v\otimes 1)^\sim$, which finishes the
proof of \sref{prop}{1.6.5}.

If $B'$ is an $A'$-algebra, the canonical isomorphism from $\Phi^*(\widetilde{B'})$ to
$(B'\otimes_{A'}A_{[\vphi]})^\sim$ is an isomorphism of $\OO_X$-algebras; if in addition $N'$
is a $B'$-module, the canonical isomorphism from $\Phi^*(\widetilde{N'})$ to
$(N'\otimes_{A'}A_{[\vphi]})^\sim$ is an isomorphism of $\Phi^*(\widetilde{B'})$-modules.

\begin{envs}[Corollary]{1.6.6}
\label{cor-1.1.6.6}
The sections of $\Phi^*(\widetilde{N'})$, the canonical images of the sections $s'$, where
$s'$ varies over the $A'$-module $\Gamma(\widetilde{N'})$, generate the $A$-module
$\Gamma(\Phi^*(N'))$.
\end{envs}

Indeed. these images identify with the elements $z'\otimes 1$ of $N$, when we identify $N'$
and $N$ with $\Gamma(\widetilde{N'})$ and $\Gamma(\widetilde{N})$ respectively
\sref{thm}{1.3.7} and $z'$ varies over $N'$.

\begin{env}{1.6.7}
\label{env-1.1.6.7}
In the proof of \sref{prop}{1.6.5}, we had proved in passing that the canonical map
(\textbf{0},~4.4.3.2) $\rho:\widetilde{N'}\to\Phi_*(\Phi^*(\widetilde{N'}))$ is none other
than the homomorphism $\widetilde{j}$,
\oldpage{96}
where $j:N'\to N'\otimes_{A'}A_{[\vphi]}$ is the homomorphism $z'\mapsto z'\otimes 1$.
Similarly, the canonical map (\textbf{0},~4.4.3.3)
$\sigma:\Phi^*(\Phi_*(\widetilde{M}))\to\widetilde{M}$ is none other than $\widetilde{p}$,
where $p:M_{[\vphi]}\otimes_{A'}A_{[\vphi]}\to M$ is the canonical homomorphism which, sends
each tensor product $z\otimes a$ ($z\in M$, $a\in A$) to $a\cdot z$; this follows immediately
from the definitions (\pref{env}{3.7.1}, \pref{env}{4.4.3}, and \sref{thm}{1.3.7}).

We conclude (\pref{env}{4.4.3} and (\textbf{0},~3.5.4.4)) that if $v:N'\to M_{[\vphi]}$ is an
$A'$-homomorphism, we have $\widetilde{v}^\sharp=(v\otimes 1)^\sim$.
\end{env}

\begin{env}{1.6.8}
\label{env-1.1.6.8}
Let $N_1'$, $N_2'$ be two $A'$-modules, and assume $N_1'$ admits a \emph{finite
presentation}; it then follows from \sref{env}{1.6.7} and (\sref{cor}{1.3.12}, (ii)) that the
canonical homomorphism \pref{env}{4.4.6}
\[
  \Phi^*(\shHom_{\widetilde{A'}}(\widetilde{N_1'},\widetilde{N_2'}))
  \longrightarrow\shHom_{\widetilde{A}}(\Phi^*(\widetilde{N_1'}),\Phi^*(\widetilde{N_2'}))
\]
is none other than $\widetilde{\gamma}$, where $\gamma$ denotes the canonical homomorphism
of $A$-modules
$\Hom_{A'}(N_1',N_2')\otimes_{A'}A\to\Hom_A(N_1'\otimes_{A'}A,N_2'\otimes_{A'}A)$.
\end{env}

\begin{env}{1.6.9}
\label{env-1.1.6.9}
Let $\mathfrak{J}'$ be an ideal of $A'$, $M$ an $A$-module; as by definition
$\widetilde{\mathfrak{J}'}\widetilde{M}$ is the image of the canonical homomorphism
$\Phi^*(\widetilde{\mathfrak{J}'})\otimes_{\widetilde{A}}\widetilde{M}\to\widetilde{M}$, it
follows from \sref{prop}{1.6.5} and (\sref{cor}{1.3.12}, (i)) that
$\widetilde{\mathfrak{J}'}\widetilde{M}$ canonically identifies with $(\mathfrak{J}'M)^\sim$;
in particular, $\Phi^*(\widetilde{\mathfrak{J}'})\widetilde{A}$ identifies with
$(\mathfrak{J}'A)^\sim$, and taking into account the right exactness of the functor $\Phi^*$,
the $\widetilde{A}$-algebra $\Phi^*((A'/\mathfrak{J}')^\sim)$ identifies with
$(A/\mathfrak{J}'A)^\sim$.
\end{env}

\begin{env}{1.6.10}
\label{env-1.1.6.10}
Let $A''$ be a third ring, $\vphi'$ a homomorphism $A''\to A'$, and put
$\vphi''=\vphi\circ\vphi'$. It follows immediately from the definitions that
${}^a\vphi''=({}^a\vphi')\circ({}^a\vphi)$, and
$\widetilde{\vphi''}=\widetilde{\vphi}\circ\widetilde{\vphi'}$ \pref{env}{1.5.7}. We conclude
that we have $\Phi''=\Phi'\circ\Phi$; in other words, $(\Spec(A),\widetilde{A})$ is a
\emph{functor} from the category of rings to that of ringed spaces.
\end{env}

\subsection{Characterisation of morphisms of affine schemes}
\label{1-schemes-1.7}

\begin{envs}[Definition]{1.7.1}
\label{defn-1.1.7.1}
We say that a ringed space $(X,\OO_X)$ is an affine scheme if it is isomorphic to a ringed
space of the form $(\Spec(A),\widetilde{A})$, where $A$ is a ring; we then say that
$\Gamma(X,\OO_X)$, which canonically identifies with the ring $A$ \sref{thm}{1.3.7} is the
ring of the affine scheme $(X,\OO_X)$, and we denote it by $A(X)$ when there is no chance of
confusion.
\end{envs}

By abuse of language, when we speak of an \emph{affine scheme $\Spec(A)$}, it will always be
the ringed space $(\Spec(A),\widetilde{A})$.

\begin{env}{1.7.2}
\label{env-1.1.7.2}
Let $A$, $B$ be two rings, $(X,\OO_X)$, $(Y,\OO_Y)$ the affine schemes corresponding to
the prime spectra $X=\Spec(A)$, $Y=\Spec(B)$. We have seen \sref{env}{1.6.1} that each ring
homomorphism $\vphi:B\to A$ corresponds to a morphism
$\Phi=({}^a\vphi,\widetilde{\vphi})=\Spec(\vphi):(X,\OO_X)\to(Y,\OO_Y)$. We note that $\vphi$
is entirely determined by $\Phi$, since we have by definition
$\vphi=\Gamma(\widetilde{\vphi}):\Gamma(\widetilde{B})
\to\Gamma({}^a\vphi_*(\widetilde{A})=\Gamma(\widetilde{A})$.
\end{env}

\begin{envs}[Theorem]{1.7.3}
\label{thm-1.1.7.3}
Let $(X,\OO_X)$, $(Y,\OO_Y)$ be two affine schemes. For a morphism of ringed spaces
$(\psi,\theta):(X,\OO_X)\to(Y,\OO_Y)$ to be of the form $({}^a\vphi,\widetilde{\vphi})$,
where $\vphi$ is a homomorphism of rings: $A(Y)\to A(X)$, it is necessary and sufficient
that, for each $x\in X$, $\theta_x^\sharp$ is a local homomorphism: $\OO_{\psi(x)}\to\OO_x$.
\end{envs}

\oldpage{97}
Put $A=A(X)$, $B=A(Y)$. The condition is necessary, since we saw \sref{env}{1.6.1} that
$\widetilde{\vphi}_x^\sharp$ is the homomorphism from $B_{{}^a\vphi(x)}$ to $A_x$ canonically
induced by $\vphi$, and by definition of ${}^a\vphi(x)=\vphi^{-1}(\mathfrak{j}_x)$, this
homomorphism is local.

We prove that the condition is sufficient. By definition, $\theta$ is a homomorphism
$\OO_Y\to\psi_*(\OO_X)$, and we canonically obtain a ring homomorphism
\[
  \vphi=\Gamma(\theta):B=\Gamma(Y,\OO_Y)
  \longrightarrow\Gamma(Y,\psi_*(\OO_X))=\Gamma(X,\OO_X)=A.
\]

The hypotheses on $\theta_x^\sharp$ allow us to deduce from this homomorphism, by passing to
quotients, a momomorphism $\theta^x$ from the residue field $\kres(\psi(x))$ to the residue
field $\kres(x)$, such that, for each section $f\in\Gamma(Y,\OO_Y)=B$, we have
$\theta^x(f(\psi(x)))=\vphi(f)(x)$. The relation $f(\psi(x))=0$ is therefore equivalent to
$\vphi(f)(x)=0$, which means that $\mathfrak{j}_{\psi(x)}=\mathfrak{j}_{{}^a\vphi(x)}$, and
we now write $\psi(x)={}^a\vphi(x)$ for each $x\in X$, or $\psi={}^a\vphi$. We also know that
the diagram
\[
  \xymatrix{
    B=\Gamma(Y,\OO_Y)\ar[r]^\vphi\ar[d] &
    \Gamma(X,\OO_X)=A\ar[d]\\
    B_{\psi(x)}\ar[r]^{\theta_x^\sharp} &
    A_x
  }
\]
is commutative \pref{env}{3.7.2}, which means that $\theta_x^\sharp$ is equal to the
homomorphism $\vphi_x:B_{\psi(x)}\to A_x$ canonically induced by $\vphi$ \pref{env}{1.5.1}.
As the data of the $\theta_x^\sharp$ completely characterize $\theta^\sharp$, and as a result
$\theta$ \pref{env}{3.7.1}, we conclude that we have $\theta=\widetilde{\vphi}$, by
definition of $\widetilde{\vphi}$ \sref{env}{1.6.1}.

We say that a mormphism $(\psi,\theta)$ of ringed spaces satisfying the condition of
\sref{thm}{1.7.3} is a \emph{morphism of affine schemes}.

\begin{envs}[Corollary]{1.7.4}
\label{cor-1.1.7.4}
If $(X,\OO_X)$, $(Y,\OO_Y)$ are affine schemes, there exists a canonical isomorphism from
the set of morphisms of affine schemes $\Hom((X,\OO_X),(Y,\OO_Y))$ to the set of
ring homomorphisms from $B$ to $A$, where $A=\Gamma(\OO_X)$ and $B=\Gamma(\OO_Y)$.
\end{envs}

Furthermore, we can say that the functors $(\Spec(A),\widetilde{A})$ in $A$ and
$\Gamma(X,\OO_X)$ in $(X,\OO_X)$ define an \emph{equivalence} between the category of
commutative rings and the opposite category of affine schemes (T, I, 1.2).

\begin{envs}[Corollary]{1.7.5}
\label{cor-1.1.7.5}
If $\vphi:B\to A$ is surjective, then the corresponding morphism
$({}^a\vphi,\widetilde{\vphi})$ is a momomorphism of ringed spaces
\emph{(cf. \sref{env}{4.1.7})}.
\end{envs}

Indeed, we know that ${}^a\vphi$ is injective \sref{env}{1.2.5}, and as $\vphi$ is
surjective, for each $x\in X$, $\vphi_x^\sharp:B_{{}^a\vphi(x)}\to A_x$, which is induced
by $\vphi$ by passing to rings of fractions, is also surjective \pref{env}{1.5.1}; hence
the conclusion \pref{env}{4.1.1}.

\subsection{Morphisms from locally ringed spaces to affine schemes}
\label{1-schemes-1.8}

\oldpage{217$'$}
Due to a remark by J.~Tate, the statements given in \sref{thm}{1.7.3} and \sref{prop}{2.2.4}
can be generalized as follows:\footnote{[Trans] The following section (I.1.8) was added in
the errata of EGA~II, hence the temporary change in page numbers, which refer to EGA~II.}

\begin{envs}[Proposition]{1.8.1}
\label{prop-1.1.8.1}
Let $(S,\OO_S)$ be an affine scheme, $(X,\OO_X)$ a locally ringed space. Then there is a
canonical bijection from the set of ring homomorphisms
\oldpage{218$'$}
$\Gamma(S,\OO_S)\to\Gamma(X,\OO_X)$ to the set of morphisms of ringed spaces
$(\psi,\theta):(X,\OO_X)\to(S,\OO_S)$ such that, for each $x\in X$, $\theta_x^\sharp$ is a
local homomorphism: $\OO_{\psi(x)}\to\OO_x$.
\end{envs}

We note first that if $(X,\OO_X)$, $(S,\OO_S)$ are any two ringed spaces, a morphism
$(\psi,\theta)$ from $(X,\OO_X)$ to $(S,\OO_S)$ canonically defines a ring homomorphism
$\Gamma(\theta):\Gamma(S,\OO_S)\to\Gamma(X,\OO_X)$, hence a first map
\[
  \rho:\Hom((X,\OO_X),(S,\OO_S))\longrightarrow\Hom(\Gamma(S,\OO_S),\Gamma(X,\OO_X)).
  \tag{1.8.1.1}
\]
Conversely, under the stated hypotheses, we set $A=\Gamma(S,\OO_S)$, and consider a ring
homomorphism $\vphi:A\to\Gamma(X,\OO_X)$. For each $x\in X$, it is clear that the set of the
$f\in A$ such that $\vphi(f)(x)=0$ is a \emph{prime ideal} of $A$, since
$\OO_x/\mathfrak{m}_x=\kres(x)$ is a field; it is therefore an element of $S=\Spec(A)$, which
we denote ${}^a\vphi(x)$. In addition, for each $f\in A$, we have by definition
\pref{env}{5.5.2} that ${}^a\vphi(D(f))=X_f$, which proves that ${}^a\vphi$ is a
\emph{continuous map} $X\to S$. We define then a homomorphism
\[
  \widetilde{\vphi}:\OO_S\longrightarrow{}^a\vphi_*(\OO_X)
\]
of $\OO_S$-modules; for each $f\in A$, we have $\Gamma(D(f),\OO_S)=A_f$ \sref{prop}{1.3.6};
for each $s\in A$, we correspond to $s/f\in A_f$ the element
$(\vphi(s)|X_f)(\vphi(f)|X_f)^{-1}$ of $\Gamma(X_f,\OO_X)=\Gamma(D(f),{}^a\vphi(\OO_X))$, and
we check immediately (by passing from $D(f)$ to $D(fg)$) that this is a well-defined
homomorphism of $\OO_S$-modules, hence a morphism $({}^a\vphi,\widetilde{\vphi})$ of ringed
spaces. In addition, with the same notations, and setting $y={}^a\vphi(x)$ for purposes of
simplification, we see immediately \pref{env}{3.7.1} that we have
$\widetilde{\vphi}_x^\sharp(s_y/f_y)=(\vphi(s)_x)(\vphi(f)_x)^{-1}$; as the relation
$s_y\in\mathfrak{m}_y$ is by definition equivalent to $\vphi(s)_x\in\mathfrak{m}_x$, we see
that $\widetilde{\vphi}_x^\sharp$ is a \emph{local} homomorphism $\OO_y\to\OO_x$, and we have
so defined a second map $\sigma:\Hom(\Gamma(S,\OO_S),\Gamma(X,\OO_X))\to\mathfrak{L}$, where
$\mathfrak{L}$ is the set of the morphisms $(\psi,\theta):(X,\OO_X)\to(S,\OO_S)$ such that
$\theta_x^\sharp$ is local for each $x\in X$. It remains to prove that $\sigma$ and $\rho$
(restricted to $\mathfrak{L}$) are inverses of each other; the definition of
$\widetilde{\vphi}$ immediately shows that $\Gamma(\widetilde{\vphi})=\vphi$, and as a result
$\rho\circ\sigma$ is the identity. To see that $\sigma\circ\rho$ is the identity, start with a
morphism $(\psi,\theta)\in\mathfrak{L}$ and let $\vphi=\Gamma(\theta)$; the hypotheses on
$\theta_x^\sharp$ allows us to induce from this morphism, by passing to quotients, a
monomorphism $\theta^x:\kres(\psi(x))\to\kres(x)$ such that for each section
$f\in A=\Gamma(S,\OO_S)$, we have $\theta^x(f(\psi(x)))=\vphi(f)(x)$; the relation
$f(\vphi(x))=0$ is therefore equivalent to $\vphi(f)(x)=0$, which proves that
${}^a\vphi=\psi$. On the other hand, the definitions imply that the diagram
\[
  \xymatrix{
    A\ar[r]^\vphi\ar[d] &
    \Gamma(X,\OO_X)\ar[d]\\
    A_{\psi(x)}\ar[r]^{\theta_x^\sharp} &
    \OO_x
  }
\]
is commutative, and it is the same for the analogous diagram where $\theta_x^\sharp$ is
replaced by $\widetilde{\vphi}_x^\sharp$, hence $\widetilde{\vphi}_x^\sharp=\theta_x^\sharp$
\pref{env}{1.2.4}, and as a result $\widetilde{\vphi}=\theta$.

\begin{env}{1.8.2}
\label{env-1.1.8.2}
When $(X,\OO_X)$ and $(Y,\OO_Y)$ are \emph{locally} ringed spaces, we will consider the
morphisms $(\psi,\theta):(X,\OO_X)\to(Y,\OO_Y)$ such that, for each $x\in X$,
$\theta_x^\sharp$ is a \emph{local} homomorphism: $\OO_{\psi(x)}\to\OO_x$. Henceforth when
we speak
\oldpage{219$'$}
of a \emph{morphism of locally ringed spaces}, it will always be a morphism such as the
above; with this definition of morphisms, it is clear that the locally ringed spaces form a
\emph{category}; for two objects $X$, $Y$ of this category, $\Hom(X,Y)$ thus denotes the set
of morphisms of locally ringed spaces from $X$ to $Y$ (the set denoted $\mathfrak{L}$ in
\sref{prop}{1.8.1}); when we consider the set of \emph{morphisms of ringed spaces} from $X$
to $Y$, we will denote it by $\Hom_\text{rs}(X,Y)$ to avoid any confusion. The map (1.8.1.1)
is then written as
\[
  \rho:\Hom_\text{rs}(X,Y)\longrightarrow\Hom(\Gamma(Y,\OO_Y),\Gamma(X,\OO_X))
  \tag{1.8.2.1}
\]
and its restriction 
\[
  \rho':\Hom(X,Y)\longrightarrow\Hom(\Gamma(Y,\OO_Y),\Gamma(X,\OO_X))
  \tag{1.8.2.2}
\]
is a \emph{functorial} map in $X$ and $Y$ on the category of locally ringed spaces.
\end{env}

\begin{envs}[Corollary]{1.8.3}
\label{cor-1.1.8.3}
Let $(Y,\OO_Y)$ be a locally ringed space. For $Y$ to be an affine scheme, it is necessary
and sufficient that for each locally ringed space $(X,\OO_X)$, the map \emph{(1.8.2.2)} is
bijective.
\end{envs}

Proposition \sref{prop}{1.8.1} shows that the condition is necessary. Conversely, if we
suppose that the condition is satisfied and if we put $A=\Gamma(Y,\OO_Y)$, it follows from
the hypotheses and from \sref{prop}{1.8.1} that the functors $X\mapsto\Hom(X,Y)$ and
$X\mapsto\Hom(X,\Spec(A))$, from the category of locally ringed spaces to that of sets, are
\emph{isomorphic}. We know that this implies the existence of a canonical isomorphism
$X\to\Spec(A)$ (cf. \textbf{0},~8).

\begin{env}{1.8.4}
\label{env-1.1.8.4}
Let $S=\Spec(A)$ be an affine scheme; denote by $(S',A')$ the ringed space whose underlying
space is \emph{reduced to a point} and the structure sheaf $A'$ is the (necessarily simple)
sheaf on $S'$ defined by the ring $A$. Let $\pi:S\to S'$ be the unique map from $S$ to $S'$;
on the other hand, we note that for each open subset $U$ of $S$, we have a canonical map
$\Gamma(S',A')=\Gamma(S,\OO_S)\to\Gamma(U,\OO_S)$ which thus defines a \emph{$\pi$-morphism}
$\iota:A'\to\OO_S$ of sheaves of rings. We have thus canonically defined a \emph{morphism of
ringed spaces $i=(\pi,\iota):(S,\OO_S)\to(S',A')$}. For each $A$-module $M$, we denote by
$M'$ the simple sheaf on $S'$ defined by $M$, which is evidently an $A'$-module. It is clear
that we have $i_*(\widetilde{M})=M'$ \sref{thm}{1.3.7}.
\end{env}

\begin{envs}[Lemma]{1.8.5}
\label{lem-1.1.8.5}
With the notations of \sref{env}{1.8.4}, for each $A$-module $M$, the canonical functorial
$\OO_S$-homomorphism \emph{(\textbf{0}, 4.4.3.3)}
\[
  i^*(i_*(\widetilde{M}))\longrightarrow\widetilde{M}
  \tag{1.8.5.1}
\]
is an isomorphism.
\end{envs}

Indeed, the two parts of (1.8.5.1) are right exact (the functor
$M\mapsto i_*(\widetilde{M})$ being evidently exact) and commute with direct sums; by
considering $M$ as the cokernel of a homomorphism $A^{(I)}\to A^{(J)}$, we reduce to proving
the lemma for the case where $M=A$, and it is evident in this case.

\begin{envs}[Corollary]{1.8.6}
\label{cor-1.1.8.6}
Let $(X,\OO_X)$ be a ringed space, $u:X\to S$ a morphism of ringed spaces.
\oldpage{220$'$}
For each $A$-module $M$, we have (with the notations of \sref{env}{1.8.4}) a canonical
functorial isomorphism of $\OO_X$-modules
\[
  u^*(\widetilde{M})\isoto u^*(i^*(M')).
  \tag{1.8.6.1}
\]
\end{envs}

\begin{envs}[Corollary]{1.8.7}
\label{cor-1.1.8.7}
Under the hypotheses of \sref{cor}{1.8.6}, we have, for each $A$-module $M$ and each
$\OO_X$-module $\sh{F}$, a canonical functorial isomorphism in $M$ and $\sh{F}$
\[
  \Hom_{\OO_S}(\widetilde{M},u_*(\sh{F}))\isoto\Hom_A(M,\Gamma(X,\sh{F})).
  \tag{1.8.7.1}
\]
\end{envs}
We have in fact, according to \pref{env}{4.4.3} and Lemma \sref{lem}{1.8.5}, a canonical
isomorphism of bifunctors
\[
  \Hom_{\OO_S}(\widetilde{M},u_*(\sh{F}))\isoto\Hom_{A'}(M',i_*(u_*(\sh{F})))
\]
and it is clear that the right hand side is none other than $\Hom_A(M,\Gamma(X,\sh{F}))$. We
note that the canonical homomorphism (1.8.7.1) sends each $\OO_S$-homomorphism
$h:\widetilde{M}\to u_*(\sh{F})$ (in other words, each $u$-morphism $\widetilde{M}\to\sh{F}$)
to the $A$-homomorphism $\Gamma(h):M\to\Gamma(S,u_*(\sh{F}))=\Gamma(X,\sh{F})$.

\begin{env}{1.8.8}
\label{env-1.1.8.8}
With the notations of \sref{env}{1.8.4}, it is clear \pref{env}{4.1.1} that each morphism
of ringed spaces $(\psi,\theta):X\to S'$ is equivalent to the data of a ring homomorphism
$A\to\Gamma(X,\OO_X)$. We can thus interpret \sref{env}{1.8.1} as defining a canonical
bijection $\Hom(X,S)\isoto\Hom(X,S')$ (where we understand that the right hand side are
morphisms of ringed spaces, since in general $A$ is not a local ring). More generally, if
$X$, $Y$ are two locally ringed spaces and if $(Y',A')$ is the ringed space whose underlying
space is reduced to a point and whose sheaf of rings $A'$ is the simple sheaf defined by the
ring $\Gamma(Y,\OO_Y)$, we can interpret (1.8.2.1) as a map
\[
  \rho:\Hom_\text{rs}(X,Y)\longrightarrow\Hom(X,Y').
  \tag{1.8.8.1}
\]
The result of \sref{cor}{1.8.3} is interepreted by saying that affine schemes are
characterized among locally ringed spaces as those for which the restriction o $\rho$ to
$\Hom(X,Y)$:
\[
  \rho':\Hom(X,Y)\longrightarrow\Hom(X,Y')
  \tag{1.8.8.2}
\]
is \emph{bijective} for \emph{each} locally ringed space $X$. In the following chapter, we
generalize this definition, which allows us to associate to \emph{any} ringed space $Z$ (and
not only to a ringed space whose underlying space is reduced to a point) a locally ringed
space which we will call $\Spec(Z)$; this will be the starting point for a ``relative''
theory of preschemes over any ringed space, extending the results of Chapter~I.
\end{env}

\begin{env}{1.8.9}
\label{env-1.1.8.9}
We can consider the pairs $(X,\sh{F})$ consisting of a locally ringed space $X$ and an
$\OO_X$-module $\sh{F}$ as forming a category, a \emph{morphism} of this category being a
pair $(u,h)$ consisting of a morphism of locally ringed spaces
\oldpage{221$'$}
$u:X\to Y$ and a $u$-morphism $h:\sh{G}\to\sh{F}$ of modules; these morphisms (for
$(X,\sh{F})$ and $(Y,\sh{G})$ fixed) form a set which we denote by
$\Hom((X,\sh{F}),(Y,\sh{G}))$; the map $(u,h)\mapsto(\rho'(u),\Gamma(h))$ is a canonical
map
\[
  \Hom((X,\sh{F}),(Y,\sh{G}))
  \longrightarrow\Hom((\Gamma(Y,\OO_Y),\Gamma(Y,\sh{G})),(\Gamma(X,\OO_X),\Gamma(X,\sh{F})))
  \tag{1.8.9.1}
\]
\emph{functorial} in $(X,\sh{F})$ and $(Y,\sh{G})$, the right hand side being the set of
di-homomorphisms corresponding to the rings and modules considered \pref{env}{1.0.2}.
\end{env}

\begin{envs}[Corollary]{1.8.10}
\label{cor-1.1.8.10}
Let $Y$ be a locally ringed space, $\sh{G}$ an $\OO_Y$-module. For $Y$ to be an affine scheme
and $\sh{G}$ to be a quasi-coherent $\OO_Y$-module, it is necessary and sufficient that for
each pair $(X,\sh{F})$ consisting of a locally ringed space $X$ and an $\OO_X$-module
$\sh{F}$, the canonical map \emph{(1.8.9.1)} is bijective.
\end{envs}

We leave the reader to give the proof, which is modeled on that of \sref{cor}{1.8.3}, and
using \sref{prop}{1.8.1} and \sref{cor}{1.8.7}.

\begin{env}[Remark]{1.8.11}
\label{rmk-1.1.8.11}
The statements \sref{thm}{1.7.3}, \sref{cor}{1.7.4}, and \sref{prop}{2.2.4} are particular
cases of \sref{prop}{1.8.1}, as well as the definition in \sref{env}{1.6.1}; similarly,
\sref{prop}{2.2.5} follows from \sref{cor}{1.8.7}. Corollary \sref{1.8.7} also implies
\sref{prop}{1.6.3} (and as a result \sref{cor}{1.6.4}) as a particular case, since if $X$ is
an affine scheme and $\Gamma(X,\sh{F})=N$, the functors
$M\mapsto\Hom_{\OO_S}(\widetilde{M},u_*(\widetilde{N}))$ and
$M\mapsto\Hom_{\OO_S}(\widetilde{M},(N_{[\vphi]})^\sim)$ (where $\vphi:A\to\Gamma(X,\OO_X)$
corresponds to $u$) are isomorphic by \sref{cor}{1.8.7} and \sref{cor}{1.3.8}. Finally,
\sref{prop}{1.6.5} (and as a result \sref{cor}{1.6.6}) follow from \sref{cor}{1.8.6} and the
fact that for each $f\in A$, the $A_f$-modules $N'\otimes_{A'}A_f$ and $(N'\otimes_{A'}A)_f$
(notations of \sref{prop}{1.6.5}) are canonically isomorphic.
\end{env}

\section{Preschemes and morphisms of preschemes}
\label{1-schemes-2}

\subsection{Definition of preschemes}
\label{1-schemes-2.1}

\begin{env}{2.1.1}
\label{env-1.2.1.1}
Given a ringed space $(X,\OO_X)$, we say that an open subset
$V$ of $X$ is an \emph{affine open} if the ringed space $(V,\OO_X|V)$ is an
affine scheme \sref{env}{1.7.1}.
\end{env}

\begin{envs}[Definition]{2.1.2}
\label{defn-1.2.1.2}
We define a prescheme to be a ringed space
$(X,\OO_X)$ such that every point of $X$ admits an affine open neighbourhood.
\end{envs}

\begin{envs}[Proposition]{2.1.3}
\label{prop-1.2.1.3}
\oldpage{98}
If $(X,\OO_X)$ is a prescheme then the affine opens give a basis for the topology of $X$.
\end{envs}

Indeed, if $V$ is an arbitrary open neighbourhood of $x\in X$, then there
exists by hypothesis an open neighbourhood $W$ of $x$ such that $(W,\OO_X|W)$ is
an affine scheme; we write $A$ to mean its ring. In the space $W$, $V\cap W$ is
an open neighbourhood of $x$; thus there exists $f\in A$ such that $D(f)$ is an
open neighbourhood of $x$ contained inside $V\cap W$ (\sref{prop}{1.1.10}, (i)). The ringed
space $(D(f),\OO_X|D(f))$ is thus an affine scheme, isomorphic to $A_f$
\sref{env}{1.3.6}, whence the proposition.

\begin{envs}[Proposition]{2.1.4}
\label{prop-1.2.1.4}
The underlying space of a prescheme is a Kolmogoroff space.
\end{envs}

Indeed, if $x$, $y$ are two distinct points of a prescheme $X$ then it is clear
that there exists an open neighbourhood of one of these points that does not
contain the other if $x$ and $y$ are not in the same affine open; and if they
are in the same affine open, this is a result of \sref{cor}{1.1.8}.

\begin{envs}[Proposition]{2.1.5}
\label{prop-1.2.1.5}
If $(X,\OO_X)$ is a prescheme then every closed
irreducible subset of $X$ admits exactly one generic point, and the map
$x\mapsto\overline{\{x\}}$ is thus a bijection of $X$ onto its set of closed
irreducible subsets.
\end{envs}

Indeed, if $Y$ is a closed irreducible subset of $X$ and $y\in Y$, and if $U$
is an open affine neighbourhood of $y$ in $X$, then $U\cap Y$ is everywhere
dense in $Y$, as well as irreducible (\pref{env}{2.1.1} and \pref{env}{2.1.4}); thus by
\sref{cor}{1.1.14}, $U\cap Y$ is the closure in $U$ of a point $x$, and then
$Y\subset\overline{U}$ is the closure of $x$ in $X$. The uniqueness of the
generic point of $X$ is a result of \sref{prop}{2.1.4} and (\pref{env}{2.1.3}).

\begin{env}{2.1.6}
\label{env-1.2.1.6}
If $Y$ is a closed irreducible subset of $X$ and $y$ its
generic point then the local ring $\OO_y$, also written $\OO_{X/Y}$, is called the
\emph{local ring of $X$ along $Y$}, or the \emph{local ring of $Y$ in $X$}.

If $X$ itself is irreducible and $x$ its generic point then we say that
$\OO_x$ is the \emph{ring of rational functions on $X$} (cf. \textsection7).
\end{env}

\begin{envs}[Proposition]{2.1.7}
\label{prop-1.2.1.7}
If $(X,\OO_X)$ is a prescheme then the ringed
space $(U,\OO_X|U)$ is a prescheme for every open subset $U$.
\end{envs}

This follows directly from Definition \sref{defn}{2.1.2} and Proposition \sref{prop}{2.1.3}.

We say that $(U,\OO_X|U)$ is the prescheme \emph{induced} on $U$ by
$(X,\OO_X)$, or the \emph{restriction} of $(X,\OO_X)$ to $U$.

\begin{env}{2.1.8}
\label{env-1.2.1.8}
We say that a prescheme $(X,\OO_X)$ is \emph{irreducible}
(resp. \emph{connected}) if the underlying space $X$ is irreducible (resp.
connected). We say that a prescheme is \emph{integral} if it is
\emph{irreducible and reduced} (cf. \sref{env}{5.1.4}). We say that a prescheme
$(X,\OO_X)$ is \emph{locally integral} if each $x\in X$ admits an open
neighbourhood $U$ such that the prescheme induced on $U$ by $(X,\OO_X)$ is integral.
\end{env}

\subsection{Morphisms of preschemes}
\label{1-schemes-2.2}

\begin{envs}[Definition]{2.2.1}
\label{defn-1.2.2.1}
Given two preschemes $(X,\OO_X)$, $(Y,\OO_Y)$, we
define a morphism (of preschemes) of $(X,\OO_X)$ to $(Y,\OO_Y)$ to be a morphism
of ringed spaces $(\psi,\theta)$ such that, for all $x\in X$, $\theta_x^\sharp$ is a
local homomorphism $\OO_{\psi(x)}\to\OO_x$.
\end{envs}

\oldpage{99}
By passing to quotients, the map $\OO_{\psi(x)}\to\OO_x$ gives us a monomorphism
$\theta^x:\kres(\psi(x))\to\kres(x)$, which lets us consider $\kres(x)$ as an
\emph{extension} of the field $\kres(\psi(x))$.

\begin{env}{2.2.2}
\label{env-1.2.2.2}
The composition $(\psi'',\theta'')$ of two morphisms
$(\psi,\theta)$, $(\psi',\theta')$ of preschemes is also a morphism of
preschemes, since it is given by the formula
${\theta''}^\sharp=\theta^\sharp\circ\psi^*({\theta'}^\sharp)$ \pref{env}{3.5.5}. From this
we conclude that preschemes form a \emph{category}; using the usual notation, we
will write $\Hom(X,Y)$ to mean the set of morphisms from a prescheme $X$ to a
prescheme $Y$.
\end{env}

\begin{env}[Example]{2.2.3}
\label{exm-1.2.2.3}
If $U$ is an open subset of $X$ then the canonical
injection \pref{env}{4.1.2} of the induced prescheme $(U,\OO_X|U)$ into
$(X,\OO_X)$ is a morphism of preschemes; it is further a \emph{monomorphism} of
ringed spaces (and \emph{a fortiori} a monomorphism of preschemes), which
follows rapidly from \pref{env}{4.1.1}.
\end{env}

\begin{envs}[Proposition]{2.2.4}
\label{prop-1.2.2.4}
Let $(X,\OO_X)$ be a prescheme, and $(S,\OO_S)$ an
affine scheme associated to a ring $A$. Then there exists a canonical bijective
correspondence between morphisms of preschemes from $(X,\OO_X)$ to $(S,\OO_S)$ and
ring homomorphisms from $A$ to $\Gamma(X,\OO_X)$.
\end{envs}

Note first of all that, if $(X,\OO_X)$ and $(Y,\OO_Y)$ are two arbitrary ringed spaces,
a morphism $(\psi,\theta)$ from $(X,\OO_X)$ to $(Y,\OO_Y)$ canonically defines a ring
homomorphism
$\Gamma(\theta):\Gamma(Y,\OO_Y)\to\Gamma(Y,\psi_*(\OO_X))=\Gamma(X,\OO_X)$.
In the case that we consider, everything boils down to showing that any
homomorphism $\vphi:A\to\Gamma(X,\OO_X)$ is of the form $\Gamma(\theta)$
for one and only one $\theta$. Now, by hypothesis there is a covering
$(V_\alpha)$ of $X$ by affine opens; by composing of $\vphi$ with the
restriction homomorphism $\Gamma(X,\OO_X)\to\Gamma(V_\alpha,\OO_X|V_\alpha)$ we
obtain a homomorphism $\vphi_\alpha:A\to\Gamma(V_\alpha,\OO_X|V_\alpha)$
that corresponds to a unique morphism $(\psi_\alpha,\theta_\alpha)$ from the
prescheme $(V_\alpha,\OO_X|V_\alpha)$ to $(S,\OO_S)$, thanks to \sref{env}{1.7.3}.
Furthermore, for each pair of indices $(\alpha,\beta)$, each point of
$V_\alpha\cap V_\beta$ admits an open affine neighbourhood $W$ contained inside
$V_\alpha\cap V_\beta$ \sref{prop}{2.1.3}; it is clear that that, by composing
$\vphi_\alpha$ and $\vphi_\beta$ with the restriction homomorphisms to $W$,
we obtain the same homomorphism $\Gamma(S,\OO_S)\to\Gamma(W,\OO_X|W)$, so, thanks
to the relations $(\theta_\alpha^\sharp)_x=(\vphi_\alpha)_x$ for all $x\in
V_\alpha$ and all $\alpha$ \sref{env}{1.6.1}, the restriction to $W$ of the morphisms
$(\psi_\alpha,\theta_\alpha)$ and $(\psi_\beta,\theta_\beta)$ coincide. From
this we conclude that there is a morphism
$(\psi,\theta):(X,\OO_X)\to(S,\OO_S)$ of ringed spaces, and only one such
that its restriction to each $V_\alpha$ is $(\psi_\alpha,\theta_\alpha)$, and it
is clear that this morphism is a morphism of preschemes and such that
$\Gamma(\theta)=\vphi$.

Let $u:A\to\Gamma(X,\OO_X)$ be a ring homomorphism, and $v=(\psi,\theta)$
the corresponding morphism $(X,\OO_X)\to(S,\OO_S)$. For each $f\in A$ we have
that
\[
  \psi^{-1}(D(f))=X_{u(f)}
  \tag{2.2.4.1}
\]
with the notation of \pref{env}{5.5.2} relative to the locally-free sheaf
$\OO_X$. In fact, it suffices to verify this formula when $X$ itself is affine,
and then this is nothing but (1.2.2.2).

\begin{envs}[Proposition]{2.2.5}
\label{prop-1.2.2.5}
Under the hypotheses of \sref{prop}{2.2.4}, let
$\vphi:A\to\Gamma(X,\OO_X)$ be a ring homomorphism,
$f:(X,\OO_X)\to(S,\OO_S)$ the corresponding morphism of preschemes,
$\sh{G}$ (resp. $\sh{F}$) an $\OO_X$-module (resp. $\OO_S$-module), and
$M=\Gamma(S,\sh{F})$. Then there exists a canonical bijective
\oldpage{100}
correspondence between $f$-morphisms $\sh{F}\to\sh{G}$ \pref{env}{4.4.1} and
$A$-homomorphisms $M\to(\Gamma(X,\sh{G}))_{[\vphi]}$.
\end{envs}

Indeed, reasoning as in \sref{prop}{2.2.4}, we are rapidly led to the case where $X$ is
affine, and the proposition then follows from \sref{prop}{1.6.3} and \sref{env}{1.3.8}.

\begin{env}{2.2.6}
\label{env-1.2.2.6}
We say that a morphism of preschemes
$(\psi,\theta):(X,\OO_X)\to(Y,\OO_Y)$ is \emph{open} (resp. \emph{closed})
if, for all open subsets $U$ of $X$ (resp. all closed subsets $F$ of $X$),
$\psi(U)$ is open (resp. $\psi(F)$ is closed) in $Y$. We say that
$(\psi,\theta)$ is \emph{dominant} if $\psi(X)$ is dense in $Y$, and
\emph{surjective} if $\psi$ is surjective. We will point out that these
conditions rely only on the continuous map $\psi$.
\end{env}

\begin{envs}[Proposition]{2.2.7}
\label{prop-1.2.2.7}
Let
\begin{gather*}
  f=(\psi,\theta):(X,\OO_X)\to(Y,\OO_Y);\\
  g=(\psi',\theta'):(Y,\OO_Y)\to(Z,\OO_Z)
\end{gather*}
be two morphisms of
preschemes.
\begin{enumerate}[label=\rm{(\roman*)}]
  \item If $f$ and $g$ are both open (resp. closed, dominant, surjective),
        then so is $g\circ f$.
  \item If $f$ is surjective and $g\circ f$ closed, then $g$ is closed.
  \item If $g\circ f$ is surjective, then $g$ is surjective.
\end{enumerate}
\end{envs}
Claims (i) and (iii) are evident. Write $g\circ f=(\psi'',\theta'')$.
If $F$ is closed in $Y$ then $\psi^{-1}(F)$ is closed in $X$, so
$\psi''(\psi^{-1}(F))$ is closed in $Z$; but since $\psi$ is surjective,
$\psi(\psi^{-1}(F))=F$, so $\psi''(\psi^{-1}(F))=\psi'(F)$, which proves (ii).

\begin{envs}[Proposition]{2.2.8}
\label{prop-1.2.2.8}
Let $f=(\psi,\theta)$ be a morphism
$(X,\OO_X)\to(Y,\OO_Y)$, and $(U_\alpha)$ an open cover of $Y$. For $f$ to be
open (resp. closed, surjective, dominant), it is necessary and sufficient that
its restriction to each induced prescheme
$(\psi^{-1}(U_\alpha),\OO_X|\psi^{-1}(U_\alpha))$, considered as a morphism of
preschemes from this induced prescheme to the induced prescheme
$(U_\alpha,\OO_Y|U_\alpha)$ is open (resp. closed, surjective, dominant).
\end{envs}

The proposition follows immediately from the definitions, taking into
account the fact that a subset $F$ of $Y$ is closed (resp. open, dense) in $Y$
if and only if each of the $F\cap U_\alpha$ are closed (resp. open, dense) in
$U_\alpha$.

\begin{env}{2.2.9}
\label{env-1.2.2.9}
Let $(X,\OO_X)$ and $(Y,\OO_Y)$ be two preschemes; suppose that
$X$ (resp. $Y$) has a finite number of irreducible components $X_i$ (resp.
$Y_i$) ($1\leqslant i\leqslant n$); let $\xi_i$ (resp. $\eta_i$) be the generic
point of $X_i$ (resp. $Y_i$) \sref{env}{2.1.5}. We say that a morphism
\[
  f=(\psi,\theta):(X,\OO_X)\to(Y,\OO_Y)
\]
is \emph{birational} if, for all $i$, $\psi^{-1}(\eta_i)=\{\xi_i\}$ and
$\theta_{\xi_i}^\sharp:\OO_{\eta_i}\to\OO_{\xi_i}$ is an \emph{isomorphism}. It
is clear that a birational morphism is dominant \pref{env}{2.1.8}, and so is
surjective if it is also closed.
\end{env}

\begin{env}[Notational conventions]{2.2.10}
\label{rmk-1.2.2.10}
In all that follows, when there is
no risk of confusion, we \emph{suppress} the structure sheaf (resp. the morphism
of structure sheaves) from the notation of a prescheme (resp. morphism of
preschemes). If $U$ is an open subset of the underlying space $X$ of a
prescheme, then whenever we speak of $U$ as a prescheme we always mean the
induced prescheme on $U$.
\end{env}

\subsection{Gluing preschemes}
\label{1-schemes-2.3}

\begin{env}{2.3.1}
\label{env-1.2.3.1}
\oldpage{101}
It follows from definition \sref{defn}{2.1.2} that every ringed space obtained by
\emph{gluing} preschemes \pref{env}{4.1.6} is again a prescheme. In particular, although
every prescheme admits, by definition, a cover by affine open sets, we see that every
prescheme can actually be obtained by \emph{gluing affine schemes}.
\end{env}

\begin{env}[Example]{2.3.2}
\label{exm-1.2.3.2}
Let $K$ be a field, and $B=K[s]$, $C=K[t]$ be two polynomial rings in one indeterminate over
$K$, and define $X_1=\Spec(B)$, $X_2=\Spec(C)$, which are two isomorphic affine schemes. In
$X_1$ (resp. $X_2$), let $U_{12}$ (resp. $U_{21}$) be the affine open $D(s)$ (resp. $D(t)$)
where the ring $B_s$ (resp. $C_t$) is formed of rational fractions of the form $f(s)/s^m$
(resp. $g(t)/t^n$) with $f\in B$ (resp. $g\in C$). Let $u_{12}$ be the isomorphism of
preschemes $U_{21}\to U_{12}$ corresponding \sref{prop}{2.2.4} to the isomorphism from $B$ to
$C$ that, to $f(s)/s^m$, associates the rational fraction $f(1/t)/(1/t^m)$. We can glue $X_1$
and $X_2$ along $U_{12}$ and $U_{21}$ by using $u_{12}$, because there is clearly no gluing
condition. We later show that the prescheme $X$ obtained in this manner is a particular case
of a general method of construction (\textbf{II},~2.4.3). Here we only show that $X$ \emph{is
not an affine scheme}; this will follow from the fact that the ring $\Gamma(X,\OO_X)$ is
\emph{isomorphic} to $K$, and so its spectrum reduces to a point. Indeed, a section of
$\OO_X$ over $X$ has a restriction over $X_1$ (resp. $X_2$), identified to an affine open of
$X$, that is a polynomial $f(s)$ (resp. $g(t)$), and it follows from the definitions that we
should have $g(t)=f(1/t)$, which is not possible if $f=g\in K$.
\end{env}

\subsection{Local schemes}
\label{1-schemes-2.4}

\begin{env}{2.4.1}
\label{env-1.2.4.1}
We say that an affine scheme is a \emph{local scheme} if
it is the affine scheme associated to a local ring $A$; then there exists in
$X=\Spec(A)$ a single \emph{closed point $a\in X$}, and for all other $b\in X$
we have that $a\in\overline{\{b\}}$ \sref{cor}{1.1.7}.
\end{env}
    
For all preschemes $Y$ and points $y\in Y$, the local scheme $\Spec(\OO_y)$
is called the \emph{local scheme of $Y$ at the point $y$}. Let $V$ be an affine
open of $Y$ containing $y$, and $B$ the ring of the affine scheme $V$; then
$\OO_y$ is canonically identified with $B_y$ \sref{env}{1.3.4}, and the canonical
homomorphism $B\to B_y$ thus corresponds \sref{env}{1.6.1} to a morphism of preschemes
$\Spec(\OO_y)\to V$. If we compose this morphism with the canonical injection
$V\to Y$, then we obtain a morphism $\Spec(\OO_y)\to Y$, which is
\emph{independent} of the affine open $V$ (containing $y$) that we chose:
indeed, if $V'$ is some other affine open containing $y$, then there exists a
third affine open $W$ containing $y$ and such that $W\subset V\cap V'$ \sref{prop}{2.1.3};
we can thus assume that $V\subset V'$, and then if $B'$ is the ring of $V'$,
everything comes down to remarking that the diagram
\[
  \xymatrix{
    B'\ar[rr]\ar[dr] & & B\ar[dl]\\
    & \OO_y &
  }
\]
is commutative \pref{env}{1.5.1}. The morphism
\[
  \Spec(\OO_y)\longrightarrow Y
\]
thus defined is said to be \emph{canonical}.
    
\begin{envs}[Proposition]{2.4.2}
\label{prop-1.2.4.2}
\oldpage{102}
Let $(Y,\OO_Y)$ be a prescheme; for all $y\in Y$, let $(\psi,\theta)$ be the canonical
morphism $(\Spec(\OO_y),\widetilde{\OO}_y)\to(Y,\OO_Y)$. Then $\psi$ is a homeomorphism
from $\Spec(\OO_y)$ to the subspace $S_y$ of $Y$ given by the $z$ such that
$y\in\overline{\{z\}}$ (\emph{or, equivalently, the generalizations of $y$
\pref{env}{2.1.2}}; furthermore, if $z=\psi(\mathfrak{p})$, then
$\theta_z^\sharp:\OO_z\to(\OO_y)_\mathfrak{p}$ is an isomorphism; $(\psi,\theta)$
is thus a monomorphism of ringed spaces.
\end{envs}
    
As the unique closed point $a$ of $\Spec(\OO_y)$ is contained in the closure of any
point of this space, and since $\psi(a)=y$, the image of $\Spec(\OO_y)$ under
the continuous map $\psi$ is contained in $S_y$. Since $S_y$ is contained in
every affine open containing $y$, one can consider just the case where $Y$ is an
affine scheme; but then this proposition follows from \sref{env}{1.6.2}.
    
\emph{We see \sref{prop}{2.1.5} that there is a bijective correspondence between
$\Spec(\OO_y)$ and the set of closed irreducible subsets of $Y$ containing $y$.}
    
\begin{envs}[Corollary]{2.4.3}
\label{cor-1.2.4.3}
For $y\in Y$ to be the generic point of an
irreducible component of $Y$, it is necessary and sufficient that the only prime
ideal of the local ring $\OO_y$ is its maximal ideal (\emph{in other words, that
$\OO_y$ is of \emph{dimension zero}}).
\end{envs}
    
\begin{envs}[Proposition]{2.4.4}
\label{prop-1.2.4.4}
Let $(X,\OO_X)$ be a local scheme of a ring
$A$, $a$ its unique closed point, and $(Y,\OO_Y)$ a prescheme. Every morphism
$u=(\psi,\theta):(X,\OO_X)\to(Y,\OO_Y)$ then factorizes uniquely as
$X\to\Spec(\OO_{\psi(a)})\to Y$, where the second arrow denotes the canonical
morphism, and the first corresponds to a local homomorphism $\OO_{\psi(a)}\to A$.
This establishes a canonical bijective correspondence between the set of
morphisms $(X,\OO_X)\to(Y,\OO_Y)$ and the set of local homomorphisms $\OO_y\to A$
for ($y\in Y$).
\end{envs}
    
Indeed, for all $x\in X$, we have that $a\in\overline{\{x\}}$, so
$\psi(a)\in\overline{\{\psi(x)\}}$, which shows that $\psi(X)$ is contained in
every affine open containing $\psi(a)$. So it suffices to consider the case
where $(Y,\OO_Y)$ is an affine scheme of ring $B$, and we then have that
$u=({}^a\vphi,\tilde{\vphi})$, where $\vphi\in\Hom(B,A)$ \sref{thm}{1.7.3}. Further,
we have that $\vphi^{-1}(\mathfrak{j}_a)=\mathfrak{j}_{\psi(a)}$, and hence
that the image under $\vphi$ of any element of
$B-\mathfrak{j}_{\psi(a)}$ is invertible in the local ring $A$; the
factorization in the result follows from the universal property of the ring of
fractions \pref{env}{1.2.4}. Conversely, to each local homomorphism
$\OO_y\to A$ there exists a unique corresponding morphism
$(\psi,\theta):X\to\Spec(\OO_y)$ such that $\psi(a)=y$ \sref{env}{1.7.3}, and,
by composing with the canonical morphism $\Spec(\OO_y)\to Y$, we obtain a morphism
$X\to Y$, which proves the proposition.
    
\begin{env}{2.4.5}
\label{env-1.2.4.5}
The affine schemes whose ring is a field $K$ have an
underlying space that is just a point. If $A$ is a local ring with maximal
ideal $\mathfrak{m}$, then each local homomorphism $A\to K$ has kernel equal to
$\mathfrak{m}$, and so factorizes as $A\to A/\mathfrak{m}\to K$, where the
second arrow is a monomorphism. The morphisms $\Spec(K)\to\Spec(A)$ thus
correspond bijectively to monomorphisms of fields $A/\mathfrak{m}\to K$.
\end{env}
    
Let $(Y,\OO_Y)$ be a prescheme; for each $y\in Y$ and each ideal
$\mathfrak{a}_y$ of $\OO_y$, the canonical homomorphism
$\OO_y\to\OO_y/\mathfrak{a}_y$ defines a morphism
$\Spec(\OO_y/\mathfrak{a}_y)\to\Spec(\OO_y)$; if we compose this with the
canonical morphism $\Spec(\OO_y)\to Y$, then we obtain a morphism
$\Spec(\OO_y/\mathfrak{a}_y)\to Y$, again said to be \textit{canonical}. For
$\mathfrak{a}_y=\mathfrak{m}_y$, this says that $\OO_y/\mathfrak{a}_y=\kres(y)$, and
so Proposition \sref{prop}{2.4.4} says that:
    
\begin{envs}[Corollary]{2.4.6}
\label{cor-1.2.4.6}
\oldpage{103}
Let $(X,\OO_X)$ be a local scheme whose ring $K$ is a field, $\xi$ be the unique point of
$X$, and $(Y,\OO_Y)$ a prescheme. Then each morphism $u:(X,\OO_X)\to(Y,\OO_Y)$ factorizes
uniquely as $X\to\Spec(\kres(\psi(\xi)))\to Y$, where the second arrow denotes the canonical
morphism, and the first corresponds to a monomorphism $\kres(\psi(\xi))\to K$.
This establishes a canonical bijective correspondance between the set of
morphisms $(X,\OO_X)\to (Y,\OO_Y)$ and the set of monomorphisms $\kres(y)\to K$ (for
$y\in Y$).
\end{envs}
    
\begin{envs}[Corollary]{2.4.7}
\label{cor-1.2.4.7}
For all $y\in Y$, each canonical morphism
$\Spec(\OO_y/\mathfrak{a}_y)\to Y$ is a monomorphism of ringed spaces.
\end{envs}
    
We have already seen this when $\mathfrak{a}_y=0$ \sref{prop}{2.4.2}, and it suffices
to apply \sref{env}{1.7.5}.
    
\begin{env}[Remark]{2.4.8}
\label{rmk-1.2.4.8}
Let $X$ be a local scheme, and $a$ its unique
closed point. Since every affine open containing $a$ is necessarily in the
whole of $X$, every \emph{invertible} $\OO_X$-module \pref{env}{5.4.1} is
necessarily \emph{isomorphic to $\OO_X$} (or, as we say, again, \emph{trivial}).
This property doesn't hold in general, for an arbitrary affine scheme
$\Spec(A)$; we will see in Chapter~V that if $A$ is a normal ring then this is
true when $A$ is \unsure{\emph{factorial}}.
\end{env}

\subsection{Preschemes over a prescheme}
\label{1-schemes-2.5}

\begin{envs}[Definition]{2.5.1}
\label{defn-1.2.5.1}
Given a prescheme $S$, we say that the data of a
prescheme $X$ and a morphism of preschemes $\vphi:X\to S$ defines a
prescheme $X$ \emph{over the prescheme $S$}, or an \emph{$S$-prescheme}; we say
that $S$ is the \emph{base prescheme} of the $S$-prescheme $X$. The morphism
$\vphi$ is called the \emph{structure morphism} of the $S$-prescheme $X$.
When $S$ is an affine scheme of ring $A$, we also say that $X$ endowed with
$\vphi$ is a prescheme \emph{over the ring $A$} (or an \emph{$A$-prescheme}).
\end{envs}

It follows from \sref{prop}{2.2.4} that the data of a prescheme over a ring $A$ is
equivalent to the data of a prescheme $(X,\OO_X)$ whose structure sheaf $\OO_X$ is
a sheaf of \emph{$A$-algebras}. \emph{An arbitrary prescheme can always be
considered as a $\bb{Z}$-prescheme in a unique way.}

If $\vphi:X\to S$ is the structure morphism of an $S$-prescheme $X$, we
say that a point $x\in X$ is \emph{over a point $s\in S$} if $\vphi(x)=s$. We
say that $X$ \emph{dominates} $S$ if $\vphi$ is a dominant morphism \sref{env}{2.2.6}.

\begin{env}{2.5.2}
\label{env-1.2.5.2}
Let $X$ and $Y$ be two $S$-preschemes; we say that a morphism
of preschemes $u:X\to Y$ is a \emph{morphism of preschemes over $S$} (or
an \emph{$S$-morphism}) if the diagram
\[
  \xymatrix{
    X \ar[rr]^u \ar[dr] & & Y\ar[dl]\\
    & S &
  }
\]
(where the diagonal arrows are the structure morphisms) is
commutative: this ensures that, for all $s\in S$ and $x\in X$ over $s$, $u(x)$
is also above $s$.
\end{env}

From this definition it follows immediately that the composition of two
$S$-morphisms is an $S$-morphism; $S$-preschemes thus form a \emph{category}.

We denote by $\Hom_S(X,Y)$ the set of $S$-morphisms from an $S$-prescheme $X$ to
an $S$-prescheme $Y$; the identity morphism of an $S$-prescheme is denoted by
$1_X$.

When $S$ is an affine scheme of ring $A$, we will also say \emph{$A$-morphism}
instead of $S$-morphism.

\begin{env}{2.5.3}
\label{env-1.2.5.3}
\oldpage{104}
If $X$ is an $S$-prescheme, and $v:X'\to X$ a morphism of preschemes,
then the composition $X'\to X\to S$ endows $X'$ with the structure of an $S$-prescheme;
in particular, every prescheme induced by an open set $U$ of $X$ can be considered as an
$S$-prescheme by the canonical injection.
\end{env}

If $u:X\to Y$ is an $S$-morphism of $S$-preschemes, then the restriction
of $u$ to any prescheme induced by an open subset $U$ of $X$ is also an
$S$-morphism $U\to Y$. Conversely, let $(U_\alpha)$ be an open cover of $X$,
and for each $\alpha$ let $u_\alpha:U_\alpha\to Y$ be an $S$-morphism; if,
for all pairs of indices $(\alpha,\beta)$, the restrictions of $u_\alpha$ and
$u_\beta$ to $U_\alpha\cap U_\beta$ agree, then there exists an $S$-morphism
$X\to Y$, and only one such that the restriction to each $U_\alpha$ is
$u_\alpha$.

If $u:X\to Y$ is an $S$-morphism such that $u(X)\subset V$, where $V$ is
an open subset of $Y$, then $u$, considered as a morphism from $X$ to $V$, is
also an $S$-morphism.

\begin{env}{2.5.4}
\label{env-1.2.5.4}
Let $S'\to S$ be a morphism of preschemes; for all
$S'$-preschemes, the composition $X\to S'\to S$ endows $X$ with the structure of
an $S$-prescheme. Conversely, suppose that $S'$ is the induced prescheme of an
open subset of $S$; let $X$ be an $S$-prescheme and suppose that the structure
morphism $f:X\to S$ is such that $f(X)\subset S'$; then we can consider
$X$ as an $S'$-prescheme. In this latter case, if $Y$ is another $S$-prescheme
whose structure morphism sends the underlying space to $S'$, then every
$S$-morphism from $X$ to $Y$ is also an $S'$-morphism.
\end{env}

\begin{env}{2.5.5}
\label{env-1.2.5.5}
If $X$ is an $S$-prescheme, with structure morphism
$\vphi:X\to S$, we define an \emph{$S$-section of $X$} to be an
$S$-morphism from $S$ to $X$, that is to say a morphism of preschemes
$\psi:S\to X$ such that $\vphi\circ\psi$ is the identity on $S$. We
denote by $\Gamma(X/S)$ the set of $S$-sections of $X$.
\end{env}

\section{Products of preschemes}
\label{1-schemes-3}

\subsection{Sums of preschemes}
\label{1-schemes-3.1}

Let $(X_\alpha)$ be any family of preschemes; let $X$ be a topological space which is the
\emph{sum} of the underlying spaces $X_\alpha$; $X$ is then the union of the pairwise
disjoint open subspaces $X_\alpha'$, and for each $\alpha$ there is a homomorphism
$\vphi_\alpha$ from $X_\alpha$ to $X_\alpha'$. If we equip each of the $X_\alpha'$ with the
sheaf $(\vphi_\alpha)_*(\OO_{X_\alpha})$, it is clear that $X$ becomes a prescheme, which
we call the \emph{sum} of the family of preschemes $(X_\alpha)$ and which we denote
$\amalg_\alpha X_\alpha$. If $Y$ is a prescheme, the map $f\mapsto(f\circ\vphi_\alpha)$ is a
\emph{bijection} from the set $\Hom(X,Y)$ to the product set $\Pi_\alpha\Hom(X_\alpha,Y)$.
In particular, if the $X_\alpha$ are $S$-preschemes, with structure morphisms $\psi_\alpha$,
$X$ is an $S$-prescheme by the unique morphism $\psi:X\to S$ such that
$\psi\circ\vphi_\alpha=\psi_\alpha$ for each $\alpha$. The sum of two preschemes $X$, $Y$ is
denoted by $X\amalg Y$. It is immediate that if $X=\Spec(A)$, $Y=\Spec(B)$, $X\amalg Y$
canonically identifies with $\Spec(A\times B)$.

\subsection{Products of preschemes}
\label{1-schemes-3.2}

\begin{envs}[Definition]{3.2.1}
\label{defn-1.3.2.1}
Given two $S$-preschemes $X$, $Y$, we say that a triple $(Z,p_1,p_2)$ consisting of an
$S$-prescheme $Z$ and of two $S$-morphisms $p_1:Z\to X$, $p_2:Z\to Y$, is a product of the
\oldpage{105}
$S$-preschemes $X$ and $Y$, if, for each $S$-prescheme $T$, the map
$f\mapsto(p_1\circ f,p_2\circ f)$ is a bijection from the set of $S$-morphisms from $T$ to
$Z$, to the set of pairs consisting of an $S$-morphism $T\to X$ and an $S$-morphism $T\to Y$
(in other words, a bijection
\[
  \Hom_S(T,Z)\isoto\Hom_S(T,X)\times\Hom_S(T,Y)).
\]
\end{envs}

This is therefore a general notion of a \emph{product} of two objects in a category, applied
to the category of $S$-preschemes (T, I, 1.1); in particular, a product of two $S$-preschemes
is \emph{unique} up to a unique $S$-isomorphism. Because of this uniqueness, most of the time
we will denote a product of two $S$-preschemes $X$, $Y$ by the notation $X\times_S Y$ (or
simply $X\times Y$ when there is no chance of confusion), the morphisms $p_1$, $p_2$ (called
the \emph{canonical projections} of $X\times_S$ to $X$ and $Y$, respectively) are suppressed
in the notation. If $g:T\to X$, $h:T\to Y$ are two $S$-morphisms, we denote by $(g,h)_S$, or
simply $(g,h)$, the $S$-morphism $f:T\to X\times_S Y$ such that $p_1\circ f=g$,
$p_2\circ f=h$. If $X'$, $Y'$ are two $S$-preschemes, $p_1'$, $p_2'$ the canonical
projections of $X'\times_S Y'$ (assumed to exist), $u:X'\to X$, $v:Y'\to Y$ two
$S$-morphisms, then we write $u\times_S v$ (or simply $u\times v$) for the $S$-morphism
$(u\circ p_1',v\circ p_2')_S$ from $X'\times_S Y'$ to $X\times_S Y$.

When $S$ is an affine scheme of ring $A$, we often replace $S$
by $A$ is the above notations.

\begin{envs}[Proposition]{3.2.2}
\label{prop-1.3.2.2}
Let $X$, $Y$, $S$ be three affine schemes, $B$, $C$, $A$ their respective rings. Let
$Z=\Spec(B\otimes_A C)$, $p_1$, $p_2$ the $S$-morphisms corresponding \sref{prop}{2.2.4} to
the canonical $A$-homomorphisms $u:b\mapsto b\otimes 1$ and $v:c\mapsto 1\otimes c$ from $B$
and $C$ to $B\otimes_A C$; then $(Z,p_1,p_2)$ is a product of $X$ and $Y$.
\end{envs}

According to \sref{prop}{2.2.4}, it suffices to check that if, to each $A$-homomorphism
$f:B\otimes_A C\to L$ (where $L$ is an $A$-algebra), we associate the pair
$(f\circ u,f\circ v)$, then we define a bijection
$\Hom_A(B\otimes_A C,L)\isoto\Hom_A(B,L)\times\Hom_A(C,L)$,\footnote{The notation $\Hom_A$
denotes here the set of homomorphisms of \emph{$A$-algebras}.} which follows immediately
from the definitions and the relation $b\otimes c=(b\otimes 1)(1\otimes c)$.

\begin{envs}[Corollary]{3.2.3}
\label{cor-1.3.2.3}
Let $T$ be an affine scheme of ring $D$, $\alpha=({}^a\rho,\widetilde{\rho})$
(resp. $\beta=({}^a\sigma,\widetilde{\sigma})$) an $S$-morphism $T\to X$ (resp. $T\to Y$),
where $\rho$ (resp. $\sigma$) is an $A$-homomorphism from $B$ (resp. $C$) to $D$; then
$(\alpha,\beta)_S=({}^a\tau,\widetilde{\tau})$, where $\tau$ is the homomorphism
$B\otimes_A C\to D$ such that $\tau(b\otimes c)=\rho(b)\sigma(c)$.
\end{envs}

\begin{envs}[Proposition]{3.2.4}
\label{prop-1.3.2.4}
Let $f:S'\to S$ be a \emph{monomorphism} of preschemes \normalfont{(T, I, 1.1)}, $X$, $Y$ two
$S'$-preschemes, which are also considered as $S$-preschemes by means of $f$. Each product of
$S$-preschemes $X$, $Y$ is then a product of $S'$-preschemes $X$, $Y$, and conversely.
\end{envs}

Let $\vphi:X\to S'$, $\psi:Y\to S'$ be the structure morphisms. If $T$ is an $S$-prescheme,
$u:T\to X$, $v:T\to Y$ two $S$-morphisms, we have by definition
$f\circ\vphi\circ u=f\circ\psi\circ v=\theta$, the structure morphism of $T$; the hypotheses
on $f$ imply that $\vphi\circ u=\psi\circ v=\theta'$, and we can consider $T$ as an
$S'$-prescheme with structure morphism $\theta'$, $u$ and $v$ as $S'$-morphisms. The
conclusion of the proposition follows immediately, taking into account \sref{defn}{3.2.1}.

\begin{envs}[Corollary]{3.2.5}
\label{cor-1.3.2.5}
Let $X$, $Y$ be two $S$-preschemes, $\vphi:X\to S$, $\psi:Y\to S$ their structure morphisms,
$S'$ an open subset of $S$ wuch that $\vphi(X)\subset S'$, $\psi(Y)\subset S'$. Each product
of $S$-preschemes $X$, $Y$ is then also a product of $S'$-preschemes $X$, $Y$, and
conversely.
\end{envs}

\oldpage{106}
It suffices to apply \sref{prop}{3.2.4} to the canonical injection $S'\to S$.

\begin{envs}[Theorem]{3.2.6}
\label{thm-1.3.2.6}
Given two $S$-preschemes $X$, $Y$, there exists a product $X\times_S Y$.
\end{envs}

We proceed in several steps.

\begin{envs}[Lemma]{3.2.6.1}
\label{lem-1.3.2.6.1}
Let $(Z,p,q)$ be a product of $X$ and $Y$, $U$, $V$ two open subsets of $X$, $Y$,
respectively. If we put $W=p^{-1}(U)\cap q^{-1}(V)$, then the triple consisting of $W$ and
the restrictions of $p$ and $q$ to $W$ \emph{(considered as the morphisms $W\to U$, $W\to V$,
respectively)} is a product of $U$ and $V$.
\end{envs}

Indeed, if $T$ is an $S$-prescheme, we can identify the $S$-morphisms $T\to W$ and the
$S$-morphisms $T\to Z$ mapping $T$ to $W$. If then $g:T\to U$, $h:T\to V$ are any two
$S$-morphisms, we can consider them as $S$-morphisms from $T$ to $X$ and $Y$ respectively,
and by hypothesis there is then a unique $S$-morphism $f:T\to Z$ such that $g=p\circ f$,
$h=q\circ f$. As $p(f(Y))\subset U$, $q(f(T))\subset V$, we have
\[
  f(T)\subset p^{-1}(U)\cap q^{-1}(V)=W,
\]
hence our assertion.

\begin{envs}[Lemma]{3.2.6.2}
\label{lem-1.3.2.6.2}
Let $Z$ be an $S$-prescheme, $p:Z\to X$, $q:Z\to Y$ two $S$-morphisms, $(U_\alpha)$ an open
cover of $X$, $(V_\lambda)$ an open cover of $Y$. Suppose that for each pair
$(\alpha,\lambda)$, the $S$-prescheme
$W_{\alpha\lambda}=p^{-1}(U_\alpha)\cap q^{-1}(V_\lambda)$ and the restrictions of $p$ and
$q$ to $W_{\alpha\lambda}$ form a product of $U_\alpha$ and $V_\lambda$. Then $(Z,p,q)$ is a
product of $X$ and $Y$.
\end{envs}

We first show that, if $f_1$, $f_2$ are two $S$-morphisms $T\to Z$, then the relations
$p\circ f_1=p\circ f_2$ and $q\circ f_1=q\circ f_2$ imply $f_1=f_2$. Indeed, $Z$ is the union
of the $W_{\alpha\lambda}$, so the $f_1^{-1}(W_{\alpha\lambda})$ form an open cover of $T$,
and similarly for $f_2^{-1}(W_{\alpha\lambda})$. In addition, we have
\[
  f_1^{-1}(W_{\alpha\lambda})=f_1^{-1}(p^{-1}(U_\alpha))\cap f_1^{-1}(q^{-1}(V_\lambda))
  =f_2^{-1}(p^{-1}(U_\alpha))\cap f_2^{-1}(q^{-1}(V_\lambda))=f_2^{-1}(W_{\alpha\lambda})
\]
by hypothesis, and it reduces to seeing that the the restrictions of $f_1$ and $f_2$ to
$f_1^{-1}(W_{\alpha\lambda})=f_2^{-1}(W_{\alpha\lambda})$ are identical for each pair of
indices. But as these restrictions can be considered as $S$-morphisms from
$f_1^{-1}(W_{\alpha\lambda})$ to $W_{\alpha\lambda}$, our assertion follows from the
hypotheses and Definition \sref{defn}{3.2.1}.

Suppose now that we are given two $S$-morphisms $g:T\to X$, $h:T\to Y$. Put
$T_{\alpha\lambda}=g^{-1}(U_\alpha)\cap h^{-1}(V_\lambda)$; the $T_{\alpha\lambda}$ form an
open cover of $T$. By hypothesis, there exists an $S$-morphism $f_{\alpha\lambda}$ such that
$p\circ f_{\alpha\lambda}$ and $q\circ f_{\alpha\lambda}$ are the respective restrictions of
$g$ and $h$ to $T_{\alpha\lambda}$. In addition, we show that the restrictions of
$f_{\alpha\lambda}$ and $f_{\beta\mu}$ to $T_{\alpha\lambda}\cap T_{\beta\mu}$ coincide,
which would finish the proof of \sref{lem}{3.2.6.2}. The images of
$T_{\alpha\lambda}\cap T_{\beta\mu}$ under $f_{\alpha\lambda}$ and $f_{\beta\mu}$ are
contained in $W_{\alpha\lambda}\cap W_{\beta\mu}$ by definition. As
\[
  W_{\alpha\lambda}\cap W_{\beta\mu}
  =p^{-1}(U_\alpha\cap U_\beta)\cap q^{-1}(V_\lambda\cap V_\mu),
\]
it follows from \sref{lem}{3.2.6.1} that $W_{\alpha\lambda}\cap W_{\beta\mu}$ and the
restrictions to this prescheme of $p$ and $q$ form a \emph{product} of $U_\alpha\cap U_\beta$
and $V_\lambda\cap V_\mu$. As $p\circ f_{\alpha\lambda}$ and $p\circ f_{\beta\mu}$ coincide
on $T_{\alpha\lambda}\cap T_{\beta\mu}$ and similarly for $q\circ f_{\alpha\lambda}$ and
$q\circ f_{\beta\mu}$, we see that $f_{\alpha\lambda}$ and $f_{\beta\mu}$ coincide on
$T_{\alpha\lambda}\cap T_{\beta\mu}$, q.e.d.

\begin{envs}[Lemma]{3.2.6.3}
\label{lem-1.3.2.6.3}
\oldpage{107}
Let $(U_\alpha)$ be an open cover of $X$, $(V_\lambda)$ an open cover of $Y$, and suppose
that for each pair $(\alpha,\lambda)$, there exists a product of $U_\alpha$ and $V_\lambda$;
then there exists a product of $X$ and $Y$.
\end{envs}

Applying Lemma \sref{lem}{3.2.6.1} to the open sets $U_\alpha\cap U_\beta$ and
$V_\lambda\cap V_\mu$, we see that there exists a product of $S$-preschemes induced
respectively by $X$ and $Y$ on these open sets; in addition, the uniqueness of the product
shows that, if we set $i=(\alpha,\lambda)$, $j=(\beta,\mu)$, then there is a canonical
isomorphism $h_{ij}$ (resp. $h_{ji}$) from this product to an $S$-prescheme $W_{ij}$
(resp. $W_{ji}$) induced by $U_\alpha\times_S V_\lambda$ (resp. $U_\beta\times_S V_\mu$) on
an open set; $f_{ij}=h_{ij}\circ h_{ji}^{-1}$ is then an isomorphism from $W_{ji}$ to
$W_{ij}$. In addition, for a third pair $k=(\gamma,\nu)$, we have
$f_{ik}=f_{ij}\circ f_{jk}$ on
$W_{ki}\cap W_{kj}$, as it follows from applying \sref{lem}{3.2.6.1} to the open sets
$U_\alpha\cap U_\beta\cap U_\gamma$ and $V_\lambda\cap V_\mu\cap V_\nu$ in $U_\beta$ and
$V_\mu$, respectively. It follows that we have a prescheme $Z$, an open cover $(Z_i)$ of the
underlying space of $Z$, and for each $i$ and isomorphism $g_i$ from the induced prescheme
$Z_i$ to the prescheme $U_\alpha\times_S V_\lambda$, so that for each pair $(i,j)$, we have
$f_{ij}=g_i\circ g_j^{-1}$ \sref{env}{2.3.1}; in addition, we have $g_i(Z_i\cap Z_j)=W_{ij}$.
If $p_i$, $q_i$, $\theta_i$ are the projections and the structure morphism of the
$S$-prescheme $U_\alpha\times_S V_\lambda$, we immediately note that
$p_i\circ g_i=p_j\circ g_j$ on $Z_i\cap Z_j$, and similarly for the two other morphisms. We
can thus define the morphisms of preschemes $p:Z\to X$ (resp. $q:Z\to Y$, $\theta:Z\to S$) by
the condition that $p$ (resp. $q$, $\theta$) coincide with $p_i\circ g_i$
(resp. $q_i\circ g_i$, $\theta_i\circ g_i$) on each of the $Z_i$; $Z$, equipped with
$\theta$, is then an $S$-prescheme. We now show that
$Z_i'=p^{-1}(U_\alpha)\cap q^{-1}(V_\lambda)$ is equal to $Z_i$. For each index
$j=(\beta,\mu)$, we have $Z_j\cap Z_i'=g_j^{-1}(p_j^{-1}(U_\alpha)\cap q_j^{-1}(V_\lambda))$.
We have
\[
  p_j^{-1}(U_\alpha)\cap q_j^{-1}(V_\lambda)
  =p_j^{-1}(U_\alpha\cap U_\beta)\cap q_j^{-1}(V_\lambda\cap V_\mu);
\]
according to \sref{lem}{3.2.6.1}, the restrictions of $p_j$ and $q_j$ to
$p_j^{-1}(U_\alpha)\cap q_j^{-1}(V_\lambda)$ define on this $S$-prescheme the structure of a
product of $U_\alpha\cap U_\beta$ and $V_\lambda\cap V_\mu$; but the uniqueness of the
product then implies that $p_j^{-1}(U_\alpha)\cap q_j^{-1}(V_\lambda)=W_{ji}$. As a result we
have $Z_j\cap Z_i'=Z_j\cap Z_i$ for each $j$, hence $Z_i'=Z_i$. We then deduce from
\sref{lem}{3.2.6.2} that $(Z,p,q)$ is a product of $X$ and $Y$.

\begin{envs}[Lemma]{3.2.6.4}
\label{lem-1.3.2.6.4}
Let $\vphi:X\to S$, $\psi:Y\to S$ be the structure morphisms of $X$ and $Y$, $(S_i)$ and open
cover of $S$, and set $X_i=\vphi^{-1}(S_i)$, $Y_i=\psi^{-1}(S_i)$. If each of the products
$X_i\times_S Y_i$ exists, then $X\times_S Y$ exists.
\end{envs}

According to \sref{lem}{3.2.6.3}, everything comes down to proving that the products
$X_i\times_S Y_i$ exists for any $i$ and $j$. Set
$X_{ij}=X_i\cap X_j=\vphi^{-1}(S_i\cap S_j)$, $Y_{ij}=Y_i\cap Y_j=\psi^{-1}(S_i\cap S_j)$;
according to \sref{lem}{3.2.6.1}, the product $Z_{ij}=X_{ij}\times_S Y_{ij}$ exists. We now
note that if $T$ is an $S$-prescheme and if $g:T\to X_i$, $h:T\to Y_j$ are $S$-morphisms,
then we necessarily have that $\vphi(g(T))=\psi(h(T))\subset S_i\cap S_j$ according to the
definition of an $S$-morphisms, thus $g(T)\subset X_{ij}$ and $h(T)\subset Y_{ij}$; it is
then immediate that $Z_{ij}$ is the product of $X_i$ and $Y_j$.

\begin{env}{3.2.6.5}
\label{env-1.3.2.6.5}
We can now complete the proof of Theorem \sref{thm}{3.2.6}. If $S$ is an \emph{affine
scheme}, there are covers $(U_\alpha)$, $(V_\lambda)$ of $X$ and $Y$ respectively,
consisting of affine opens; as $U_\alpha\times_S V_\lambda$ exists according to
\sref{prop}{3.2.2}, there similarly exists $X\times_S Y$ by \sref{lem}{3.2.6.3}. If $S$
is any prescheme, there is a cover $(S_i)$ of $S$ consisting of affine opens. If
$\vphi:X\to S$, $\psi:Y\to S$ are the structure morphisms, and if we set
$X_i=\vphi^{-1}(S_i)$, $Y_i=\psi^{-1}(S_i)$, the products $X_i\times_{S_i}Y_i$ exist
according to the
\oldpage{108}
above; but then the products $X_i\times_S Y_i$ also exist \sref{cor}{3.2.5}, therefore
$X\times_S Y$ similarly exists by \sref{lem}{3.2.6.4}.
\end{env}

\begin{envs}[Corollary]{3.2.7}
\label{cor-1.3.2.7}
Let $Z=X\times_S Y$ be the product of two $S$-preschemes, $p$, $q$ the projections from $Z$
to $X$ and $Y$, $\vphi$ (resp. $\psi$) the structure morphism of $X$ (resp. $Y$). Let $S'$ be
an open subset of $S$, $U$ (resp. $V$) an open subset of $X$ (resp. $Y$) contained in
$\vphi^{-1}(S')$ (resp. $\psi^{-1}(S')$). Then the product $U\times_{S'}V$ canonically
identifies with the prescheme induced on $Z$ by $p^{-1}(U)\cap q^{-1}(V)$ (considered as a
$S'$-prescheme). In addition, if $f:T\to X$, $g:T\to Y$ are $S$-morphisms such that
$f(T)\subset U$, $g(T)\subset V$, the $S'$-morphism $(f,g)_{S'}$ identifies with the
restriction of $(f,g)_S$ to $p^{-1}(U)\cap q^{-1}(V)$.
\end{envs}

This follows from \sref{cor}{3.2.5} and \sref{lem}{3.2.6.1}.

\begin{env}{3.2.8}
\label{env-1.3.2.8}
Let $(X_\alpha)$, $(Y_\lambda)$ be two familes of $S$-preschemes, $X$ (resp. $Y$) the sum
of the family $(X_\alpha)$ (resp. $(Y_\lambda)$) (3.1). Then $X\times_S Y$ identifies with
the \emph{sum} of the family $(X_\alpha\times_S Y_\lambda)$; this follows immediately from
\sref{lem}{3.2.6.3}.
\end{env}

\begin{env}{3.2.9}
\label{env-1.3.2.9}
\footnote{[Trans] \sref{env}{3.2.9} is from the errata of EGA~II, on page 221.}
\oldpage{221$'$}
It follows from \sref{prop}{1.8.1} that we can state \sref{prop}{3.2.2} in the following
manner: $Z=\Spec(B\otimes_A C)$ is not only a product of $X=\Spec(B)$ and $Y=\Spec(C)$ in the
category of \emph{$S$-preschemes}, but also in the category of \emph{locally ringed spaces
over $S$} (with a definition of $S$-morphisms modeled on that of \sref{env}{2.5.2}). The
proof of \sref{thm}{3.2.6} also proves that for any two $S$-preschemes $X$, $Y$, the
prescheme $X\times_S Y$ is not only the product of $X$ and $Y$ in the category of
$S$-preschemes, but also in the category of locally ringed spaces over the prescheme $S$.
\end{env}

\subsection{Formal properties of the product; change of the base prescheme}
\label{1-schemes-3.3}

\begin{env}{3.3.1}
\label{env-1.3.3.1}
The reader will notice that all the properties stated in this section, except
\sref{env}{3.3.13} and \sref{env}{3.3.15}, are true without modification in any
category, whenever the products involved in the statements exist (since it is
clear that the notions of an $S$-object and of an $S$-morphism can be defined
exactly as in (2.5) for any object $S$ of the category).
\end{env}

\begin{env}{3.3.2}
\label{env-1.3.3.2}
First, $X\times_S Y$ is a \emph{covariant bifunctor} in $X$ and $Y$ on the
category of $S$-preschemes: it suffices in fact to note that the diagram
\[
  \xymatrix{
    X\times Y\ar[r]^{f\times 1}\ar[d] &
    X'\times Y\ar[r]^{f'\times 1}\ar[d] &
    X''\times Y\ar[d]\\
    X\ar[r]^f &
    X'\ar[r]^{f'} &
    X''
  }
\]
is commutative.
\end{env}

\begin{envs}[Proposition]{3.3.3}
\label{prop-1.3.3.3}
For each $S$-prescheme $X$, the first (resp. second) projection from
$X\times_S S$ (resp. $S\times_S X$) is a functorial isomorphism from
$X\times_S S$ (resp. $S\times_S X$) to $X$, whose inverse isomorphism is
$(1_X,\vphi)_S$ (resp. $(\vphi,1_X)_S$), where we denote by $\vphi$ the
structure morphism $X\to S$; therefore we can write, up to a canonical
isomorphism,
\[
  X\times_S S=S\times_S X=X.
\]
\end{envs}

It suffices to prove that the triple $(X,1_X,\vphi)$ is a product of $X$ and
$S$. If $T$ is an $S$-prescheme, the only $S$-morphism from $T$ to $S$ is
necessarily the structure morphism $\psi:T\to S$. If $f$ is an $S$-morphism from
$T$ to $X$, we necessarily have $\psi=\vphi\circ f$, hence our assertion.

\begin{envs}[Corollary]{3.3.4}
\label{cor-1.3.3.4}
Let $X$, $Y$ be two $S$-preschemes, $\vphi:X\to S$, $\psi:Y\to S$ their
structure morphisms. If we canonically identify $X$ with $X\times_S S$ and $Y$
with $S\times_S Y$, the projections $X\times_S Y\to X$ and $X\times_S Y\to Y$
identify respectively with $1_X\times\psi$ and $\vphi\times 1_Y$.
\end{envs}

The proof is immediate and is left to the reader.

\begin{env}{3.3.5}
\label{env-1.3.3.5}
We can define in a manner similar to (3.2) the product of a
\oldpage{109}
finite number $n$ of $S$-preschemes, the existence of these products following
from \sref{thm}{3.2.6} by induction on $n$, and noting that
$(X_1\times_S X_2\times_S\cdots\times_S X_{n-1})\times_S X_n$ satisfies the
definition of a product. The uniqueness of the product implies, as in any
category, its \emph{commutativity} and \emph{associativity} properties. If, for
example, $p_1$, $p_2$, $p_3$ denote the projections from
$X_1\times_S X_2\times_S X_3$, and if we identify this prescheme with
$(X_1\times_S X_2)\times_S X_3$, then the projection to $X_1\times_S X_2$ is
identified with $(p_1,p_2)_S$.
\end{env}

\begin{env}{3.3.6}
\label{env-1.3.3.6}
Let $S$, $S'$ be two preschemes, $\vphi:S'\to S$ a morphism, which makes $S'$ an
$S$-prescheme. For each $S$-prescheme $X$, consider the product $X\times_S S'$,
and let $p$ and $\pi'$ be the projections to $X$ and $S'$ respectively. Equipped
with $\pi'$, this product is an $S'$-prescheme; when we consider it as such, we
denote it by $X_{(S')}$ or $X_{(\vphi)}$, and we say that this is the prescheme
obtained by \emph{base change} from $S$ to $S'$, by means of the morphism
$\vphi$, or the \emph{inverse image} of $X$ by $\vphi$. We note that if $\pi$ is
the structure morphism of $X$, $\theta$ the structure morphism of
$X\times_S S'$, considered as an $S$-prescheme, then the diagram
\[
  \xymatrix{
    X\ar[d]_\pi &
    X_{(S')}\ar[l]_p\ar[ld]_\theta\ar[d]^{\pi'}\\
    S &
    S'\ar[l]_\vphi
  }
\]
is commutative.
\end{env}

\begin{env}{3.3.7}
\label{env-1.3.3.7}
With the notations of \sref{env}{3.3.6}, for each $S$-morphism $f:X\to Y$, we
denote by $f_{(S')}$ the $S'$-morphism $f\times_S 1:X_{(S')}\to Y_{(S')}$, and
we say that $f_{(S')}$ is the \emph{base change} (or \emph{inverse image}) of
$f$ by $\vphi$. Therefore, $X_{(S')}$ is a \emph{covariant functor} in $X$, from
the category of $S$-preschemes to that of $S'$-preschemes.
\end{env}

\begin{env}{3.3.8}
\label{env-1.3.3.8}
The prescheme $X_{(S')}$ can be considered as a solution to a \emph{universal
mapping problem}: each $S'$-prescheme $T$ is also an $S$-prescheme via $\vphi$;
each $S$-morphism $g:T\to X$ is then uniquely written as $g=p\circ f$, where $f$
is an $S'$-morphism $T\to X_{(S')}$, as it follows from the definition of the
product applied to the $S$-morphisms $f$ and $\psi:T\to S'$ (the structure
morphism of $T$).
\end{env}

\begin{envs}[Proposition]{3.3.9}
\label{prop-1.3.3.9}
\emph{(``Transitivity of base change'')}. Let $S''$ be a prescheme,
$\vphi':S''\to S$ a morphism. For each $S$-prescheme $X$, there exists an
canonical functorial isomorphism from the $S''$-prescheme
$(X_{(\vphi)})_{(\vphi')}$ to the $S''$-prescheme $X_{(\vphi\circ\vphi')}$.
\end{envs}

Indeed, let $T$ be a $S''$-prescheme, $\psi$ its structure morphism, and $g$ an
$S$-morphism from $T$ to $X$ ($T$ being considered as an $S$-prescheme with
structure morphism $\vphi\circ\vphi'\circ\psi$). As $T$ is also a $S'$-prescheme
with structure morphism $\vphi'\circ\psi$, we can write $g=p\circ g'$, where
$g'$ is an $S'$-morphism $T\to X_{(\vphi)}$, and then $g'=p'\circ g''$, where
$g''$ is an $S''$-morphism $T\to(X_{(\vphi)})_{(\vphi')}$:
\[
  \xymatrix{
    X\ar[d]_\pi &
    X_{(\vphi)}\ar[l]_p\ar[d]_{\pi'} &
    (X_{(\vphi)})_{(\vphi')}\ar[l]_{p'}\ar[d]^{\pi''}\\
    S &
    S\ar[l]_\vphi &
    S''\ar[l]_{\vphi'}.
  }
\]
\oldpage{110}
Hence the result follows by the uniqueness of the solution to a universal
mapping problem.

This result can be written as the equality (up to a canonical isomorphism)
$(X_{(S')})_{(S'')}=X_{(S'')}$, if there is no chance of confusion, or also
\[
  (X\times_S S')\times_{S'}S''=X\times_S S'';
  \tag{3.3.9.1}
\]
the functorial nature of the isomorphism defined in \sref{prop}{3.3.9} can
similarly be expressed by the transitivity formula for base change morphisms
\[
  (f_{(S')})_{(S'')}=f_{(S'')}
  \tag{3.3.9.2}
\]
for each $S$-morphism $f:X\to Y$.

\begin{envs}[Corollary]{3.3.10}
\label{cor-1.3.3.10}
If $X$ and $Y$ are two $S$-preschemes, then there exists a canonical functorial
isomorphism from the $S'$-prescheme $X_{(S')}\times_{S'}Y_{(S')}$ to the
$S'$-prescheme $(X\times_S Y)_{(S')}$.
\end{envs}

Indeed, we have, up to canonical isomorphism,
\[
  (X\times_S S')\times_{S'}(Y\times_S S')
  =X\times_S(Y\times_S S')=(X\times_S Y)\times_S S'
\]
according to (3.3.9.1) and the associativity of products of $S$-preschemes.

The functorial nature of the isomorphism defined in \sref{cor}{3.3.10} can be
expressed by the formula
\[
  (u_{(S')},v_{(S')})_{S'}=((u,v)_S)_{(S')}
  \tag{3.3.10.1}
\]
for each pair of $S$-morphisms $u:T\to X$, $v:T\to Y$.

In other words, the base change functor $X_{(S')}$ \emph{commutes with
products}; it also commutes with sums \sref{env}{3.2.8}.

\begin{envs}[Corollary]{3.3.11}
\label{cor-1.3.3.11}
Let $Y$ be an $S$-prescheme, $f:X\to Y$ a morphism which makes $X$ a
$Y$-prescheme (and as a result also an $S$-prescheme). The prescheme $X_{(S')}$
then identifies with the product $X\times_Y Y_{(S')}$, the projection
$X\times_Y Y_{(S')}\to Y_{(S')}$ identifying with $f_{(S')}$.
\end{envs}

Let $\psi:Y\to S$ be the structure morphism of $Y$; we have the commutative
diagram
\[
  \xymatrix{
    S'\ar[d] &
    Y_{(S')}\ar[l]\ar[d] &
    X_{(S')}\ar[l]_{f_{(S')}}\ar[d]\\
    S &
    Y\ar[l]_\psi &
    X\ar[l]_f.
  }
\]
We have that $Y_{(S')}$ identifies with $S_{(\psi)}'$ and $X_{(S')}$ with
$S_{(\psi\circ f)}'$; taking into account \sref{prop}{3.3.9} and
\sref{cor}{3.3.4}, we deduce the corollary.

\begin{env}{3.3.12}
\label{env-1.3.3.12}
Let $f:X\to X'$, $g:Y\to Y'$ be two $S$-morphisms which are \emph{monomorphisms}
of preschemes (T, I, 1.1); then $f\times_S g$ is a \emph{monomorphism}. Indeed,
if $p$ and $q$ are the projections of $X\times_S Y$, $p'$, $q'$ those of
$X'\times_S Y'$, and $u$, $v$ two $S$-morphisms $T\to X\times_S Y$, then the
relation $(f\times_S g)\circ u=(f\times_S g)\circ v$ implies that
$p'\circ(f\times_S g)\circ u=p'\circ(f\times_S g)\circ v$, in other words,
$f\circ p\circ u=f\circ p\circ v$, and as $f$ is a monomorphism,
$p\circ u=p\circ v$; using the fact that $g$ is a monomorphism, we similarly
obtain $q\circ u=q\circ v$, hence $u=v$.

\oldpage{111}
It follows that for each base change $S'\to S$,
\[
  f_{(S')}:X_{(S')}\longrightarrow Y_{(S')}
\]
is a monomorphism.
\end{env}

\begin{env}{3.3.13}
\label{env-1.3.3.13}
Let $S$, $S'$ be two affine schemes of rings $A$, $A'$ respectively; a morphism
$S'\to S$ then corresponds to a ring homomorphism $A\to A'$. If $X$ is an
$S$-prescheme, we denote by $X_{(A')}$ or $X\otimes_A A'$ the $S'$-prescheme
$X_{(S')}$; when $X$ is also affine of ring $B$, $X_{(A')}$ is affine of ring
$B_{(A')}=B\otimes_A A'$ obtained by extension by scalars from the $A$-algebra
$B$ to $A'$.
\end{env}

\begin{env}{3.3.14}
\label{env-1.3.3.14}
With the notations of \sref{env}{3.3.6}, for each \emph{$S$-morphism}
$f:S'\to X$, $f'=(f,1_{S'})_S$ is an $S'$-morphism $S'\to X'=X_{(S')}$ such that
$p\circ f'=f$, $\pi'\circ f'=1_{S'}$, in other words an \emph{$S'$-section of
of $X'$}; conversely, if $f'$ is such an $S'$-section, $f=p\circ f'$ is an
$S$-morphism $S'\to X$. We thus define a canonical
\emph{bijective correspondence}
\[
  \Hom_S(S',X)\isoto\Hom_{S'}(S',X').
\]
We say that $f'$ is the \emph{graph morphism} of $f$, and we denote it by
$\Gamma_f$.
\end{env}

\begin{env}{3.3.15}
\label{env-1.3.3.15}
Given a prescheme $X$, which we can always consider it as a $\bb{Z}$-prescheme,
it follows in particular from \sref{env}{3.3.14} that the \emph{$X$-sections} of
$X\otimes_\bb{Z}\bb{Z}[T]$ (where $T$ is an indeterminate) bijectively
correspond to \emph{morphisms} $\bb{Z}[T]\to X$. Let us show that these
$X$-sections also bijectively correspond to \emph{sections of the structure
sheaf $\OO_X$ over $X$}. Indeed, let $(U_\alpha)$ be a cover of $X$ by the
affine opens; let $u:X\to X\otimes_\bb{Z}\bb{Z}[T]$ be an $X$-morphism and let
$u_\alpha$ be its restriction to $U_\alpha$; if $A_\alpha$ is the ring of the
affine scheme $U_\alpha$, then $U_\alpha\otimes_\bb{Z}\bb{Z}[T]$ is an affine
scheme of ring $A_\alpha[T]$ \sref{prop}{3.2.2}, and $u_\alpha$ canonically
corresponds to an $A_\alpha$-homomorphism $A_\alpha[T]\to A_\alpha$
\sref{thm}{1.7.3}. Now, as such a homomorphism is completely determined by the
data of the image of $T$ in $A_\alpha$, let
$s_\alpha\in A_\alpha=\Gamma(U_\alpha,\OO_X)$, and if we suppose that the
restrictions of $u_\alpha$ and $u_\beta$ to an open affine
$V\subset U_\alpha\cap U_\beta$ coincide, then we see immediately that
$s_\alpha$ and $s_\beta$ coincide on $V$; thus the family $(s_\alpha)$ consists
of the restrictions to $U_\alpha$ of a section $s$ of $\OO_X$ over $X$;
convsersely, it is clear that such a section defines a family $(u_\alpha)$ of
morphisms which are the restrictions to $U_\alpha$ of an $X$-morphism
$X\to X\otimes_\bb{Z}\bb{Z}[T]$. This result is generalized in
(\textbf{II},~1.7.12).
\end{env}

\section{Subpreschemes and immersion morphisms}
\label{1-schemes-4}

\section{Reduced preschemes; separation conditions}
\label{1-schemes-5}

\section{Finiteness conditions}
\label{1-schemes-6}

\section{Rational maps}
\label{1-schemes-7}

\section{Chevalley schemes}
\label{1-schemes-8}

\subsection{Allied local rings}
\label{1-schemes-8.1}

For each local ring $A$, we denote by $\mathfrak{m}(A)$ the maximal ideal of
$A$.

\begin{envs}[Lemma]{8.1.1}
\label{lem-1.8.1.1}
Let $A$ and $B$ be two local rings such that $A\subset B$;
then the following conditions are equivalent: \emph{(i)}
$\mathfrak{m}(B)\cap A=\mathfrak{m}(A)$; \emph{(ii)}
$\mathfrak{m}(A)\subset\mathfrak{m}(B)$; \emph{(iii)} $1$ is not an element of
the ideal of $B$ generated by $\mathfrak{m}(A)$.
\end{envs}

It's evident that (i) implies (ii), and (ii) implies (iii); lastly, if (iii) is
true, then $\mathfrak{m}(B)\cap A$ contains $\mathfrak{m}(A)$ and doesn't
contain $1$, and is thus equal to $\mathfrak{m}(A)$.

When the equivalent conditions of \sref{lem}{8.1.1} are satisfied, we say that $B$
\emph{dominates} $A$; this is equivalent to saying that the injection $A\to B$
is a \emph{local} homomorphism. It is clear that, in the set of local subrings
of a ring $R$, the relation given by domination is an \unsure{order}.

\begin{env}{8.1.2}
\label{env-1.8.1.2}
Now consider a \emph{field} $R$. For all subrings $A$ of
$R$, we denote by $L(A)$ the set of local rings $A_\mathfrak{p}$, where
$\mathfrak{p}$ runs over the prime spectrum of $A$; they are identified with the
subrings of $R$ containing $A$. Since
$\mathfrak{p}=(\mathfrak{p}A_\mathfrak{p})\cap A$, the map
$\mathfrak{p}\mapsto A_\mathfrak{p}$ from $\Spec(A)$ to $L(A)$ is bijective.
\end{env}

\begin{envs}[Lemma]{8.1.3}
\label{lem-1.8.1.3}
Let $R$ be a field, and $A$ a subring of $R$. For a
local subring $M$ of $R$ to dominate a ring $A_\mathfrak{p}\in L(A)$ it is
necessary and sufficient that $A\subset M$; the local ring $A_\mathfrak{p}$
dominated by $M$ is then unique, and corresponds to
$\mathfrak{p}=\mathfrak{m}(M)\cap A$.
\end{envs}

Indeed, if $M$ dominates $A_\mathfrak{p}$, then $\mathfrak{m}(M)\cap
A_\mathfrak{p}=\mathfrak{p}A_\mathfrak{p}$, by \sref{lem}{8.1.1}, whence the
uniqueness of $\mathfrak{p}$; on the other hand, if $A\subset M$, then
$\mathfrak{m}M\cap A=\mathfrak{p}$ is prime in $A$, and since
$A-\mathfrak{p}\subset M$, we have that $A_\mathfrak{p}\subset M$ and
$\mathfrak{p}A_\mathfrak{p}\subset\mathfrak{m}(M)$, so $M$ dominates
$A_\mathfrak{p}$

\begin{envs}[Lemma]{8.1.4}
\label{lem-1.8.1.4}
\oldpage{165}
Let $R$ be a field, $M$ and $N$ two local
subrings of $R$, and $P$ the subring of $R$ generated by $M\cup N$. Then the
following conditions are equivalent:
\begin{enumerate}[label=\rm{(\roman*)}]
  \item There exists a prime ideal $\mathfrak{p}$ of $P$ such that
        $\mathfrak{m}(M)=\mathfrak{p}\cap M$ and $\mathfrak{m}(N)=\mathfrak{p}\cap N$.
  \item The ideal $\mathfrak{a}$ generated in $P$ by $\mathfrak{m}(M)\cup\mathfrak{m}(N)$ is
        distinct from $P$.
  \item There exists a local subring $Q$ of $R$ simultaneously dominating both $M$ and $N$.
\end{enumerate}
\end{envs}

It is clear that (i) implies (ii); conversely, if $\mathfrak{a}\neq P$, then
$\mathfrak{a}$ is contained in a maximal ideal $\mathfrak{n}$ of $P$, and since
$1\not\in\mathfrak{n}$, $\mathfrak{n}\cap M$ contains $\mathfrak{m}(M)$ and is
distinct from $M$, so $\mathfrak{n}\cap M=\mathfrak{m}(M)$, and similarly
$\mathfrak{n}\cap N=\mathfrak{m}(N)$. It is clear that, if $Q$ dominates both
$M$ and $N$, then $P\subset Q$ and
$\mathfrak{m}(M)=\mathfrak{m}(Q)\cap M=(\mathfrak{m}(Q)\cap P)\cap M$, and
$\mathfrak{m}(N)=(\mathfrak{m}(Q)\cap P)\cap N$, so (iii) implies (i); the inverse is evident
when we take $Q=P_\mathfrak{p}$.

When the conditions of \sref{lem}{8.1.4} are satisfied, we say, with C.~Chevalley,
that the local rings $M$ and $N$ are \emph{allied}.

\begin{envs}[Proposition]{8.1.5}
\label{prop-1.8.1.5}
Let $A$ and $B$ be two subrings of a field $R$,
and $C$ the subring of $R$ generated by $A\cup B$. Then the following
conditions are equivalent:
\begin{enumerate}[label=\rm{(\roman*)}]
  \item For every local ring $Q$ containing $A$ and $B$, we have that
        $A_\mathfrak{p}=B_\mathfrak{q}$, where $\mathfrak{p}=\mathfrak{m}(Q)\cap A$ and
        $\mathfrak{q}=\mathfrak{m}(Q)\cap B$.
  \item For all prime ideals $\mathfrak{r}$ of $C$, we have that
        $A_\mathfrak{p}=B_\mathfrak{q}$, where $\mathfrak{p}=\mathfrak{r}\cap A$ and
        $\mathfrak{q}=\mathfrak{r}\cap B$.
  \item If $M\in L(A)$ and $N\in L(B)$ are allied, then they are identical.
  \item $L(A)\cap L(B)=L(C)$.
\end{enumerate}
\end{envs}

Lemmas \sref{lem}{8.1.3} and \sref{lem}{8.1.4} prove that (i) and (iii) are equivalent; it
is clear that (i) implies (ii) by taking $Q=C_\mathfrak{r}$; conversely, (ii)
implies (i), because if $Q$ contains $A\cup B$ then it contains $C$, and if
$\mathfrak{r}=\mathfrak{m}(Q)\cap C$ then $\mathfrak{p}=\mathfrak{r}\cap A$ and
$\mathfrak{q}=\mathfrak{r}\cap B$, from \sref{lem}{8.1.3}. It is immediate that (iv)
implies (i), because if $Q$ contains $A\cup B$ then it dominates a local ring
$C_\mathfrak{r}\in L(C)$ by \sref{lem}{8.1.3}; by hypothesis we have that
$C_\mathfrak{r}\in L(A)\cap L(B)$, and \sref{lem}{8.1.1} and \sref{lem}{8.1.3} prove that
$C_\mathfrak{r}=A_\mathfrak{p}=B_\mathfrak{q}$. We prove finally that (iii)
implies (iv). Let $Q\in L(C)$; $Q$ dominates some $M\in L(A)$ and some $N\in
L(B)$ \sref{lem}{8.1.3}, so $M$ and $N$, being allied, are identical by hypothesis.
As we then have that $C\subset M$, we know that $M$ dominates some $Q'\in L(C)$
\sref{lem}{8.1.3}, so $Q$ dominates $Q'$, whence necessarily \sref{lem}{8.1.3} $Q=Q'=M$,
so $Q\in L(A)\cap L(B)$. Conversely, if $Q\in L(A)\cap L(B)$, then $C\subset
Q$, so \sref{lem}{8.1.3} $Q$ dominates some $Q''\in L(C)\subset L(A)\cap L(B)$; $Q$
and $Q''$, being allied, are identical, so $Q''=Q\in L(C)$, which completes the
proof.

\subsection{Local rings of an integral scheme}
\label{1-schemes-8.2}

\begin{env}{8.2.1}
\label{env-1.8.2.1}
Let $X$ be an \emph{integral} prescheme, and $R$ its field of
rational functions, identical to the local ring of the generic point $a$ of $X$;
for all $x\in X$, we know that $\OO_x$ can be canonically identified with a
subring of $R$ \sref{env}{7.1.5}, and for every rational function $f\in R$, the
domain of definition $\delta(f)$ of $f$ is the open set of $x\in X$ such that
$f\in\OO_x$. It thus follows from \sref{env}{7.2.6} that, for every open $U\subset X$,
we have
\[
  \Gamma(U,\OO_X)=\bigcap_{x\in U}\OO_x.
  \tag{8.2.1.1}
\]
\end{env}

\begin{envs}[Proposition]{8.2.2}
\label{prop-1.8.2.2}
\oldpage{166}
Let $X$ be an integral prescheme,
and $R$ its field of rational fractions. For $X$ to be a scheme, it is
necessary and sufficient that the relation ``$\OO_x$ and $\OO_y$ are allied''
\sref{lem}{8.1.4}, for points $x$, $y$ of $X$, implies that $x=y$.
\end{envs}

Suppose that this condition is verified, and aim to show that $X$ is separated.
Let $U$ and $V$ be two distinct affine opens of $X$, with rings $A$ and $B$,
identified with subrings of $R$; $U$ (resp.$V$) is thus identified \sref{env}{8.1.2}
with $L(A)$ (resp.$L(B)$), and the hypothesis tells us \sref{env}{8.1.5} that $C$ is
the subring of $R$ generated by $A\cup B$, and $W=U\cap V$ is identified with
$L(A)\cap L(B)=L(C)$. Further, we know
(\cite{1}, p.~\unsure{5-03},~4~\emph{bis}) that every subring $E$ of $R$ is
equal to the intersection of the local rings belonging to $L(E)$; $C$ is thus
identified with the intersection of the rings $\OO_z$ for $z\in W$, or,
equivalently (8.2.1.1) with $\Gamma(W,\OO_X)$. So consider the subprescheme
induced by $X$ on $W$; to the \unsure{identity} morphism $\vphi:
C\to\Gamma(W,\OO_X)$ there corresponds \sref{env}{2.2.4} a morphism
$\Phi=(\psi,\theta):W\to\Spec(C)$; we will see that $\Phi$ is an
\emph{isomorphism} of preschemes, whence $W$ is an \emph{affine} open. The
identification of $W$ with $L(C)=\Spec(C)$ shows that $\psi$ is
\emph{bijective}.On the other hand, for all $x\in W$, $\theta_x^\sharp$ is the
injection $C_\mathfrak{r}\to\OO_x$, where $\mathfrak{r}=\mathfrak{m}_x\cap C$,
and by definition $C_\mathfrak{r}$ is identified with $\OO_x$, so $\theta_x^\sharp$
is bijective. It thus remains to show that $\psi$ is a \emph{homeomorphism},
i.e. that for every closed subset $F\subset W$, $\psi(F)$ is closed in
$\Spec(C)$. But $F$ is the \unsure{trace over} $W$ of closed subspace of $U$,
of the form $V(\mathfrak{a})$, where $\mathfrak{a}$ is an ideal of $A$; we show
that $\psi(F)=V(\mathfrak{a}C)$, which proves our claim. In fact, the prime
ideals of $C$ containing $\mathfrak{a}C$ are the prime ideals of $C$ containing
$\mathfrak{a}$, and so are the ideals of the form $\psi(x)=\mathfrak{m}_x\cap
C$, where $\mathfrak{a}\subset\mathfrak{m}_x$ and $x\in W$; since
$\mathfrak{a}\subset\mathfrak{m}_x$ is equivalent to $x\in V(\mathfrak{a})=W\cap
F$ for $x\in U$, we do indeed have that $\psi(F)=V(\mathfrak{a}C)$.

It follows that $X$ is separated, because $U\cap V$ is affine and its ring $C$
is generated by the union $A\cup B$ of the rings of $U$ and $V$ \sref{env}{5.5.6}.

Conversely, suppose that $X$ is separated, and let $x$, $y$ be two points of $X$
such that $\OO_x$ and $\OO_y$ are allied. Let $U$ (resp. $V$) be an affine open
containing $x$ (resp. $y$), of ring $A$ (resp. $B$); we then know that $U\cap V$
is affine and that its ring $C$ is generated by $A\cup B$ \sref{env}{5.5.6}. If
$\mathfrak{p}=\mathfrak{m}_x\cap A$ and $\mathfrak{q}=\mathfrak{m}_y\cap B$,
then $A_\mathfrak{p}=\OO_x$ and $B_\mathfrak{q}=\OO_y$, and since $A_\mathfrak{p}$
and $B_\mathfrak{q}$ are allied, there exists a prime ideal $\mathfrak{r}$ of
$C$ such that $\mathfrak{p}=\mathfrak{r}\cap A$ and
$\mathfrak{q}=\mathfrak{r}\cap B$ \sref{lem}{8.1.4}. But then there exists a point
$z\in U\cap V$ such that $\mathfrak{r}=\mathfrak{m}_z\cap C$, since $U\cap V$ is
affine, and so evidently $x=z$ and $y=z$, whence $x=y$.

\begin{envs}[Corollary]{8.2.3}
\label{cor-1.8.2.3}
Let $X$ be an integral scheme, and $x$, $y$ two
points of $X$.In order that $x\in\overline{\{y\}}$, it is necessary and
sufficient that $\OO_x\subset\OO_y$, or, equivalently, that every rational
function defined at $x$ is also defined at $y$.
\end{envs}

The condition is evidently necessary because the domain of definition
$\delta(f)$ of a rational function $f\in R$ is open; we now show that it is
sufficient.If $\OO_x\subset\OO_y$, then there exists a prime ideal
$\mathfrak{p}$ of $\OO_x$ such that $\OO_y$ dominates $(\OO_x)_\mathfrak{p}$
\sref{lem}{8.1.3}; but \sref{env}{2.4.2} there exists $z\in X$ such that
$x\in\overline{\{z\}}$ and $\OO_z=(\OO_x)_\mathfrak{p}$; since $\OO_z$ and $\OO_y$
are allied, we have that $z=y$ by \sref{prop}{8.2.2}, whence the corollary.

\begin{envs}[Corollary]{8.2.4}
\label{cor-1.8.2.4}
If $X$ is an integral scheme then the map
$x\to\OO_x$ is injective; equivalently, if $x$ and $y$ are two distinct points of
$X$, then there exists a rational function defined at one of these points but
not the other.
\end{envs}

\oldpage{167}
This follows from \sref{cor}{8.2.3} and the axiom ($T_0$) \sref{prop}{2.1.4}.

\begin{envs}[Corollary]{8.2.5}
\label{cor-1.8.2.5}
Let $X$ be an integral scheme whose underlying
space is Noetherian; letting $f$ run over the field $R$ of rational functions on
$X$, the sets $\delta(f)$ generate the topology of $X$.
\end{envs}

In fact, every closed subset of $X$ is thus a finite union of irreducible closed
subsets, i.e. of the form $\overline{\{y\}}$ \sref{env}{2.1.5}. But, if
$x\not\in\overline{\{y\}}$, then there exists a rational function $f$ defined at
$x$ but not at $y$ \sref{cor}{8.2.3}, or, equivalently, we have that $x\in\delta(f)$
and $\delta(f)$ is not contained in $\overline{\{y\}}$. The complement of
$\overline{\{y\}}$ is thus a union of sets of the form $\delta(f)$, and by
virtue of the first remark, every open subset of $X$ is the union of finite
intersections of open sets of the form $\delta(f)$.

\begin{env}{8.2.6}
\label{env-1.8.2.6}
Corollary \sref{cor}{8.2.5} shows that the topology of $X$ is
entirely characterised by the data of the local rings $(\OO_x)_{x\in X}$ that
have $R$ as their field of fractions. It amounts to the same to say that the
closed subsets of $X$ are defined in the following manner: given a finite subset
$\{x_1,\ldots,x_n\}$ of $X$, consider the set of $y\in X$ such that
$\OO_y\subset\OO_{x_i}$ for at least one index $i$, and these sets (over all
choices of $\{x_1,\ldots,x_n\}$) are the closed subsets of $X$. Further, once
the topology on $X$ is known, the structure sheaf $\OO_X$ is also determined by
the family of the $\OO_x$, since $\Gamma(U,\OO_X)=\bigcap_{x\in U}\OO_x$ by
(8.2.1.1). The family $(\OO_X)_{x\in X}$ thus completely determines the
prescheme $X$ when $X$ is an integral scheme whose underlying space is
Noetherian.
\end{env}

\begin{envs}[Proposition]{8.2.7}
\label{prop-1.8.2.7}
Let $X$, $Y$ be two integral schemes, $f:X\to Y$ a dominant morphism
\sref{env}{2.2.6}, and $K$ (resp.$L$) the field of rational
functions on $X$ (resp.$Y$). Then $L$ can be identified with a subfield of
$K$, and for all $x\in X$, $\OO_{f(x)}$ is the unique local ring of $Y$ dominated
by $\OO_x$.
\end{envs}

In fact, if $f=(\psi,\theta)$ and $a$ is the generic point of $X$, then
$\psi(a)$ is the generic point of $Y$ \pref{env}{2.1.5}; $\theta_a^\sharp$ is then
a monomorphism of fields, from $L=\OO_{\psi(a)}$ to $K=\OO_a$. Since every
non-empty affine open $U$ of $Y$ contains $\psi(a)$, it follows from
\sref{env}{2.2.4} that the homomorphism $\Gamma(U,\OO_Y)\to\Gamma(\psi^{-1}(U),\OO_X)$
corresponding to $f$ is the restriction of $\theta_a^\sharp$ to $\Gamma(U,\OO_Y)$.
So, for every $x\in X$, $\theta_x^\sharp$ is the restriction to $\OO_{\psi(a)}$ of
$\theta_a^\sharp$, and is thus a monomorphism. We also know that $\theta_x^\sharp$ is a
local homomorphism, so, if we identify $L$ with a subfield of $K$ by
$\theta_a^\sharp$, $\OO_{\psi(x)}$ is dominated by $\OO_x$ \sref{lem}{8.1.1}; it is also
the only local ring of $Y$ dominated by $\OO_x$, since two local rings of $Y$
that are allied are identical \sref{prop}{8.2.2}.

\begin{envs}[Proposition]{8.2.8}
\label{prop-1.8.2.8}
Let $X$ be an \emph{irreducible} prescheme; and
$f:X\to Y$ a local immersion (\emph{resp.} a local isomorphism); and
suppose further that $f$ is separated. Then $f$ is an immersion (\emph{resp.}
an open immersion).
\end{envs}

Let $f=(\psi,\theta)$; it suffices, in both cases, to prove that $\psi$ is a
\emph{homeomorphism} from $X$ to $\psi(X)$ \sref{env}{4.5.3}. Replacing $f$ by
$f_\mathrm{red}$ (\sref{env}{5.1.6} and \sref{env}{5.5.1}, (vi)), we can assume that $X$
and $Y$ are \emph{reduced}. If $Y'$ is the closed reduced subprescheme of $Y$
having $\overline{\psi(X)}$ as its underlying space, then $f$ factorizes as
$X\xrightarrow{f'}Y'\xrightarrow{j}Y$, where $j$ is the canonical injection
\sref{env}{5.2.2}. It follows from (\sref{env}{5.5.1}, (v)) that $f'$ is again a
separated morphism; further, $f'$ is again
\oldpage{168}
a local immersion (resp. a local isomorphism), because, since the condition is local on $X$
and $Y$, we can reduce ourselves to the case where $f$ is a closed immersion (resp. open
immersion), and then our claim follows immediately from \sref{env}{4.2.2}.

We can thus suppose that $f$ is a \emph{dominant} morphism, which leads to the
fact that $Y$ is, itself, irreducible \pref{env}{2.1.5}, and so $X$ and $Y$
are both \emph{integral}. Further, the condition being local on $Y$, we can
suppose that $Y$ is an affine scheme; since $f$ is separated, $X$ is a scheme
(\sref{env}{5.5.1}, (ii)), and we are finally at the hypotheses of \sref{prop}{8.2.7}.
Then, for all $x\in X$, $\theta_x^\sharp$ is injective; but the hypothesis that $f$
is a local immersion implies that $\theta_x^\sharp$ is surjective \sref{env}{4.2.2}, so
$\theta_x^\sharp$ is bijective, or, equivalently (with the identification of
\sref{prop}{8.2.7}) we have that $\OO_{\psi(x)}=\OO_x$. This implies, by \sref{cor}{8.2.4},
that $\psi$ is an \emph{injective} map, which already proves the proposition
when $f$ is a local isomorphism \sref{env}{4.5.3}. When we suppose that $f$ is only
a local immersion, for all $x\in X$ there exists an open neighbourhood $U$ of
$x$ in $X$ and an open neighbourhood $V$ of $\psi(x)$ in $Y$ such that the
restriction of $\psi$ to $U$ is a homeomorphism from $U$ to a \emph{closed}
subset of $V$. But $U$ is dense in $X$, so $\psi(U)$ is dense in $Y$ and
\emph{a fortiori} in $V$, which proves that $\psi(U)=V$; since $\psi$ is
injective, $\psi^{-1}(V)=U$ and this proves that $\psi$ is a homeomorphism from
$X$ to $\psi(X)$.

\subsection{Chevalley schemes}
\label{1-schemes-8.3}

\begin{env}{8.3.1}
\label{env-1.8.3.1}
Let $X$ be a \emph{Noetherian} integral scheme, and $R$ its
field of rational functions; we denote by $X'$ the set of local subrings
$\OO_x\subset R$, where $x$ runs over all points of $X$. The set $X'$ verifies
the three following conditions:
\begin{enumerate}
  \item[(Sch.~1)] For all $M\in X'$, $R$ is the field of fractions of $M$.
  \item[(Sch.~2)] There exists a finite set of Noetherian subrings $A_i$ of $R$
  such that $X'=\bigcup_i L(A_i)$, and, for all pairs of indices $i$, $j$, the subring
  $A_{ij}$ of $R$ generated by $A_i\cup A_j$ is an algebra of finite type over $A_i$.
  \item[(Sch.~3)] Two elements $M$ and $N$ of $X'$ that are allied are identical.
\end{enumerate}
\end{env}

We have basically seen in \sref{env}{8.2.1} that (Sch.~1) is satisfied, and (Sch.~3)
follows from \sref{env}{8.2.2}. To show (Sch.~2), it suffices to cover $X$ by a
finite number of affine opens $U_i$, whose rings are Noetherian, and to take
$A_i=\Gamma(U_i,\OO_X)$; the hypothesis that $X$ is a scheme implies that
$U_i\cap U_j$ is affine, and also that $\Gamma(U_i\cap U_j,\OO_X)=A_{ij}$
\sref{env}{5.5.6}; further, since the space $U_i$ is Noetherian, the immersion
$U_i\cap U_j\to U_i$ is of finite type \sref{env}{6.3.5}, so $A_{ij}$ is an
$A_i$-algebra of finite type \sref{env}{6.3.3}.

\begin{env}{8.3.2}
\label{env-1.8.3.2}
The structures whose axioms are (Sch.~1), (Sch.~2), and
(Sch.~3), generalise ``schemes'' in the sense of C.~Chevalley, who supposes
furthermore that $R$ is an extension of finite type of a field $K$, and that the
$A_i$ are $K$-algebras of finite type (which renders a part of (Sch.~2) useless)
\cite{1}. Conversely, if we have such a structure on a set $X'$, then we can
associate to it an integral scheme $X$ by using the remarks from \sref{env}{8.2.6}:
the underlying space of $X$ is equal to $X'$ endowed with the topology defined
in \sref{env}{8.2.6}, and with the sheaf $\OO_X$ such that
$\Gamma(U,\OO_X)=\bigcap_{x\in U}\OO_x$ for all open $U\subset X$, with the
evident definition of restriction homomorphisms. We leave to the reader the
task of verifying that we obtain thusly an integral scheme, whose local rings
are the elements of $X'$; we will not use this result in what follows.
\end{env}

\section{Supplement on quasi-coherent sheaves}
\label{1-schemes-9}

\subsection{Tensor product of quasi-coherent sheaves}
\label{1-schemes-9.1}

\begin{envs}[Proposition]{9.1.1}
\label{prop-1.9.1.1}
\oldpage{169}
Let $X$ be a prescheme (\emph{resp.} a locally Noetherian prescheme). Let $\sh{F}$ and
$\sh{G}$ be two quasi-coherent (\emph{resp.} coherent) $\OO_X$-modules; then
$\sh{F}\otimes_{\OO_X}\sh{G}$ is quasi-coherent (\emph{resp.} coherent) and
of finite type if $\sh{F}$ and $\sh{G}$ are of finite type. If
$\sh{F}$ admits a finite presentation and if $\sh{G}$ is quasi-coherent
(\emph{resp.} coherent), then $\shHom(\sh{F},\sh{G})$ is quasi-coherent
(\emph{resp.} coherent).
\end{envs}

Being a local property, we can suppose that $X$ is affine (resp. Noetherian
affine); further, if $\sh{F}$ is coherent, then we can assume that it is the
cokernel of a homomorphism $\OO_X^m\to\OO_X^n$. The claims pertaining to
quasi-coherent sheaves then follow from \sref{cor}{1.3.12} and \sref{cor}{1.3.9}; the
claims pertaining to coherent sheaves follow from \sref{thm}{1.5.1} and from the fact
that, if $M$ and $N$ are modules of finite type over a Noetherian ring $A$,
$M\otimes_A N$ and $\Hom_A(M,N)$ are $A$-modules of finite type.

\begin{envs}[Definition]{9.1.2}
\label{defn-1.9.1.2}
Let $X$ and $Y$ be two $S$-preschemes, $p$ and
$q$ the projections of $X\times_S Y$, and $\sh{F}$ (resp.$\sh{G}$) a
quasi-coherent $\OO_X$-module (resp. quasi-coherent $\OO_Y$-module). We define the
tensor product of $\sh{F}$ and $\sh{G}$ over $\OO_S$ (\emph{or} over $S$),
denoted by $\sh{F}\otimes_{\OO_S}\sh{G}$ (\emph{or}
$\sh{F}\otimes_S\sh{G}$) to be the tensor product
$p^*(\sh{F})\otimes_{\OO_{X\times_S Y}}q^*(\sh{G})$ over the
prescheme $X\times_S Y$.
\end{envs}

If $X_i$ ($1\leqslant i\leqslant n$) are $S$-preschemes, and $\sh{F}_i$ are quasi-coherent
$\OO_{X_i}$-modules ($1\leqslant i\leqslant n$), then we define similarly the tensor product
$\sh{F}_1\otimes_S\sh{F}_2\otimes_S\cdots\otimes_S\sh{F}_n$ over the
prescheme $Z=X_1\times_S X_2\times_S\cdots\times_S X_n$; it is a
\emph{quasi-coherent} $\OO_Z$-module by virtue of \sref{prop}{9.1.1} and
\pref{env}{5.1.4}; it is \emph{coherent} if the $\sh{F}_i$ are coherent and
$Z$ is \emph{locally Noetherian}, by virtue of \sref{prop}{9.1.1},
\pref{env}{5.3.11}, and \sref{env}{6.1.1}.

Note that if we take $X=Y=S$ then definition \sref{defn}{9.1.2} gives us back the tensor
product of $\OO_S$-modules. Furthermore, as $q^*(\OO_Y)=\OO_{X\times_S Y}$
\pref{env}{4.3.4}, the product $\sh{F}\otimes_S\OO_Y$ is canonically
identified with $p^*(\sh{F})$, and, in the same way,
$\OO_X\otimes_S\sh{G}$ is canonically identified with $q^*(\sh{G})$. In
particular, if we take $Y=S$ and denote by $f$ the structure morphism $X\to Y$,
we have that $\OO_X\otimes_Y\sh{G}=f^*(\sh{G})$: the ordinary tensor
product and the inverse image thus appear as particular cases of the general
tensor product.

Definition \sref{defn}{9.1.2} leads immediately to the fact that, for fixed $X$ and
$Y$, $\sh{F}\otimes_S\sh{G}$ is an \emph{additive covariant bifunctor that
is right exact} in $\sh{F}$ and $\sh{G}$.

\begin{envs}[Proposition]{9.1.3}
\label{prop-1.9.1.3}
Let $S$, $X$, $Y$ be three affine schemes of rings
$A$, $B$, $C$ (respectively), with $B$ and $C$ being $A$-algebras. Let $M$
(\emph{resp.} $N$) be a $B$-module (\emph{resp.} $C$-module), and
$\sh{F}=\widetilde{M}$ (\emph{resp.} $\sh{G}=\widetilde{N}$) the
associated quasi-coherent sheaf; then $\sh{F}\otimes_S\sh{G}$ is
canonically isomorphic to the sheaf associated to the $(B\otimes_A C)$-module
$M\otimes_A N$.
\end{envs}

\oldpage{170}
In fact, by virtue of \sref{prop}{1.6.5}, $\sh{F}\otimes_S\sh{G}$
is canonically isomorphic to the sheaf associated to the $(B\otimes_A C)$-module
\[
  \big(M\otimes_B(B\otimes_A C)\big)\otimes_{B\otimes_A C}\big((B\otimes_A C)\otimes_C N\big)
\]
and by the canonical isomorphisms between tensor
products, this latter module is isomorphic to
\[
  M\otimes_B(B\otimes_A C)\otimes_C N=(M\otimes_B B)\otimes_A(C\otimes_C N)=M\otimes_A N.
\]

\begin{envs}[Proposition]{9.1.4}
\label{prop-1.9.1.4}
Let $f:T\to X$, and $g:T\to Y$ be
two $S$-morphisms, and $\sh{F}$ (\emph{resp.} $\sh{G}$) a quasi-coherent
$\OO_X$-module (\emph{resp.} quasi-coherent $\OO_Y$-module). Then
\[
  (f,g)^*_S(\sh{F}\otimes_S\sh{G})=f^*(\sh{F})\otimes_{\OO_T}g^*(\sh{G}).
\]
\end{envs}

If $p$, $q$ are the projections of $X\times_S Y$, then the formula in fact follows
from the relations $(f,g)^*_S\circ p^*=f^*$ and
$(f,g)^*_S\circ q^*=g^*$ \pref{env}{3.5.5}, and the fact that the inverse
image of a tensor product of algebraic sheaves is the tensor product of their inverse
images \pref{env}{4.3.3}.

\begin{envs}[Corollary]{9.1.5}
\label{cor-1.9.1.5}
Let $f:X\to X'$ and $g:Y\to Y'$ be
$S$-morphisms, and $\sh{F}'$ (\emph{resp.} $\sh{G}'$) a quasi-coherent
$\OO_{X'}$-module (\emph{resp.} quasi-coherent $\OO_{Y'}$-module). Then
\[
  (f,g)^*_S(\sh{F}'\otimes_S\sh{G}')=f^*(\sh{F}')\otimes_S g^*(\sh{G}')
\]
\end{envs}

This follows from \sref{prop}{9.1.4} and the fact that $f\times_S g=(f\circ p, g\circ q)_S$,
where $p$, $q$ are the projections of $X\times_S Y$.

\begin{envs}[Corollary]{9.1.6}
\label{cor-1.9.1.6}
Let $X$, $Y$, $Z$ be three $S$-preschemes, and $\sh{F}$ (\emph{resp.} $\sh{G}$, $\sh{H}$) a
quasi-coherent $\OO_X$-module (\emph{resp.} quasi-coherent $\OO_Y$-module, quasi-coherent
$\OO_Z$-module); then the sheaf $\sh{F}\otimes_S\sh{G}\otimes_S\sh{H}$ is the inverse image
of $(\sh{F}\otimes_S\sh{G})\otimes_S\sh{H}$ by the canonical isomorphism from
$X\times_S Y\times_S Z$ to $(X\times_S Y)\times_S Z$.
\end{envs}

In fact, this isomorphism is given by $(p_1,p_2)_S\times_S p_3$, where $p_1$, $p_2$, $p_3$
are the projections of $X\times_S Y\times_S Z$.

Similarly, the inverse image of $\sh{G}\otimes_S\sh{F}$ by the canonical isomorphism from
$X\times_S Y$ to $Y\times_S X$ is $\sh{F}\otimes_S\sh{G}$.

\begin{envs}[Corollary]{9.1.7}
\label{cor-1.9.1.7}
If $X$ is an $S$-prescheme, then every quasi-coherent $\OO_X$-module $\sh{F}$ is the inverse
image of $\sh{F}\otimes_S\OO_S$ by the canonical isomorphism from $X$ to $X\times_S S$
\sref{prop}{3.3.3}.
\end{envs}

In fact, this isomorphism is $(1_X,\vphi)_S$, where $\vphi$ is the structure morphism
$X\to S$, and the corollary follows from \sref{prop}{9.1.4} and the fact that
$\vphi^*(\OO_S)=\OO_X$.

\begin{env}{9.1.8}
\label{env-1.9.1.8}
Let $X$ be an $S$-prescheme, $\sh{F}$ a quasi-coherent
$\OO_X$-module, and $\vphi:S'\to S$ a morphism; we denote by
$\sh{F}_{(\vphi)}$ or $\sh{F}_{(S')}$ the quasi-coherent sheaf
$\sh{F}\otimes_S\OO_{S'}$ over $X\times_S S'=X_{(\vphi)}=X_{(S')}$; so
$\sh{F}_{(S')}=p^*(\sh{F})$, where $p$ is the projection $X_{(S')}\to X$.
\end{env}

\begin{envs}[Proposition]{9.1.9}
\label{prop-1.9.1.9}
Let $\vphi'':S''\to S'$ be a morphism.
For every quasi-coherent $\OO_X$-module $\sh{F}$ on the $S$-prescheme $X$,
$(\sh{F}_{(\vphi)})_{(\vphi')}$ is the inverse image of
$\sh{F}_{(\vphi\circ\vphi')}$ by the canonical isomorphism
$(X_{(\vphi)})_{(\vphi')}\isoto X_{(\vphi\circ\vphi')}$
\sref{prop}{3.3.9}.
\end{envs}

This follows immediately from the definitions and from \sref{prop}{3.3.9}, and is
written
\[
  (\sh{F}\otimes_S\OO_{S'})\otimes_{S'}\OO_{S''}=\sh{F}\otimes_S\OO_{S''}.
  \tag{9.1.9.1}
\]

\begin{envs}[Proposition]{9.1.10}
\label{prop-1.9.1.10}
Let $Y$ be an $S$-prescheme, and $f:X\to Y$ an $S$-morphism.
For every quasi-coherent $\OO_Y$-module and every morphism
$S'\to S$, we have that
$(f_{(S')})^*(\sh{G}_{(S')})=(f^*(\sh{G}))_{(S')}$.
\end{envs}

This follows immediately from the commutativity of the diagram
\oldpage{171}
\[
  \xymatrix{
    X_{(S')}\ar[r]^{f_{(S')}}\ar[d] & Y_{(S')}\ar[d]\\
    X\ar[r]^f & Y.
  }
\]

\begin{envs}[Corollary]{9.1.11}
\label{cor-1.9.1.11}
Let $X$ and $Y$ be $S$-preschemes, and
$\sh{F}$ (\emph{resp.} $\sh{G}$) a quasi-coherent $\OO_X$-module
(\emph{resp.} quasi-coherent $\OO_Y$-module). Then the inverse image of the sheaf
$(\sh{F}_{(S')})\otimes_{(S')}(\sh{G}_{(S')})$ by the canonical isomorphism
$(X\times_S Y)_{(S')}\isoto(X_{(S')})\times_{S'}(Y_{(S')})$
\sref{cor}{3.3.10} is equal to $(\sh{F}\otimes_S\sh{G})_{(S')}$.
\end{envs}

If $p$, $q$ are the projections of $X\times_S Y$, then the isomorphism in question
is nothing but $(p_{(S')}, q_{(S')})_{S'}$; the corollary follows from
Propositions \sref{prop}{9.1.4} and \sref{prop}{9.1.10}.

\begin{envs}[Proposition]{9.1.12}
\label{prop-1.9.1.12}
With the notation from \sref{defn}{9.1.2}, let $z$ be
a point of $X\times_S Y$, $x=p(z)$, and $y=q(z)$; the stalk
$(\sh{F}\otimes_S\sh{G})_z$ is isomorphic to
$(\sh{F}_x\otimes_{\OO_x}\OO_z)\otimes_{\OO_z}(\sh{G}_y\otimes_{\OO_y}\OO_z)
=\sh{F}_x\otimes_{\OO_x}\OO_z\otimes_{\OO_y}\otimes\sh{G}_y$.
\end{envs}

As we can reduce ourselves to the affine case, the proposition follows from
equation (1.6.5.1).

\begin{envs}[Corollary]{9.1.13}
\label{cor-1.9.1.13}
If $\sh{F}$ and $\sh{G}$ are of finite type, then we have that
\[
  \Supp(\sh{F}\otimes_S\sh{G})=p^{-1}(\Supp(\sh{F}))\cap q^{-1}(\Supp(\sh{G})).
\]
\end{envs}

Since $p^*(\sh{F})$ and $q^*(\sh{G})$ are both of finite type over
$\OO_{X\times_S Y}$, we are reduced, by \sref{prop}{9.1.12} and \pref{env}{1.7.5}, to
the case where $\sh{G}=\OO_Y$, that is, it remains to prove the following
equation:
\[
  \Supp(p^{-1}(\sh{F}))=p^{-1}(\Supp(\sh{F})).
  \tag{9.1.13.1}
\]

The same reasoning as in \pref{env}{1.7.5} leads us to prove that, for all
$z\in X\times_S Y$, we have $\OO_z/\mathfrak{m}_x\OO_z\neq0$ (with $x=p(z)$),
which follows from the fact that the homomorphism $\OO_x\to\OO_z$ is \emph{local},
by hypothesis.

We leave it to the reader to extend the results in this section to the more
general case of arbitrarily (but finitely) many factors, instead of just two.

\subsection{Direct image of a quasi-coherent sheaf}
\label{1-schemes-9.2}

\begin{envs}[Proposition]{9.2.1}
\label{prop-1.9.2.1}
Let $f:X\to Y$ be a morphism of
preschemes. We suppose that there exists a cover $(Y_\alpha)$ of $Y$ by affine
opens having the following property: every $f^{-1}(Y_\alpha)$ admits a
\emph{finite} cover $(X_{\alpha i})$ by affine opens contained in
$f^{-1}(Y_\alpha)$ such that every intersection $X_{\alpha i}\cap X_{\alpha j}$
is itself a \emph{finite} union of affine opens. With these hypotheses, for
every quasi-coherent $\OO_X$-module $\sh{F}$, $f_*(\sh{F})$ is a
quasi-coherent $\OO_Y$-module.
\end{envs}

Since this is a local condition on $Y$, we can assume that $Y$ is equal to one
of the $Y_\alpha$, and thus omit the indices $\alpha$.

\begin{enumerate}[label=(\alph*)]
  \item First, suppose that the $X_i\cap X_j$
        are themselves \emph{affine} opens. We set $\sh{F}_i=\sh{F}|X_i$ and
        $\sh{F}_{ij}=\sh{F}|(X_i\cap X_j)$, and let $\sh{F}'_i$ and
        $\sh{F}'_{ij}$ be the images of $\sh{F}_i$ and $\sh{F}_{ij}$
        (respectively) by the restriction of $f$ to $X_i$ and $X_i\cap X_j$
        (respectively); we know that the $\sh{F}'_i$ and $\sh{F}'_{ij}$ are
        quasi-coherent \sref{prop}{1.6.3}. Set $\sh{G}=\bigoplus_i\sh{F}'_i$ and
        $\sh{H}=\bigoplus_{i,j}\sh{F}'_{ij}$; $\sh{G}$ and $\sh{H}$ are
        quasi-coherent $\OO_Y$-modules; we will define a homomorphism
        $u:\sh{G}\to\sh{H}$ such that $f_*(\sh{F})$ is the
        \emph{kernel} of $u$; it will follow from this that $f_*(\sh{F})$ is
        quasi-coherent \sref{cor}{1.3.9}. It suffices to define $u$ as
\oldpage{172}
        a homomorphism of presheaves; taking into account the definitions of $\sh{G}$
        and $\sh{H}$, it thus suffices, for every open subset $W\subset Y$, to define a
        homomorphism
        \[
          u_W:\bigoplus_i\Gamma(f^{-1}(W)\cap X_i,\sh{F})
          \longrightarrow\bigoplus_{i,j}\Gamma(f^{-1}(W)\cap X_i\cap X_j,\sh{F})
        \]
        in such a way that it satisfies the usual compatibility conditions when $W$
        varies. If, for every section $s_i\in\Gamma(f^{-1}(W)\cap X_i,\sh{F})$, we
        denote by $s_{i|j}$ the restriction to $f^{-1}(W)\cap X_i\cap X_j$, then we set
        \[
          u_W\big((s_i)\big)=(s_{i|j}-s_{j|i})
        \]
        and the compatibility conditions are clearly satisfied. To prove that the kernel
        $\sh{R}$ of $u$ is $f_*(\sh{F})$, we define a homomorphism from $f_*(\sh{F})$ to
        $\sh{R}$ by sending each section $s\in\Gamma(f^{-1}(W),\sh{F})$ to the family
        $(s_i)$, where $s_i$ is the restriction of $s$ to $f^{-1}(W)\cap X_i$; the
        axioms (F1) and (F2) of sheaves (G, II, 1.1) tell us that this homomorphism is
        \emph{bijective}, which finishes the proof in this case.
  \item In the general case, the same reasoning applies once we have established that
        the $\sh{F}_{ij}$ are quasi-coherent. But, by hypothesis, $X_i\cap X_j$ is a
        finite union of affine opens $X_{ijk}$; and since the $X_{ijk}$ are affine opens
        \emph{in a scheme}, the intersection of any two of them is again an affine open
        \sref{env}{5.5.6}. We are thus led to the first case, and so we have proved
        \sref{prop}{9.2.1}.
\end{enumerate}

\begin{envs}[Corollary]{9.2.2}
\label{cor-1.9.2.2}
The conclusion of \sref{prop}{9.2.1} holds true in each of the following cases:
\begin{enumerate}[label=\rm{(\alph*)}]
  \item $f$ is separated and quasi-compact.
  \item $f$ is separated and of finite type.
  \item $f$ is quasi-compact and the underlying space of $X$ is locally Noetherian.
\end{enumerate}
\end{envs}

In case \emph{(a)}, the $X_{\alpha i}\cap X_{\alpha j}$ are affine \sref{env}{5.5.6}.
Case \emph{(b)} is a particular case of \emph{(a)} \sref{env}{6.6.3}. Finally, in case
\emph{(c)}, we can reduce to the case where $Y$ is affine and the underlying
space of $X$ is Noetherian; then $X$ admits a finite cover of affine opens
$(X_i)$, and the $X_i\cap X_j$, being quasi-compact, are finite unions of affine
opens \sref{prop}{2.1.3}.

\subsection{Extension of sections of quasi-coherent sheaves}
\label{1-schemes-9.3}

\begin{envs}[Theorem]{9.3.1}
\label{thm-1.9.3.1}
Let $X$ be a prescheme whose underlying space is Noetherian, or a scheme whose underlying
space is quasi-compact. Let $\sh{L}$ be an invertible $\OO_X$-module \pref{env}{5.4.1}, $f$ a
section of $\sh{L}$ over $X$, $X_f$ the open set of $x\in X$ such that $f(x)\neq0$
\pref{env}{5.5.1}, and $\sh{F}$ a quasi-coherent $\OO_X$-module.
\begin{enumerate}[label=\rm{(\roman*)}]
  \item If $s\in\Gamma(X,\sh{F})$ is such that $s|X_f=0$, then there exists an integer $n>0$
        such that $s\otimes f^{\otimes n}=0$.
  \item For every section $s\in\Gamma(X_f,\sh{F})$, there exists an integer $n>0$ such that
        $s\otimes f^{\otimes n}$ extends to a section of $\sh{F}\otimes\sh{L}^{\otimes n}$
        over $X$.
\end{enumerate}
\end{envs}

\begin{enumerate}[label=(\roman*)]
  \item Since the underlying space of $X$ is quasi-compact, and thus the union of
        finitely-many affine opens $U_i$ with $\sh{L}|U_i$ is isomorphic to
        $\OO_X|U_i$, we can reduce to the case where $X$ is affine and $\sh{L}=\OO_X$.
        In this case, $f$ is identified with an element of $A(X)$, and we have that
        $X_f=D(f)$; $s$ is identified with an element of an $A(X)$-module $M$, and
        $s|X_f$ to the corresponding element of $M_f$, and the result is then trivial,
        recalling the definition of a module of fractions.
\oldpage{173}
  \item Again, $X$ is a finite union of affine opens $U_i$ ($1\leqslant i\leqslant r$)
        such that $\sh{L}|U_i\cong\OO_X|U_i$, and for every $i$,
        $(s\otimes f^{\otimes n})|(U_i\cap X_f)$ is identified (by the aforementioned
        isomorphism) with $(f|(U_i\cap X_f))^n(s|(U_i\cap X_f))$. We then know
        \sref{thm}{1.4.1} that there exists an integer $n>0$ such that, for all
        $i$, $(s\otimes f^{\otimes n})|(U_i\cap X_f)$ extends to a section $s_i$ of
        $\sh{F}\otimes\sh{L}^{\otimes n}$ over $U_i$. Let $s_{i|j}$ be the restriction
        of $s_i$ to $U_i\cap U_j$; by definition we have that $s_{i|j}-s_{j|i}=0$ on
        $X_f\cap U_i\cap U_j$. But, if $X$ is a Noetherian space, then $U_i\cap U_j$ is
        quasi-compact; if $X$ is a scheme, then $U_i\cap U_j$ is an affine open
        \sref{env}{5.5.6}, and so again quasi-compact. By virtue of (i), there thus
        exists an integer $m$ (independent of $i$ and $j$) such that
        $(s_{i|j}-s_{j|i})\otimes f^{\otimes m}=0$. It immediately follows that there
        exists a section $s'$ of $\sh{F}\otimes\sh{L}^{\otimes(n+m)}$ over $X$,
        restricting to $s_i\otimes f^{\otimes m}$ over each $U_i$, and restricting to
        $s\otimes f^{\otimes(n+m)}$ over $X_f$.
\end{enumerate}

The following corollaries give an interpretation of Theorem \sref{thm}{9.3.1} in a more
algebraic language:
\begin{envs}[Corollary]{9.3.2}
\label{cor-1.9.3.2}
With the hypotheses of \sref{thm}{9.3.1}, consider the graded ring $A_*=\Gamma_*(\sh{L})$
and the graded $A_*$-module $M_*=\Gamma_*(\sh{L},\sh{F})$ \pref{env}{5.4.6}. If $f\in A_n$,
where $n\in\bb{Z}$, then there is a canonical isomorphism
$\Gamma(X_f,\sh{F})\isoto((M_*)_f)_0$ (\emph{the subgroup of the module of
fractions $(M_*)_f$ consisting of elements of degree $0$}).
\end{envs}

\begin{envs}[Corollary]{9.3.3}
\label{cor-1.9.3.3}
Suppose that the hypotheses of \sref{thm}{9.3.1} are satisfied, and suppose further that
$\sh{L}=\OO_X$. Then, setting $A=\Gamma(X,\OO_X)$ and $M=\Gamma(X,\sh{F})$, the $A_f$-module
$\Gamma(X_f,\sh{F})$ is canonically isomorphic to $M_f$.
\end{envs}

\begin{envs}[Proposition]{9.3.4}
\label{prop-1.9.3.4}
Let $X$ be a Noetherian prescheme, $\sh{F}$ a coherent $\OO_X$-module, and $\sh{J}$ a
coherent sheaf of ideals in $\OO_X$, such that the support of $\sh{F}$ is contained in that
of $\OO_X|\sh{J}$. Then there exists a whole number $n>0$ such that $\sh{J}^n\sh{F}=0$.
\end{envs}

Since $X$ is a union of finitely-many affine opens whose rings are Noetherian, we can suppose
that $X$ is affine of Noetherian ring $A$; then $\sh{F}=\widetilde{M}$, where
$M=\Gamma(X,\sh{F})$ is an $A$-module of finite type, and $\sh{J}=\widetilde{\mathfrak{J}}$,
where $\mathfrak{J}=\Gamma(X,\sh{J})$ is an ideal of $A$ (\sref{thm}{1.4.1} and
\sref{thm}{1.5.1}). Since $A$ is Noetherian, $\mathfrak{J}$ admits a finite system of
generators $f_i$ ($1\leqslant i\leqslant m$). By hypothesis, every section of $\sh{F}$ over
$X$ is zero on each of the $D(f_i)$; if $s_j$ ($1\leqslant j\leqslant q$) are sections of
$\sh{F}$ generating $M$, then there exists a whole number $h$, independent of $i$ and $j$,
such that $f_i^h s_j=0$ \sref{thm}{1.4.1}, whence $f_i^h s=0$ for all $s\in M$. We thus
conclude that if $n=mh$ then $\mathfrak{J}^n M=0$, and so the corresponding $\OO_X$-module
$\sh{J}^n\sh{F}=\widetilde{\mathfrak{J}^n M}$ \sref{env}{1.3.13} is zero.

\begin{envs}[Corollary]{9.3.5}
\label{cor-1.9.3.5}
With the hypotheses of \sref{prop}{9.3.4}, there exists a closed subprescheme $Y$ of $X$,
whose underlying space is the support of $\OO_X/\sh{J}$, such that, if $j:Y\to X$ is the
canonical injection, then $\sh{F}=j_*(j^*(\sh{F}))$.
\end{envs}

First of all, note that the supports of $\OO_X/\sh{J}$ and $\OO_X/\sh{J}^n$ are the same,
since, if $\sh{J}_x=\OO_x$, then $\sh{J}_x^n=\OO_x$, and we also have that
$\sh{J}_x^n\subset\sh{J}_x$ for all $x\in X$. We can, thanks to \sref{prop}{9.3.4}, thus
suppose that $\sh{J}\sh{F}=0$; we can then take $Y$ to be the closed subprescheme of $X$
defined by $\sh{J}$, and since $\sh{F}$ is then an $(\OO_X/\sh{J})$-module, the conclusion
follows immediately.

\subsection{Extension of quasi-coherent sheaves}
\label{1-schemes-9.4}        

\begin{env}{9.4.1}
\label{env-1.9.4.1}
Let
\oldpage{174}
$X$ be a topological space, $\sh{F}$ a sheaf of sets (resp. of groups, of rings) on $X$, $U$
an open subset of $X$, $\psi:U\to X$ the canonical injection, and $\sh{G}$ a subsheaf of
$\sh{F}|U=\psi^*(\sh{F})$. Since $\psi_*$ is left exact, $\psi_*(\sh{G})$ is a subsheaf of
$\psi_*(\psi^*(\sh{F}))$; if we denote by $\rho$ the canonical homomorphism
$\sh{F}\to\psi_*(\psi^*(\sh{F}))$ \pref{env}{3.5.3}, then we denote by $\overline{\sh{G}}$
the subsheaf $\rho^{-1}(\psi_*(\sh{G}))$ of $\sh{F}$. It follows immediately from the
definitions that, for every open subset $V$ of $X$, $\Gamma(V,\overline{\sh{G}})$ consists of
sections $s\in\Gamma(V,\sh{F})$ whose restriction to $V\cap U$ is a section of $\sh{G}$ over
$V\cap U$. We thus have that $\overline{\sh{G}}|U=\psi^*(\overline{\sh{G}})=\sh{G}$, and that
$\overline{\sh{G}}$ is the \emph{biggest} subsheaf of $\sh{F}$ that restricts to $\sh{G}$
over $U$; we say that $\overline{\sh{G}}$ is the \emph{canonical extension} of the subsheaf
$\sh{G}$ of $\sh{F}|U$ to a subsheaf of $\sh{F}$.
\end{env}

\begin{envs}[Proposition]{9.4.2}
\label{prop-1.9.4.2}
Let $X$ be a prescheme, $U$ an open subset of $X$ such that the canonical injection
$j:U\to X$ is a quasi-compact morphism \emph{(which will be the case for \emph{all} $U$ if
the underlying space of $X$ is \emph{locally Noetherian}
{\normalfont(\sref{env}{6.6.4}, (i))})}. Then:
\begin{enumerate}[label=\rm{(\roman*)}]
  \item For every quasi-coherent $(\OO_X|U)$-module $\sh{G}$, $j_*(\sh{G})$
        is a quasi-coherent $\OO_X$-module, and $j_*(\sh{G})|U=j^*(j_*(\sh{G}))=\sh{G}$.
  \item For every quasi-coherent $\OO_X$-module $\sh{F}$ and every quasi-coherent
        $(\OO_X|U)$-submodule $\sh{G}$, the canonical extension
        $\overline{\sh{G}}$ of $\sh{G}$ \sref{env}{9.4.1} is a
        quasi-coherent $\OO_X$-submodule of $\sh{F}$.
\end{enumerate}
\end{envs}

If $j=(\psi,\theta)$ ($\psi$ being the injection $U\to X$ of underlying spaces), then by
definition we have that $j_*(\sh{G})=\psi_*(\sh{G})$ for every $(\OO_X|U)$-module $\sh{G}$,
and, further, that $j^*(\sh{H})=\psi^*(\sh{H})=\sh{H}|U$ for every $\OO_X$-module $\sh{H}$,
by definition of the prescheme induced over an open subset. So (i) is thus a particular case
of (\sref{cor}{9.2.2}, (a)); for the same reason, $j_*(j^*(\sh{F}))$ is quasi-coherent, and
since $\overline{\sh{G}}$ is the inverse image of $j_*(\sh{G})$ by the homomorphism
$\rho:\sh{F}\to j_*(j^*(\sh{F}))$, (ii) follows from \sref{env}{4.1.1}.
 
Note that the hypothesis that the morphism $j:U\to X$ is quasi-compact
holds whenever the open subset $U$ is \emph{quasi-compact} and $X$ is a
\emph{scheme}: indeed, $U$ is then a union of finitely-many affine opens $U_i$,
and for every affine open $V$ of $X$, $V\cap U_i$ is an affine open \sref{env}{5.5.6}, and
thus quasi-compact.
 
\begin{envs}[Corollary]{9.4.3}
\label{cor-1.9.4.3}
Let $X$ be a prescheme, $U$ a quasi-compact open subset of $X$ such that the injection
morphism $j:U\to X$ is quasi-compact. Suppose as well that every quasi-coherent
$\OO_X$-module is the inductive limit of its quasi-coherent $\OO_X$-submodules of finite type
\emph{(which will be the case if $X$ is an \emph{affine scheme})}. Then let $\sh{F}$ be a
quasi-coherent $\OO_X$-module, and $\sh{G}$ a quasi-coherent $(\OO_X|U)$-submodule \emph{of
finite type} of $\sh{F}|U$. Then there exists a quasi-coherent $\OO_X$-submodule $\sh{G}'$ of
$\sh{F}$ \emph{of finite type} such that $\sh{G}'|U=\sh{G}$.
\end{envs}
 
Indeed, we have $\sh{G}=\overline{\sh{G}}|U$, and $\overline{\sh{G}}$ is quasi-coherent, from
\sref{prop}{9.4.2}, and so the inductive limit of its quasi-coherent $\OO_X$-submodules
$\sh{H}_\lambda$ of finite type. It follows that $\sh{G}$ is the inductive limit of the
$\sh{H}_\lambda|U$, and thus equal to one of the $\sh{H}_\lambda|U$ since it is of finite
type \pref{env}{5.2.3}.
 
\begin{env}[Remark]{9.4.4}
\label{rmk-1.9.4.4}
Suppose that for \emph{every} affine open $U\subset X$, the injection morphism $U\to X$ is
quasi-compact. Then, if the conclusion of \sref{cor}{9.4.3} holds for every affine open $U$
and every quasi-coherent $(\OO_X|U)$-submodule $\sh{G}$ of $\sh{F}|U$ of finite type, it
follows
\oldpage{175}
that $\sh{F}$ is the inductive limit of its quasi-coherent $\OO_X$-submodules of finite type.
Indeed, for every affine open $U\subset X$, we have that $\sh{F}|U=\widetilde{M}$, where $M$
is an $A(U)$-module, and since the latter is the inductive limit of its quasi-coherent
submodules of finite type, $\sh{F}|U$ is the inductive limit of its $(\OO_X|U)$-submodules of
finite type \sref{cor}{1.3.9}. But, by hypothesis, each of these submodules is induced on $U$
by a quasi-coherent $\OO_X$-submodule $\sh{G}_{\lambda,U}$ of $\sh{F}$ of finite type. The
finite sums of the $\sh{G}_{\lambda,U}$ are again quasi-coherent $\OO_X$-modules of finite
type, because the property is local, and the case where $X$ is affine was covered in
\sref{env}{1.3.10}; it is clear then that $\sh{F}$ is the inductive limit of these finite
sums, whence our claim.
\end{env}
 
\begin{envs}[Corollary]{9.4.5}
\label{cor-1.9.4.5}
Under the hypotheses of \sref{cor}{9.4.3}, for every quasi-coherent $(\OO_X|U)$-module
$\sh{G}$ of finite type, there exists a quasi-coherent $\OO_X$-module $\sh{G}'$ of finite
type such that $\sh{G}'|U=\sh{G}$.
\end{envs}

Since $\sh{F}=j_*(\sh{G})$ is quasi-coherent \sref{prop}{9.4.2} and $\sh{F}|U=\sh{G}$, it
suffices to apply \sref{cor}{9.4.3} to $\sh{F}$.

\begin{envs}[Lemma]{9.4.6}
\label{lem-1.9.4.6}
Let $X$ be a prescheme, $L$ a well-ordered set, $(V_\lambda)_{\lambda\in L}$ a cover of $X$
by affine opens, and $U$ an open subset of $X$; for all $\lambda\in L$, we set
$W_\lambda=\bigcup_{\mu<\lambda}V_\mu$. Suppose that: (1) for every $\lambda\in L$,
$V_\lambda\cap W_\lambda$ is quasi-compact; (2) the immersion morphism $U\to X$ is
quasi-compact. Then, for every quasi-coherent $\OO_X$-module $\sh{F}$ and every
quasi-coherent $(\OO_X|U)$-submodule $\sh{G}$ of $\sh{F}|U$ \emph{of finite type}, there
exists a quasi-coherent $\OO_X$-submodule $\sh{G}'$ of $\sh{F}$ \emph{of finite type} such
that $\sh{G}'|U=\sh{G}$.
\end{envs}

Let $U_\lambda=U\cup W_\lambda$; we will define a family $(\sh{G}'_\lambda)$ by induction,
where $\sh{G}'_\lambda$ is a quasi-coherent $(\OO_X|U_\lambda)$-submodule of
$\sh{F}|U_\lambda$ of finite type, such that $\sh{G}'_\lambda|U_\mu=\sh{G}'_\mu$ for
$\mu<\lambda$ and $\sh{G}'_\lambda|U=\sh{G}$. The unique $\OO_X$-submodule $\sh{G}'$ of
$\sh{F}$ such that $\sh{G}'|U_\lambda=\sh{G}'$ for all $\lambda\in L$ \pref{env}{3.3.1} gives
us what we want. So suppose that the $\sh{G}'_\mu$ are defined and have the preceding
properties for $\mu<\lambda$; if $\lambda$ does not have a predecessor then we take for
$\sh{G}'_\lambda$ the unique $(\OO_X|U_\lambda)$-submodule of $\sh{F}|U_\lambda$ such that
$\sh{G}'_\lambda|U_\mu=\sh{G}'_\mu$ for all $\mu<\lambda$, which is allowed since the $U_\mu$
with $\mu<\lambda$ then form a cover of $U_\lambda$. If, conversely, $\lambda=\mu+1$, then
$U_\lambda=U_\mu\cup V_\mu$, and it suffices to define a quasi-coherent
$(\OO_X|V_\mu)$-submodule $\sh{G}''_\mu$ of $\sh{F}|V_\mu$ of finite type such that
\[
  \sh{G}''_\mu|(U_\mu\cap V_\mu)=\sh{G}'_\mu|(U_\mu\cap V_\mu);
\]
and then to take for $\sh{G}'_\lambda$ the $(\OO_X|U_\lambda)$-submodule of
$\sh{F}|U_\lambda$ such that $\sh{G}'_\lambda|U_\mu=\sh{G}'_\mu$ and
$\sh{G}'_\lambda|V_\mu=\sh{G}''_\mu$ \pref{env}{3.3.1}. But, since $V_\mu$ is affine, the
existence of $\sh{G}''_\mu$ is guaranteed by \sref{cor}{9.4.3} as soon as we show that
$U_\mu\cap V_\mu$ is quasi-compact; but $U_\mu\cap V_\mu$ is the union of $U\cap V_\mu$ and
$W_\mu\cap V_\mu$, which are both quasi-compact by virtue of the hypothesis.

\begin{envs}[Theorem]{9.4.7}
\label{thm-1.9.4.7}
Let $X$ be a prescheme, and $U$ an open set of $X$. Suppose that one of the following
conditions is verified:
\begin{enumerate}[label=\rm{(\alph*)}]
  \item the underlying space of $X$ is locally Noetherian;
  \item $X$ is a quasi-compact scheme and $U$ is a quasi-compact open.
\end{enumerate}
Then, for every quasi-coherent $\OO_X$-module $\sh{F}$ and every quasi-coherent
$(\OO_X|U)$-submodule $\sh{G}$ of $\sh{F}|U$ \emph{of finite type}, there exists a
quasi-coherent $\OO_X$-submodule $\sh{G}'$ of $\sh{F}$ \emph{of finite type} such that
$\sh{G}'|U=\sh{G}$.
\end{envs}

Let
\oldpage{176}
$(V_\lambda)_{\lambda\in L}$ be a cover of $X$ by affine opens, with $L$ assumed to be finite
in case (b); since $L$ is equipped with the structure of a well-ordered set, it suffices to
check that the conditions of \sref{lem}{9.4.6} are satisfied. It is clear in the case of (a),
as the spaces $V_\lambda$ are Noetherian. For case (b), the $V_\lambda\cap\lambda_\mu$ are
affine \sref{env}{5.5.6}, and thus quasi-compact, and since $L$ is finite,
$V_\lambda\cap W_\lambda$ is quasi-compact. Whence the theorem.

\begin{envs}[Corollary]{9.4.8}
\label{cor-1.9.4.8}
Under the hypotheses of \sref{thm}{9.4.7}, for every quasi-coherent $(\OO_X|U)$-module
$\sh{G}$ of finite type, there exists a quasi-coherent $\OO_X$-module $\sh{G}'$ of finite
type such that $\sh{G}'|U=\sh{G}$.
\end{envs}

It suffices to apply \sref{thm}{9.4.7} to $\sh{F}=j_*(\sh{G})$, which is quasi-coherent
\sref{prop}{9.4.2} and such that $\sh{F}|U=\sh{G}$.

\begin{envs}[Corollary]{9.4.9}
\label{cor-1.9.4.9}
Let $X$ be a prescheme whose underlying space is locally Noetherian, or a quasi-compact
scheme. Then every quasi-coherent $\OO_X$-module is the inductive limit of its quasi-coherent
$\OO_X$-submodules of finite type.
\end{envs}

This follows from \sref{thm}{9.4.7} and Remark \sref{rmk}{9.4.4}.

\begin{envs}[Corollary]{9.4.10}
\label{cor-1.9.4.10}
Under the hypotheses of \sref{cor}{9.4.9}, if a quasi-coherent $\OO_X$-module $\sh{F}$ is
such that every quasi-coherent $\OO_X$-submodule of finite type of $\sh{F}$ is generated by
its sections over $X$, then $\sh{F}$ is generated by its sections over $X$.
\end{envs}

Indeed, let $U$ be an affine open neighbourhood of a point $x\in X$, and let $s$ be a
section of $\sh{F}$ over $U$; the $\OO_X$-submodule $\sh{G}$ of $\sh{F}|U$ generated by $s$
is quasi-coherent and of finite type, so there exists a quasi-coherent $\OO_X$-submodule
$\sh{G}'$ of $\sh{F}$ of finite type such that $\sh{G}'|U=\sh{G}$ \sref{thm}{9.4.7}. By
hypothesis, there is thus a finite number of sections $t_i$ of $\sh{G}'$ over $X$ and of
sections $a_i$ of $\OO_X$ over a neighbourhood $V\subset U$ of $x$ such that
$s|V=\sum_i a_i(t_i|V)$, which proves the corollary.

\subsection{Closed image of a prescheme; closure of a subprescheme}
\label{1-schemes-9.5}        

\begin{envs}[Proposition]{9.5.1}
\label{prop-1.9.5.1}
Let $f:X\to Y$ be a morphism of preschemes such that $f_*(\OO_X)$ is a quasi-coherent
$\OO_Y$-module (which will be the case if $f$ is quasi-compact and if in addition $f$ is
either separated or $X$ is locally Noetherian \sref{cor}{9.2.2}). Then there exists a smaller
subprescheme $Y'$ of $Y$ such that $f$ factors through the canonical injection $j:Y'\to Y$
(\emph{or, equivalently \sref{env}{4.4.1}, such that the subprescheme $f^{-1}(Y')$ of $X$ is
\emph{identical} to $X$}).
\end{envs}

More precisely:
\begin{envs}[Corollary]{9.5.2}
\label{cor-1.9.5.2}
Under the conditions of \sref{prop}{9.5.1}, let $f=(\psi,\theta)$, and let $\sh{J}$ be the
(quasi-coherent) kernel of the homomorphism $\theta:\OO_Y\to f_*(\OO_X)$. Then the closed
subprescheme $Y'$ of $Y$ defined by $\sh{J}$ satisfies the conditions of \sref{prop}{9.5.1}.
\end{envs}

Since the functor $\psi^*$ is exact, the canonical factorization
$\theta:\OO_Y\to\OO_Y/\sh{J}\xrightarrow{\theta'}\psi_*(\OO_X)$ gives (\textbf{0},~3.5.4.3)
a factorization
$\theta^\sharp:\psi^*(\OO_Y)\to\psi^*(\OO_Y)/\psi^*(\sh{J})
\xrightarrow{{\theta'}^\sharp}\OO_X$; since $\theta_x^\sharp$ is a local homomorphism for
every $x\in X$, the same is true of ${\theta_x'}^\sharp$; if we denote by $\psi_0$ the
continuous map $\psi$ considered as a map from $X$ to $X'$, and by $\theta_0$ the restriction
$\theta'|X':(\OO_Y/\sh{J})|X'\to\psi_*(\OO_X)|X'=(\psi_0)_*(\OO_X)$, we see that
$f_0=(\psi_0,\theta_0)$ is a morphism of preschemes $X\to X'$ \sref{defn}{2.2.1} such that
$f=j\circ f_0$. Now, if $X''$ is
\oldpage{177}
a second closed subprescheme of $Y$, defined by a quasi-coherent sheaf of ideals $\sh{J}'$ of
$\OO_Y$, such that $f$ factors through the injection $j':X''\to Y$, then we should
immediately have that $\psi(X)\subset X''$, and so $X'\subset X''$, since $X''$ is closed.
Furthermore, for all $y\in X''$, $\theta$ should factorize as
$\OO_y\to\OO_y/\sh{J}'_y\to(\psi_*(\OO_X))_y$, which by definition leads to
$\sh{J}'_y\subset\sh{J}_y$, and thus $X'$ is a closed subprescheme of $X''$
\sref{env}{4.1.10}.

\begin{envs}[Definition]{9.5.3}
\label{defn-1.9.5.3}
Whenever there exists a smaller subprescheme $Y'$ of $Y$ such that $f$ factors through the
canonical injection $j:Y'\to Y$, we
say that $Y'$ is the \emph{closed image} prescheme of $X$ by the morphism $f$.
\end{envs}

\begin{envs}[Proposition]{9.5.4}
\label{prop-1.9.5.4}
If $f_*(\OO_X)$ is a quasi-coherent $\OO_Y$-module, then the underlying space of
the closed image of $X$ by $f$ is the closure $\overline{f(X)}$ in $Y$.
\end{envs}

\subsection{Quasi-coherent sheaves of algebras; change of structure sheaf}
\label{1-schemes-9.6}        

\section{Formal schemes}
\label{1-schemes-10}



\clearpage

%%%%%%%%%%%%%%%
%% EGA I bib %%
%%%%%%%%%%%%%%%
\input{ega.i.bib}

\end{document}

