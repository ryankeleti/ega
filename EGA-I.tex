\ProvidesPackage{preamble}

\usepackage[utf8]{inputenc}
\usepackage[T1]{fontenc}
\usepackage{microtype}
\usepackage[left=0.75in,right=0.75in,top=0.75in,bottom=0.75in]{geometry}
\usepackage[all]{xy}
\usepackage{enumitem}
\usepackage{color}
\usepackage{soul}
\usepackage{fancyhdr}
\usepackage{mathtools}
\usepackage{amssymb}
\usepackage{amsthm}
\usepackage[charter,
            greekfamily=didot,
            uppercase=upright,
            greeklowercase=upright]{mathdesign}
\usepackage[compact]{titlesec}
\usepackage[colorlinks=true,hyperindex,citecolor=blue,linkcolor=magenta]{hyperref}
\usepackage{bookmark}
\usepackage[asterism]{sectionbreak}


%%%%%%%%%%%%%%
% formatting %
%%%%%%%%%%%%%%

\allowdisplaybreaks[1]
\binoppenalty=9999
\relpenalty=9999
\setitemize{nosep}

% for Chapter 0, Chapter I, etc.
% credit for ZeroRoman https://tex.stackexchange.com/questions/211414/
\newcommand{\ZeroRoman}[1]{\ifcase\value{#1}\relax 0\else\Roman{#1}\fi}
\renewcommand{\thechapter}{\ZeroRoman{chapter}}

%%%%%%%%%%%%%%%%%
% math commands %
%%%%%%%%%%%%%%%%%

% for easy changes to style
\newcommand{\sh}{\mathscr}         % sheaf font
\newcommand{\bb}{\mathbf}          % bold font
\newcommand{\cat}{\mathsf}         % category font
%
\newcommand{\rad}{\mathfrak{r}}    % radical
\newcommand{\nilrad}{\mathfrak{R}} % nilradical
\newcommand{\emp}{\varnothing}     % empty set
\newcommand{\vphi}{\phi}           % font switches \phi and \varphi, change if needed
\newcommand{\HH}{\mathrm{H}}       % cohomology
\newcommand{\dual}[1]{{#1}^\vee}   % dual
\renewcommand{\k}{\bb{k}}          % residue field
\newcommand{\K}{\cat{K}}           % category
\newcommand{\OO}{\sh{O}}           % structure sheaf
\newcommand{\F}{\sh{F}}            % sheaf F
\newcommand{\G}{\sh{G}}            % sheaf G

% operators
%\newcommand*{\sheafHom}{\mathscr{H}\text{\normalfont\kern -3pt {\calligra\large om}}\,}
\def\shHom{\sh{H}\textit{om}} % sheaf Hom
\def\Hom{{\mathop{\mathrm{Hom}}\nolimits}}
\def\Supp{{\mathop{\mathrm{Supp}}\nolimits}}
\def\img{{\mathop{\mathrm{im}}\nolimits}}
\def\Spec{{\mathop{\mathrm{Spec}}\nolimits}}

% if unsure of a translation
\newcommand{\unsure}[2][]{\hl{#2}\marginpar{#1}}
\newcommand{\completelyunsure}{\unsure{[\ldots]}}

% use to mark where original page starts
\newcommand{\oldpage}[1]{\marginpar{\textbf{#1}}\ignorespaces}

% special ref
\newcommand{\sref}[2]{\hyperref[#1-\arabic{chapter}.#2]{\normalfont{(#2)}}}

% ref prelim
\newcommand{\pref}[2]{\hyperref[#1-0.#2]{\normalfont{(\textbf{0}, #2)}}}

%% ref out of chapter
%\newcommand{\cref}[4]{\hyperref[#1-#2.#3]{\normalfont{(\textbf{#3}, #4)}}}

% currently this works as \begin{env}[optional rmk]{x.y.z}
\makeatletter
\newenvironment{env}[2][\@nil]{%
    \def\tmp{#1}%
    \ifx\tmp\@nnil
        \par\medskip\noindent\indent\textbf{(#2)}\rmfamily
    \else
        \par\medskip\noindent\indent\textit{\textbf{#1}}~\textbf{(#2)}.\,---\rmfamily
    \fi}
\makeatother

% use this for definitions, propositions, corollaries, etc.
\makeatletter
\newenvironment{envs}[2][\@nil]{
  \par\medskip\noindent\indent\textit{\textbf{#1}}~\textbf{(#2)}.\,---\itshape
}
\makeatother



\begin{document}
\title{The language of schemes (EGA~I)}
\maketitle

\phantomsection
\label{section-phantom}

\tableofcontents

\section*{Summary}
\label{section-EGA-I-summary}

\begin{tabular}{ll}
  \textsection1. & Affine schemes.\\
  \textsection2. & Preschemes and morphisms of preschemes.\\
  \textsection3. & Products of preschemes.\\
  \textsection4. & Subpreschemes and immersion morphisms.\\
  \textsection5. & Reduced preschemes; separation condition.\\
  \textsection6. & Finiteness conditions.\\
  \textsection7. & Rational maps.\\
  \textsection8. & Chevalley schemes.\\
  \textsection9. & Supplement on quasi-coherent sheaves.\\
  \textsection10. & Formal schemes.
\end{tabular}\\

\oldpage[I]{79}
The \textsection\textsection1--8 do little more than develop a language, which will be used in the following.
It should be noted, however, that in accordance with the general spirit of this treatise, \textsection\textsection7--8 will be used less than the others, and in a less essential way; we have moreover spoken of Chevalley's schemes only to make the link with the language of Chevalley \cite{I-1} and Nagata \cite{I-9}.
The \textsection9 gives definitions and results on quasi-coherent sheaves, some of which are no longer limited to a translation into a ``geometric'' language of known notions of commutative algebra, but are already of a global nature; they will be indispensable, in the following chapters, for the global study of morphisms.
Finally, \textsection10 introduces a generalization of the notion of schemes, which will be used as an intermediary in Chapter~III to formulate and prove in a convenient way the fundamental results of the cohomological study of the proper morphisms; moreover, it should be noted that the notion of formal schemes seems indispensable to express certain facts of the ``theory of modules'' (classification problems of algebraic varieties).
The results of \textsection10 will not be used before \textsection3 of Chapter~III and it is recommended to omit reading until then.
\bigskip

\input{EGA-I/EGA-I-1}
\input{EGA-I/EGA-I-2}
\input{EGA-I/EGA-I-3}
\input{EGA-I/EGA-I-4}
\input{EGA-I/EGA-I-5}
\input{EGA-I/EGA-I-6}
\input{EGA-I/EGA-I-7}
\input{EGA-I/EGA-I-8}
\input{EGA-I/EGA-I-9}
\input{EGA-I/EGA-I-10}

\bibliography{the}
\bibliographystyle{amsalpha}

\end{document}

