\section{Rings of fractions}
\label{section-rings-of-fractions}

\setcounter{subsection}{-1}
\subsection{Rings and Algebras}
\label{subsection-rings-and-algs}

\begin{env}[1.0.1]
\label{0.1.0.1}
\oldpage[0\textsubscript{I}]{11}
All the rings considered in this treatise will have a {\em unit element}; all
the modules over such a ring will be assumed to be {\em unitary}; the ring
homomorphisms will always be assumed to {\em send the unit element to a unit
element}; unless otherwise stated, a subring of a ring $A$ will be assumed to
{\em contain the unit element of $A$}. We will consider especially
{\em commutative} rings, and when we speak of a ring without specification, it
will be implied that it is commutative. If $A$ is a not-necessarily-commutative ring,
by an $A$-module we will we mean a left module unless stated otherwise.
\end{env}

\begin{env}[1.0.2]
\label{0.1.0.2}
Let $A$, $B$ be two not-necessarily-commutative rings and $\vphi:A\to B$ a
homomorphism. Any left (resp. right) $B$-module $M$ can be provided with a left
(resp. right) $A$-module structure by $a\cdot m=\vphi(a)\cdot m$
(resp. $m\cdot a=m\cdot\vphi(a)$); when it will be necessary to distinguish $M$
as an $A$-module or a $B$-module, we will denote by $M_{[\vphi]}$ the left
(resp. right) $A$-module as defined. If $L$ is an $A$-module, then a
homomorphism $u:L\to M_{[\vphi]}$ is a homomorphism of abelian groups such
that $u(a\cdot x)=\vphi(a)\cdot u(x)$ for $a\in A$, $x\in L$; we will also say
that it is a {\em $\vphi$-homomorphism} $L\to M$, and that the pair $(\vphi,u)$
(or, by abuse of language, $u$) is a {\em di-homomorphism} of $(A,L)$ in
$(B,M)$. The pairs $(A,L)$ consisting of a ring $A$ and an $A$-module $L$ thus
form a {\em category} for which the morphisms are di-homomorphisms.
\end{env}

\begin{env}[1.0.3]
\label{0.1.0.3}
Under the hypotheses of \hyperref[0.1.0.2]{(1.0.2)}, if $\mathfrak{J}$ is a left
(resp. right) ideal of $A$, we denote by $B\mathfrak{J}$ (resp. $\mathfrak{J}B$)
the left (resp. right) ideal $B\vphi(\mathfrak{J})$
(resp. $\vphi(\mathfrak{J})B$) of $B$ generated by $\vphi(\mathfrak{J})$; it is
also the image of the canonical homomorphism $B\otimes_A\mathfrak{J}\to B$
(resp. $\mathfrak{J}\otimes_A B\to B$) of left (resp. right) $B$-modules.
\end{env}

\begin{env}[1.0.4]
\label{0.1.0.4}
If $A$ is a (commutative) ring, $B$ a not-necessarily-commutative ring, then the data
of a structure of an {\em $A$-algebra} on $B$ is equivalent to the data of a
ring homomorphism $\vphi:A\to B$ such that $\vphi(A)$ is contained in the center
of $B$. For all ideals $\mathfrak{J}$ of $A$, $\mathfrak{J}B=B\mathfrak{J}$ is
then a two-sided ideal of $B$, and for every $B$-module $M$, $\mathfrak{J}M$ is
then a $B$-module equal to $(B\mathfrak{J})M$.
\end{env}

\begin{env}[1.0.5]
\label{0.1.0.5}
We will not revisit the notions of {\em module of finite type} and
(commutative) {\em algebra of finite type}; to say that an $A$-module $M$ is of
finite type means that there exists
\oldpage[0\textsubscript{I}]{12}
an exact sequence $A^p\to M\to 0$. We say that an $A$-module $M$ admits a
{\em finite presentation} if it is isomorphic to the cokernel of a homomorphism
$A^p\to A^q$, in other words, there exists an exact sequence
$A^p\to A^q\to M\to 0$. We note that for a {\em Noetherian} ring $A$, every
$A$-module of finite type admits a finite presentation.

Let us recall that an $A$-algebra $B$ is called {\em integral} over $A$ if
every element in $B$ is a root in $B$ of a monic polynomial with coefficients in
$A$; equivalently, every element of $B$ is contained in a subalgebra of $B$
which is an {\em $A$-module of finite type}. When this is so, and if $B$ is
commutative, the subalgebra of $B$ generated by a finite subset of $B$ is an
$A$-module of finite type; for a commutative algebra $B$ to be integral and of
finite type over $A$, it is necessary and sufficient that $B$ is an $A$-module
of finite type; we also say that $B$ is an {\em integral $A$-algebra of finite
type} (or simply {\em finite} if there is no chance of confusion). It should be
noted that in these definitions, it is not assumed that the homomorphism
$A\to B$ defining the $A$-algebra structure is injective.
\end{env}

\begin{env}[1.0.6]
\label{0.1.0.6}
An {\em integral} ring (or an {\em integral domain}) is a ring in which the
product of a finite family of elements $\neq 0$ is $\neq 0$; equivalently, in
such a ring we have $0\neq 1$ and the product of two elements $\neq 0$ is non
zero. A {\em prime} ideal of a ring $A$ is an ideal $\mathfrak{p}$ such that
$A/\mathfrak{p}$ is integral; this therefore implies that $\mathfrak{p}\neq A$.
For a ring $A$ to have at least one prime ideal, it is necessary and sufficient
that $A\neq\{0\}$.
\end{env}

\begin{env}[1.0.7]
\label{0.1.0.7}
A {\em local} ring is a ring $A$ in which there exists a unique maximal ideal,
which is then the complement of the invertible elements and contains all the
ideals $\neq A$. If $A$ and $B$ are two local rings, $\mathfrak{m}$ and
$\mathfrak{n}$ their respective maximal ideals, we say that a homomorphism
$\vphi:A\to B$ is {\em local} if $\vphi(\mathfrak{m})\subset\mathfrak{n}$ (or,
equivalently, if $\vphi^{-1}(\mathfrak{n})=\mathfrak{m}$). By passing to
quotients, such a homomorphism then defines a momomorphism from the residue
field $A/\mathfrak{m}$ to the residue field $B/\mathfrak{n}$. The composition of
two local homomorphisms is a local homomorphism.
\end{env}

\subsection{Radical of an ideal. Nilradical and radical of a ring}
\label{subsection-radical-and-nilradical}

\begin{env}[1.1.1]
\label{0.1.1.1}
Let $\mathfrak{a}$ be an ideal of a ring $A$; the {\em radical} of
$\mathfrak{a}$, denoted by $\rad(\mathfrak{a})$, is the set of $x\in A$ such
that $x^n\in\mathfrak{a}$ for an integer $n>0$; it is an ideal containing
$\mathfrak{a}$. We have $\rad(\mathfrak{r}(\mathfrak{a}))=\rad(\mathfrak{a})$;
the relation $\mathfrak{a}\subset\mathfrak{b}$ implies
$\rad(\mathfrak{a})\subset\rad(\mathfrak{b})$; the radical of a finite
intersection of ideals is the intersection of their radicals. If $\vphi$ is a
homomorphism of a ring $A'$ into $A$, then we have
$\rad(\vphi^{-1}(\mathfrak{a}))=\vphi^{-1}(\rad(\mathfrak{a}))$ for any ideal
$\mathfrak{a}\subset A$. For an ideal to be the radical of an ideal, it is
necessary and sufficient that it be an intersection of prime ideals. The radical
of an ideal $\mathfrak{a}$ is the intersection of the {\em minimal} prime
ideals which contain $\mathfrak{a}$; if $A$ is Noetherian, these minimal prime
ideals are finite in number.

The radical of the ideal $(0)$ is also called the {\em nilradical} of $A$; it
is the set $\nilrad$ of the nilpotent elements of $A$. It is said that the ring
$A$ is {\em reduced} if $\nilrad=(0)$; for every ring $A$, the quotient
$A/\nilrad$ of $A$ by its nilradical is a reduced ring.
\end{env}

\begin{env}[1.1.2]
\label{0.1.1.2}
Recall that the {\em nilradical} $\nilrad(A)$ of a (not necessarily
commutative) ring $A$ is the intersection of the maximal left ideals of $A$ (and
also the intersection of maximal right ideals). The nilradical of $A/\nilrad(A)$
is $(0)$.
\end{env}

\subsection{Modules and rings of fractions}
\label{subsection-modules-and-rings-of-fractions}

\begin{env}[1.2.1]
\label{0.1.2.1}
\oldpage[0\textsubscript{I}]{13}
We say that a subset $S$ of a ring $A$ is {\em multiplicative} if $1\in S$ and
if the product of two elements of $S$ is in $S$. The examples which will be the
most important in the following are: 1st the set $S_f$ of
powers $f^n$ ($n\geqslant 0$) of an element $f\in A$; 2nd the
complement $A-\mathfrak{p}$ of a {\em prime} ideal $\mathfrak{p}$ of $A$.
\end{env}

\begin{env}[1.2.2]
\label{0.1.2.2}
Let $S$ be a multiplicative subset of a ring $A$, $M$ an $A$-module; on the set
$M\times S$, the relation between pairs $(m_1,s_1)$, $(m_2,s_2)$:
\begin{center}
  ``there exists an $s\in S$ such that $s(s_1 m_2-s_2 m_1)=0$''
\end{center}
is an equivalence relation. We denote by $S^{-1}M$ the quotient set of
$M\times S$ by this relation and by $m/s$ the canonical image in $S^{-1}M$ of
the pair $(m,s)$; we call $i_M^S:m\mapsto m/1$ (also denoted $i^S$) the
{\em canonical} map from $M$ to $S^{-1}M$. This map is in general neither
injective nor surjective; its kernel is the set of $m\in M$ such that there
exists an $s\in S$ for which $sm=0$.

On $S^{-1}M$ we define an additive group law by taking
\[
  (m_1/s_1)+(m_2/s_2)=(s_2 m_1+s_1 m_2)/(s_1 s_2)
\]
(we check that it is independent of the expressions of the elements of $S^{-1}M$
considered). On $S^{-1}A$ we further define a multiplicative law by setting
$(a_1/s_1)(a_2/s_2)=(a_1 a_2)/(s_1 s_2)$, and finally an external law on
$S^{-1}M$, having $S^{-1}A$ as a set of operators, by setting
$(a/s)(m/s')=(am)/(ss')$. It is thus verified that $S^{-1}A$ is provided with a
ring structure (called {\em the ring of fractions of $A$ with denominators in
$S$}) and $S^{-1}M$ the structure of an $S^{-1}A$-module (called {\em the
module of fractions of $M$ with denominators in $S$}); for all $s\in S$, $s/1$
is invertible in $S^{-1}A$, its inverse being $1/s$. The canonical map $i_A^S$
(resp. $i_M^S$) is a ring homomorphism (resp. a homomorphism of $A$-modules,
$S^{-1}M$ being considered as an $A$-module by means of the homomorphism
$i_A^S:A\to S^{-1}A$).
\end{env}

\begin{env}[1.2.3]
\label{0.1.2.3}
If $S_f=\{f^n\}_{n\geqslant 0}$ for a $f\in A$, we write $A_f$ and $M_f$ instead
of $S_f^{-1}A$ and $S_f^{-1}M$; when $A_f$ is considered as algebra over $A$, we
can write $A_f=A[1/f]$. $A_f$ is isomorphic to the quotient algebra
$A[T]/(fT-1)A[T]$. When $f=1$, $A_f$ and $M_f$ canonically identify with $A$ and
$M$; if $f$ is nilpotent, then $A_f$ and $M_f$ are $0$.

When $S=A-\mathfrak{p}$, where $\mathfrak{p}$ is a prime ideal of $A$, we write
$A_\mathfrak{p}$ and $M_\mathfrak{p}$ instead of $S^{-1}A$ and $S^{-1}M$;
$A_\mathfrak{p}$ is a {\em local ring} whose maximal ideal $\mathfrak{q}$ is
generated by $i_A^S(\mathfrak{p})$, and we have
$(i_A^S)^{-1}(\mathfrak{q})=\mathfrak{p}$; by passing to quotients, $i_A^S$
gives a monomorphism from the integral ring $A/\mathfrak{p}$ to the field
$A_\mathfrak{p}/\mathfrak{q}$, which identifies with the field of fractions of
$A/\mathfrak{p}$.
\end{env}

\begin{env}[1.2.4]
\label{0.1.2.4}
The ring of fractions $S^{-1}A$ and the canonical homomorphism $i_A^S$ are a
solution to a {\em universal mapping problem}: any homomorphism $u$ from $A$ to
a ring $B$ such that $u(S)$ is composed of {\em invertible} elements in $B$
factorizes uniquely as
\[
  u:A\xrightarrow{i_A^S}S^{-1}A\xrightarrow{u^*}B
\]
\oldpage[0\textsubscript{I}]{14}
where $u^*$ is a ring homomorphism. Under the same hypotheses, let $M$ be an
$A$-module, $N$ a $B$-module, and $v:M\to N$ a homomorphism of $A$-modules (for
the $B$-module structure on $N$ defined by $u:A\to B$); then $v$ factorizes
uniquely as
\[
  v:M\xrightarrow{i_M^S}S^{-1}M\xrightarrow{v^*}N
\]
where $v^*$ is a homomorphism of $S^{-1}A$-modules (for the $S^{-1}A$-module
structure on $N$ defined by $u^*$).
\end{env}

\begin{env}[1.2.5]
\label{0.1.2.5}
We define a canonical isomorphism $S^{-1}A\otimes_A M\isoto S^{-1}M$ of
$S^{-1}A$-modules, sending the element $(a/s)\otimes m$ to the element $(am)/s$,
the inverse isomorphism sending $m/s$ to $(1/s)\otimes m$.
\end{env}

\begin{env}[1.2.6]
\label{0.1.2.6}
For each ideal $\mathfrak{a}'$ of $S^{-1}A$,
$\mathfrak{a}=(i_A^S)^{-1}(\mathfrak{a}')$ is an ideal of $A$, and
$\mathfrak{a}'$ is the ideal of $S^{-1}A$ generated by $i_A^S(\mathfrak{a})$,
which identifies with $S^{-1}\mathfrak{a}$ \hyperref[0.1.3.2]{(1.3.2)}. The map
$\mathfrak{p}'\mapsto(i_A^S)^{-1}(\mathfrak{p}')$ is an isomorphism, for the
order structure, from the set of {\em prime} ideals of $S^{-1}A$ to the set of
prime ideals $\mathfrak{p}$ of $A$ such that $\mathfrak{p}\cap S=\emp$. In
addition, the local rings $A_\mathfrak{p}$ and $(S^{-1}A)_{S^{-1}\mathfrak{p}}$
are canonically isomorphic \hyperref[0.1.5.1]{(1.5.1)}.
\end{env}

\begin{env}[1.2.7]
\label{0.1.2.7}
When $A$ is an {\em integral} ring, for which $K$ denotes its field of
fractions, the canonical map $i_A^S:A\to S^{-1}A$ is injective for any
multiplicative subset $S$ not containing $0$, and $S^{-1}A$ then canonically
identifies with a subring of $K$ containing $A$. In particular, for each prime
ideal $\mathfrak{p}$ of $A$, $A_\mathfrak{p}$ is a local ring containing $A$,
with maximal ideal $\mathfrak{p}A_\mathfrak{p}$, and we have
$\mathfrak{p}A_\mathfrak{p}\cap A=\mathfrak{p}$.
\end{env}

\begin{env}[1.2.8]
\label{0.1.2.8}
If $A$ is a {\em reduced} ring \hyperref[0.1.1.1]{(1.1.1)}, so is $S^{-1}A$: indeed, if
$(x/s)^n=0$ for $x\in A$, $s\in S$, then this means that there exists an
$s'\in S$ such that $s' x^n=0$, hence $(s' x)^n=0$, which, by hypothesis, implies
$s' x=0$, so $x/s=0$.
\end{env}

\subsection{Functorial properties}
\label{subsection-func-ring-of-fractions}

\begin{env}[1.3.1]
\label{0.1.3.1}
Let $M$, $N$ be two $A$-modules, $u$ an $A$-homomorphism $M\to N$. If $S$ is a
multiplicative subset of $A$, we define a $S^{-1}A$-homomorphism
$S^{-1}M\to S^{-1}N$, denoted by $S^{-1}u$, by setting $S^{-1}u(m/s)=u(m)/s$; if
$S^{-1}M$ and $S^{-1}N$ are canonically identified with $S^{-1}A\otimes_A M$ and
$S^{-1}A\otimes_A N$ \hyperref[0.1.2.5]{(1.2.5)}, $S^{-1}u$ is considered as $1\otimes u$.
If $P$ is a third $A$-module, $v$ an $A$-homomorphism $N\to P$, we have
$S^{-1}(v\circ u)=(S^{-1}v)\circ(S^{-1}u)$; in other words, $S^{-1}M$ is a
{\em covariant functor in $M$}, from the category of $A$-modules to that of
$S^{-1}A$-modules ($A$ and $S$ being fixed).
\end{env}

\begin{env}[1.3.2]
\label{0.1.3.2}
The functor $S^{-1}M$ is {\em exact}; in other words, if the following sequence
\[
  M\xrightarrow{u}N\xrightarrow{v}P
\]
is exact, so is the following sequence
\[
  S^{-1}M\xrightarrow{S^{-1}u}S^{-1}N\xrightarrow{S^{-1}v}S^{-1}P.
\]
In particular, if $u:M\to N$ is injective (resp. surjective), the same is true
for $S^{-1}u$;
\oldpage[0\textsubscript{I}]{15}
if $N$ and $P$ are two submodules of $M$, $S^{-1}N$ and $S^{-1}P$ canonically
identify with submodules of $S^{-1}M$, and we have
\[
  S^{-1}(N+P)=S^{-1}N+S^{-1}P
  \quad\text{and}\quad
  S^{-1}(N\cap P)=(S^{-1}N)\cap(S^{-1}P).
\]
\end{env}

\begin{env}[1.3.3]
\label{0.1.3.3}
Let $(M_\alpha,\vphi_{\beta\alpha})$ be an inductive system of $A$-modules; then
$(S^{-1}M_\alpha,S^{-1}\vphi_{\beta\alpha})$ is an inductive system of
$S^{-1}A$-modules. Expressing the $S^{-1}M_\alpha$ and
$S^{-1}\vphi_{\beta\alpha}$ as tensor products (\hyperref[0.1.2.5]{(1.2.5)} and
\hyperref[0.1.3.1]{(1.3.1)}), it follows from the permutability of the tensor product and
inductive limit operations that we have a canonical isomorphism
\[
  S^{-1}\varinjlim M_\alpha\isoto\varinjlim S^{-1}M_\alpha
\]
which is further expressed by saying that the functor $S^{-1}M$ (in $M$)
{\em commutes with inductive limits}.
\end{env}

\begin{env}[1.3.4]
\label{0.1.3.4}
Let $M$, $N$ be two $A$-modules; there is a canonical {\em functorial}
isomorphism (in $M$ and $N$)
\[
  (S^{-1}M)\otimes_{S^{-1}A}(S^{-1}N)\isoto S^{-1}(M\otimes_A N)
\]
which sends $(m/s)\otimes(n/t)$ to $(m\otimes n)/st$.
\end{env}

\begin{env}[1.3.5]
\label{0.1.3.5}
We also have a {\em functorial} homomorphism (in $M$ and $N$)
\[
  S^{-1}\Hom_A(M,N)\longrightarrow\Hom_{S^{-1}A}(S^{-1}M,S^{-1}N)
\]
which sends $u/s$ to the homomorphism $m/t\mapsto u(m)/st$. When $M$ has a
finite presentation, the above homomorphism is an {\em isomorphism}: it is
immediate when $M$ is of the form $A^r$, and we pass to the general case by
starting with the following exact sequence $A^p\to A^q\to M\to 0$, and using the
exactness of the functor $S^{-1}M$ and the left-exactness of the functor
$\Hom_A(M,N)$ in $M$. Note that this case always occurs when $A$ is
{\em Noetherian} and the $A$-module $M$ is {\em of finite type}.
\end{env}

\subsection{Change of multiplicative subset}
\label{subsection-change-of-mult-subset}

\begin{env}[1.4.1]
\label{0.1.4.1}
Let $S$, $T$ be two multiplicative subsets of a ring $A$ such that $S\subset T$;
there exists a canonical homomorphism $\rho_A^{T,S}$ (or simply $\rho^{T,S}$)
from $S^{-1}A$ to $T^{-1}A$, sending the element denoted $a/s$ of $S^{-1}A$ to
the element denoted $a/s$ in $T^{-1}A$; we have $i_A^T=\rho_A^{T,S}\circ i_A^S$.
For every $A$-module $M$, there exists in the same way an $S^{-1}A$-linear map
from $S^{-1}M$ to $T^{-1}M$ (the latter considered as an $S^{-1}A$-module thanks
to the homomorphism $\rho_A^{T,S}$), which sends the element $m/s$ of $S^{-1}M$
to the element $m/s$ of $T^{-1}M$; we denote this map by $\rho_M^{T,S}$, or
simply $\rho^{T,S}$, and we still have $i_M^T=\rho_M^{T,S}\circ i_M^S$; by the
canonical identification \hyperref[0.1.2.5]{(1.2.5)}, $\rho_M^{T,S}$ identifies with
$\rho_A^{T,S}\otimes 1$. The homomorphism $\rho_M^{T,S}$ is a {\em functorial
morphism} (or natural transformation) from the functor $S^{-1}M$ to the functor
$T^{-1}M$, in other words, the diagram
\[
  \xymatrix{
  S^{-1}M\ar[r]^{S^{-1}u}\ar[d]_{\rho_M^{T,S}} &
  S^{-1}N\ar[d]^{\rho_N^{T,S}}\\
  T^{-1}M\ar[r]^{T^{-1}u} &
  T^{-1}N
  }
\]
\oldpage[0\textsubscript{I}]{16}
is commutative, for every homomorphism $u:M\to N$; $T^{-1}u$ is entirely
determined by $S^{-1}u$, since for $m\in M$ and $t\in T$, we have
\[
  (T^{-1}u)(m/t)=(t/1)^{-1}\rho^{T,S}((S^{-1}u)(m/1)).
\]
\end{env}

\begin{env}[1.4.2]
\label{0.1.4.2}
With the same notations, for two $A$-modules $M$, $N$, the diagrams
(cf. \hyperref[0.1.3.4]{(1.3.4)} and \hyperref[0.1.3.5]{(1.3.5)})
\[
  \xymatrix{
    (S^{-1}M)\otimes_{S^{-1}A}(S^{-1}N)\ar[r]^\sim \ar[d] &
    S^{-1}(M\otimes_A N)\ar[d] & &
    S^{-1}\Hom_A(M,N)\ar[r]\ar[d] &
    \Hom_{S^{-1}A}(S^{-1}M,S^{-1}N)\ar[d]\\
    (T^{-1}M)\otimes_{T^{-1}A}(T^{-1}N)\ar[r]^\sim &
    T^{-1}(M\otimes_A N) & &
    T^{-1}\Hom_A(M,N)\ar[r] &
    \Hom_{T^{-1}A}(T^{-1}M,T^{-1}N)
  }
\]
are commutative.
\end{env}

\begin{env}[1.4.3]
\label{0.1.4.3}
There is an important case in which the homomorphism $\rho^{T,S}$ is
{\em bijective}, where then we know that every element of $T$ is divisor of an
element of $S$; we then identify the modules $S^{-1}M$ and $T^{-1}M$ via
$\rho^{T,S}$. We say that $S$ is {\em saturated} if every divisor in $A$ of an
element of $S$ is in $S$; by replacing $S$ with the set $T$ of all the divisors
of the elements of $S$ (a set which is multiplicative and saturated), we see
that we can always, if we wish, be limited to the consideration of modules of
fractions $S^{-1}M$, where $S$ is saturated.
\end{env}

\begin{env}[1.4.4]
\label{0.1.4.4}
If $S$, $T$, $U$ are three multiplicative subsets of $A$ such that
$S\subset T\subset U$, we have
\[
  \rho^{U,S}=\rho^{U,T}\circ\rho^{T,S}.
\]
\end{env}

\begin{env}[1.4.5]
\label{0.1.4.5}
Consider an {\em increasing filtered family} $(S_\alpha)$ of multiplicative
subsets of $A$ (we write $\alpha\leqslant\beta$ for $S_\alpha\subset S_\beta$),
and let $S$ be the multiplicative subset $\bigcup_\alpha S_\alpha$; let us put
$\rho_{\beta\alpha}=\rho_A^{S_\beta,S_\alpha}$ for $\alpha\leqslant\beta$;
according to \hyperref[0.1.4.4]{(1.4.4)}, the homomorphisms $\rho_{\beta\alpha}$ define a
ring $A'$ as the {\em inductive limit} of the inductive system of rings
$(S_\alpha^{-1}A,\rho_{\beta\alpha})$. Let $\rho_\alpha$ be the canonical map
$S_\alpha^{-1}A\to A'$, and let $\vphi_\alpha=\rho_A^{S,S_\alpha}$; as
$\vphi_\alpha=\vphi_\beta\circ\rho_{\beta\alpha}$ for $\alpha\leqslant\beta$
according to \hyperref[0.1.4.4]{(1.4.4)}, we can uniquely define a homomorphism
$\vphi:A'\to S^{-1}A$ such that the diagram
\[
  \xymatrix{
    & S_\alpha^{-1}A\ar[ddl]_{\rho_\alpha}
                    \ar[d]^{\rho_{\beta\alpha}}
                    \ar[rdd]^{\vphi_\alpha}\\
    & S_\beta^{-1}A\ar[ld]^{\rho_\beta}
                   \ar[rd]_{\vphi_\beta}
    & & (\alpha\leqslant\beta)\\
    A'\ar[rr]_\vphi
    & & S^{-1}A
  }
\]
is commutative. In fact, $\vphi$ is an {\em isomorphism}; it is indeed
immediate by construction that $\vphi$ is surjective. On the other hand, if
$\rho_\alpha(a/s_\alpha)\in A'$ is such that $\vphi(\rho_\alpha(a/s_\alpha))=0$,
this means that $a/s_\alpha=0$ in $S^{-1}A$, that is to say that there exists an
$s\in S$ such that $sa=0$; but there is a $\beta\geqslant\alpha$ such that
$s\in S_\beta$, and consequently, as
$\rho_\alpha(a/s_\alpha)=\rho_\beta(sa/ss_\alpha)=0$, we find that $\vphi$ is
injective. The case for an $A$-module $M$ is treated likewise, and thus we have
defined canonical isomorphisms
\[
  \varinjlim S_\alpha^{-1}A\isoto(\varinjlim S_\alpha)^{-1}A,\quad
  \varinjlim S_\alpha^{-1}M\isoto(\varinjlim S_\alpha)^{-1}M,
\]
the second being {\em functorial} in $M$.
\end{env}

\begin{env}[1.4.6]
\label{0.1.4.6}
\oldpage[0\textsubscript{I}]{17}
Let $S_1$, $S_2$ be two multiplicative subsets of $A$; then $S_1 S_2$ is also a
multiplicative subset of $A$. Let us denote by $S_2'$ the canonical image of
$S_2$ in the ring $S_1^{-1}A$, which is a multiplicative subset of this ring.
For each $A$-module $M$ there is then a functorial isomorphism
\[
  {S_2'}^{-1}(S_1^{-1}M)\isoto(S_1 S_2)^{-1}M
\]
which sends $(m/s_1)/(s_2/1)$ to the element $m/(s_1 s_2)$.
\end{env}

\subsection{Change of ring}
\label{subsection-change-of-ring}

\begin{env}[1.5.1]
\label{0.1.5.1}
Let $A$, $A'$ be two rings, $\vphi$ a homomorphism $A'\to A$, $S$ (resp. $S'$)
a multiplicative subset of $A$ (resp. $A'$), such that $\vphi(S')\subset S$; the
composition homomorphism $\smash{A'\xrightarrow{\vphi}A\to S^{-1}A}$ factors as
$\smash{A'\to{S'}^{-1}\xrightarrow{\vphi^{S'}} S^{-1}A}$ by virtue of
\hyperref[0.1.2.4]{(1.2.4)}; where $\vphi^{S'}(a'/s')=\vphi(a')/\vphi(s')$. If
$A=\vphi(A')$ and $S=\vphi(S')$, then $\vphi^{S'}$ is {\em surjective}. If
$A'=A$ and if $\vphi$ is the identity, then $\vphi^{S'}$ is none other than the
homomorphism $\rho_A^{S,S'}$ defined in \hyperref[0.1.4.1]{(1.4.1)}.
\end{env}

\begin{env}[1.5.2]
\label{0.1.5.2}
Under the hypotheses of \hyperref[0.1.5.1]{(1.5.1)}, let $M$ be an $A$-module. There
exists a canonical functorial morphism
\[
  \sigma:{S'}^{-1}(M_{[\vphi]})\longrightarrow(S^{-1}M)_{[\vphi^{S'}]}
\]
of ${S'}^{-1}A'$-modules, sending each element $m/s'$ of
${S'}^{-1}(M_{[\vphi]})$ to the element $m/\vphi(s')$ of
$(S^{-1}M)_{[\vphi^{S'}]}$; in fact, we verify immediately that this definition
does not depend on the expression $m/s'$ of the element considered. {\em When
$S=\vphi(S')$}, the homomorphism $\sigma$ is {\em bijective}. When $A'=A$ and
$\vphi$ is the identity, $\sigma$ is none other than the homomorphism
$\rho_M^{S,S'}$ defined in \hyperref[0.1.4.1]{(1.4.1)}.

When $M=A$ is taken in particular, the homomorphism $\vphi$ defines on $A$ an
$A'$-algebra structure; ${S'}^{-1}(A_{[\vphi]})$ is then provided with a ring
structure, for which it identifies with $(\vphi(S'))^{-1}A$, and the
homomorphism $\sigma:{S'}^{-1}(A_{[\vphi]})\to S^{-1}A$ is a homomorphism of
${S'}^{-1}A'$-algebras.
\end{env}

\begin{env}[1.5.3]
\label{0.1.5.3}
Let $M$ and $N$ be two $A$-modules; by composing the homomorphisms defined in
\hyperref[0.1.3.4]{(1.3.4)} and \hyperref[0.1.5.2]{(1.5.2)}, we obtain a homomorphism
\[
  (S^{-1}M\otimes_{S^{-1}A}S^{-1}N)_{[\vphi^{S'}]}
  \longleftarrow{S'}^{-1}((M\otimes A)_{[\vphi]})
\]
which is an isomorphism when $\vphi(S')=S$. Similarly, by composing the
homomorphisms in \hyperref[0.1.3.5]{(1.3.5)} and \hyperref[0.1.5.2]{(1.5.2)}, we obtain a
homomorphism
\[
  {S'}^{-1}((\Hom_A(M,N))_{[\vphi]})
  \longrightarrow(\Hom_{S^{-1}A}(S^{-1}M,S^{-1}N))_{[\vphi^{S'}]}
\]
which is an isomorphism when $\vphi(S')=S$ and $M$ admits a finite presentation.
\end{env}

\begin{env}[1.5.4]
\label{0.1.5.4}
Let us now consider an $A'$-module $N'$, and form the tensor product
$N'\otimes_{A'}A_{[\vphi]}$, which can be considered as an $A$-module by setting
$a\cdot(n'\otimes b)=n'\otimes(ab)$. There is a functorial isomorphism of
$S^{-1}A$-modules
\[
  \tau:({S'}^{-1}N')\otimes_{{S'}^{-1}A'}(S^{-1}A)_{[\vphi^{S'}]}
  \isoto S^{-1}(N'\otimes_{A'}A_{[\vphi]})
\]
\oldpage[0\textsubscript{I}]{18}
which sends the element $(n'/s')\otimes(a/s)$ to the element
$(n'\otimes a)/(\vphi(s')s)$; indeed, we verify separately that when we replace
$n'/s'$ (resp. $a/s$) by another expression of the same element,
$(n'\otimes a)/(\vphi(s')s)$ does not change; on the other hand, we can define
an inverse homomorphism of $\tau$ by sending $(n'\otimes a)/s$ to the element
$(n'/1)\otimes(a/s)$: we use the fact that $S^{-1}(N'\otimes_{A'}A_{[\vphi]})$
is canonically isomorphic to $(N'\otimes_{A'}A_{[\vphi]})\otimes_A S^{-1}A$
\hyperref[0.1.2.5]{(1.2.5)}, so also to $N'\otimes_{A'}(S^{-1}A)_{[\psi]}$, where we
denote by $\psi$ the composite homomorphism $a'\mapsto\vphi(a')/1$ from $A'$ to
$S^{-1}A$.
\end{env}

\begin{env}[1.5.5]
\label{0.1.5.5}
If $M'$ and $N'$ are two $A'$-modules, by composing the isomorphisms
\hyperref[0.1.3.4]{(1.3.4)} and \hyperref[0.1.5.4]{(1.5.4)}, we obtain an isomorphism
\[
  {S'}^{-1}M\otimes_{{S'}^{-1}A'}{S'}^{-1}N'\otimes_{{S'}^{-1}A'}S^{-1}A
  \isoto S^{-1}(M'\otimes_{A'}N'\otimes_{A'}A).
\]
Likewise, if $M'$ admits a finite presentation, we have by \hyperref[0.1.3.5]{(1.3.5)} and
\hyperref[0.1.5.4]{(1.5.4)} an isomorphism
\[
  \Hom_{{S'}^{-1}A'}({S'}^{-1}M',{S'}^{-1}N')\otimes_{{S'}^{-1}A'}S^{-1}A
  \isoto S^{-1}(\Hom_{A'}(M',N')\otimes_{A'}A).
\]
\end{env}

\begin{env}[1.5.6]
\label{0.1.5.6}
Under the hypotheses of \hyperref[0.1.5.1]{(1.5.1)}, let $T$ (resp. $T'$) be a second
multiplicative subset of $A$ (resp. $A'$) such that $S\subset T$
(resp. $S'\subset T'$) and $\vphi(T')\subset T$. Then the diagram
\[
  \xymatrix{
    {S'}^{-1}A'\ar[r]^{\vphi^{S'}}\ar[d]_{\rho^{T',S'}} &
    S^{-1}A\ar[d]^{\rho^{T,S}}\\
    {T'}^{-1}A'\ar[r]^{\vphi^{T'}} &
    T^{-1}A
  }
\]
is commutative. If $M$ is an $A$-module, the diagram
\[
  \xymatrix{
    {S'}^{-1}(M_{[\vphi]})\ar[r]^\sigma\ar[d]_{\rho^{T',S'}} &
    (S^{-1}M)_{[\vphi^{S'}]}\ar[d]^{\rho^{T,S}}\\
    {T'}^{-1}(M_{[\vphi]})\ar[r]^\sigma &
    (T^{-1}M)_{[\vphi^{T'}]}
  }
\]
is commutative. Finally, if $N'$ is an $A'$-module, the diagram
\[
  \xymatrix{
    ({S'}^{-1}N')\otimes_{{S'}^{-1}A'}(S^{-1}A)_{[\vphi^{S'}]}
      \ar[r]^\sim_\tau\ar[d] &
    S^{-1}(N'\otimes_{A'}A_{[\vphi]})\ar[d]^{\rho^{T,S}}\\
    ({T'}^{-1}N')\otimes_{{T'}^{-1}A'}(T^{-1}A)_{[\vphi^{T'}]}
      \ar[r]^\sim_\tau &
    T^{-1}(N'\otimes_{A'}A_{[\vphi]})
  }
\]
is commutative, the left vertical arrow obtained by applying
$\rho_{N'}^{T',S'}$ to ${S'}^{-1}N'$ and $\rho_A^{T,S}$ to $S^{-1}A$.
\end{env}

\begin{env}[1.5.7]
\label{0.1.5.7}
\oldpage[0\textsubscript{I}]{19}
Let $A''$ be a third ring, $\vphi':A''\to A'$ a ring homomorphism, $S''$ a
multiplicative subset of $A''$ such that $\vphi'(S'')\subset S'$. Set
$\vphi''=\vphi\circ\vphi'$; then we have
\[
  {\vphi''}^{S''}=\vphi^{S'}\circ{\vphi'}^{S''}.
\]
Let $M$ be an $A$-module; evidently we have
$M_{[\vphi'']}=(M_{[\vphi]})_{[\vphi']}$; if $\sigma'$ and $\sigma''$ are the
homomorphisms defined by $\vphi'$ and $\vphi''$ as $\sigma$ is defined in
\hyperref[0.1.5.2]{(1.5.2)} by $\vphi$, we have the transitivity formula
\[
  \sigma''=\sigma\circ\sigma'.
\]

Finally, let $N''$ be an $A''$-module; the $A$-module
$N''\otimes_{A''}A_{[\vphi'']}$ canonically identifies with
$(N''\otimes_{A''}{A'}_{[\vphi']})\otimes_{A'}A_{[\vphi]}$, and likewise the
$S^{-1}A$-module
$({S''}^{-1}N'')\otimes_{{S''}^{-1}A''}(S^{-1}A)_{[{\vphi''}^{S''}]}$ identifies
canonically with
$(({S''}^{-1}N'')\otimes_{{S''}^{-1}A''}({S'}^{-1}A')_{[{\vphi'}^{S''}]})
  \otimes_{{S'}^{-1}A'}(S^{-1}A)_{[\vphi^{S'}]}$. With these identifications, if
$\tau'$ and $\tau''$ are the isomorphisms defined by $\vphi'$ and $\vphi''$ as
$\tau$ is defined in \hyperref[0.1.5.4]{(1.5.4)} by $\vphi$, we have the transitivity
formula
\[
  \tau''=\tau\circ(\tau'\otimes 1).
\]
\end{env}

\begin{env}[1.5.8]
\label{0.1.5.8}
Let $A$ be a subring of a ring $B$; for every {\em minimal} prime ideal
$\mathfrak{p}$ of $A$, there exists a minimal prime ideal $\mathfrak{q}$ of $B$
such that $\mathfrak{p}=A\cap\mathfrak{q}$. Indeed, $A_\mathfrak{p}$ is a
subring of $B_\mathfrak{p}$ \hyperref[0.1.3.2]{(1.3.2)} and has {\em a single prime
ideal $\mathfrak{p}'$} \hyperref[0.1.2.6]{(1.2.6)}; since $B_\mathfrak{p}$ is not
$0$, it has at least one prime ideal $\mathfrak{q}'$ and we have necessarily
$\mathfrak{q}'\cap A_\mathfrak{p}=\mathfrak{p}'$; the prime ideal
$\mathfrak{q}_1$ of $B$, an inverse image of $\mathfrak{q}'$ is thus such that
$\mathfrak{q}_1\cap A=\mathfrak{p}$, and {\em a fortiori} we have
$\mathfrak{q}\cap A=\mathfrak{p}$ for every minimal prime ideal $\mathfrak{q}$
of $B$ contained in $\mathfrak{q}_1$.
\end{env}

\subsection{Identification of the module $M_f$ as an inductive limit}
\label{subsection-localization-as-inductive-limit}

\begin{env}[1.6.1]
\label{0.1.6.1}
Let $M$ be an $A$-module, $f$ an element of $A$. Consider a sequence $(M_n)$ of
$A$-modules, all identical to $M$, and for each pair of integers $m\leqslant n$,
let $\vphi_{nm}$ be the homomorphism $z\mapsto f^{n-m}z$ from $M_m$ to $M_n$; it
is immediate that $((M_n),(\vphi_{nm}))$ is an {\em inductive system} of
$A$-modules; let $N=\varinjlim M_n$ be the inductive limit of this system. We
define a canonical {\em functorial} $A$-isomorphism from $N$ to $M_f$. For this
reason, let us note that, for all $n$, $\theta_n:z\mapsto z/f^n$ is an
$A$-homomorphism from $M=M_n$ to $M_f$, and it follows from the definitions that
we have $\theta_n\circ\vphi_{nm}=\theta_m$ for $m\leqslant n$. There therefore
exists an $A$-homomorphism $\theta:N\to M_f$ such that, if $\vphi_n$ denotes the
canonical homomorphism $M_n\to N$, we have $\theta_n=\theta\circ\vphi_n$ for all
$n$. Since, by hypothesis, every element of $M_f$ is of the form $z/f^n$ for at
least one $n$, it is clear that $\theta$ is surjective. On the other hand, if
$\theta(\vphi_n(z))=0$, in other words $z/f^n=0$, then there exists an integer
$k>0$ such that $f^k z=0$, so $\vphi_{n+k,n}(z)=0$, which gives $\vphi_n(z)=0$.
We can therefore identify $M_f$ and $\varinjlim M_n$ by means of $\theta$.
\end{env}

\begin{env}[1.6.2]
\label{0.1.6.2}
Now write $M_{f,n}$, $\vphi_{nm}^f$ and $\vphi_n^f$ instead of $M_n$,
$\vphi_{nm}$ and $\vphi_n$. Let $g$ be a second element of $A$. As $f^n$ divides
$f^n g^n$, we have a functorial homomorphism
\[
  \rho_{fg,f}:M_f\longrightarrow M_{fg}
  \quad(\hyperref[0.1.4.1]{(1.4.1)}\text{ and }\hyperref[0.1.4.3]{(1.4.2)});
\]
\oldpage[0\textsubscript{I}]{20}
if we identify $M_f$ and $M_{fg}$ with $\varinjlim M_{f,n}$ and
$\varinjlim M_{fg,n}$ respectively, $\rho_{fg,f}$ identifies with the
{\em inductive limit} of the maps $\rho_{fg,f}^n:M_{f,n}\to M_{fg,n}$, defined
by $\rho_{fg,f}^n(z)=g^n z$. Indeed, this follows immediately from the
commutativity of the diagram
\[
  \xymatrix{
    M_{f,n}\ar[r]^{\rho_{fg,f}^n}\ar[d]_{\vphi_n^f} &
    M_{fg,n}\ar[d]^{\vphi_n^{fg}}\\
    M_f\ar[r]^{\rho_{fg,f}} &
    M_{fg}.
  }
\]
\end{env}

\subsection{Support of a module}
\label{subsection-support-of-module}

\begin{env}[1.7.1]
\label{0.1.7.1}
Given an $A$-module $M$, we call the {\em support} of $M$ and denote by
$\Supp(M)$ the set of prime ideals $\mathfrak{p}$ of $A$ such that
$M_\mathfrak{p}\neq 0$. For $M=0$ to be true, it is necessary and sufficient
that $\Supp(M)=\emp$, because if $M_\mathfrak{p}=0$ for all $\mathfrak{p}$, the
annihilator of an element $x\in M$ cannot be contained in any prime ideal of
$A$, so $A$ is \unsure{total}.
\end{env}

\begin{env}[1.7.2]
\label{0.1.7.2}
If $0\to N\to M\to P\to 0$ is an exact sequence of $A$-modules, we have
\[
  \Supp(M)=\Supp(N)\cup\Supp(P)
\]
because for every prime ideal $\mathfrak{p}$ of $A$, the sequence
$0\to N_\mathfrak{p}\to M_\mathfrak{p}\to P_\mathfrak{p}\to 0$ is exact
\hyperref[0.1.3.2]{(1.3.2)} and for $M_\mathfrak{p}=0$, it is necessary and sufficient
that $N_\mathfrak{p}=P_\mathfrak{p}=0$.
\end{env}

\begin{env}[1.7.3]
\label{0.1.7.3}
If $M$ is the sum of a family $(M_\lambda)$ of submodules, then $M_\mathfrak{p}$
is the sum of the $(M_\lambda)_\mathfrak{p}$ for every prime ideal
$\mathfrak{p}$ of $A$ (\hyperref[0.1.3.3]{(1.3.3)} and \hyperref[0.1.3.2]{(1.3.2)}),
so $\Supp(M)=\bigcup_\lambda\Supp(M_\lambda)$.
\end{env}

\begin{env}[1.7.4]
\label{0.1.7.4}
If $M$ is an $A$-module {\em of finite type}, then $\Supp(M)$ is the set of prime
ideals {\em containing the annihilator of} $M$. Indeed, if $M$ is cyclic and
generated by $x$, we say that $M_\mathfrak{p}=0$ means that there exists an
$s\not\in\mathfrak{p}$ such that $s\cdot x=0$, so that $\mathfrak{p}$ does not
contain the annihilator of $x$. If now $M$ admits a finite system
$(x_i)_{1\leqslant i\leqslant n}$ of generators and if $\mathfrak{a}_i$ is the
annihilator of $x_i$, then it follows from \hyperref[0.1.7.3]{(1.7.3)} that $\Supp(M)$ is
the set of $\mathfrak{p}$ containing one of the $\mathfrak{a}_i$, or,
equivalently, the set of $\mathfrak{p}$ containing
$\mathfrak{a}=\bigcap_i\mathfrak{a}_i$, which is the annihilator of $M$.
\end{env}

\begin{env}[1.7.5]
\label{0.1.7.5}
If $M$ and $N$ are two $A$-modules {\em of finite type}, we have
\[
  \Supp(M\otimes_A N)=\Supp(M)\cap\Supp(N).
\]
It can be seen that if $\mathfrak{p}$ is a prime ideal of $A$, the condition
$M_\mathfrak{p}\otimes_{A_\mathfrak{p}}N_\mathfrak{p}\neq 0$ is equivalent to
``$M_\mathfrak{p}\neq 0$ and $N_\mathfrak{p}\neq 0$'' (taking into account
\hyperref[0.1.3.4]{(1.3.4)}). In other words, it is about seeing that if $P$, $Q$ are two
modules of finite type over a {\em local} ring $B\neq 0$, then
$P\otimes_B Q\neq 0$. Let $\mathfrak{m}$ be the maximal ideal of $B$. By virtue
of Nakayama's Lemma, the vector spaces $P/\mathfrak{m}P$ and $Q/\mathfrak{m}Q$
are not $0$, so it is the same for the tensor product
$(P/\mathfrak{m}P)\otimes_{B/\mathfrak{m}}(Q/\mathfrak{m}Q)
  =(P\otimes_B Q)\otimes_B(B/\mathfrak{m})$, hence the conclusion.

In particular, if $M$ is an $A$-module of finite type, $\mathfrak{a}$ an ideal
of $A$, then $\Supp(M/\mathfrak{a}M)$ is the set of prime ideals containing both
$\mathfrak{a}$ and the annihilator $\mathfrak{n}$ of $M$ \hyperref[0.1.7.4]{(1.7.4)}, that
is, the set of prime ideals containing $\mathfrak{a}+\mathfrak{n}$.
\end{env}

