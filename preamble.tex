\documentclass{amsart}

\usepackage[all]{xy}
\usepackage[T1]{fontenc}
\usepackage{xstring}
\usepackage{xparse}
\usepackage{xr-hyper}
\usepackage[colorlinks=true,hyperindex,citecolor=blue,linkcolor=magenta]{hyperref}
%\usepackage[left=0.75in,right=0.75in,top=0.75in,bottom=0.75in]{geometry}
\usepackage[margin=0.75in]{geometry}
\usepackage[charter,ttscaled=false,greekfamily=didot,uppercase=upright,greeklowercase=upright]{mathdesign}
\usepackage{tikz-cd}
\usepackage{enumitem}

\externaldocument[intro-]{intro}
\externaldocument[prelim-]{prelim}
\externaldocument[schemes-]{schemes}

\newtheoremstyle{ega-env-style}%
  {}{}{\rmfamily}{}{\bfseries}{.}{ }{\thmnote{(#3)}}%

\newtheoremstyle{ega-thm-env-style}%
  {}{}{\itshape}{}{\bfseries}{. --- }{ }{\thmname{#1}\thmnote{ (#3)}}%

\newtheoremstyle{ega-defn-env-style}%
  {}{}{\rmfamily}{}{\bfseries}{. --- }{ }{\thmname{#1}\thmnote{ (#3)}}%

\theoremstyle{ega-env-style}
\newtheorem*{env}{---}

\theoremstyle{ega-thm-env-style}
\newtheorem*{thm}{Theorem}
\newtheorem*{prop}{Proposition}
\newtheorem*{lem}{Lemma}
\newtheorem*{cor}{Corollary}

\theoremstyle{ega-defn-env-style}
\newtheorem*{defn}{Definition}
\newtheorem*{exm}{Example}
\newtheorem*{rmk}{Remark}
\newtheorem*{nota}{Notational conventions}

%\allowdisplaybreaks[1]
%\binoppenalty=9999
%\relpenalty=9999

% for Chapter 0, Chapter I, etc.
% credit for ZeroRoman https://tex.stackexchange.com/questions/211414/
% added into scripts/make_book.py
%\newcommand{\ZeroRoman}[1]{\ifcase\value{#1}\relax 0\else\Roman{#1}\fi}
%\renewcommand{\thechapter}{\ZeroRoman{chapter}}

\def\mathcal{\mathscr}
\def\sh{\mathcal}                 % sheaf font
\def\bb{\mathbf}                  % bold font
\def\cat{\mathtt}                 % category font
\def\fk{\mathfrak}                % mathfrak font
\def\rad{\fk{r}}                  % radical
\def\nilrad{\fk{R}}               % nilradical
\def\emp{\varnothing}             % empty set
\def\vphi{\phi}                   % for switching \phi and \varphi, change if needed
\def\HH{\mathrm{H}}               % cohomology H
\def\dual#1{{#1}^\vee}            % dual
\def\kres{k}                      % residue field k
\def\isoto{\xrightarrow{\sim}}    % isomorphism \to
\def\C{\cat{C}}                   % category C
\def\op{^\cat{op}}                % opposite category
\def\Set{\cat{Set}}               % category of sets
\def\CHom{\cat{Hom}}              % functor category
\def\OO{\sh{O}}                   % structure sheaf O

\def\shHom{\sh{H}\!{om}}          % sheaf Hom
\def\Hom{{\mathop{\mathrm{Hom}}\nolimits}}
\def\Tor{{\mathop{\mathrm{Tor}}\nolimits}}
\def\Supp{{\mathop{\mathrm{Supp}}\nolimits}}
\def\Ker{{\mathop{\mathrm{Ker}}\nolimits}\,}
\def\Im{{\mathop{\mathrm{Im}}\nolimits}\,}
\def\Coker{{\mathop{\mathrm{Coker}}\nolimits}\,}
\def\Spec{{\mathop{\mathrm{Spec}}\nolimits}}
\def\Spf{{\mathop{\mathrm{Spf}}\nolimits}}
\def\grad{{\mathop{\mathrm{grad}}\nolimits}}
\def\inf{{\mathop{\mathrm{inf}}\nolimits}}

% if unsure of a translation
%\newcommand{\unsure}[2][]{\hl{#2}\marginpar{#1}}
%\newcommand{\completelyunsure}{\unsure{[\ldots]}}
\def\unsure#1{#1 {\color{red}(?)}}
\def\completelyunsure{{\color{red}(???)}}

% use to mark where original page starts
\newcommand{\oldpage}[2][--]{{\marginpar{\textbf{#1}~|~#2}}}
\def\sectionbreak{\begin{center}***\end{center}}

% for referencing environments.
% use as \sref{chapter-number.x.y.z}, with optional args
% for volume and indices, e.g. \sref[volume]{chapter-number.x.y.z}[i].
\NewDocumentCommand{\sref}{o m o}{%
  \IfNoValueTF{#1}%
    {\IfNoValueTF{#3}%
      {\hyperref[#2]{\normalfont{(\StrGobbleLeft{#2}{2})}}}%
      {\hyperref[#2]{\normalfont{(\StrGobbleLeft{#2}{2},~{#3})}}}}%
    {\IfNoValueTF{#3}%
      {\hyperref[#2]{\normalfont{(\textbf{#1},~\StrGobbleLeft{#2}{2})}}}%
      {\hyperref[#2]{\normalfont{(\textbf{#1},~\StrGobbleLeft,~{#3})}}}}%
}

