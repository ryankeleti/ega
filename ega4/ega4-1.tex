\section{Relative finiteness conditions. Constructible sets of preschemes}
\label{section:relative-finiteness-conditions-constructible-sets-of-preschemes}

In this section. we will resume the expos\'e of ``finiteness conditions'' for a morphism of preschemes $f:X\to Y$ given in~(\textbf{I},~\hyperref[subsection:1.6.3]{6.3}~and~\hyperref[subsection:1.6.6]{6.6}).
There are essentially two notions of ``finiteness'' of a \emph{global} nature on $X$, that of \emph{quasi-compact} morphism (defined in~\sref[I]{1.6.6.1}) and that of a \emph{quasi-separated} morphism; on the other hand, there are two notions of ``finiteness'' of a \emph{local} nature on $X$, that of a morphism \emph{locally of finite type} (defined in~\sref[I]{1.6.6.2}) and that of a morphism \emph{locally of finite presentation}.
By combining these local notions with the preceding global notions, we obtain the notion of a morphism \emph{of finite type} (defined in~\sref[I]{1.6.3.1}) and of a morphism \emph{of finite presentation}.
For the convenience of the reader, we will give again in this section the properties stated in~(\textbf{I},~\hyperref[subsection:1.6.3]{6.3}~and~\hyperref[subsection:1.6.6]{6.6}), referring to their labels in Chapter~I for their proofs.

In~n\textsuperscript{os}\hyperref[subsection:4.1.8]{1.8} and~\hyperref[subsection:4.1.9]{1.9}, we complete, in the context of preschemes, and making use of the previous notions of finiteness, the results on constructible sets given in~(\textbf{0}\textsubscript{III},~\textsection\hyperref[section:1.9]{9}).

\subsection{Quasi-compact morphisms}
\label{subsection:quasi-compact-morphisms}

\begin{defn}[1.1.1]
\label{4.1.1.1}
We say that a morphism of preschemes $f:X\to Y$ is \emph{quasi-compact} if the continous map $f$ from the topological space $X$ to the topological space $Y$ is quasi-compact~\sref[0]{0.9.1.1}, in other words, if the inverse image $f^{-1}(U)$ of every quasi-compact open subset $U$ of $Y$ is quasi-compact~(cf.~\sref[I]{1.6.6.1}).
\end{defn}

If $\mathfrak{B}$ is a basis for the topology of $Y$ consisting of affine open sets, then for $f$ to be quasi-compact, it is necessary and sufficient that for all $V\in\mathfrak{B}$, $f^{-1}(V)$ is a \emph{finite union of affine open sets}.
For example, if $Y$ is affine and $X$ is quasi-compact, \emph{every} morphism $f:X\to Y$ is quasi-compact~\sref[I]{1.6.6.1}.

If $f:X\to Y$ is a quasi-compact morphism, then it is clear that for every open subset $V$ of $Y$, the restriction of $f$ to $f^{-1}(V)$ is a quasi-compact morphism $f^{-1}(V)\to V$.
Conversely, if $(U_\alpha)$ is an open cover of $Y$ and $f:X\to Y$ is a morphism such that the restrictions $f^{-1}(U_\alpha)\to U_\alpha$ are quasi-compact, then $f$ is quasi-compact.
As
\oldpage[IV-1]{225}
a result, if $f:X\to Y$ is an $S$-morphism of $S$-preschemes, and if there exists an open cover $(S_\lambda)$ of $S$ such that the restrictions $g^{-1}(S_\lambda)\to h^{-1}(S_\lambda)$ of $f$ (where $g$ and $h$ are the structure morphisms) are quasi-compact, then $f$ is quasi-compact.



