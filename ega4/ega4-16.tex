%!TEX root = ../ega4.tex
%magic spell to make sublime text work

\setcounter{section}{15}
\section{Differential invariants. Differentially smooth morphisms}
\label{section:IV.16}

\oldpage[IV-4]{5}
In this paragraph we will present, in global form, some notions of differential calculus particularly useful in algebraic geometry.
We will ignore many classic developments in differential geometry (connections, infinitesimal transformations associated to vector fields, jets, etc.), although these notions are translated in a particularly natural way for schemes.
We will similarly ignore phenomena exclusive to characteristic $p>0$ (some of which are seen, in the affine case, in \hyperref[section:0.21]{(\textbf{0}, 21)}.
For certain complements to the differential formalism for preschemes the reader may consult Expos\'es~II and VII of \cite{IV-42} as well as subsequent chapters of this treatise. 

\subsection{Normal invariants of an immersion}
\label{IV.16.1}

\begin{env}[16.1.1]
\label{IV.16.1.1}

Let $(X, \sh{O}_X), (Y, \sh{O}_Y)$ be two ringed spaces and $f = (\psi, \theta): Y \to X$ a morphism of ringed spaces \sref[0]{0.4.1.1} such that the homomorphism
\[
  \theta^\#: \psi^*(\sh{O}_X) \to \sh{O}_Y
\]
is surjective, so that $\sh{O}_Y$ is identified with a sheaf of quotient rings $\psi^*(\sh{O}_X)/\sh{I}_f$. 
We can then endow $\psi^*(\sh{O}_X)$ with the $\sh{I}_f$-preadic filtration.
\end{env}

\begin{definition}[16.1.2]
\label{IV.16.1.2}
The $\sh{O}_Y$-augmented sheaf of rings $\psi^*(\sh{O}_X)/\sh{I}_f^{n+1}$ is called the $n$'th \emph{normal invariant} of $f$;
the ringed space $(Y, \psi^*(\sh{O}_X)/\sh{I}_f^{n+1})$ is called the $n$'th \emph{infinitesimal neighborhood} of $Y$ along $f$ and is denoted by $Y^{(n)}_f$ or simply $Y^{(n)}$.
The sheaf of graded rings associated to the sheaf of filtered rings $\psi^*(\sh{O}_X)$
\[
  \label{IV.16.1.2.1}
  \shGr_\bullet(f) = \bigoplus_{n \geq 0}(\sh{I}_f^{n}/\sh{I}_f^{n+1} )
  \tag{16.1.2.1}
\]
is called the sheaf of graded rings \emph{associated to} $f$. The sheaf $\shGr_1(f) = \sh{I}_f/\sh{I}_f^{2}$ is called the \emph{conormal sheaf} of $f$ (that will be denoted by $\sh{N}_{Y/X}$ when there is no risk of confusion). 
\end{definition}

It is clear that the $\sh{O}_{Y^{(n)}} = \psi^*(\sh{O}_X)/\sh{I}_f^{n+1}$ (that we also denote $\sh{O}_{Y_f^{(n)}})$ form a
\oldpage[IV-4]{6}
projective system of sheaves of rings on $Y$, the transition homomorphism $\phi_{nm}:\sh{O}_{Y^{(m)}} \to \sh{O}_{Y^{(n)}}$ for $n \leq m$ identifies $\sh{O}_{Y^{(n)}}$ with the quotient of $\sh{O}_{Y^{(m)}}$ by the power $(\sh{I}_f/\sh{I}_f^{n+1} )^m$ of the \emph{agumentation ideal} of $\sh{O}_{Y^{(n)}}$, kernel of $\phi_{0n}: \sh{O}_{Y^{(n)}} \to \sh{O}_{Y}$.
The $Y^{(n)}$ therefore form a inductive system of ringed spaces, all having underlying space $Y$, and we have canonical morphisms of ringed spaces $h_n: Y^{(n)} \to X$ equal to $(\psi, \theta_n)$, where $\theta^\#_n$ is the canonical morphism $\psi^*(\sh{O}_X) \to \psi^*(\sh{O}_X)/\sh{I}_f^{n+1}$.
It is clear that the sheaf $\shGr_\bullet(f)$ is a sheaf of graded algebras over the sheaf of rings $\sh{O}_Y = \shGr_0(f)$ and the $\shGr_k(f)$ of $\sh{O}_Y$-modules.

As with every sheaf of filtered rings, we have a \emph{canonical surjective homomorphism} of graded $\sh{O}_Y$-algebras
\[
  \label{IV.16.1.2.2}
  \bb{S}_{\sh{O}_Y}^\bullet(\shGr_1(f)) \to \shGr_\bullet(f)
  \tag{16.1.2.2}
\]
which coincide in degrees $0$ and $1$ with the identities.

\begin{examples}[16.1.3]
\label{IV.16.1.3}
\begin{enumerate}
  \item[(i)] Suppose that $X$ is a locally ringed space, $Y$ is reduced to a single point $y$ (endowed with a ring $\sh{O}_y$) and that, if $x = \psi(y)$, $\theta^\#:\sh{O}_x \to \sh{O}_y$ is a \emph{surjective} homomorphism of rings having as kernel the maximal ideal $\mathfrak{m}_x$ of $\sh{O}_x$.
  So the $\sh{O}_{Y^{(n)}}$ are identified with the rings $\sh{O}_x/\mathfrak{m}_x^{n+1}$ and $\shGr_\bullet(f)$ with the graded ring associated with the local ring $\sh{O}_x$ endowed with the $\mathfrak{m}_x$-preadic filtration.
  \item[(ii)] Suppose that $Y$ is a closed subset of an open subspace $U$ of $X$ and that the $\sh{O}_Y$ is induced on $Y$ by a quotient sheaf $\sh{O}_U/\sh{I}$, where $\sh{I}$ is an ideal of $\sh{O}_U$ such that $\sh{I}_x = \sh{O}_x$ for every $x \not\in Y$;
  if $X$ is a locally ringed space we also suppose that $\sh{I}_x \neq \sh{O}_x$ for $y \in Y$ so that $(Y, \sh{O}_Y)$ is a locally ringed space.
  
  Let $\psi_0: Y \to U$ be the canonical injection and denote by $\theta_0: \sh{O}_U \to (\psi_0)_*(\sh{O}_Y)$ the homomorphism such that $\theta_0^\#$ is the canonical homomorphism $\psi^*_0(\sh{O}_U) = \sh{O}_U|Y \to (\sh{O}_U/\sh{I})|Y$, so that $j_0=(\psi_0, \theta_0):Y \to U$ is a morphism of ringed spaces (and of locally ringed spaces if $X$ is a locally ringed space);
  if $i:U \to X$ is the canonical injection (morphism of ringed spaces), $j = i\circ j_0$ is the morphism $(\psi, \theta)$ of $Y$ to $X$ where $\psi: Y \to X$ is the canonical injection and $\theta:\sh{O}_X \to \psi_*(\sh{O}_Y)$ is the homomorphism such that $\theta^\# = \theta_0^\#$.
  Since $\theta^\#$ is surjective we can apply the previous definitions;
  $\sh{O}_{Y^{(n)}}$ is equal to $\psi^*_0(\sh{O}_U/\sh{I}^{n+1})$, and we have $(\psi_0)_*(\sh{O}_{Y^{(n)}} ) = \sh{O}_U/\sh{I}^{n+1}$, and $\shGr_n(j) = \shGr_n(j_0) = \psi^*_0(\sh{I}^n/\sh{I}^{n+1}) = j^*_0(\sh{I}^n/\sh{I}^{n+1})$.
  %I am pretty sure it should be \psi^* instead of j^*_0 in the last line... ~solov-t
\end{enumerate}
\end{examples}

\begin{env}[16.1.4]
\label{IV.16.1.4}
The example \sref{IV.16.1.3}[ii] shows that in general the $\sh{O}_{Y^{(n)}}$ are \emph{not canonically endowed with a structure of $\sh{O}_Y$-module}, or \emph{a fortiori} with a structure of $\sh{O}_Y$-algebra.
The data of such structure is equivalent to the data of a homomorphism of sheaves of rings $\lambda_n:\sh{O}_Y \to \sh{O}_{Y^{(n)}}$, right inverse to the augmentation morphism $\phi_{0n}$;
it is also equivalent to the data of a morphism of ringed spaces $(I_Y, \lambda_n): Y^{(n)} \to Y$ right inverse to the canonical morphism $(I_Y, \phi_{0n}): Y \to Y^{(n)}$.
\end{env}

\begin{proposition}[16.1.5]
\label{IV.16.1.5}
Let $f = (\psi, \theta): Y \to X$ be an immersion of preschemes. We have:
\begin{enumerate}
  \item[(i)] $\shGr_\bullet(f)$ is a quasi-coherent graded $\sh{O}_Y$-algebra.
\oldpage[IV-4]{7}
  \item[(ii)] The $Y^{(n)}$ are preschemes, canonically isomorphic to subpreschemes of $X$.
  \item[(iii)] Every homomorphism of sheaves of rings $\lambda_n: \sh{O}_Y \to \sh{O}_{Y^{(n)}}$, right inverse to the augmentation homomorphism $\phi_{0n}$, makes the $\sh{O}_{Y^{(n)}}$ and $\sh{O}_{Y^{(k)}}$ for $k\leq n$ quasi-coherent $\sh{O}_Y$-algebras;
  the structure of $\sh{O}_Y$-module deducted from the preceding structures on the $\shGr_k(f)$ for $k \leq n$ coincide with the ones defined in \sref{IV.16.1.2}.
\end{enumerate}
\end{proposition}

\begin{proof}
(i) Since the question is local on $X$ and $Y$, we can reduce to the case where $Y$ is a closed subpreschemes of $X$ defined by an quasi-coherent ideal $\sh{I}$ of $\sh{O}_X$;
since $\sh{O}_Y$ is the restriction to $Y$ of $\sh{O}_X/\sh{I}$ the assertion (i) is evident, and $Y^{(n)}$ is the closed subprescheme of $X$ defined by the quasi-coherent ideal $\sh{I}^{n+1}$ of $\sh{O}_X$.
Finally, to prove (iii) we notice that the data of $\lambda_n$ makes the ideal $\sh{I}/\sh{I}^n$ of the augmentation $\phi_{0n}$ and their quotients $\sh{I}/\sh{I}^{k+1} (1\leq k \leq n)$ $\sh{O}_Y$-modules, and it suffices to prove by induction on $k$ that the $\sh{I}/\sh{I}^{k+1}$ are quasi-coherent $\sh{O}_Y$-modules and the structure of quotient $\sh{O}_Y$-module induced on $\sh{I}^k/\sh{I}^{k+1}$ is the same as defined on \sref{IV.16.1.2}.
The second assertion is immediate, $\sh{I}^k/\sh{I}^{k+1}$ being killed by $\sh{I}/\sh{I}^{n+1}$;
the first result, by induction on $k$, is trivial for $k=1$ and for $\sh{I}/\sh{I}^{k+1}$ being an extension of $\sh{I}/\sh{I}^{k}$ by $\sh{I}^k/\sh{I}^{k+1}$ \hyperref[section:III.1.4.17]{(\textbf{III}, 1.4.17)}.
\end{proof}

\begin{corollary}[16.1.6]
\label{IV.16.1.6}
Under the general hypothesis of \sref{IV.16.1.5}, if the immersion $f$ is locally of finite presentation then the $\shGr_n(f)$ are quasi-coherent $\sh{O}_Y$-modules of finite type.
\end{corollary}

\begin{proof}
Indeed, with the notation from the proof of \sref{IV.16.1.5}, $\sh{I}$ is an ideal of finite type of $\sh{O}_X$ \sref{IV.1.4.7}, therefore the $\sh{I}^n/\sh{I}^{n+1}$ are $\sh{O}_Y$-modules of finite type, hence the conclusion.
\end{proof}

\begin{corollary}[16.1.7]
\label{IV.16.1.7}
Under the general hypotheses of \sref{IV.16.1.5}, let $g:X \to Y$ be a morphism of preschemes, left inverse to $f$.
Therefore, for every $n$, the composite morphism $(I, \lambda_n): Y^{(n)}\xrightarrow{h_n} X \xrightarrow{g} Y$ defines a homomorphism of sheaves of rings $\lambda_n: \sh{O}_Y \to \sh{O}_{Y^{(n)}}$ right inverse to the augmentation $\phi_{0n}$, making $\sh{O}_{Y^{(n)}}$ a quasi-coherent $\sh{O}_Y$-algebra;
via these homomorphisms, the transition homomorphism $\phi_{nm}:\sh{O}_{Y^{(m)}} \to \sh{O}_{Y^{(n)}}$ ($n\leq m$) are homomorphisms of $\sh{O}_Y$-algebras. 
Also, if $g$ is locally of finite type, then the $\sh{O}_{Y^{(n)}}$ are quasi-coherent $\sh{O}_Y$-modules of finite type.
\end{corollary}

\begin{proof}
The first assertion is an immediate result from the definitions and \sref{IV.16.1.5}.
On the other hand, if $g$ is locally of finite type, then $f$ is locally of finite presentation \sref{IV.1.4.3}[(v)];
the $\shGr_n(f)$ being then quasi-coherent $\sh{O}_Y$-modules of finite type by \sref{IV.16.1.6}, the same goes for the $\sh{O}_Y$-modules $\sh{I}/\sh{I}^{n+1}$, being extensions of a finite number of the $\shGr_k(f)$ \sref[III]{III.1.4.17}.
\end{proof}

\begin{proposition}[16.1.8]
\label{IV.16.1.8}
Let $X$ be a locally Noetherian prescheme, $j:Y \to X$ an immersion;
Then the $Y^{(n)}$ are locally Noetherian preschemes, the $\shGr_n(j)$ are coherent $\sh{O}_Y$-modules and the $\shGr_\bullet(j)$ is a coherent sheaf of rings over the space $Y$.
\end{proposition}

\begin{proof}
Everything is local on $X$ and $Y$, so we reduce to the case where $X$ is affine and $j$ is a closed immersion and therefore all the assertions are evident except for the last, which results from the fact that if $A$ is a Noetherian ring and $\mathfrak{I}$ is an ideal of $A$, $\gr_\mathfrak{I}^\bullet(A)$ is a Noetherian ring, taking into account the exactness of the functor $\psi^*$ and \sref[0]{0.5.3.7}.
\end{proof}

\begin{proposition}[16.1.9]
\label{IV.16.1.9}
\oldpage[IV-4]{8}
Let $X$ be a prescheme, $j: Y \to X$ an immersion locally of finite presentation, $y$ a point of $Y$. The following conditions are equivalent:
\begin{enumerate}
  \item[(a)] There exists an open neighborhood $U$ of y in $Y$ such that $j|U$ is a homeomorphism of $U$ onto an open set of $X$.
  \item[(b)] There is an integer $n>0$ such that the canonical homomorphism
  \[
    (\phi_{n-1,n})_y: \sh{O}_{Y^{(n)},y} \to \sh{O}_{Y^{(n-1)},y}
  \]
  is bijective.
  \item[(c)] There is an integer $n>0$ such that $(\shGr_n(j))_y = 0$.
  
  In addition, if the integer $n$ satisfies \emph{(b)} or \emph{(c)}, then there is a neighborhood $V$ of $y$ in $Y$ such that $\shGr_m(j)|V = 0$ for $m \geq n$ and that $\phi_{nm}|V: \sh{O}_{Y^{(m)}}|V \to \sh{O}_{Y^{(n)}}|V$ is bijective for $m \geq n$. 
\end{enumerate}
\end{proposition}

\begin{proof}
The question being local on $Y$, we can restrict ourselves to the case where $j$ is a closed immersion, $Y$ being defined by a quasi-coherent ideal \emph{of finite type} $\mathfrak{I}$ of $\sh{O}_X$.
The equivalence of (b) and (c), for a given $n$, is immediate;
also, since $\sh{I}^n/\sh{I}^{n+1}$ is an $\sh{O}_X$-module of finite type, there is an open neighborhood $U$ of $y$ in $X$ such that $\sh{I}^n|U = \sh{I}^{n+1}|U$ \sref[0]{0.5.2.2}, so we also have $\sh{I}^n|U = \sh{I}^m|U$ for $m \geq n$ proving the last assertions.
To prove that (a) implies (b), we can restrict ourselves to the cases where the underlying space of $Y$ is equal to the underlying space of $X$ and where $\sh{I}$ is generated by a finite number of sections over $X$:
since $\sh{I}$ is contained in the nilradical $\sh{N}$ of $\sh{O}_X$ \sref[I]{I.5.1.2}, it is now nilpotent which proves b).
Finally, to prove that (b) implies (a), we can restrict ourselves to the case where $\sh{I}^n = \sh{I}^m$; 
therefore, for every $y \in Y$, since $\sh{I}_y \subset \mathfrak{m}_y$, maximal ideal of $\sh{O}_{X,y}$, we must have $\sh{I}^n_y = 0$ because of Nakayama's lemma, since $\sh{I}_y$ is an ideal of finite type.
The set of $x \in X$ such that $\sh{I}^n_x = 0$ is an open $U$ of $X$ contained in $Y$ \sref[0]{0.5.2.2};
since on the other hand $\sh{I}_x \neq 0$ for $x \notin Y$, we must have $U = Y$.
\end{proof}

\begin{corollary}[16.1.10]
\label{IV.16.1.10}
For a restriction of the immersion $j$ to an open neighborhood of $y$ in $Y$ to be an open immersion (in other words, for $j$ to be a \emph{local isomorphism} on the point $y$), it is necessary and sufficient that $(\shGr_1(j))_y = (\sh{N}_{Y/X})_y = 0$.
\end{corollary}

\begin{proof}
The condition is clearly necessary, and the previous reasoning applied to $n=1$ proves that it is sufficient.
\end{proof}

\begin{remark}[16.1.11]
\label{IV.16.1.11}
\begin{enumerate}
  \item[(i)] Under the conditions of the definition \sref{IV.16.1.1}, the projective limit of the projective system $(\sh{O}_{Y^{(n)}}, \phi_{nm})$ of sheaves of rings over $Y$ is called the \emph{normal invariant of infinite order} of $f$, and sometimes denoted by $\sh{O}_{Y^{(\infty)}}$.
  When $X$ is a locally noetherian prescheme, $j:Y \to X$ a closed immersion, $Y$ then is a closed subprescheme of $X$ defined by a coherent ideal $\sh{I}$ and $\sh{O}_{Y^{(\infty)}}$ is exactly the \emph{formal completion} of $\sh{O}_X$ along $Y$ \sref[I]{I.10.8.4}, and $Y^{(\infty)} = (Y, \sh{O}_{Y^{(\infty)}})$ is the formal prescheme that is the \emph{completion} of $X$ along $Y$ \sref[I]{I.10.8.5}.
  In all cases, we could say that $Y^{(\infty)}$ is the \emph{formal neighborhood} of $Y$ in $X$ (via the morphism $f$).
  In the particular case we have just considered, it is the formal prescheme that is the inductive limit of the infinitesimal neighborhoods of order $n$.
  \item[(ii)] Note that for a morphism of preschemes $f=(\psi, \theta): Y \to X$, it can happen that the homomorphism $\theta^\#:\psi^*(\sh{O}_X) \to \sh{O}_Y$ is surjective without $f$ being a local 
\oldpage[IV-4]{9}
  immersion and without $f$ being injective.
  We have an example by taking $Y$ to be a sum of preschemes $Y_\lambda$ all isomorphic to $\Spec(\sh{O}_x)$, where $x \in X$, ad taking $f$ to be the morphism equal to the canonical morphism in each of the $Y_\lambda$.
\end{enumerate}
\end{remark}

\subsection{Functorial properties of the normal invariants of an immersion}
\label{IV.16.2}

\begin{env}[16.2.1]
\label{IV.16.2.1}
Let $f = (\psi, \theta): Y \to X$ and $f' = (\psi', \theta'): Y' \to X'$ by two morphisms of ringed spaces such that $\theta^\#$ and $\theta'^\#$ are surjective;
consider a commutative diagram of morphisms of ringed spaces
\[
  \label{IV.16.2.1.1}
  \xymatrix{
    Y \ar[r]^f & X \\
    Y'\ar[r]_{f'} \ar[u]^u & X'\ar[u]_v  
  }
  \tag{16.2.1.1}
\]

Let $u = (\rho, \lambda), v = (\sigma, \mu)$. 
We have $\rho^*(\psi^*(\sh{O}_X)) = \psi'^*(\sigma^*(\sh{O}_X))$ and as a result a commutative diagram of homomorphisms of sheaves of rings over $Y'$
\[
  \xymatrix{
    \rho^*(\psi^*(\sh{O}_X)) = \psi'^*(\sigma^*(\sh{O}_X)) \ar[r]^-{\psi'^*(\mu^\#)}\ar[d]_{\rho^*(\theta^\#)} & \psi'^*(\sh{O}_{X'}) \ar[d]^{\theta'^\#} \\
    \rho^*(\sh{O}_Y)\ar[r]_-{\lambda^\#}  & \sh{O}_{Y'}  
  }
\]
from which we conclude, if $\sh{I}$ and $\sh{I'}$ are the kernels of $\theta^\#$ and $\theta'^\#$, that we have $\psi'^*(\mu^\#)(\rho^*(\sh{I})) \subset \sh{I'}$, having in mind the exactness of the functor $\rho^*$.
We deduce that, for every integer $n$, $\psi'^*(\mu^\#)(\rho^*(\sh{I}^n)) \subset \sh{I'}^n$, which shows that $\psi'^*(\mu^\#)$ defines, passing to the quotients, a homomorphism of sheaves of rings
\[
  \label{IV.16.2.1.2}
  \nu_n: \rho^*(\psi^*(\sh{O}_X)/\sh{I}^{n+1}) \to \psi'^*(\sh{O}_{X'})/\sh{I'}^{n+1}
  \tag{16.2.1.2}
\]
and therefore a morphism of ringed spaces $w_n = (\rho, \nu_n): Y'^{(n)} \to Y^{(n)}$ (which, for $n = 0$, is none other than $u$).
It results immediately from this definition that the diagrams
\[
  \xymatrix@R=1pc{
    Y^{(n)} \ar[r]^-{h_{mn}} & Y^{(m)} \ar[r]^-{h_m} & X \\
    & & & (n \leq m) \\
    Y'^{(n)} \ar[r]_-{h'_{mn}} \ar[uu]^-{w_n} & Y'^{(m)} \ar[r]_-{h'_m} \ar[uu]^-{w_m} & X' \ar[uu]_-v \\
  }
\]
(where the horizontal arrows are the canonical morphisms \sref{IV.16.1.2}) are commutative.

By passage to the quotients via the morphisms \sref{IV.16.2.1.2}, and taking into
\oldpage[IV-4]{10}
account the exactness of the functor $\rho^*$, we obtain a di-homomorphism of graded algebras (relative to the morphism $\lambda^\#: \rho^*(\sh{O}_Y) \to \sh{O}_{Y'}$)
\[
  \label{IV.16.2.1.3}
  \gr(u): \rho^*(\shGr_\bullet(f)) \to \shGr_\bullet(f')
  \tag{16.2.1.3}
\]
(or, if you like, a $\rho$-morphism \sref[0]{0.3.5.1} $\shGr_\bullet(f) \to \shGr_\bullet(f')$), and in particular a di-homomorphism of conormal sheafs
\[
  \gr_1(u): \rho^*(\shGr_1(f)) \to \shGr_1(f').
\]

It is also immediate that these homomorpisms give rise to a commutative diagram
\[
  \label{IV.16.2.1.4}
  \xymatrix{
    \rho^*(\bb{S}_{\sh{O}_Y}^\bullet(\shGr_1(f)) ) \ar[r] \ar[d]_-{\bb{S}(\gr_1(u))} & \rho^*(\shGr_\bullet(f)) \ar[d]^-{\gr(u)}\\
    \bb{S}_{\sh{O}_Y}^\bullet(\shGr_1(f')) \ar[r] & \shGr_\bullet(f')
  }
  \tag{16.2.1.4}
\]
where the horizontal arrow are the canonical morphisms \sref{IV.16.1.2.2}.

Finally, if we have a commutative diagram of morphisms of ringed spaces
\[
  \xymatrix{
    Y \ar[r]^{f} & X \\
    Y' \ar[r]_{f'} \ar[u]^u & X' \ar[u]_v\\
    Y'' \ar[r]_{f''} \ar[u]^{u'} & X'' \ar[u]_{v'}\\
  }
\]
where $f'' = (\psi'', \theta'')$ is such that $\theta''^\#$ is surjective, and if $w_n'$ and $w_n''$ are defined from $u'$, $v'$ for one and $u'' = u \circ u'$, $v'' = v \circ v'$ for the other, we have $w_n'' = w_n \circ w_n'$, which follows immediately from the definitions and from \sref[0]{0.3.5.5};
we have also $\gr(u'') = \gr(u') \circ \rho'^*(\gr(u))$ if $u' = (\rho', \lambda')$.
Therefore we can say that $Y^{(n)}$ and $\shGr_\bullet(f)$ \emph{depend functorially} on $f$. 
\end{env}

\begin{proposition}[16.2.2]
\label{IV.16.2.2}
With the notation and hypotheses of \sref{IV.16.2.1}, suppose also that $f$, $f'$, $u$ and $v$ are morphisms of preschemes. We have:
\begin{enumerate}
  \item[(i)] The morphisms $w_n:Y'^{(n)} \to Y^{(n)}$ are morphisms of preschemes.
  \item[(ii)] If $Y' = Y \times_X X'$, $u$ and $f'$ the canonical projections, and if $f$ is an immersion or if $v$ is flat, we have $Y'^{(n)} = Y^{(n)} \times_X X'$.
  \item[(iii)] If $Y' = Y \times_X X'$ and if $v$ is flat (resp. if $f$ is an immersion), the homomorphism 
  \[
    \Gr(u) = \gr(u)\otimes I : \shGr_\bullet(f)\otimes_{\sh{O}_Y}\sh{O}_{Y'} \to \shGr_\bullet(f')
  \]
  is bijective (resp. surjective).
\end{enumerate}
\end{proposition}

\begin{proof}
\begin{enumerate}
  \item[(i)] The hypothesis grant us immediately that, for every $y' \in Y'$, $\rho_{y'}^*(\theta_{\psi'(y')}^\#)$ is a \emph{local} homomorphism \sref[I]{I.1.6.2}, so $w_n$ is a morphism of preschemes \sref[I]{I.2.2.1}.
  \oldpage[IV-4]{11}
  \item[(ii) and (iii)] If $f$ is an immersion, we can restrict ourselves to the case where $f$ is a closed immersion, $Y$ being defined by a quasi-coherent ideal $\sh{I}$ of $\sh{O}_X$ and $Y^{(n)}$ by the ideal $\sh{I}^{n+1}$;
  the assertions results from \sref[I]{I.4.4.5}.

  Second, suppose that $v$ is flat;
  we can restrict ourselves to the case where $X = \Spec(A)$, $Y = \Spec(B)$, $X' = \Spec(A')$ are affines, $A'$ being a flat $A$-module;
  so $Y' = \Spec(B')$ where $B' = B \otimes_A A'$;
  in addition, if $\mathfrak{I}$ is the kernel of the homomorphism $A \to B$, the kernel $\mathfrak{I'}$ of $A' \to B'$ is identified with $\mathfrak{I}\otimes_A A'$ by flatness, and $\sh{I}'^n/\sh{I'}^{n+1}$ is equal to
  \begin{align*}
    \psi'^*(\sigma^*((\mathfrak{I}^n/\mathfrak{I}^{n+1})^\sim) \otimes_{\sigma^*(\sh{O}_X)} \sh{O}_{X'}) =& \\
    \psi'^*(\sigma^*((\mathfrak{I}^n/\mathfrak{I}^{n+1} ))^\sim) \otimes_{\psi'^*(\sigma^*(\sh{O}_X))} &\psi'^*(\sh{O}_{X'}) = \rho^*(\sh{I}^n/\sh{I}^{n+1})\otimes_{\rho^*(\psi^*(\sh{O}_X))} \psi'^*(\sh{O}_{X'}) 
  \end{align*}
  and in particular for $n = 0$, we have
  \[
    \sh{O}_{Y'} = \rho^*(\sh{O}_Y) \otimes_{\rho^*(\psi^*(\sh{O}_X))} \psi'^*(\sh{O}_{X'})
  \]
  from which we have canonical isomorphism of $\sh{I}'^n/\sh{I'}^{n+1}$ with
  \[
    \rho^*(\sh{I}^n/\sh{I}^{n+1})\otimes_{\rho^*(\sh{O}_Y)} \sh{O}_{Y'} = (\sh{I}^n/\sh{I}^{n+1}) \otimes_{\sh{O}_Y} \sh{O}_{Y'}
  \]
  which proves (iii).
  Let now $C_n = \Gamma(Y, \sh{O}_{Y^{(n)}}), C'_n = \Gamma(Y', \sh{O}_{Y'^{(n)}})$.
  As $Y^{(n)}$ and $Y'^{(n)}$ are affine schemes \sref{IV.16.1.5}, the kernel $\mathfrak{K}_n$ (resp. $\mathfrak{K}'_n$) of the homomorphism $C_n \to C_{n-1}$ (resp. $C'_n \to C'_{n-1}$) is $\Gamma(Y, \sh{I}^n/\sh{I}^{n+1})$ (resp. $\Gamma(Y, \sh{I}'^n/\sh{I'}^{n+1})$);
  therefore we can deduce from the preceding results that $\mathfrak{K}'_n = \mathfrak{K}_n \otimes_A A'$.
  Now, we have a commutative diagram
  \[
    \xymatrix{
      0 \ar[r] & \mathfrak{K}_n \ar[d]^-r\ar[r]  \otimes_A A' & C_n \otimes_A A' \ar[d]^-{s_n}\ar[r]  & C_{n-1} \otimes_A A' \ar[d]^-{s_{n-1}}\ar[r]  & 0 \\
      0 \ar[r] & \mathfrak{K}'_n \ar[r]   & C'_n \ar[r]  & C'_{n-1} \ar[r]  & 0
    }
  \]
  where the vertical arrow of the left is bijective and the two lines are exact ($A'$ being a flat $A$-module).
  We deduce by induction that $s_n$ is bijective for every $n$, because it's true by hypothesis for $n = 0$, and we deduce by application of the five lemma for all $n$.
  That proves the second assertion of (ii).
\end{enumerate}
\end{proof}

\begin{corollary}[16.2.3]
\label{IV.16.2.3}
Let $g: X \to Y$, $u: Y' \to Y$ be two morphisms of preschemes, $X' = X \times_Y Y'$, $g': X' \to Y'$ and $v: X' \to X$ by the canonical projections. Let $f: Y \to X$ by a $Y$-section of $X$ (and therefore an immersion), $f' = f_{(Y')}: Y' \to X'$ the $Y'$-section of $X'$ deduced from $f$ by the base change $u$.
We have:
\begin{enumerate}
  \item[(i)] The morphism $w_n:{Y'}_{f'}^{(n)} \to Y_f^{(n)}$ corresponding to $f$, $f'$, $u$, $v$ \sref{IV.16.2.1} and the canonical morphism $h'_n: {Y'}_{f'}^{(n)} \to X'$ identifies $ {Y'}_{f'}^{(n)}$ with the product $Y_f^{(n)} \times_X X'$.
  \item[(ii)] If we endow $\sh{O}_{Y_f^{(n)}}$ (resp. $\sh{O}_{{Y'}_{f'}^{(n)}}$) with the structure of $\sh{O}_Y$-algebra defined by $g$ (resp. with the structure of $\sh{O}_{Y'}$-algebra defined by $g'$ ) \sref{IV.16.1.5}[(iii)],
  % The original citation is IV.16.1.6, but he clearly meant 16.1.5 item (iii)
  the homomorphism of $\sh{O}_{Y'}$-algebras
  \[
    \label{IV.16.2.3.1}
      \rho^*(\sh{O}_{Y_f^{(n)}})\otimes_{\sh{O}_Y} \sh{O}_{Y'} \to \sh{O}_{{Y'}_{f'}^{(n)}}
    \tag{16.2.3.1}
  \]
  \oldpage[IV-4]{12}
  deduced from the homomorphism $\nu_n$ \sref{IV.16.2.1.2} is bijective.
  Also, the homomorphism of $\sh{O}_{Y'}$-module
  \[
    \label{IV.16.2.3.2}
    \Gr_1(u): \shGr_1(f)\otimes_{\sh{O}_Y} \sh{O}_{Y'} \to \shGr_1(f')
    \tag{16.2.3.2}
  \]
  is bijective.
 \end{enumerate} 
\end{corollary}

\begin{proof}
\begin{enumerate}
  \item[(i)] Let us first note that $f': Y' \to X'$ and $u: Y' \to Y$  identifies $Y'$ with the product $Y \times_X X'$ (via the structural morphisms $f:Y \to X$ and $v: X' \to X$) \sref{IV.14.5.12.1}.
  The conclusion of (i) now follows from \sref{IV.16.2.2}[(ii)], the morphism $g$ being an immersion.
  \item[(ii)] The commutative diagram
  \[
  \xymatrix{
    Y_f^{(n)} \ar[d]^{h_n}  & {Y'}_{f'}^{(n)} \ar[d]^-{h'_n} \ar[l]^{w_n}\\  
    X         \ar[d]^{g}    & X'              \ar[d]^{g'} \ar[l]^v \\  
    Y                       & Y' \ar[l]^u \\  
  }
  \]
  identifies ${Y'}_{f'}^{(n)}$ with the product $Y_f^{(n)} \times_X X'$, so \sref[I]{I.3.3.9} it identifies (via the morphisms $g'\circ h'_n$ and $w_n$) ${Y'}_{f'}^{(n)}$ to the product $Y_f^{(n)} \times_Y Y'$.
  Since $Y_f^{(n)}$ (resp. ${Y'}_{f'}^{(n)}$) is the affine prescheme over $Y$ (resp. over $Y'$) associated with the $\sh{O}_Y$-algebra $\sh{O}_{Y_f^{(n)}}$ (resp. to the $\sh{O}_{Y'}$-algebra $\sh{O}_{{Y'}_{f'}^{(n)}}$), the fact that the canonical homomorphism \sref{IV.16.2.3.1} is bijective results from \sref[II]{II.1.5.2}.
  Finally, the canonical homomorphism \sref{IV.16.2.3.1} is compatible with the augmentations $\sh{O}_{Y_f^{(n)}} \to \sh{O}_Y$ and $\sh{O}_{{Y'}_{f'}^{(n)}} \to \sh{O}_{Y'}$;
  since $\sh{O}_{Y_f^{(n)}}$ is a direct sum (as an $\sh{O}_Y$-module) of $\sh{O}_Y$ and the augmentation ideal $\sh{I}/\sh{I}^{n+1}$, we can therefore see that the canonical homomorphism \sref{IV.16.2.3.1}, restricted to $\sh{I}/\sh{I}^{n+1} \otimes_{\sh{O}_Y} \sh{O}_{Y'}$, is a bijection of the latter onto $\sh{I}'/\sh{I}'^{n+1}$. For $n=1$ this shows that $\Gr_1(u)$ is bijective.
\end{enumerate}
\end{proof}

We note that, under the hypothesis of \sref{IV.16.2.3}, the homomorphisms $\Gr_n(u)$ are \emph{surjective} in view of the above, but are not bijective in general for $n \geq 2$. However:

\begin{corollary}[16.2.4]
\label{IV.16.2.4}
Under the hypothesis of \sref{IV.16.2.3}, suppose that $u: Y' \to Y$ is a flat morphism (resp. that the $\shGr_n(f)$ are flat $\sh{O}_Y$-modules for $n \leq m$).
Then the homomorphism 
\[
  \Gr_n(u):\shGr_n(f) \otimes_{\sh{O}_Y}\sh{O}_{Y'} \to \shGr_n(f')
\]
is bijective for all $n$ (resp. for $n \leq m $).

\end{corollary}

\begin{proof}
If $u$ is flat, then we deduce by base change that the same is true for $v:X' \to X$, and we already know in this case that $\Gr(u)$ is bijective \sref{IV.16.2.2}[(iii)].
If the $\shGr_n(f)$ are flat for $n\leq m$, if first see by induction on $n$ that the same goes for $\sh{I}/\sh{I}^{n+1}$ for $n\leq m$, because of the exact sequences
  \[
    \xymatrix{
      0 \ar[r] & \sh{I}^n/\sh{I}^{n+1} \ar[r] & \sh{I}/\sh{I}^{n+1} \ar[r] & \sh{I}/\sh{I}^{n} \ar[r] & 0
    }
  \]
  \oldpage[IV-4]{13}
  \sref[0]{0.6.1.2};
  in addition, we have the commutative diagram
  \[
    \xymatrix{
      0 \ar[r] & (\sh{I}^n/\sh{I}^{n+1}) \otimes_{\sh{O}_Y}\sh{O}_{Y'} \ar[d]\ar[r] &( \sh{I}/\sh{I}^{n+1}) \otimes_{\sh{O}_Y}\sh{O}_{Y'} \ar[d]\ar[r] & (\sh{I}/\sh{I}^{n}) \otimes_{\sh{O}_Y}\sh{O}_{Y'} \ar[d]\ar[r] & 0 \\
      0 \ar[r] & \sh{I'}^n/\sh{I'}^{n+1} \ar[r] & \sh{I'}/\sh{I'}^{n+1} \ar[r] & \sh{I'}/\sh{I'}^{n} \ar[r] & 0
    }
  \]
  in which the lines are exact (the first from flatness \sref[0]{0.6.1.2}) and the two last vertical arrows are bijectives in the light of \sref{IV.16.2.2}[(ii)];
  from this we conclude.
\end{proof}

\begin{remarks}[16.2.5]
\label{IV.16.2.5}
\begin{enumerate}
  \item[(i)] The reasoning of \sref{IV.16.2.2}[(i)] still applies to \sref{IV.16.2.1.1} when these are morphisms of \emph{locally ringed spaces} \sref[I]{I.1.8.2}.
  \item[(ii)] In \sref{IV.16.2.2}[(ii)], the conclusion is no longer necessairly valid if we only suppose that $v$ and $f$ are morphisms of preschemes ($f$ verifying the condition of \sref{IV.16.1.1}).
  For example (with the notations of the demonstration of \sref{IV.16.2.2}[(ii)]), it can happen that $\mathfrak{I} = 0$ but the kernel $\mathfrak{I}'$ of $A' \to B' = B \otimes_A A' $ is not zero and that $B' \neq 0$, in which case we have $Y^{(n)} = Y$ for all $n$, but ${Y'}^{(n)} \neq Y'$.
  We have an example of this taking $A = \bb{Z}$, $B = \bb{Q}$, $A' = \prod_{h = 1}^\infty (\bb{Z}/m^h\bb{Z})$ where $m>1$.
\end{enumerate}
\end{remarks}

\begin{env}[16.2.6]
\label{IV.16.2.6}
Consider the particular case of \sref{IV.16.2.1.1} where $X' = X$, $v$ being the identity, $X$ a prescheme, $Y$ a subprescheme of $X$, $Y'$ a subprescheme of $Y$, $f$, $u$, $f' = f \circ u$ the canonical injections;
the di-homomorphism \sref{IV.16.2.1.3} gives us, by tensorization $\sh{O}_{Y'}$ over $\rho^*(\sh{O}_Y)$, a homomorphism of graded $\sh{O}_{Y'}$-algebras
\[
  \label{IV.16.2.6.1}
  u^*(\shGr_\bullet(f)) \to \shGr_\bullet(f').
  \tag{16.2.6.1}
\]
On the other hand, we identify $\sh{O}_Y$ to $\psi^*(\sh{O}_X)/\sh{I}_f$ and $\sh{O}_{Y'}$ to $\rho^*(\sh{O}_Y)/\sh{I}_u$;
since $\rho^*$ is an exact functor, we have $\rho^*(\sh{O}_Y) = \rho^*(\psi^*(\sh{O}_X))/\rho^*(\sh{I}_f) = \psi'^*(\sh{O}_X)/\rho^*(\sh{I}_f)$, and since $\sh{O}_{Y'}$ is moreover identified with $\psi'^*{\sh{O}_X}/\sh{I}_{f'}$, we see that $\sh{I}_u = \sh{I}_{f'}/\rho^*(\sh{I}_f)$.
We deduce that for every integer $n$ there is a canonical homomorphism $\sh{I}_{f'}^n/\sh{I}_{f'}^{n+1} \to \sh{I}_{u}^n/\sh{I}_{u}^{n+1}$, from which we have canonical morphism of graded $\sh{O}_{Y'}$-algebras
\[
  \label{IV.16.2.6.2}
  \shGr_\bullet(f') \to \shGr_\bullet(u).
  \tag{16.2.6.2}
\]
\end{env}

\begin{proposition}[16.2.7]
\label{IV.16.2.7}
Let $X$ be a prescheme, $Y$ a subprescheme of $X$, $Y'$ a subprescheme of $Y$, $j:X \to Y$ the canonical injection.
We have an exact sequence of conormal sheaves ($\sh{O}_{Y'}$-modules)
\[
  \label{IV.16.2.7.1}
  \xymatrix{
    j^*(\sh{N}_{Y/X}) \ar[r] & \sh{N}_{Y'/X} \ar[r] & \sh{N}_{Y'/Y} \ar[r] & 0
  }
  \tag{16.2.7.1}
\]
where the arrows are the degree $1$ components of the canonical homomorphisms \sref{IV.16.2.6.1} and \sref{IV.16.2.6.2}.
\end{proposition}

\begin{proof}
  The problem being local, we can restrict ourselves to the case where $X = \Spec(A)$, $Y = \Spec(B)$ and $Y = \Spec(A/\mathfrak{I})$, $Y' = \Spec(A/\mathfrak{K})$, $\mathfrak{I}$ and $\mathfrak{K}$ being ideals of $A$ such that $\mathfrak{I} \subset \mathfrak{K}$;
  everything reduces to seeing 
  \oldpage[IV-4]{14}
  that the sequence of canonical morphisms $\mathfrak{I}/\mathfrak{K}\mathfrak{I} \to \mathfrak{K}/\mathfrak{K}^2 \to (\mathfrak{K}\mathfrak{I} )/ (\mathfrak{K}\mathfrak{I} )^2 \to 0$ is exact, which is immediate given that the image of $\mathfrak{I}/\mathfrak{K}\mathfrak{I} $ in $\mathfrak{K}/\mathfrak{K}^2$ is $(\mathfrak{I} + \mathfrak{K}^2)/\mathfrak{K}^2$ and that $(\mathfrak{K}\mathfrak{I} )/ (\mathfrak{K}\mathfrak{I} )^2$ is identified with $\mathfrak{K}/ (\mathfrak{I} + \mathfrak{K}^2)$.
\end{proof}

It is easy to give examples where the sequence \sref{IV.16.2.7.1} extended on the right by $0$ is not exact;
with the previous notation, just take $A = k[T]$, $\mathfrak{I} = AT^2$, $\mathfrak{K} = AT$, because then $(\mathfrak{I} + \mathfrak{K}^2)/\mathfrak{K}^2 = 0$ and $\mathfrak{I}/\mathfrak{K}\mathfrak{I} \neq 0$.
See however \sref{IV.16.9.13} and \sref{IV.19.1.5} for some cases there the extended sequence is indeed exact.