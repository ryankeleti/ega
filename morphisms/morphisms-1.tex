\section{Affine morphisms}
\label{section-affine-morphisms}

\subsection{$S$-preschemes and $\mathcal{O}_S$-algebras}
\label{subsection-s-preschemes-algebras}

\begin{env}[1.1.1]
\label{2.1.1.1}
Let $S$ be a prescheme, $X$ an $S$-prescheme, and $f:X\to S$ its structure morphism.
We know \sref[0]{0.4.2.4} that the direct image $f_*(\OO_X)$ is an $\OO_S$-algebra, which we
\oldpage[II]{6}
denote $\sh{A}(X)$ when there is little chance of confusion; if $U$ is an open subset of $S$, then we have
\[
  \sh{A}(f^{-1}(U))=\sh{A}(X)|U.
\]
Similarly, for every $\OO_X$-module $\sh{F}$ (resp. every $\OO_X$-algebra $\sh{B}$), we write $\sh{A}(\sh{F})$ (resp. $\sh{A}(\sh{B})$) for the direct image $f_*(\sh{F})$ (resp. $f_*(\sh{B})$) which is an $\sh{A}(X)$-module (resp. an $\sh{A}(X)$-algebra) and not only an $\OO_S$-module (resp. an $\OO_S$-algebra).
\end{env}

\begin{env}[1.1.2]
\label{2.1.1.2}
Let $Y$ be a second $S$-prescheme, $g:Y\to S$ its structure morphism, and $h:X\to Y$ an $S$-morphism; we then have the commutative diagram
\[
  \xymatrix{
    X\ar[rr]^h\ar[rd]_f & &
    Y\ar[ld]^g\\
    & S.
  }
  \tag{1.1.2.1}
\]

We have by definition $h=(\psi,\theta)$, where $\theta:\OO_Y\to h_*(\OO_X)=\psi_*(\OO_X)$ is a homomorphism of sheaves of rings; we induce \sref[0]{0.4.2.2} a homomorphism of $\OO_S$-algebras $g_*(\theta):g_*(\OO_Y)\to g_*(h_*(\OO_X))=f_*(\OO_X)$, in other words, a homomorphism of $\OO_S$-algebras $\sh{A}(Y)\to\sh{A}(X)$, which we denote by $\sh{A}(h)$.
If $h':Y\to Z$ is a second $S$-morphism, then it is immediate that $\sh{A}(h'\circ h)=\sh{A}(h)\circ\sh{A}(h')$.
We havve thus define a \emph{contravariant functor $\sh{A}(X)$} from the category of $S$-preschemes to the category of $\OO_S$-algebras.

Now let $\sh{F}$ be an $\OO_X$-module, $\sh{G}$ an $\OO_Y$-module, and $u:\sh{G}\to\sh{F}$ be an $h$-morphism, that is \sref[0]{0.4.4.1} a homomorphism of $\OO_Y$-modules $\sh{G}\to h_*(\sh{F})$.
Then $g_*(u):g_*(\sh{G})\to g_*(h_*(\sh{F}))=f_*(\sh{F})$ is a homomorphism $\sh{A}(\sh{G})\to\sh{A}(\sh{F})$ of $\OO_S$-modules, which we denote by $\sh{A}(u)$; in addition, the pair $(\sh{A}(h),\sh{A}(u))$ form a \emph{di-homomorphism} from the $\sh{A}(Y)$-module $\sh{A}(\sh{G})$ to the $\sh{A}(X)$-module $\sh{A}(\sh{F})$.
\end{env}

\begin{env}[1.1.3]
\label{2.1.1.3}
If we fix the prescheme $S$, then we can consider the pairs $(X,\sh{F})$, where $X$ is an $S$-prescheme and $\sh{F}$ is an $\OO_X$-module, as forming a \emph{category}, by defining a \emph{morphism} $(X,\sh{F})\to(Y,\sh{G})$ as a pair $(h,u)$, where $h:X\to Y$ is an $S$-morphism and $u:\sh{G}\to\sh{F}$ is an $h$-morphism.
We can theen say that $(\sh{A}(X),\sh{A}(\sh{F}))$ is a \emph{contravariant functor} with values in the category whose objects are pairs consisting of an $\OO_S$-algebra and a module over that algebra, and the morphisms are the di-homomorphisms.
\end{env}

\subsection{Affine preschemes over a prescheme}
\label{subsection-affine-preschemes-over-a-prescheme}

