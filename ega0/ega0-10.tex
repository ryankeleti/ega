\section{Supplement on flat modules}
\label{section:supplement-on-flat-modules}

For the proofs that are missing in \hyperref[subsection:relations-between-flat-modules-and-free-modules]{(10.1)} and \hyperref[subsection:local-flatness-criteria]{(10.2)}, we refer the reader to Bourbaki, \emph{Alg. comm.}, chap.~II and III.

\subsection{Relations between flat modules and free modules}
\label{subsection:relations-between-flat-modules-and-free-modules}

\begin{env}[10.1.1]
\label{0.10.1.1}
Let $A$ be a ring, $\fk{J}$ an ideal of $A$, and $M$ an $A$-module;
for every integer $p\geq0$, we have a canonical homomorphism of $(A/\fk{J})$-modules
\begin{equation*}
\label{0.10.1.1.1}
    \vphi_p: (M/\fk{J}M) \otimes_{A/\fk{J}} (\fk{J}^p/\fk{J}^{p+1}) \to \fk{J}^pM/\fk{J}^{p+1}M\tag{10.1.1.1}
\end{equation*}
which is evidently \emph{surjective}.
We denote by $\gr(A)=\oplus_{p\geq0}\fk{J}^p/\fk{J}^{p+1}$ the graded ring associated to $A$ filtered by the $\fk{J}^p$, and by $\gr(M)=\oplus_{p\geq0}\fk{J}^pM/\fk{J}^{p+1}M$ the graded $\gr(A)$-module associated to $M$ filtered by the $\fk{J}^pM$;
we then have $\gr_p(A)=\fk{J}^p/\fk{J}^{p+1}$, and $\gr_p(M)=\fk{J}^pM/\fk{J}^{p+1}M$;
the $\vphi$ define a \emph{surjective} homomorphism of graded $\gr(A)$-modules
\begin{equation*}
\label{0.10.1.1.2}
    \vphi: \gr_0(M) \otimes_{\gr_0(A)} \gr(A) \to \gr(M).\tag{10.1.1.2}
\end{equation*}
\end{env}

\oldpage[0\textsubscript{III}]{18}
\begin{env}[10.1.2]
\label{0.10.1.2}
Suppose that \emph{one} of the following hypotheses is satisfied:
\begin{enumerate}[label=\emph{(\roman*)}]
    \item $\fk{J}$ is nilpotent;
    \item $A$ is Noetherian, $\fk{J}$ is contained in the radical of $A$, and $M$ is of finite type.
\end{enumerate}
Then the following properties are equivalent.
\begin{enumerate}[label=\emph{(\alph*)}]
    \item $M$ is a free $A$-module.
    \item $M/\fk{J}M=N\otimes_A(A/\fk{J})$ is a free $(A/\fk{J})$-module, and $\Tor_1^A(M,A/\fk{J})=0$.
    \item $M/\fk{J}M$ is a free $(A/\fk{J})$-module, and the canonical homomorphism \sref{0.10.1.1.2} is injective (and thus bijective).
\end{enumerate}
\end{env}

\begin{env}[10.1.3]
\label{0.10.1.3}
Suppose that $A/\fk{J}$ is a \emph{field} (in other words, that $\fk{J}$ is maximal), and that one of the hypotheses, (i) and (ii), of \sref{0.10.1.2} is satisfied.
Then the following properties are equivalent.
\begin{enumerate}[label=\emph{(\alph*)}]
    \item $M$ is a free $A$-module.
    \item $M$ is a projective $A$-module.
    \item $M$ is a flat $A$-module.
    \item $\Tor_1^A(M,A/\fk{J})=0$.
    \item The canonical homomorphism \sref{0.10.1.1.2} is bijective.
\end{enumerate}
\end{env}

This result can be applied, in particular, to the following two cases:
\begin{enumerate}[label=(\roman*)]
    \item $M$ is an \emph{arbitrary} module, over a local ring $A$ whose maximal ideal $\fk{J}$ is \emph{nilpotent} (for example, a local Artinian ring);
    \item $M$ is a module \emph{of finite type} over a \emph{local Noetherian} ring.
\end{enumerate}

\subsection{Local flatness criteria}
\label{subsection:local-flatness-criteria}

\begin{env}[10.2.1]
\label{0.10.2.1}
\end{env}
