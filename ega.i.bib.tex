
\label{bib-section}

%\renewcommand\refname{Bibliography}
\begin{thebibliography}{22}\oldpage{214}
\bibitem{1}
H. Cartan and C. Chevalley,
S\'eminaire de l'\'Ecole Normale Sup\'erieure,
8\textsuperscript{th} year (1955--56),
\emph{G\'eom\'etrie alg\'ebrique}.

\bibitem{2}
H. Cartan and S. Eilenberg,
\emph{Homological Algebra},
Princeton Math. Series (Princeton University Press),
1956.

\bibitem{3}
W. L. Chow and J. Igusa,
Cohomology theory of varities over rings,
\emph{Proc. Nat. Acad. Sci. U.S.A.},
t. XLIV (1958),
p. 1244--1248.

\bibitem{4}
R. Godement,
\emph{Th\'eorie des faisceaux},
Actual. Scient. et Ind.,
n\textsuperscript{o} 1252,
Paris (Hermann),
1958.

\bibitem{5}
H. Grauert,
Ein Theorem der analytischen Garbentheorie und die Moldulr\"aume komplexer Strukturen,
\emph{Publ. Math. Inst. Hautes \'Etudes Scient.},
n\textsuperscript{o}5,
1960.

\bibitem{6}
A. Grothendieck,
Sur quelques points d'alg\`ebre homologique,
\emph{T\^ohoku Math. Journ.},
t. IX (1957),
p. 119--221.

\bibitem{7}
A. Grothendieck,
Cohomology theory of abstract algebraic varieties,
\emph{Proc. Intern. Congress of Math.},
p. 103--118,
Edinburgh (1958).

\bibitem{8}
A. Grothendieck,
G\'eom\'etrie formelle et g\'eom\'etrie alg\'ebrique,
\emph{S\'eminaire Bourbaki},
11\textsuperscript{th} year (1958--59),
expos\'e 182.

\bibitem{9}
M. Nagata,
A general theory of algebraic geometry over Dedekind domains,
\emph{Amer. Math. Journ.}:
I,
t. LXXVIII,
p. 78--116 (1956);
II,
t. LXXX,
p. 382--420 (1958).

\bibitem{10}
D. G. Northcott,
\emph{Ideal theory},
Cambridge Univ. Press,
1953.

\bibitem{11}
P. Samuel,
\emph{Commutative algebra} (Notes by D. Herzig),
Cornell Univ.,
1953.

\bibitem{12}
P. Samuel,
\emph{Alg\`ebre locale},
M\'em. Sci. Math.,
n\textsuperscript{o}123,
Paris,
1953.

\bibitem{13}
P. Samuel and O. Zariski,
\emph{Commutative algebra},
2 vol.,
New York (Van Nostrand),
1968--60.

\bibitem{14}
J.-P. Serre,
Faisceaux alg\'ebriques coh\'erents,
\emph{Ann. of Math.},
t. LXI (1955),
p. 197--278.

\bibitem{15}
J.-P. Serre,
Sur la cohomologie des vari\'et\'es alg\'ebriques,
\emph{Journ. of Math.} (9),
t. XXXVI (1957),
p. 1--16.

\bibitem{16}
J.-P. Serre,
G\'eom\'etrie alg\'ebrique and g\'eom\'etrie analytique,
\emph{Ann. Inst. Fourier},
t. VI (1955--56),
p. 1--42.

\bibitem{17}
J.-P. Serre,
Sur la dimension homologique des anneaux et des modules noeth\'eriens,
\emph{Proc. Intern. Symp. on Alg. Number theory},
p. 176--189,
Tokyo--Nikko,
1955.

\bibitem{18}
A. Weil,
\emph{Foundations of algebraic geometry},
Amer. Math. Soc. Coll. Publ.,
n\textsuperscript{o}29,
1946.

\bibitem{19}
A. Weil,
Numbers of solutions of equations in finite fields,
\emph{Bull. Amer. Math. Soc.},
t. LV (1949),
p. 497--508.

\bibitem{20}
O. Zariski,
\emph{Theory and applications of holomorphic functions on algebraic varieties over arbitrary ground fields},
Mem. Amer. Math. Soc.,
n\textsuperscript{o}5 (1951).

\bibitem{21}
O. Zariski,
A new proof of Hilbert's Nullstellensatz,
\emph{Bull. Amer. Math. Soc.},
t. LIII (1947),
p. 362--368.

\bibitem{22}
E. K\"ahler,
Geometria Arithmetica,
\emph{Ann. di Mat.} (4),
t. XLV (1958),
p. 1--368.

\end{thebibliography}
