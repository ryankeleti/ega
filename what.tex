\begin{document}
\title{What this is}
\maketitle

\phantomsection
\label{section-phantom}


\noindent
{\em This section is written by the translators.}

\noindent
This is a community translation of Grothendieck's \'El\'ements de g\'eom\'etrie alg\'ebrique (EGA).
As it is a work in progress by multiple people, there will probably be a few mistakes---if you spot any then please do \href{https://github.com/ryankeleti/en.ega/issues}{let us know}\footnote{\url{https://github.com/ryankeleti/en.ega/issues}}.

\noindent
To contribute, please visit
\begin{center}
  \url{https://github.com/ryankeleti/en.ega}.
\end{center}

\noindent
{\em On est d\'esol\'es, Grothendieck.}

\section*{Notes from the translators}
Grothendieck's writing style in EGA is quite particular, most notably for its long sentence structure.
As translators, we have tried to give the best possible approximation of this style in English, resisting the temptation to ``streamline'' things in places where the language is more dense than usual.

Whenever a note is made by the translators, it will be prefaced by ``[Trans]''.

Along the margins we have provided the page numbers corresponding to the original text.
Due to EGA being a collection of volumes (one non-preliminary chapter per volume), the page numbers reset at every new chapter.
In addition, the preliminary section is stretched out over multiple volumes.
To combat this, we label the pages as
\begin{center}
  \textbf{X}~|~$n$,
\end{center}
referring to Chapter~X, page $n$.
In the case of the preliminaries, the preliminaries from volume~Y are denoted as \textbf{0\textsubscript{Y}}.

\section*{Mathematical notes}
EGA uses {\em prescheme} for what is now usually called a scheme, and {\em scheme} for what is now usually called a separated scheme.

\end{document}

