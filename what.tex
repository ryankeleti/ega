\ProvidesPackage{preamble}

\usepackage[utf8]{inputenc}
\usepackage[T1]{fontenc}
\usepackage{microtype}
\usepackage[left=0.75in,right=0.75in,top=0.75in,bottom=0.75in]{geometry}
\usepackage[all]{xy}
\usepackage{enumitem}
\usepackage{color}
\usepackage{soul}
\usepackage{fancyhdr}
\usepackage{mathtools}
\usepackage{amssymb}
\usepackage{amsthm}
\usepackage[charter,
            greekfamily=didot,
            uppercase=upright,
            greeklowercase=upright]{mathdesign}
\usepackage[compact]{titlesec}
\usepackage[colorlinks=true,hyperindex,citecolor=blue,linkcolor=magenta]{hyperref}
\usepackage{bookmark}
\usepackage[asterism]{sectionbreak}


%%%%%%%%%%%%%%
% formatting %
%%%%%%%%%%%%%%

\allowdisplaybreaks[1]
\binoppenalty=9999
\relpenalty=9999
\setitemize{nosep}

% for Chapter 0, Chapter I, etc.
% credit for ZeroRoman https://tex.stackexchange.com/questions/211414/
\newcommand{\ZeroRoman}[1]{\ifcase\value{#1}\relax 0\else\Roman{#1}\fi}
\renewcommand{\thechapter}{\ZeroRoman{chapter}}

%%%%%%%%%%%%%%%%%
% math commands %
%%%%%%%%%%%%%%%%%

% for easy changes to style
\newcommand{\sh}{\mathscr}         % sheaf font
\newcommand{\bb}{\mathbf}          % bold font
\newcommand{\cat}{\mathsf}         % category font
%
\newcommand{\rad}{\mathfrak{r}}    % radical
\newcommand{\nilrad}{\mathfrak{R}} % nilradical
\newcommand{\emp}{\varnothing}     % empty set
\newcommand{\vphi}{\phi}           % font switches \phi and \varphi, change if needed
\newcommand{\HH}{\mathrm{H}}       % cohomology
\newcommand{\dual}[1]{{#1}^\vee}   % dual
\renewcommand{\k}{\bb{k}}          % residue field
\newcommand{\K}{\cat{K}}           % category
\newcommand{\OO}{\sh{O}}           % structure sheaf
\newcommand{\F}{\sh{F}}            % sheaf F
\newcommand{\G}{\sh{G}}            % sheaf G

% operators
%\newcommand*{\sheafHom}{\mathscr{H}\text{\normalfont\kern -3pt {\calligra\large om}}\,}
\def\shHom{\sh{H}\textit{om}} % sheaf Hom
\def\Hom{{\mathop{\mathrm{Hom}}\nolimits}}
\def\Supp{{\mathop{\mathrm{Supp}}\nolimits}}
\def\img{{\mathop{\mathrm{im}}\nolimits}}
\def\Spec{{\mathop{\mathrm{Spec}}\nolimits}}

% if unsure of a translation
\newcommand{\unsure}[2][]{\hl{#2}\marginpar{#1}}
\newcommand{\completelyunsure}{\unsure{[\ldots]}}

% use to mark where original page starts
\newcommand{\oldpage}[1]{\marginpar{\textbf{#1}}\ignorespaces}

% special ref
\newcommand{\sref}[2]{\hyperref[#1-\arabic{chapter}.#2]{\normalfont{(#2)}}}

% ref prelim
\newcommand{\pref}[2]{\hyperref[#1-0.#2]{\normalfont{(\textbf{0}, #2)}}}

%% ref out of chapter
%\newcommand{\cref}[4]{\hyperref[#1-#2.#3]{\normalfont{(\textbf{#3}, #4)}}}

% currently this works as \begin{env}[optional rmk]{x.y.z}
\makeatletter
\newenvironment{env}[2][\@nil]{%
    \def\tmp{#1}%
    \ifx\tmp\@nnil
        \par\medskip\noindent\indent\textbf{(#2)}\rmfamily
    \else
        \par\medskip\noindent\indent\textit{\textbf{#1}}~\textbf{(#2)}.\,---\rmfamily
    \fi}
\makeatother

% use this for definitions, propositions, corollaries, etc.
\makeatletter
\newenvironment{envs}[2][\@nil]{
  \par\medskip\noindent\indent\textit{\textbf{#1}}~\textbf{(#2)}.\,---\itshape
}
\makeatother



\begin{document}
\title{What this is}
\maketitle

\phantomsection
\label{section-phantom}

\noindent
\emph{This section is written by the translators.}

\noindent
This is a community translation of Grothendieck's \'El\'ements de g\'eom\'etrie alg\'ebrique (EGA).
As it is a work in progress by multiple people, there will probably be a few mistakes---if you spot any then please do \href{https://github.com/ryankeleti/ega/issues}{let us know}\footnote{\url{https://github.com/ryankeleti/ega/issues}}.

\noindent
To contribute, please visit
\begin{center}
  \url{https://github.com/ryankeleti/ega}.
\end{center}

\noindent
\emph{On est d\'esol\'es, Grothendieck.}

\section*{Notes from the translators}
Grothendieck's writing style in EGA is quite particular, most notably for its long sentence structure.
As translators, we have tried to give the best possible approximation of this style in English, resisting the temptation to ``streamline'' things in places where the language is more dense than usual.

\sectionbreak

Whenever a note is made by the translators, it will be prefaced by ``[Trans]''.

\sectionbreak

Along the margins we have provided the page numbers corresponding to the original text.
Due to EGA being a collection of volumes (one non-preliminary chapter per volume), the page numbers reset at every new chapter.
In addition, the preliminary section is stretched out over multiple volumes.
To combat this, we label the pages as
\begin{center}
  \textbf{X}~|~$n$,
\end{center}
referring to Chapter~X, page $n$.
In the case of the preliminaries, the preliminaries from volume~Y are denoted as \textbf{0\textsubscript{Y}}.

\sectionbreak

Later volumes (such as EGAs III and IV) include errata for earlier chapters.
Where possible, we have used these to `update' our translation, and entirely replace whatever mistakes might have been in the original copies of EGAs I and II, though we will try to include footnotes pointing out when this has taken place.

\section*{Mathematical warnings}
EGA uses \emph{prescheme} for what is now usually called a scheme, and \emph{scheme} for what is now usually called a separated scheme.

\nocite{*}
\bibliography{the}
\bibliographystyle{amsalpha}

\end{document}

