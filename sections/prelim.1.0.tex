\documentclass[../main.tex]{subfiles}

\begin{document}

\begin{cx}{1.0.1}
\oldpage{11}All the rings considered in this Treatise will have a \emph{unit element}; all the modules
on such a ring will be assumed to be \emph{unitary}; the ring homomorphisms will always be
assumed to \emph{transform the unit element into a unit element}; unless otherwise stated,
a sub-ring of a ring $A$ will be assumed to \emph{contain the unit element of} $A$. We will
consider especially \emph{commutative} rings, and when we speak of a ring without
specification, it
will be implied that it is commutative. If $A$ is a ring not necessarily commutative, by
$A$-module we will we mean a left module, unless stated otherwise.
\end{cx}

\begin{cx}{1.0.2}
Let $A$, $B$ be two rings, not necessarily commutative, $\varphi:A\to B$ a homomorphism.
Any left (resp. right) $B$-module $M$ can be provided with a left (resp. right) $A$-module
structure by $a\cdot m=\varphi(a)\cdot m$ (resp. $m\cdot a=m\cdot\varphi(a)$); when it will
be necessary to distinguish $M$ as an $A$-module or a $B$-module, we will denote by
$M_{[\varphi]}$ the left (resp. right) $A$-module as defined. If $L$ is an $A$-module, then
a homomorphism $u:L\to M_{[\varphi]}$ is a homomorphism of commutative groups such that
$u(a\cdot x)=\varphi(a)\cdot u(x)$ for $a\in A$, $x\in L$; we will also say that it is a
$\varphi$-\emph{homomorphism} $L\to M$,
and that the pair $(\varphi,u)$ (or, by misuse of langauge, $u$)
is a \emph{di-homomorphism} of $(A,L)$ in $(B,M)$. The pairs $(A,L)$ formed by a ring $A$
and an $A$-module $L$ thus form a \emph{category} for which the morphisms are
di-homomorphisms.
\end{cx}

\begin{cx}{1.0.3}
Under the hypothesis of (1.0.2), if $\mf{J}$ is a left (resp. right) ideal of $A$, we
denote by $B\mf{J}$ (resp. $\mf{J}B$) the left (resp. right) ideal $B\varphi(\mf{J})$
(resp. $\varphi(\mf{J})B$) of $B$ generated by $\varphi(\mf{J})$; it is also the image
of the canonical homomorphism $B\otimes_A\mf{J}\to B$ (resp. $\mf{J}\otimes_A B\to B$)
of left (resp. right) $B$-modules.
\end{cx}

\begin{cx}{1.0.4}
If $A$ is a (commutative) ring, $B$ a non necessarily commutative ring, the data of
a structure of an $A$-\emph{algebra} on $B$ is equivalent to the data of a ring
homomorphism $\varphi:A\to B$ such that $\varphi(A)$ is contained in the center of $B$.
For all ideals $\mf{J}$ of $A$, $\mf{J}B=B\mf{J}$ is then a two-sided ideal of $B$, and
for every $B$-module $M$, $\mf{J}M$ is then a $B$-module equal to $(B\mf{J})M$.
\end{cx}

\begin{cx}{1.0.5}
We will not return to the notions of \emph{module finite type} and
\emph{algebra} (commutative) \emph{of finite type}; to say that an $A$-module $M$
is of finite type means that there exists \oldpage{12}an exact sequence $A^p\to M\to 0$. We say that
an $A$-module $M$ admits a \emph{finite presentation} if it is isomorphic to the cokernel
of a homomorphism $A^p\to A^q$, in other words, there exists an exact sequence
$A^p\to A^q\to M\to 0$. We note that for a \emph{Noetherian} ring $A$, every $A$-module
of finite type admits a finite presentation.

Let us recall that an $A$-algebra $B$ is called \emph{integral} over $A$ if every element
in $B$ is a root in $B$ of a monic polynomial with coefficients in $A$; equivalently, every
element of $B$ is contained in a subalgebra of $B$ which is an $A$-\emph{module of finite type}.
When this is so, and $B$ is commutative, the subalgebra of $B$ generated by a finite part of
$B$ is an $A$-module of finite type; for a commutative algebra $B$ to be integral and of finite
type over $A$, it is necessary and therefore sufficient that $B$ be an $A$-module of finite
type; we also say that $B$ is an \emph{integral} $A$-\emph{algebra of finite type} (or
simply \emph{finite} if there is no confusion). It will be observed that in these definitions,
it is not assumed that the homomorphism $A\to B$ defining the structure of an $A$-algebra
is injective.
\end{cx}

\begin{cx}{1.0.6}
An \emph{integral domain} is a ring in which the product of a finite family of elements
$\neq 0$ is $\neq 0$; equivalently, in such a ring we have $0\neq 1$ and the product of
two elements $\neq 0$ is non zero. A \emph{prime} ideal of a ring $A$ is an ideal $\mf{p}$
such that $A/\mf{p}$ is integral; this therefore entails $\mf{p}\neq A$. For a ring $A$ to
have at least one prime ideal, it is necessary and sufficent that $A\neq\{0\}$.
\end{cx}

\begin{cx}{1.0.7}
A \emph{local} ring is a ring $A$ in which there exists a single maximal ideal, which is then
the complement of the invertible elements and contains all the ideals $\neq A$. If $A$ and $B$
are two local rings, $\mf{m}$ and $\mf{n}$ their respective maximal ideals, we say that a
homomorphism $\varphi:A\to B$ is \emph{local} if $\varphi(\mf{m})\su\mf{n}$ (or, equivalently,
$\varphi^{-1}(\mf{n})=\mf{m}$). By passing to quotients, such a homomorphism then defines
a momomorphism of the residue field $A/\mf{m}$ into the residue field $B/\mf{n}$. The
composition of two local homomorphisms is a local homomorphism.
\end{cx}

\end{document}

