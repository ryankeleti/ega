\documentclass[../main.tex]{subfiles}

\begin{document}

\begin{cx}{8.3.1}
    Let $X$ be a \emph{Noetherian} integral scheme, and $R$ its field of rational functions; we denote by $X'$ the set of local subrings $\O_x\subset R$, where $x$ runs over all points of $X$.
    The set $X'$ verifies the three following conditions:
    \begin{enumerate}
        \item[(Sch. 1)] For all $M\in X'$, $R$ is the field of fractions of $M$.
        \item[(Sch. 2)] There exists a finite set of Noetherian subrings $A_i$ of $R$ such that $X'=\bigcup_i L(A_i)$, and, for all pairs of indices $i,j$, the subring $A_{ij}$ of $R$ generated by $A_i\cup A_j$ is an algebra of finite type over $A_i$.
        \item[(Sch. 3)] Two elements $M$ and $N$ of $X'$ that are allied are identical.
    \end{enumerate}
\end{cx}

We have basically seen in (8.2.1) that (Sch.~1) is satisfied, and (Sch.~3) follows from (8.2.2).
To show (Sch.~2), it suffices to cover $X$ by a finite number of affine opens $U_i$, whose rings are Noetherian, and to take $A_i=\Gamma(U_i,\O_X)$; the hypothesis that $X$ is a scheme implies that $U_i\cap U_j$ is affine, and also that $\Gamma(U_i\cap U_j,\O_X)=A_{ij}$ (5.5.6); further, since the space $U_i$ is Noetherian, the immersion $U_i\cap U_j\to U_i$ is of finite type (6.3.5), so $A_{ij}$ is an $A_i$-algebra of finite type (6.3.3).

\begin{cx}{8.3.2}
    The structures whose axioms are (Sch.~1), (Sch.~2), and (Sch.~3), generalise ``schemes'' in the sense of C.~Chevalley, who supposes furthermore that $R$ is an extension of finite type of a field $K$, and that the $A_i$ are $K$-algebras of finite type (which renders a part of (Sch.~2) useless) [1].
    Conversely, if we have such a structure on a set $X'$, then we can associate to it an integral scheme $X$ by using the remarks from (8.2.6): the underlying space of $X$ is equal to $X'$ endowed with the topology defined in (8.2.6), and with the sheaf $\O_X$ such that $\Gamma(U,\O_X)=\bigcap_{x\in U}\O_x$ for all open $U\subset X$, with the evident definition of restriction homomorphisms.
    We leave to the reader the task of verifying that we obtain thusly an integral scheme, whose local rings are the elements of $X'$; we will not use this result in what follows.
\end{cx}

\end{document}
