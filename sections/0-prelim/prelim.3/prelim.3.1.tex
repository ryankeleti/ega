\documentclass[../main.tex]{subfiles}

\begin{document}

\begin{env}{3.1.1}
Let $\K$ be a category, $(A_\alpha)_{\alpha\in I}$,
$(A_{\alpha\beta})_{(\alpha,\beta)\in I\times I}$ two families of objects
of $\K$ such that $A_{\beta\alpha}=A_{\alpha\beta}$,
$(\rho_{\alpha\beta})_{(\alpha,\beta)\in I\times I}$ a family of morphisms
$\rho_{\alpha\beta}\colon A_\alpha\to A_{\alpha\beta}$. We say that a pair formed by
an object $A$ of $\K$ and a family of morphisms $\rho_\alpha\colon A\to A_\alpha$ 
is a \emph{solution to the universal problem} defined by the data of the families
$(A_\alpha)$, $(A_{\alpha\beta})$, and $(\rho_{\alpha\beta})$ if, for every object $B$
of $\K$, the mapping which, at all $f\in\Hom(B,A)$ matches the
family $(\rho_\alpha\circ f)\in\prod_\alpha\Hom(B,A_\alpha)$ is a \emph{bijection}
of $\Hom(B,A)$ to the set of all $(f_\alpha)$ such that
$\rho_{\alpha\beta}\circ f_\alpha=\rho_{\beta\alpha}\circ f_\beta$ for any pair of
indices $(\alpha,\beta)$. If such a solution exists, it is unique up to an isomorphism.
\end{env}

\begin{env}{3.1.2}
We will not recall the defintion of a \emph{presheaf} $U\mapsto\sheaf{F}(U)$ on a
topological space $X$ with values in a category $\K$ (G, I, 1.9); we say that
such a presheaf is a \emph{sheaf with values in} $\K$ if it satifies the following
axiom:\\

(F) \emph{For any covering $(U_\alpha)$ of an open set $U$ of $X$ by open sets
   $U_\alpha$ contained in $U$, if we denote by $\rho_\alpha$ (resp. $\rho_{\alpha\beta}$) the
   restriction morphism}
   \[
     \sheaf{F}(U)\to\sheaf{F}(U_\alpha)\quad(\text{\emph{resp. }}\sheaf{F}(U_\alpha)\to\sheaf{F}(U_\alpha\cap U_\beta)),
   \]
   \oldpage{24}\emph{the pair formed by $\sheaf{F}(U)$ and the family $(\rho_\alpha)$ are a solution to
   the universal problem for $(\sheaf{F}(U_\alpha))$, $(\sheaf{F}(U_\alpha\cap U_\beta))$, and $(\rho_{\alpha\beta})$
   in} \eref{3.1.1}\footnote{This is a special case of the more general notion of a
   \emph{projective limit} (non-filtered) (\emph{see} (T, I, 1.8) and the book in
   preparation announced in the Introduction).}.\\

Equivalently, we can say that, for each object $T$ of $\K$, the family
$U\mapsto\Hom(T,\sheaf{F}(U))$ is a \emph{sheaf of sets}.
\end{env}

\begin{env}{3.1.3}
Assume that $\K$ is the category defined by a ``type of structure
with morphisms" $\Sigma$, the objects of $\K$ being the sets with structures
of type $\Sigma$ and morphisms those of $\Sigma$. Suppose that the category $\K$ also satisfies
the following condition:\\

(E) If $(A,(\rho_\alpha))$ is a solution of a universal mapping problem \emph{in the category} $\K$
for families $(A_\alpha)$, $(A_{\alpha\beta})$, $(\rho_{\alpha\beta})$, then it is also a solution of the
universal mapping problem for the same families \emph{in the category of sets} (that is, when we
consider $A$, $A_\alpha$, and $A_{\alpha\beta}$ as sets, $\rho_\alpha$ and $\rho_{\alpha\beta}$ as functions)
\footnote{It can be proved that it also means that the canonical functor $\K\to(\mathrm{Ens})$
\emph{commutes with projective limits} (not necessarily filtered).}.\\

Under these conditions, the condition (F) gives that, when considered as a presheaf
\emph{of sets}, $U\mapsto\sheaf{F}(U)$ is a \emph{sheaf}. In addition, for a map $u\colon T\to\sheaf{F}(U)$
to be a morphism of $\K$, it is necessary and sufficient, under (F), that each map $\rho_\alpha\circ u$
is a morphism $T\to\sheaf{F}(U_\alpha)$, which means that the structure of type $\Sigma$ on $\sheaf{F}(U)$
is the \emph{initial structure} for the morphisms $\rho_\alpha$ . Conversely, suppose a presheaf
$U\mapsto\sheaf{F}(U)$ on $X$, with values in $\K$, is a \emph{sheaf of sets} and satisfies the previous
condition; it is then clear that it satisfies (F), so it is a \emph{sheaf with values in} $\K$.
\end{env}

\begin{env}{3.1.4}
When $\Sigma$ is a type of a group or ring structure, the fact that
the presheaf $U\mapsto\sheaf{F}(U)$ with values in $\K$ is a sheaf of \emph{sets} leads \emph{ipso facto}
that it is a sheaf with values in $\K$ (in other words, a sheaf of groups or rings
within the meaning of (G))\footnote{This is because in the category $\K$, any morphism that is a
\emph{bijection} (as a map of sets) is an \emph{isomorphism}. This is no longer true when $\K$
is the category of topological spaces, for example.}. But it is not the same when, for example,
$\K$ is the category of \emph{topological rings} (with morphisms as continuous homomorphisms): a sheaf
with values in $\K$ is a sheaf of rings $U\mapsto\sheaf{F}(U)$ such that for any open $U$
and any covering of $U$ by open sets $U_\alpha\subset U$, the topology of the ring $\sheaf{F}(U)$
is to be \emph{the least fine}, making the homomorpisms $\sheaf{F}(U)\to\sheaf{F}(U_\alpha)$ continuous. We will say in
this case that $U\mapsto\sheaf{F}(U)$, considered as a sheaf of rings (without a topology), is
\emph{underlying} the sheaf of topological rings $U\mapsto\sheaf{F}(U)$. Morphisms $u_V\colon\sheaf{F}(V)\to\sheaf{G}(V)$
($V$ an arbitrary open subset of $X$) of sheaves of topological rings are therefore homomorphisms of the
underlying sheaves of rings, such that $u_V$ be \emph{continuous} for all
open $V\subset X$; to distinguish them from any homomorphisms of the sheaves
of the underlying rings, we will call them continuous homomorphisms of sheaves of topological rings.
We have similar definitions and conventions for sheaves of topological spaces or topological groups.
\end{env}

\begin{env}{3.1.5}
\oldpage{25}It is clear that for any category $\K$, if there is a presheaf (respectively a
sheaf) $\sheaf{F}$ on $X$ with values in $\K$ and $U$ is an open set of $X$, the $\sheaf{F}(V)$ for
open $V\subset U$ constitute a presheaf (or a sheaf) with values in $\K$, which we call
the presheaf (or sheaf) \emph{induced} by $\sheaf{F}$ on $U$ and denote it by $\sheaf{F}|U$.

For any morphism $u\colon\sheaf{F}\to\sheaf{G}$ of presheaves on $X$ with values in $\K$, we
denote by $u|U$ the morphism $\sheaf{F}|U\to\sheaf{G}|U$ formed by the $u_V$ for $V\subset U$.
\end{env}

\begin{env}{3.1.6}
Suppose now that the category $\K$ admits \emph{inductive limits} (T, 1.8);
then, for any presheaf (and in particular any sheaf) $\sheaf{F}$ on $X$ with
values in $\K$ and all $x\in X$, we can define the \emph{stalk} $\sheaf{F}_x$ as the object of $\K$ defined
by the inductive limit of the $\sheaf{F}(U)$ with respect to the filtering set
(for $\supset$) of the open neighborhoods $U$ of $x$ in $X$, and for the morphisms $\rho_U^V\colon\sheaf{F}(V)\to\sheaf{F}(U)$.
If $u\colon\sheaf{F}\to\sheaf{G}$ is a morphism of presheaves with values in $\K$, we define for all
$x\in X$ the morphism $u_x\colon\sheaf{F}_x\to\sheaf{G}_x$ as the inductive limit of $u_U\colon\sheaf{F}(U)\to\sheaf{G}(U)$ with respect to all
open neighborhoods of $x$; we thus define $\sheaf{F}_x$ as a covariant functor in $\sheaf{F}$, with values in $\K$, for
all $x\in X$.

When $\K$ is further defined by a kind of structure with morphisms $\Sigma$,
we call \emph{sections over $U$} of a \emph{sheaf} $\sheaf{F}$ with values in $\K$ the elements
of $\sheaf{F}(U)$, and we write $\Gamma(U,\sheaf{F})$ instead of $\sheaf{F}(U)$; for $s\in\Gamma(U,\sheaf{F})$, $V$ an open set
contained in $U$, we write $s|V$ instead of $\rho_V^U(s)$; for all $x\in U$, the canonical image
of $s$ in $\sheaf{F}_x$ is the \emph{germ} of $s$ at the point $x$, denoted by $s_x$ (\emph{we will never replace
the notation $s(x)$ in this sense,} this notation being reserved for another notion relating to sheaves
which will be considered in this Treaty \eref{5.5.1}).

If then $u\colon\sheaf{F}\to\sheaf{G}$ is a morphism of sheaves with values in $\K$, we will write $u(s)$
instead of $u_V(s)$ for all $s\in\Gamma(U,\sheaf{F})$.

If $\sheaf{F}$ is a sheaf of commutative groups, or rings, or modules, we say
that the set of $x\in X$ such that $\sheaf{F}_x\neq\{0\}$ is the \emph{support} of $\sheaf{F}$, denoted
$\Supp(\sheaf{F})$; this set is not necessarily closed in $X$.

When $\K$ is defined by a type of structure with morphisms, \emph{we
systematically refrain from using the point of view of ``{\'e}tal{\'e} spaces''} in terms of
relating to sheaves with values in $\K$; in other words, we will never consider
a sheaf as a topological space (nor even as the whole union of its
fibers), and we will not consider also a morphism $u\colon\sheaf{F}\to\sheaf{G}$ of such sheaves
on $X$ as a continuous map of topological spaces.
\end{env}

\end{document}

