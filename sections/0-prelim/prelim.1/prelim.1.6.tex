
\begin{env}{1.6.1}
\label{env-0.1.6.1}
Let $M$ be an $A$-module, $f$ an element of $A$. Consider a sequence
$(M_n)$ of $A$-modules, all identical to $M$, and for each pair of integers
$m\leq n$, let $\varphi_{nm}$ be the homomorphism $z\mapsto f^{n-m}z$ of $M_m$
into $M_n$; it is immediate that $((M_n),(\varphi_{nm}))$ is an \emph{inductive system}
of $A$-modules; let $N=\varinjlim M_n$ be the inductive limit of this system. We define
a canonical $A$-isomorphism, \emph{functorial} of $N$ on $M_f$. For this reason, let us
note that, for all $n$, $\theta_n\colon z\mapsto z/f^n$ is an $A$-homomorphism of $M=M_n$ into
$M_f$, and it follows from the definitions that we have $\theta_n\circ\varphi{nm}=\theta_m$
for $m\leq n$. There exists therefore an $A$-homomorphism $\theta\colon N\to M_f$ such that, if
$\varphi_n$ denotes the canonical homomorphism $M_n\to N$, we have $\theta_n=\theta\circ\varphi_n$
for all $n$. Since, by hypothesis, every element of $M_f$ is of the form $z/f^n$ for at least $n$,
it is clear that $\theta$ is surjective. On the other hand, if $\theta(\varphi_n(z))=0$,
in other words $z/f^n=0$, there exists an integer $k>0$ such that $f^k z=0$, so $\varphi_{n+k,n}(z)=0$,
which results in $\varphi_n(z)=0$. We can therefore identify $M_f$ and $\varinjlim M_n$ by means
of $\theta$.
\end{env}

\begin{env}{1.6.2}
\label{env-0.1.6.2}
Now write $M_{f,n}$, $\varphi_{nm}^f$ and $\varphi_n^f$ instead of $M_n$, $\varphi_{nm}$ and
$\varphi_n$. Let $g$ be a second element of $A$. As $f^n$ divides $f^n g^n$, we have a functorial
homomorphism
\[
  \rho_{fg,f}\colon M_f\longrightarrow M_{fg}\quad(\eref{1.4.1}\text{ and }\eref{1.4.3});
\]
\oldpage{20}if we indentify $M_f$ and $M_{fg}$ with $\varinjlim M_{f,n}$ and $\varinjlim M_{fg,n}$
respectively, $\rho_{fg,f}$ identifies with the \emph{inductive limit} of the maps
$\rho_{fg,f}^n\colon M_{f,n}\to M_{fg,n}$, defined by $\rho_{fg,f}^n(z)=g^n z$. Indeed, this follows
immediately from the commutivity of the diagram
\[
  \xymatrix{
    M_{f,n}\ar[r]^{\rho_{fg,f}^n}\ar[d]_{\varphi_n^f} & M_{fg,n}\ar[d]^{\varphi_n^{fg}}\\
    M_f\ar[r]^{\rho_{fg,f}} & M_{fg}.
  }
\]
\end{env}

