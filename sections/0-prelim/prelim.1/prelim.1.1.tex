\documentclass[../main.tex]{subfiles}

\begin{document}

\begin{cx}{1.1.1}
Let $\mf{a}$ be an ideal of a ring $A$; the \emph{radical} of $\mf{a}$, denoted
by $\mf{r}(a)$, is the set of $x\in A$ such that $x^n\in\mf{a}$ for an integer $n>0$; it is an
ideal containing $\mf{a}$. We have $\mf{r}(\mf{r}(a))=\mf{r}(a)$; the relation $\mf{a}\su\mf{b}$
leads to $\mf{r}(a)\su\mf{r}(b)$; the radical of a finite intersection of ideals is the
intersection of their radicals. If $\varphi$  is a homomorphism of a ring $A'$ into $A$, then we
have $\mf{r}(\varphi^{-1}(a))=\varphi^{-1}(\mf{r}(a))$
for any ideal $\mf{a}\su A$. For an ideal to be the radical of an ideal,
it is necessary and sufficient that it be an intersection of prime ideals. The radical of an
ideal $\mf{a}$ is the intersection of the
\emph{minimal} prime ideals among those containing $\mf{a}$; if $A$ is
Noetherian, these minimal prime ideals are finite in number.

The radical of the ideal $(0)$ is also called the \emph{nilradical} of $A$; it is the set
$\rad$ of the nilpotent elements of $A$. It is said that the ring $A$ is \emph{reduced} if
$\rad=(0)$; for every ring $A$, the quotient $A/\rad$ of $A$ by its nilradical is a
reduced ring.
\end{cx}

\begin{cx}{1.1.2}
Recall that the \emph{nilradical} $\rad(A)$ of a ring $A$ (not necessarily commutative) is the
intersection of the maximal left ideals of $A$ (and also the intersection of maximal
right ideals). The nilradical of $A/\rad(A)$ is $(0)$.
\end{cx}

\end{document}

