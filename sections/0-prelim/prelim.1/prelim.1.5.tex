\documentclass[../main.tex]{subfiles}

\begin{document}

\begin{cx}{1.5.1}
Let $A$, $A'$ be two rings, $\varphi$ a homomorphism $A'\to A$, $S$ (resp. $S'$)
a multiplicative subset of $A$ (resp. $A'$), such that $\varphi(S')\su S$; the
composition homomorphism $\smash{A'\xrightarrow{\varphi} A\to S^{-1}A}$ factors as
$\smash{A'\to{S'}^{-1}\xrightarrow{\varphi^{S'}} S^{-1}A}$ by virtue of (1.2.4);
where $\varphi^{S'}(a'/s')=\varphi(a')/\varphi(s')$. If $A=\varphi(A')$ and
$S=\varphi(S')$, $\varphi^{S'}$ is \emph{surjective}. If $A'=A$ and if $\varphi$
is the identity, $\varphi^{S'}$ is none other than the homomorphism $\rho_A^{S,S'}$
defined in (1.4.1).
\end{cx}

\begin{cx}{1.5.2}
Under the hypothesis of (1.5.1), let $M$ be an $A$-module. There exists a canonical
functorial morphism
\[\TI
  \sigma:{S'}^{-1}(M_{[\varphi]})\ar[r] & (S^{-1}M)_{[\varphi^{S'}]}
\TE\]
of ${S'}^{-1}A'$-modules, sending each element $m/s'$ of ${S'}^{-1}(M_{[\varphi]})$ to
the element $m/\varphi(s')$ of $(S^{-1}M)_{[\varphi^{S'}]}$; in fact, we verify
immediately that this definition does not depend on the expression $m/s'$ of the element
considered. \emph{When} $S=\varphi(S')$, the homomorphism $\sigma$ is \emph{bijective}.
When $A'=A$ and $\varphi$ is the identity, $\sigma$ is none other that the homomorphism
$\rho_M^{S,S'}$ defined in (1.4.1).

When $M=A$ is taken in particular, the homomorphism $\varphi$ defines on $A$ an $A'$-algebra
structure; ${S'}^{-1}(A_{[\varphi]})$ is then provided with a ring structure, for which it
identifies with $(\varphi(S'))^{-1}A$, and the homomorphism
${\sigma:{S'}^{-1}(A_{[\varphi]})\to S^{-1}A}$ is a homomorphism of ${S'}^{-1}A'$-algebras.
\end{cx}

\begin{cx}{1.5.3}
Let $M$ and $N$ be two $A$-modules; by composing the homomorphisms defined in (1.3.4) and
(1.5.2), we obtain a homomorphism
\[\TI
  (S^{-1}M\otimes_{S^{-1}A}S^{-1}N)_{[\varphi^{S'}]} & {S'}^{-1}((M\otimes A)_{[\varphi]})\ar[l]
\TE\]
which is an isomorphism when $\varphi(S')=S$. Similarly, by composing the homorphisms in (1.3.5)
and (1.5.2), we obtain a homomorphism
\[\TI
  {S'}^{-1}((\Hom_A(M,N))_{[\varphi]})\ar[r] & (\Hom_{S^{-1}A}(S^{-1}M,S^{-1}N))_{[\varphi^{S'}]}
\TE\]
which is an isomorphism when $\varphi(S')=S$ and $M$ admits a finite presentation.
\end{cx}

\begin{cx}{1.5.4}
Let us now consider an $A'$-module $N'$, and form the tensor product $N'\otimes_{A'}A_{[\varphi]}$,
which can be considered as an $A$-module by setting $a\cdot(n'\otimes b)=n'\otimes(ab)$. There is a
functorial isomorphism of $S^{-1}A$-modules
\[\TI
  \tau:({S'}^{-1}N')\otimes_{{S'}^{-1}A'}(S^{-1}A)_{[\varphi^{S'}]}\ar[r,"\sim"] &
  S^{-1}(N'\otimes_{A'}A_{[\varphi]})
\TE\]
\oldpage{18}which maps the element $(n'/s')\otimes(a/s)$ to the element $(n'\otimes a)/(\varphi(s')s)$;
indeed, we verify separately that when we replace $n'/s'$ (resp. $a/s$) by another expression of the
same element, $(n'\otimes a)/(\varphi(s')s)$ does not change; on the other hand, we can define a
reciprocal homomorphism of $\tau$ by sending $(n'\otimes a)/s$ to the element $(n'/1)\otimes(a/s)$:
we use the fact that $S^{-1}(N'\otimes_{A'}A_{[\varphi]})$ is canonically isomorphic to
$(N'\otimes_{A'}A_{[\varphi]})\otimes_A S^{-1}A$ (1.2.5), so also to $N'\otimes_{A'}(S^{-1}A)_{[\varphi]}$,
by designating by $\psi$ the composite homomorphism $a'\mapsto\varphi(a')/1$ of $A'$ into $S^{-1}A$.
\end{cx}

\begin{cx}{1.5.5}
If $M'$ and $N'$ are two $A'$-modules, by composing the isomorphisms (1.3.4) and (1.5.4), we obtain
an isomorphism
\[\TI
  {S'}^{-1}M\otimes_{{S'}^{-1}A'}{S'}^{-1}N'\otimes_{{S'}^{-1}A'}S^{-1}A\ar[r,"\sim"] &
  S^{-1}(M'\otimes_{A'}N'\otimes_{A'}A).
\TE\]
Likewise, if $M'$ admits a finite presentation, we have by (1.3.5) and (1.5.4) an isomorphism
\[\TI
  \Hom_{{S'}^{-1}A'}({S'}^{-1}M',{S'}^{-1}N')\otimes_{{S'}^{-1}A'}S^{-1}A\ar[r,"\sim"] &
  S^{-1}(\Hom_{A'}(M',N')\otimes_{A'}A).
\TE\]
\end{cx}

\begin{cx}{1.5.6}
Under the hypothesis of (1.5.1), let $T$ (resp. $T'$) be a second multiplicative subset of $A$
(resp. $A'$) such that $S\su T$ (resp. $S'\su T'$) and $\varphi(T')\su T$. Then the diagram
\[\TI
  {S'}^{-1}A'\ar[r,"\varphi^{S'}"]\ar[d,"\rho^{T',S'}"'] & S^{-1}A\ar[d,"\rho^{T,S}"]\\
  {T'}^{-1}A'\ar[r,"\varphi^{T'}"'] & T^{-1}A
\TE\]
is commutative. If $M$ is an $A$-module, the diagram
\[\TI
  {S'}^{-1}(M_{[\varphi]})\ar[r,"\sigma"]\ar[d,"\rho^{T',S'}"'] &
  (S^{-1}M)_{[\varphi^{S'}]}\ar[d,"\rho^{T,S}"]\\
  {T'}^{-1}(M_{[\varphi]})\ar[r,"\sigma"'] & (T^{-1}M)_{[\varphi^{T'}]}
\TE\]
is commutative. Finally, if $N'$ is an $A'$-module, the diagram
\[\TI
  ({S'}^{-1}N')\otimes_{{S'}^{-1}A'}(S^{-1}A)_{[\varphi^{S'}]}
  \ar[r,"\sim","\tau"'] \ar[d] &
  S^{-1}(N'\otimes_{A'}A_{[\varphi]})\ar[d,"\rho^{T,S}"]\\
  ({T'}^{-1}N')\otimes_{{T'}^{-1}A'}(T^{-1}A)_{[\varphi^{T'}]}
  \ar[r,"\sim","\tau"'] & T^{-1}(N'\otimes_{A'}A_{[\varphi]})
\TE\]
is commutative, the left vertical arrow obtained by applying
$\rho_{N'}^{T',S'}$ to ${S'}^{-1}N'$ and $\rho_A^{T,S}$ to $S^{-1}A$.
\end{cx}

\begin{cx}{1.5.7}
\oldpage{19}Let $A''$ be a third ring, $\varphi':A''\to A'$ a ring homomorphism,
$S''$ a multiplicative subset of $A''$ such that $\varphi'(S'')\su S'$. Set
$\varphi''=\varphi\circ\varphi'$; then we have
\[
  {\varphi''}^{S''}=\varphi^{S'}\circ{\varphi'}^{S''}.
\]
Let $M$ be an $A$-module; evidently we have $M_{[\varphi'']}=(M_{[\varphi]})_{[\varphi']}$;
if $\sigma'$ and $\sigma''$ are the homomorphisms defined by $\varphi'$ and $\varphi''$ as
$\sigma$ is defined in (1.5.2) by $\varphi$, we have the transitivity formula
\[
  \sigma''=\sigma\circ\sigma'.
\]

Finally, let $N''$ be an $A''$-module; the $A$-module $N''\otimes_{A''}A_{[\varphi'']}$
identifies canonically with
$(N''\otimes_{A''}{A'}_{[\varphi']})\otimes_{A'}A_{[\varphi]}$,
and likewise the $S^{-1}A$-module
${({S''}^{-1}N'')\otimes_{{S''}^{-1}A''}(S^{-1}A)_{[{\varphi''}^{S''}]}}$ identifies
canonically with
$(({S''}^{-1}N'')$ \smash{$\otimes_{{S''}^{-1}A''}({S'}^{-1}A')_{[{\varphi'}^{S''}]})
  \otimes_{{S'}^{-1}A'}(S^{-1}A)_{[\varphi^{S'}]}$}. With these identifications, if $\tau'$
and $\tau''$ are the isomorphisms defined by $\varphi'$ and $\varphi''$ as $\tau$ is defined
in (1.5.4) by $\varphi$, we have the transitivity formula
\[
  \tau''=\tau\circ(\tau'\otimes 1).
\]
\end{cx}

\begin{cx}{1.5.8}
Let $A$ be a subring of a ring $B$; for every \emph{minimal} prime ideal $\mf{p}$ of $A$, there
exists a minimal prime ideal $\mf{q}$ of $B$ such that $\mf{p}=A\cap\mf{q}$. Indeed, $A_\mf{p}$
is a subring of $B_\mf{p}$ (1.3.2) and has \emph{a single} prime ideal $\mf{p}'$ (1.2.6); since
$B_\mf{p}$ is not reduced to $0$, it has at least one prime ideal $\mf{q}'$ and we have
necessarily $\mf{q}'\cap A_\mf{p}=\mf{p}'$; the prime ideal $\mf{q}_1$ of $B$, a reciprocal image
of $\mf{q}'$ is thus such that $\mf{q}_1\cap A=\mf{p}$, and $\emph{a fortiori}$ we have
$\mf{q}\cap A=\mf{p}$ for every minimal prime ideal $\mf{q}$ of $B$ contained in $\mf{q}_1$.
\end{cx}

\end{document}

