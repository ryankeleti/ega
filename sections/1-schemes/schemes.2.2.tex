\documentclass[../main.tex]{subfiles}

\begin{document}

\begin{cx}[Definition]{2.2.1}
    Given two preschemes $(X,\O_X)$, $(Y,\O_Y)$, we define a morphism (of preschemes) of $(X,\O_X)$ to $(Y,\O_Y)$ to be a morphism of ringed spaces $(\psi,\theta)$ such that, for all $x\in X$, $\theta_x^\#$ is a local homomorphism $\O_{\psi(x)}\to\O_x$.
\end{cx}

\oldpage{99}By passing to quotients, the map $\O_{\psi(x)}\to\O_x$ gives us a monomorphism $\theta^x\colon k(\psi(x))\to k(x)$, which lets us consider $k(x)$ as an \emph{extension} of the field $k(\psi(x))$.

\begin{cx}{2.2.2}
    The composition $(\psi'',\theta'')$ of two morphisms $(\psi,\theta)$, $(\psi',\theta')$ of preschemes is also a morphism of preschemes, since it is given by the formula $\theta''^\#=\theta^\#\circ\psi^*(\theta'^\#)$ (\textbf{0},~3.5.5).
    From this we conclude that preschemes form a \emph{category}; using the usual notation, we will write $\Hom(X,Y)$ to mean the set of morphisms from a prescheme $X$ to a prescheme $Y$.
\end{cx}

\begin{cx}[Example]{2.2.3}
    If $U$ is an open subset of $X$ then the canonical injection (\textbf{0},~4.1.2) of the induced prescheme $(U,\O_X|U)$ into $(X,\O_X)$ is a morphism of preschemes; it is further a \emph{monomorphism} of ringed spaces (and \emph{a fortiori} a monomorphism of preschemes), which follows rapidly from (\textbf{0},~4.1.1).
\end{cx}

\begin{cx}[Proposition]{2.2.4}
    Let $(X,\O_X)$ be a prescheme, and $(S,\O_S)$ an affine scheme associated to a ring $A$.
    Then there exists a canonical bijective correspondence between morphisms of preschemes from $(X,\O_X)$ to $(S,\O_S)$ and ring homomorphisms from $A$ to $\Gamma(X,\O_X)$.
\end{cx}

\unsure{TODO}

\end{document}
