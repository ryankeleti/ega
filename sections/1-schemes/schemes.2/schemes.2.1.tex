
\begin{env}{2.1.1}
\label{env-1.2.1.1}
Given a ringed space $(X,\O_X)$, we say that an open subset
$V$ of $X$ is an \emph{affine open} if the ringed space $(V,\O_X|V)$ is an
affine scheme \eref{1.7.1}.
\end{env}

\begin{envr}[Definition]{2.1.2}
\label{defn-1.2.1.2}
We define a prescheme to be a ringed space
$(X,\O_X)$ such that every point of $X$ admits an affine open neighbourhood.
\end{envr}

\begin{env}[Proposition]{2.1.3}
\label{prop-2.1.3}
\oldpage{98}If $(X,\O_X)$ is a prescheme then
the affine opens give a base for the topology of $X$.
\end{env}

In effect, if $V$ is an arbitrary open neighbourhood of $x\in X$, then there
exists by hypothesis an open neighbourhood $W$ of $x$ such that $(W,\O_X|W)$ is
an affine scheme; we write $A$ to mean its ring.  In the space $W$, $V\cap W$ is
an open neighbourhood of $x$; thus there exists $f\in A$ such that $D(f)$ is an
open neighbourhood of $x$ contained inside $V\cap W$ \eref{1.1.10} (i).  The ringed
space $(D(f),\O_X|D(f))$ is thus an affine scheme, isomorphic to $A_f$
\eref{1.3.6}, whence the proposition.

\begin{env}[Proposition]{2.1.4}
\label{prop-1.2.1.4}
The underlying space of a prescheme is a Kolmogoroff space.
\end{env}

In effect, if $x,y$ are two distinct points of a prescheme $X$ then it is clear
that there exists an open neighbourhood of one of these points that does not
contain the other if $x$ and $y$ are not in the same affine open; and if they
are in the same affine open, this is a result of \eref{1.1.8}.

\begin{env}[Proposition]{2.1.5}
\label{prop-1.2.1.5}
If $(X,\O_X)$ is a prescheme then every closed
irreducible subset of $X$ admits exactly one generic point, and the map
$x\mapsto\overline{\{x\}}$ is thus a bijection of $X$ onto its set of closed
irreducible subsets.
\end{env}

In effect, if $Y$ is a closed irreducible subset of $X$ and $y\in Y$, and if $U$
is an open affine neighbourhood of $y$ in $X$, then $U\cap Y$ is everywhere
dense in $Y$, as well as irreducible (\textbf{0},~2.1.1 and 2.1.4); thus by
(1.1.14), $U\cap Y$ is the closure in $U$ of a point $x$, and then
$Y\subset\overline{U}$ is the closure of $x$ in $X$.  The uniqueness of the
generic point of $X$ is a result of \sref{prop}{2.1.4} and (\textbf{0},~2.1.3).

\begin{env}{2.1.6}
\label{env-1.2.1.6}
If $Y$ is a closed irreducible subset of $X$ and $y$ its
generic point then the local ring $\O_y$, also written $\O_{X/Y}$, is called the
\emph{local ring of $X$ along $Y$}, or the \emph{local ring of $Y$ in $X$}.

If $X$ itself is irreducible and $x$ its generic point then we say that
$\O_x$ is the \emph{ring of rational functions on $X$} (cf.~s.~7).
\end{env}

\begin{env}[Proposition]{2.1.7}
\label{prop-1.2.1.7}
If $(X,\O_X)$ is a prescheme then the ringed
space $(U,\O_X|U)$ is a prescheme for every open subset $U$.
\end{env}

This follows directly from definition~\sref{defn}{2.1.2} and
proposition~\sref{prop}{2.1.3}.

We say that $(U,\O_X|U)$ is the prescheme \emph{induced} on $U$ by
$(X,\O_X)$, or the \emph{restriction} of $(X,\O_X)$ to~$U$.

\begin{env}{2.1.8}
\label{env-1.2.1.8}
We say that a prescheme $(X,\O_X)$ is \emph{irreducible}
(resp. \emph{connected}) if the underlying space $X$ is irreducible (resp.
connected).  We say that a prescheme is \emph{integral} if it is
\emph{irreducible and reduced} (cf.~\eref{5.1.4}).  We say that a prescheme
$(X,\O_X)$ is \emph{locally integral} if every $x\in X$ admits an open
neighbourhood $U$ such that the prescheme induced on $U$ by $(X,\O_X)$ is
integral.
\end{env}

