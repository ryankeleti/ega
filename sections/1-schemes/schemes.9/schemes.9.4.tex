\documentclass[../main.tex]{subfiles}

\begin{document}

\subsection{Extension of quasi-coherent sheaves}

\begin{cx}{9.4.1}
    Let\oldpage{174} $X$ be a topological space, $\scr{F}$ a sheaf of sets (resp. of groups, of rings) on $X$, $U$ an open subset of $X$, $\psi\colon U\to X$ the canonical injection, and $\scr{G}$ a sub-sheaf of $\scr{F}|U=\psi^*(\scr{F})$.
    Since $\psi_*$ is left exact, $\psi_*(\scr{G})$ is a sub-sheaf of $\psi_*(\psi^*(\scr{F}))$; if we denote by $\rho$ the canonical homomorphism $\scr{F}\to\psi_*(\psi^*(\scr{F}))$ (\textbf{0},~3.5.3), then we denote by $\overline{\scr{G}}$ the sub-sheaf $\rho^{-1}(\psi_*(\scr{G}))$ of $\scr{F}$.
    It follows immediately from the definitions that, for every open subset $V$ of $X$, $\Gamma(V,\overline{\scr{G}})$ consists of sections $s\in\Gamma(V,\scr{F})$ whose restriction to $V\cap U$ is a section of $\scr{G}$ over $V\cap U$.
    We thus have that $\overline{\scr{G}}|U=\psi^*(\overline{\scr{G}})=\scr{G}$, and that $\overline{\scr{G}}$ is the \emph{biggest} sub-sheaf of $\scr{F}$ that restricts to $\scr{G}$ over $U$; we say that $\overline{\scr{G}}$ is the \emph{canonical extension} of the sub-sheaf $\scr{G}$ of $\scr{F}|U$ to a sub-sheaf of $\scr{F}$.
\end{cx}

\begin{cx}[Proposition]{9.4.2}
    Let $X$ be a prescheme, $U$ an open subset of $X$ such that the canonical injection $j\colon U\to X$ is a quasi-compact morphism \emph{(which will be the case for \emph{all} $U$ if the underlying space of $X$ is \emph{locally Noetherian} {\normalfont(6.6.4,~(i))})}.
    Then:
    \begin{enumerate}[label=\normalfont(\roman*)]
        \item For every quasi-coherent $(\O_X|U)$-module $\scr{G}$, $j_*(\scr{G})$ is a quasi-coherent $\O_X$-module, and $j_*(\scr{G})|U=j^*(j_*(\scr{G}))=\scr{G}$.
        \item For every quasi-coherent $\O_X$-module $\scr{F}$ and every quasi-coherent sub-$(\O_X|U)$-module $\scr{G}$, the canonical extension $\overline{\scr{G}}$ of $\scr{G}$ {\normalfont(9.4.1)} is a quasi-coherent sub-$\O_X$-module of $\scr{F}$.
    \end{enumerate}
\end{cx}

If $j=(\psi,\theta)$ ($\psi$ being the injection $U\to X$ of underlying spaces), then by definition we have that $j_*(\scr{G})=\psi_*(\scr{G})$ for every $(\O_X|U)$-module $\scr{G}$, and, further, that $j^*(\scr{H})=\psi^*(\scr{H})=\scr{H}|U$ for every $\O_X$-module $\scr{H}$, by definition of the prescheme induced over an open subset.
So (i) is thus a particular case of (9.2.2,~\emph{a})); for the same reason, $j_*(j^*(\scr{F}))$ is quasi-coherent, and since $\overline{\scr{G}}$ is the inverse image of $j_*(\scr{G})$ by the homomorphism $\rho\colon\scr{F}\to j_*(j^*(\scr{F}))$, (ii) follows from (4.1.1).

Note that the hypothesis that the morphism $j\colon U\to X$ is quasi-compact holds whenever the open subset $U$ is \emph{quasi-compact} and $X$ is a \emph{scheme}: indeed, $U$ is then a union of finitely-many affine opens $U_i$, and for every affine open $V$ of $X$, $V\cap U_i$ is an affine open (5.5.6), and thus quasi-compact.

\begin{cx}[Corollary]{9.4.3}
    Let $X$ be a prescheme, $U$ a quasi-compact open subset of $X$ such that the injection morphism $j\colon U\to X$ is quasi-compact.
    Suppose as well that every quasi-coherent $\O_X$-module is the inductive limit of its quasi-coherent sub-$\O_X$-modules of finite type \emph{(which will be the case if $X$ is an \emph{affine scheme})}.
    Then let $\scr{F}$ be a quasi-coherent $\O_X$-module, and $\scr{G}$ a quasi-coherent sub-$(\O_X|U)$-module \emph{of finite type} of $\scr{F}|U$.
    Then there exists a quasi-coherent sub-$\O_X$-module $\scr{G}'$ of $\scr{F}$ \emph{of finite type} such that $\scr{G}'|U=\scr{G}$.
\end{cx}

Indeed, we have $\scr{G}=\overline{\scr{G}}|U$, and $\overline{\scr{G}}$ is quasi-coherent, from (9.4.2), and so the inductive limit of its quasi-coherent sub-$\O_X$-modules $\scr{H}_\lambda$ of finite type.
It follows that $\scr{G}$ is the inductive limit of the $\scr{H}_\lambda|U$, and thus equal to one of the $\scr{H}_\lambda|U$ since it is of finite type (\textbf{0},~5.2.3).

\begin{cx}[Remark]{9.4.4}
    Suppose that for \emph{every} affine open $U\subset X$, the injection morphism $U\to X$ is quasi-compact.
    Then, if the conclusion of (9.4.3) holds for every affine open $U$ and every quasi-coherent sub-$(\O_X|U)$-module $\scr{G}$ of $\scr{F}|U$ of finite type, it follows\oldpage{175} that $\scr{F}$ is the inductive limit of its quasi-coherent sub-$\O_X$-modules of finite type.
    Indeed, for every affine open $U\subset X$, we have that $\scr{F}|U=\widetilde{M}$, where $M$ is an $A(U)$-module, and since the latter is the inductive limit of its quasi-coherent sub-modules of finite type, $\scr{F}|U$ is the inductive limit of its sub-$(\O_X|U)$-modules of finite type (1.3.9).
    But, by hypothesis, each of these sub-modules is induced on $U$ by a quasi-coherent sub-$\O_X$-module $\scr{G}_{\lambda,U}$ of $\scr{F}$ of finite type.
    The finite sums of the $\scr{G}_{\lambda,U}$ are again quasi-coherent $\O_X$-modules of finite type, because the property is local, and the case where $X$ is affine was covered in (1.3.10); it is clear then that $\scr{F}$ is the inductive limit of these finite sums, whence our claim.
\end{cx}

\begin{cx}[Corollary]{9.4.5}
    Under the hypotheses of (9.4.3), for every quasi-coherent $(\O_X|U)$-module $\scr{G}$ of finite type, there exists a quasi-coherent $\O_X$-module $\scr{G}'$ of finite type such that $\scr{G}'|U=\scr{G}$.
\end{cx}

Since $\scr{F}=j_*(\scr{G})$ is quasi-coherent (9.4.2) and $\scr{F}|U=\scr{G}$, it suffices to apply (9.4.3) to $\scr{F}$.

\begin{cx}[Lemma]{9.4.6}
    Let $X$ be a prescheme, $L$ a well-ordered set, $(V_\lambda)_{\lambda\in L}$ a cover of $X$ by affine opens, and $U$ an open of $X$; for all $\lambda\in L$, we set $W_\lambda=\bigcup_{\mu<\lambda}V_\mu$.
    Suppose that: (1) for every $\lambda\in L$, $V_\lambda\cap W_\lambda$ is quasi-compact; (2) the immersion morphism $U\to X$ is quasi-compact.
    Then, for every quasi-coherent $\O_X$-module $\scr{F}$ and every quasi-coherent sub-$(\O_X|U)$-module $\scr{G}$ of $\scr{F}|U$ \emph{of finite type}, there exists a quasi-coherent sub-$\O_X$-module $\scr{G}'$ of $\scr{F}$ \emph{of finite type} such that $\scr{G}'|U=\scr{G}$.
\end{cx}

Let $U_\lambda=U\cup W_\lambda$; we will define a family $(\scr{G}'_\lambda)$ by recurrence, where $\scr{G}'_\lambda$ is a quasi-coherent sub-$(\O_X|U_\lambda)$-module of $\scr{F}|U_\lambda$ of finite type, such that $\scr{G}'_\lambda|U_\mu=\scr{G}'_\mu$ for $\mu<\lambda$ and $\scr{G}'_\lambda|U=\scr{G}$.
The unique sub-$\O_X$-module $\scr{G}'$ of $\scr{F}$ such that $\scr{G}'|U_\lambda=\scr{G}'$ for all $\lambda\in L$ (\textbf{0},~3.3.1) gives us what we want.
So suppose that the $\scr{G}'_\mu$ are defined and have the preceding properties for $\mu<\lambda$; if $\lambda$ does not have a predecessor then we take for $\scr{G}'_\lambda$ the unique sub-$(\O_X|U_\lambda)$-module of $\scr{F}|U_\lambda$ such that $\scr{G}'_\lambda|U_\mu=\scr{G}'_\mu$ for all $\mu<\lambda$, which is allowed since the $U_\mu$ with $\mu<\lambda$ then form a cover of $U_\lambda$.
If, conversely, $\lambda=\mu+1$, then $U_\lambda=U_\mu\cup V_\mu$, and it suffices to define a quasi-coherent sub-$(\O_X|V_\mu)$-module $\scr{G}''_\mu$ of $\scr{F}|V_\mu$ of finite type such that
\begin{equation*}
    \scr{G}''_\mu|(U_\mu\cap V_\mu)=\scr{G}'_\mu|(U_\mu\cap V_\mu);
\end{equation*}
and then to take for $\scr{G}'_\lambda$ the sub-$(\O_X|U_\lambda)$-module of $\scr{F}|U_\lambda$ such that $\scr{G}'_\lambda|U_\mu=\scr{G}'_\mu$ and $\scr{G}'_\lambda|V_\mu=\scr{G}''_\mu$ (\textbf{0},~3.3.1).
But, since $V_\mu$ is affine, the existence of $\scr{G}''_\mu$ is guaranteed by (9.4.3) as soon as we show that $U_\mu\cap V_\mu$ is quasi-compact; but $U_\mu\cap V_\mu$ is the union of $U\cap V_\mu$ and $W_\mu\cap V_\mu$, which are both quasi-compact by virtue of the hypothesis.

\begin{cx}[Theorem]{9.4.7}
    Let $X$ be a prescheme, and $U$ an open of $X$.
    Suppose that one of the following conditions is verified:
    \begin{enumerate}[label=\normalfont\alph*)]
        \item the underlying space of $X$ is locally Noetherian;
        \item $X$ is a quasi-compact scheme and $U$ is a quasi-compact open.
    \end{enumerate}
    Then, for every quasi-coherent $\O_X$-module $\scr{F}$ and every quasi-coherent sub-$(\O_X|U)$-module $\scr{G}$ of $\scr{F}|U$ \emph{of finite type}, there exists a quasi-coherent sub-$\O_X$-module $\scr{G}'$ of $\scr{F}$ \emph{of finite type} such that $\scr{G}'|U=\scr{G}$.
\end{cx}

Let\oldpage{176} $(V_\lambda)_{\lambda\in L}$ be a cover of $X$ by affine opens, with $L$ supposed finite in case \emph{b)}; since $L$ is equipped with the structure of a well-ordered set, it suffices to verify that the conditions of (9.4.6) are satisfied.
It is clear in the case of \emph{a)}, as the spaces $V_\lambda$ are Noetherian.
For case \emph{b)}, the $V_\lambda\cap\lambda_\mu$ are affine (5.5.6), and thus quasi-compact, and since $L$ is finite, $V_\lambda\cap W_\lambda$ is quasi-compact.
Whence the theorem.

\begin{cx}[Corollary]{9.4.8}
    Under the hypotheses of \emph{(9.4.7)}, for every quasi-coherent $(\O_X|U)$-module $\scr{G}$ of finite type, there exists a quasi-coherent $\O_X$-module $\scr{G}'$ of finite type such that $\scr{G}'|U=\scr{G}$.
\end{cx}

It suffices to apply (9.4.7) to $\scr{F}=j_*(\scr{G})$, which is quasi-coherent (9.4.2) and such that $\scr{F}|U=\scr{G}$.

\begin{cx}[Corollary]{9.4.9}
    Let $X$ be a prescheme whose underlying space is locally Noetherian, or a quasi-compact scheme.
    Then every quasi-coherent $\O_X$-module is the inductive limit of its quasi-coherent sub-$\O_X$-modules of finite type.
\end{cx}

This follows from (9.4.7) and the remark (9.4.4).

\begin{cx}[Corollary]{9.4.10}
    Under the hypotheses of (9.4.9), if a quasi-coherent $\O_X$-module $\scr{F}$ is such that every quasi-coherent sub-$\O_X$-module of finite type of $\scr{F}$ is generated by its sections over $X$, then $\scr{F}$ is generated by its sections over $X$.
\end{cx}

In fact, let $U$ be an affine open neighbourhood of a point $x\in X$, and let $s$ be a section of $\scr{F}$ over $U$; the sub-$\O_X$-module $\scr{G}$ of $\scr{F}|U$ generated by $s$ is quasi-coherent and of finite type, so there exists a quasi-coherent sub-$\O_X$-module $\scr{G}'$ of $\scr{F}$ of finite type such that $\scr{G}'|U=\scr{G}$ (9.4.7).
By hypothesis, there is thus a finite number of sections $t_i$ of $\scr{G}'$ over $X$ and of sections $a_i$ of $\O_X$ over a neighbourhood $V\subset U$ of $x$ such that $s|V=\sum_i a_i(t_i|V)$, which proves the corollary.

\end{document}
