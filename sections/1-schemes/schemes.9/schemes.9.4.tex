\documentclass[../main.tex]{subfiles}

\begin{document}

\begin{cx}{9.4.1}
    Let\oldpage{174} $X$ be a topological space, $\scr{F}$ a sheaf of sets (resp. of groups, of rings) on $X$, $U$ an open subset of $X$, $\psi\colon U\to X$ the canonical injection, and $\scr{G}$ a sub-sheaf of $\scr{F}|U=\psi^*(\scr{F})$.
    Since $\psi_*$ is left exact, $\psi_*(\scr{G})$ is a sub-sheaf of $\psi_*(\psi^*(\scr{F}))$; if we denote by $\rho$ the canonical homomorphism $\scr{F}\to\psi_*(\psi^*(\scr{F}))$ (\textbf{0},~3.5.3), then we denote by $\overline{\scr{G}}$ the sub-sheaf $\rho^{-1}(\psi_*(\scr{G}))$ of $\scr{F}$.
    It follows immediately from the definitions that, for every open subset $V$ of $X$, $\Gamma(V,\overline{\scr{G}})$ consists of sections $s\in\Gamma(V,\scr{F})$ whose restriction to $V\cap U$ is a section of $\scr{G}$ over $V\cap U$.
    We thus have that $\overline{\scr{G}}|U=\psi^*(\overline{\scr{G}})=\scr{G}$, and that $\overline{\scr{G}}$ is the \emph{biggest} sub-sheaf of $\scr{F}$ that restricts to $\scr{G}$ over $U$; we say that $\overline{\scr{G}}$ is the \emph{canonical extension} of the sub-sheaf $\scr{G}$ of $\scr{F}|U$ to a sub-sheaf of $\scr{F}$.
\end{cx}

\begin{cx}[Proposition]{9.4.2}
    Let $X$ be a prescheme, $U$ an open subset of $X$ such that the canonical injection $j\colon U\to X$ is a quasi-compact morphism \emph{(which will be the case for \emph{all} $U$ if the underlying space of $X$ is \emph{locally Noetherian} {\normalfont(6.6.4,~(i))})}.
    Then:
    \begin{enumerate}[label=\normalfont(\roman*)]
        \item For every quasi-coherent $(\O_X|U)$-module $\scr{G}$, $j_*(\scr{G})$ is a quasi-coherent $\O_X$-module, and $j_*(\scr{G})|U=j^*(j_*(\scr{G}))=\scr{G}$.
        \item For every quasi-coherent $\O_X$-module $\scr{F}$ and every quasi-coherent sub-$(\O_X|U)$-module $\scr{G}$, the canonical extension $\overline{\scr{G}}$ of $\scr{G}$ {\normalfont(9.4.1)} is a quasi-coherent sub-$\O_X$-module of $\scr{F}$.
    \end{enumerate}
\end{cx}

If $j=(\psi,\theta)$ ($\psi$ being the injection $U\to X$ of underlying spaces), then by definition we have that $j_*(\scr{G})=\psi_*(\scr{G})$ for every $(\O_X|U)$-module $\scr{G}$, and, further, that $j^*(\scr{H})=\psi^*(\scr{H})=\scr{H}|U$ for every $\O_X$-module $\scr{H}$, by definition of the prescheme induced over an open subset.
So (i) is thus a particular case of (9.2.2,~\emph{a})); for the same reason, $j_*(j^*(\scr{F}))$ is quasi-coherent, and since $\overline{\scr{G}}$ is the inverse image of $j_*(\scr{G})$ by the homomorphism $\rho\colon\scr{F}\to j_*(j^*(\scr{F}))$, (ii) follows from (4.1.1).

Note that the hypothesis that the morphism $j\colon U\to X$ is quasi-compact holds whenever the open subset $U$ is \emph{quasi-compact} and $X$ is a \emph{scheme}: indeed, $U$ is then a union of finitely-many affine opens $U_i$, and for every affine open $V$ of $X$, $V\cap U_i$ is an affine open (5.5.6), and thus quasi-compact.

\begin{cx}[Corollary]{9.4.3}
    Let $X$ be a prescheme, $U$ a quasi-compact open subset of $X$ such that the injection morphism $j\colon U\to X$ is quasi-compact.
    Suppose as well that every quasi-coherent $\O_X$-module is the inductive limit of its quasi-coherent sub-$\O_X$-modules of finite type \emph{(which will be the case if $X$ is an \emph{affine scheme})}.
    Then let $\scr{F}$ be a quasi-coherent $\O_X$-module, and $\scr{G}$ a quasi-coherent sub-$(\O_X|U)$-module \emph{of finite type} of $\scr{F}|U$.
    Then there exists a quasi-coherent sub-$\O_X$-module $\scr{G}'$ of $\scr{F}$ \emph{of finite type} such that $\scr{G}'|U=\scr{G}$.
\end{cx}

Indeed, we have $\scr{G}=\overline{\scr{G}}|U$, and $\overline{\scr{G}}$ is quasi-coherent, from (9.4.2), and so the inductive limit of its quasi-coherent sub-$\O_X$-modules $\scr{H}_\lambda$ of finite type.
It follows that $\scr{G}$ is the inductive limit of the $\scr{H}_\lambda|U$, and thus equal to one of the $\scr{H}_\lambda|U$ since it is of finite type (\textbf{0},~5.2.3).

\begin{cx}[Remark]{9.4.4}
    Suppose that for \emph{every} affine open $U\subset X$, the injection morphism $U\to X$ is quasi-compact.
    Then, if the conclusion of (9.4.3) holds for every affine open $U$ and every quasi-coherent sub-$(\O_X|U)$-module $\scr{G}$ of $\scr{F}|U$ of finite type, it follows\oldpage{175} that $\scr{F}$ is the inductive limit of its quasi-coherent sub-$\O_X$-modules of finite type.
    Indeed, for every affine open $U\subset X$, we have that $\scr{F}|U=\widetilde{M}$, where $M$ is an $A(U)$-module, and since the latter is the inductive limit of its quasi-coherent sub-modules of finite type, $\scr{F}|U$ is the inductive limit of its sub-$(\O_X|U)$-modules of finite type (1.3.9).
    But, by hypothesis, each of these sub-modules is induced on $U$ by a quasi-coherent sub-$\O_X$-module $\scr{G}_{\lambda,U}$ of $\scr{F}$ of finite type.
    The finite sums of the $\scr{G}_{\lambda,U}$ are again quasi-coherent $\O_X$-modules of finite type, because the property is local, and the case where $X$ is affine was covered in (1.3.10); it is clear then that $\scr{F}$ is the inductive limit of these finite sums, whence our claim.
\end{cx}

\begin{cx}[Corollary]{9.4.5}
    Under the hypotheses of (9.4.3), for every quasi-coherent $(\O_X|U)$-module $\scr{G}$ of finite type, there exists a quasi-coherent $\O_X$-module $\scr{G}'$ of finite type such that $\scr{G}'|U=\scr{G}$.
\end{cx}

Since $\scr{F}=j_*(\scr{G})$ is quasi-coherent (9.4.2) and $\scr{F}|U=\scr{G}$, it suffices to apply (9.4.3) to $\scr{F}$.

\begin{cx}[Lemma]{9.4.6}
    Let
\end{cx}

\end{document}
