\documentclass[../main.tex]{subfiles}

\begin{document}

\begin{env}[Theorem]{9.3.1}
    Let $X$ be a prescheme whose underlying space is Noetherian, or a scheme whose underlying space is quasi-compact.
    Let $\sheaf{L}$ be an invertible $\O_X$-module {\normalfont(\textbf{0},~5.4.1)}, $f$ a section of $\sheaf{L}$ over $X$, $X_f$ the open set of $x\in X$ such that $f(x)\neq0$ {\normalfont(\textbf{0},~5.5.1)}, and $\sheaf{F}$ a quasi-coherent $\O_X$-module.
    \begin{enumerate}[label=\normalfont(\roman*)]
        \item If $s\in\Gamma(X,\sheaf{F})$ is such that $s|X_f=0$, then there exists a whole number $n>0$ such that $s\otimes f^{\otimes n}=0$.
        \item For every section $s\in\Gamma(X_f,\sheaf{F})$, there exists a whole number $n>0$ such that $s\otimes f^{\otimes n}$ extends to a section of $\sheaf{F}\otimes\sheaf{L}^{\otimes n}$ over $X$.
    \end{enumerate}
\end{env}

\begin{enumerate}[label=\normalfont(\roman*)]
    \item Since the underlying space of $X$ is quasi-compact, and thus the union of finitely-many affine opens $U_i$ with $\sheaf{L}|U_i$ is isomorphic to $\O_X|U_i$, we can reduce to the case where $X$ is affine and $\sheaf{L}=\O_X$.
    In this case, $f$ is identified with an element of $A(X)$, and we have that $X_f=D(f)$; $s$ is identified with an element of an $A(X)$-module $M$, and $s|X_f$ to the corresponding element of $M_f$, and the result is then trivial, recalling the definition of a module of fractions.
    \item Again\oldpage{173}, $X$ is a finite union of affine opens $U_i$ ($1\leq i\leq r$) such that $\sheaf{L}|U_i\cong\O_X|U_i$, and for every $i$, $(s\otimes f^{\otimes n})|(U_i\cap X_f)$ is identified (by the aforementioned isomorphism) with $(f|(U_i\cap X_f))^n(s|(U_i\cap X_f))$.
    We then know (1.4.1) that there exists a whole number $n>0$ such that, for all $i$, $(s\otimes f^{\otimes n})|(U_i\cap X_f)$ extends to a section $s_i$ of $\sheaf{F}\otimes\sheaf{L}^{\otimes n}$ over $U_i$.
    Let $s_{i|j}$ be the restriction of $s_i$ to $U_i\cap U_j$; by definition we have that $s_{i|j}-s_{j|i}=0$ in $X_f\cap U_i\cap U_j$.
    But, if $X$ is a Noetherian space, then $U_i\cap U_j$ is quasi-compact; if $X$ is a scheme, then $U_i\cap U_j$ is an affine open (5.5.6), and so again quasi-compact.
    By virtue of (i), there thus exists a whole number $m$ (independent of $i$ and $j$) such that $(s_{i|j}-s_{j|i})\otimes f^{\otimes m}=0$.
    It immediately follows that there exists a section $s'$ of $\sheaf{F}\otimes\sheaf{L}^{\otimes(n+m)}$ over $X$, restricting to $s_i\otimes f^{\otimes m}$ over each $U_i$, and restricting to $s\otimes f^{\otimes(n+m)}$ over $X_f$.
\end{enumerate}
The following corollaries give an interpretation of theorem (9.3.1) in a more algebraic language:

\begin{env}[Corollary]{9.3.2}
    With the hypotheses of {\normalfont(9.3.1)}, consider the graded ring $A_*=\Gamma_*(\sheaf{L})$ and the graded $A_*$-module $M_*=\Gamma_*(\sheaf{L},\sheaf{F})$ {\normalfont(\textbf{0},~5.4.6)}.
    If $f\in A_n$, where $n\in\Z$, then there is a canonical isomorphism $\Gamma(X_f,\sheaf{F})\xrightarrow{\sim}((M_*)_f)_0$ (\emph{the subgroup of the module of fractions $(M_*)_f$ consisting of elements of degree $0$}).
\end{env}
\begin{env}[Corollary]{9.3.3}
    Suppose that the hypotheses of {\normalfont(9.3.1)} are satisfied, and suppose further that $\sheaf{L}=\O_X$.
    Then, setting $A=\Gamma(X,\O_X)$ and $M=\Gamma(X,\sheaf{F})$, the $A_f$-module $\Gamma(X_f,\sheaf{F})$ is canonically isomorphic to $M_f$.
\end{env}
\begin{env}[Proposition]{9.3.4}
    Let $X$ be a Noetherian prescheme, $\sheaf{F}$ a coherent $\O_X$-module, and $\sheaf{J}$ a coherent sheaf of ideals in $\O_X$, such that the support of $\sheaf{F}$ is contained in that of $\O_X|\sheaf{J}$.
    Then there exists a whole number $n>0$ such that $\sheaf{J}^n\sheaf{F}=0$.
\end{env}

Since $X$ is a union of finitely-many affine opens whose rings are Noetherian, we can suppose that $X$ is affine of Noetherian ring $A$; then $\sheaf{F}=\widetilde{M}$, where $M=\Gamma(X,\sheaf{F})$ is an $A$-module of finite type, and $\sheaf{J}=\widetilde{\mathfrak{J}}$, where $\mathfrak{J}=\Gamma(X,\sheaf{J})$ is an ideal of $A$ (1.4.1 and 1.5.1).
Since $A$ is Noetherian, $\mathfrak{J}$ admits a finite system of generators $f_i$ ($1\leq i\leq m$).
By hypothesis, every section of $\sheaf{F}$ over $X$ is zero in each of the $D(f_i)$; if $s_j$ ($1\leq j\leq q$) are sections of $\sheaf{F}$ generating $M$, then there exists a whole number $h$, independent of $i$ and $j$, such that $f_i^h s_j=0$. (1.4.1), whence $f_i^h s=0$ for all $s\in M$.
We thus conclude that if $n=mh$ then $\mathfrak{J}^n M=0$, and so the corresponding $\O_X$-module $\sheaf{J}^n\sheaf{F}=\widetilde{\mathfrak{J}^nM}$ (1.3.13) is zero.

\begin{env}[Corollary]{9.3.5}
    With the hypotheses of {\normalfont(9.3.4)}, there exists a closed sub-prescheme $Y$ of $X$, whose underlying space is the support of $\O_X/\sheaf{J}$, such that, if $j\colon Y\to X$ is the canonical injection, then $\sheaf{F}=j_*(j^*(\sheaf{F}))$.
\end{env}

First of all, note that the supports of $\O_X/\sheaf{J}$ and $\O_X/\sheaf{J}^n$ are the same, since, if $\sheaf{J}_x=\O_x$, then $\sheaf{J}_x^n=\O_x$, and we also have that $\sheaf{J}_x^n\subset\sheaf{J}_x$ for all $x\in X$.
We can, thanks to (9.3.4), thus suppose that $\sheaf{J}\sheaf{F}=0$; we can then take $Y$ to be the closed sub-prescheme of $X$ defined by $\sheaf{J}$, and since $\sheaf{F}$ is then an $(\O_X/\sheaf{J})$-module, the conclusion follows immediately.

\end{document}
