\documentclass[../main.tex]{subfiles}

\begin{document}

For every local ring $A$, we denote by $\mathfrak{m}(A)$ the maximal ideal of $A$.

\begin{cx}[Lemma]{8.1.1}
    Let $A$ and $B$ be two local rings such that $A\subset B$; then the following conditions are equivalent: \emph{(i)} $\mathfrak{m}(B)\cap A=\mathfrak{m}(A)$; \emph{(ii)} $\mathfrak{m}(A)\subset\mathfrak{m}(B)$; \emph{(iii)} $1$ is not an element of the ideal of $B$ generated by $\mathfrak{m}(A)$.
\end{cx}

It's evident that (i) implies (ii), and (ii) implies (iii); lastly, if (iii) is true, then $\mathfrak{m}(B)\cap A$ contains $\mathfrak{m}(A)$ and doesn't contain $1$, and is thus equal to $\mathfrak{m}(A)$.

When the equivalent conditions of (8.1.1) are satisfied, we say that $B$ \emph{dominates} $A$; this is equivalent to saying that the injection $A\to B$ is a \emph{local} homomorphism.
It is clear that, in the set of local sub-rings of a ring $R$, the relation given by domination is an \unsure{order}.

\unsure{TODO}

\end{document}
