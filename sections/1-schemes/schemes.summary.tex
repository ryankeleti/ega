\documentclass[../main.tex]{subfile}

\begin{document}

\begin{tabular}{rrl}
  \textsection & 1. & Affine schemes.\\
  \textsection & 2. & Preschemes and morphisms of preschemes.\\
  \textsection & 3. & Products of preschemes.\\
  \textsection & 4. & Subpreschemes and immersion morphisms.\\
  \textsection & 5. & Reduced preschemes; separation condition.\\
  \textsection & 6. & Finiteness conditions.\\
  \textsection & 7. & Rational maps.\\
  \textsection & 8. & Chevalley schemes.\\
  \textsection & 9. & Details on quasi-coherent sheaves.\\
  \textsection & 10. & Formal schemes.
\end{tabular}\\

<<<<<<< HEAD
\oldpage{79}The \textsection\textsection 1-8 do little more than develop a language, which will be used in the following.
It should be noted, however, that in accordance with the general spirit of this Treaty, \textsection\textsection
7-8 will be used less than the others, and in a less essential way; we have moreover spoken of Chevalley's schemes
only to make the link with the language of Chevalley [1] and Nagata [9]. The \textsection 9 gives definitions and results
on quasi-coherent sheaves, some of which are no longer limited to a translation into a ``geometric'' language of known notions
of commutative algebra, but are already of a global nature; they will be indispensable, from the following chapters,
in the global study of morphisms. Finally, \textsection 10 introduces a generalization of the notion of schemes, which will
be used as an intermediary in Chapter III to formulate and demonstrate in a convenient way the fundamental results of the
cohomological study of the proper morphisms; moreover, it should be noted that the notion of formal schemes seems indispensable
to express certain facts of the ``theory of modules'' (classification problems of algebraic varieties). The results of
\textsection 10 will not be used before \textsection 3 of Chapter III and it is recommended to omit reading until then.
=======
\bigskip

\oldpage{79}Sections 1 to 8 intend only to develop a language, which will be used in all that follows.
We note, however, that following the general spirit of this Treatise, sections 7 and 8 will be less used than the others, and in a less essential manner; we speak of Chevalley schemes only in order to be able to link to the language of Chevalley [1] and Nagata [9].
Section 9 gives some definitions and results about quasi-coherent sheaves, of which some are no longer limited to being simply translations of known ideas in commutative algebra into a ``geometric'' language, but instead are already global in nature; they will be indispensable, after the next chapters, in the global study of morphisms.
Finally, section~10 introduces a generalisation of the notion of a scheme, that will serve as a bridge to chapter~III to formulate and prove, in a convenient way, the fundamental results of the study of cohomology of proper morphisms; also, we note that the idea of a formal scheme seems vital in order to be able to express certain facts about the ``theory of modules'' (classification problems of algebraic varieties).
The results of section~10 will not be used before section~3 of chapter~III, and it is recommended to skip over them when first reading until reaching this later point.
>>>>>>> b254cdeae7fbbfbc0debe382b9986ad68ca9ca98

\end{document}

