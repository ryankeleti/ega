
\begin{env}{8.2.1}
\label{env-1.8.2.1}
Let $X$ be an \emph{integral} prescheme, and $R$ its field of
rational functions, identical to the local ring of the generic point $a$ of $X$;
for all $x\in X$, we know that $\O_x$ can be canonically identified with a
subring of $R$ \eref{7.1.5}, and for every rational function $f\in R$, the
domain of definition $\delta(f)$ of $f$ is the open set of $x\in X$ such that
$f\in\O_x$. It thus follows from \eref{7.2.6} that, for every open $U\subset X$,
we have
\[
  \Gamma(U,\O_X)=\bigcap_{x\in U}\O_x.\tag{8.2.1.1}
\]
\end{env}

\begin{env}[Proposition]{8.2.2}
\label{prop-1.8.2.2}
\oldpage{166}Let $X$ be an integral prescheme,
and $R$ its field of rational fractions. For $X$ to be a scheme, it is
necessary and sufficient that the relation ``$\O_x$ and $\O_y$ are allied''
\sref{lem}{8.1.4}, for points $x,y$ of $X$, implies that $x=y$.
\end{env}

Suppose that this condition is verified, and aim to show that $X$ is separated.
Let $U$ and $V$ be two distinct affine opens of $X$, with rings $A$ and $B$,
identified with subrings of $R$; $U$ (resp.$V$) is thus identified \eref{8.1.2}
with $L(A)$ (resp.$L(B)$), and the hypothesis tells us \eref{8.1.5} that $C$ is
the subring of $R$ generated by $A\cup B$, and $W=U\cap V$ is identified with
$L(A)\cap L(B)=L(C)$. Further, we know
([1],~p.~\unsure{5-03},~prop.~4~\emph{bis}) that every subring $E$ of $R$ is
equal to the intersection of the local rings belonging to $L(E)$; $C$ is thus
identified with the intersection of the rings $\O_z$ for $z\in W$, or,
equivalently (8.2.1.1) with $\Gamma(W,\O_X)$. So consider the subprescheme
induced by $X$ on $W$; to the \unsure{identity} morphism $\varphi\colon
C\to\Gamma(W,\O_X)$ there corresponds \eref{2.2.4} a morphism
$\Phi=(\psi,\theta)\colon W\to\Spec(C)$; we will see that $\Phi$ is an
\emph{isomorphism} of preschemes, whence $W$ is an \emph{affine} open. The
identification of $W$ with $L(C)=\Spec(C)$ shows that $\psi$ is
\emph{bijective}.On the other hand, for all $x\in W$, $\theta_x^\#$ is the
injection $C_\mathfrak{r}\to\O_x$, where $\mathfrak{r}=\mathfrak{m}_x\cap C$,
and by definition $C_\mathfrak{r}$ is identified with $\O_x$, so $\theta_x^\#$
is bijective.It thus remains to show that $\psi$ is a \emph{homeomorphism},
i.e.\ that for every closed subset $F\subset W$, $\psi(F)$ is closed in
$\Spec(C)$.But $F$ is the \unsure{trace over} $W$ of closed subspace of $U$,
of the form $V(\mathfrak{a})$, where $\mathfrak{a}$ is an ideal of $A$; we show
that $\psi(F)=V(\mathfrak{a}C)$, which proves our claim. In fact, the prime
ideals of $C$ containing $\mathfrak{a}C$ are the prime ideals of $C$ containing
$\mathfrak{a}$, and so are the ideals of the form $\psi(x)=\mathfrak{m}_x\cap
C$, where $\mathfrak{a}\subset\mathfrak{m}_x$ and $x\in W$; since
$\mathfrak{a}\subset\mathfrak{m}_x$ is equivalent to $x\in V(\mathfrak{a})=W\cap
F$ for $x\in U$, we do indeed have that $\psi(F)=V(\mathfrak{a}C)$.

It follows that $X$ is separated, because $U\cap V$ is affine and its ring $C$
is generated by the union $A\cup B$ of the rings of $U$ and $V$ \eref{5.5.6}.

Conversely, suppose that $X$ is separated, and let $x,y$ be two points of $X$
such that $\O_x$ and $\O_y$ are allied. Let $U$ (resp.$V$) be an affine open
containing $x$ (resp.$y$), of ring $A$ (resp.$B$); we then know that $U\cap V$
is affine and that its ring $C$ is generated by $A\cup B$ \eref{5.5.6}. If
$\mathfrak{p}=\mathfrak{m}_x\cap A$ and $\mathfrak{q}=\mathfrak{m}_y\cap B$,
then $A_\mathfrak{p}=\O_x$ and $B_\mathfrak{q}=\O_y$, and since $A_\mathfrak{p}$
and $B_\mathfrak{q}$ are allied, there exists a prime ideal $\mathfrak{r}$ of
$C$ such that $\mathfrak{p}=\mathfrak{r}\cap A$ and
$\mathfrak{q}=\mathfrak{r}\cap B$ \sref{lem}{8.1.4}. But then there exists a point
$z\in U\cap V$ such that $\mathfrak{r}=\mathfrak{m}_z\cap C$, since $U\cap V$ is
affine, and so evidently $x=z$ and $y=z$, whence $x=y$.

\begin{env}[Corollary]{8.2.3}
\label{cor-1.8.2.3}
Let $X$ be an integral scheme, and $x,y$ two
points of $X$.In order that $x\in\overline{\{y\}}$, it is necessary and
sufficient that $\O_x\subset\O_y$, or, equivalently, that every rational
function defined at $x$ is also defined at $y$.
\end{env}    

The condition is evidently necessary because the domain of definition
$\delta(f)$ of a rational function $f\in R$ is open; we now show that it is
sufficient.If $\O_x\subset\O_y$, then there exists a prime ideal
$\mathfrak{p}$ of $\O_x$ such that $\O_y$ dominates $(\O_x)_\mathfrak{p}$
\sref{lem}{8.1.3}; but \eref{2.4.2} there exists $z\in X$ such that
$x\in\overline{\{z\}}$ and $\O_z=(\O_x)_\mathfrak{p}$; since $\O_z$ and $\O_y$
are allied, we have that $z=y$ by \sref{prop}{8.2.2}, whence the corollary.

\begin{env}[Corollary]{8.2.4}
\label{cor-1.8.2.4}
If $X$ is an integral scheme then the map
$x\to\O_x$ is injective; equivalently, if $x$ and $y$ are two distinct points of
$X$, then there exists a rational function defined at one of these points but
not the other.
\end{env}

\oldpage{167}This follows from \sref{cor}{8.2.3} and the axiom ($\mathrm{T}_0$)
\eref{2.1.4}.

\begin{env}[Corollary]{8.2.5}
\label{cor-1.8.2.5}
Let $X$ be an integral scheme whose underlying
space is Noetherian; letting $f$ run over the field $R$ of rational functions on
$X$, the sets $\delta(f)$ generate the topology of $X$.
\end{env}

In fact, every closed subset of $X$ is thus a finite union of irreducible closed
subsets, i.e.of the form $\overline{\{y\}}$ \eref{2.1.5}. But, if
$x\not\in\overline{\{y\}}$, then there exists a rational function $f$ defined at
$x$ but not at $y$ \sref{cor}{8.2.3}, or, equivalently, we have that $x\in\delta(f)$
and $\delta(f)$ is not contained in $\overline{\{y\}}$. The complement  of
$\overline{\{y\}}$ is thus a union of sets of the form $\delta(f)$, and by
virtue of the first remark, every open subset of $X$ is the union of finite
intersections of open sets of the form $\delta(f)$.

\begin{env}{8.2.6}
\label{env-1.8.2.6}
Corollary~\sref{cor}{8.2.5} shows that the topology of $X$ is
entirely characterised by the data of the local rings $(\O_x)_{x\in X}$ that
have $R$ as their field of fractions. It amounts to the same to say that the
closed subsets of $X$ are defined in the following manner: given a finite subset
$\{x_1,\ldots,x_n\}$ of $X$, consider the set of $y\in X$ such that
$\O_y\subset\O_{x_i}$ for at least one index $i$, and these sets (over all
choices of $\{x_1,\ldots,x_n\}$) are the closed subsets of $X$. Further, once
the topology on $X$ is known, the structure sheaf $\O_X$ is also determined by
the family of the $\O_x$, since $\Gamma(U,\O_X)=\bigcap_{x\in U}\O_x$ by
(8.2.1.1). The family $(\O_X)_{x\in X}$ thus completely determines the
prescheme $X$ when $X$ is an integral scheme whose underlying space is
Noetherian.
\end{env}

\begin{env}[Proposition]{8.2.7}
\label{prop-1.8.2.7}
Let $X,Y$ be two integral schemes, $f\colon X\to Y$ a dominant morphism
\eref{2.2.6}, and $K$ (resp.$L$) the field of rational
functions on $X$ (resp.$Y$). Then $L$ can be identified with a subfield of
$K$, and for all $x\in X$, $\O_{f(x)}$ is the unique local ring of $Y$ dominated
by $\O_x$.
\end{env}

In fact, if $f=(\psi,\theta)$ and $a$ is the generic point of $X$, then
$\psi(a)$ is the generic point of $Y$ (\textbf{0},~2.1.5); $\theta_a^\#$ is then
a monomorphism of fields, from $L=\O_{\psi(a)}$ to $K=\O_a$. Since every
non-empty affine open $U$ of $Y$ contains $\psi(a)$, it follows from
\eref{2.2.4} that the homomorphism $\Gamma(U,\O_Y)\to\Gamma(\psi^{-1}(U),\O_X)$
corresponding to $f$ is the restriction of $\theta_a^\#$ to $\Gamma(U,\O_Y)$.
So, for every $x\in X$, $\theta_x^\#$ is the restriction to $\O_{\psi(a)}$ of
$\theta_a^\#$, and is thus a monomorphism. We also know that $\theta_x^\#$ is a
local homomorphism, so, if we identify $L$ with a subfield of $K$ by
$\theta_a^\#$, $\O_{\psi(x)}$ is dominated by $\O_x$ \sref{lem}{8.1.1}; it is also
the only local ring of $Y$ dominated by $\O_x$, since two local rings of $Y$
that are allied are identical \sref{prop}{8.2.2}.

\begin{env}[Proposition]{8.2.8}
\label{prop-1.8.2.8}
Let $X$ be an \emph{irreducible} prescheme; and
$f\colon X\to Y$ a local immersion (\emph{resp.} a local isomorphism); and
suppose further that $f$ is separated. Then $f$ is an immersion (\emph{resp.}
an open immersion).
\end{env}

Let $f=(\psi,\theta)$; it suffices, in both cases, to prove that $\psi$ is a
\emph{homeomorphism} from $X$ to $\psi(X)$ \eref{4.5.3}. Replacing $f$ by
$f_\mathrm{red}$ (\eref{5.1.6} and \eref{5.5.1},~(vi)), we can assume that $X$
and $Y$ are \emph{reduced}. If $Y'$ is the closed reduced subprescheme of $Y$
having $\overline{\psi(X)}$ as its underlying space, then $f$ factorises as
$X\xrightarrow{f'}Y'\xrightarrow{j}Y$, where $j$ is the canonical injection
\eref{5.2.2}. It follows from (\eref{5.5.1},~(v)) that $f'$ is again a
separated morphism; further, $f'$ is again \oldpage{168}a local immersion (resp.
a local isomorphism), because, since the condition is local on $X$ and $Y$, we
can reduce ourselves to the case where $f$ is a closed immersion (resp.open
immersion), and then our claim follows immediately from \eref{4.2.2}.

We can thus suppose that $f$ is a \emph{dominant} morphism, which leads to the
fact that $Y$ is, itself, irreducible (\textbf{0},~2.1.5), and so $X$ and $Y$
are both \emph{integral}. Further, the condition being local on $Y$, we can
suppose that $Y$ is an affine scheme; since $f$ is separated, $X$ is a scheme
(\eref{5.5.1},~(ii)), and we are finally at the hypotheses of \sref{prop}{8.2.7}.
Then, for all $x\in X$, $\theta_x^\#$ is injective; but the hypothesis that $F$
is a local immersion implies that $\theta_x^\#$ is surjective \eref{4.2.2}, so
$\theta_x^\#$ is bijective, or, equivalently (with the identification of
\eref{8.2.7}) we have that $\O_{\psi(x)}=\O_x$. This implies, by \sref{cor}{8.2.4},
that $\psi$ is an \emph{injective} map, which already proves the proposition
when $f$ is a local isomorphism \eref{4.5.3}. When we suppose that $f$ is only
a local immersion, for all $x\in X$ there exists an open neighbourhood $U$ of
$x$ in $X$ and an open neighbourhood $V$ of $\psi(x)$ in $Y$ such that the
restriction of $\psi$ to $U$ is a homeomorphism from $U$ to a \emph{closed}
subset of $V$. But $U$ is dense in $X$, so $\psi(U)$ is dense in $Y$ and
\emph{a fortiori} in $V$, which proves that $\psi(U)=V$; since $\psi$ is
injective, $\psi^{-1}(V)=U$ and this proves that $\psi$ is a homeomorphism from
$X$ to $\psi(X)$.

