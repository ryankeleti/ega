\section{Cohomology of affine schemes}
\label{section-cohomology-of-affine-schemes}

\subsection{Review of the exterior algebra complex}
\label{subsection-review-exterior-algebra-complex}

\begin{env}[1.1.1]
\label{3.1.1.1}
Let $A$ be a ring, $\mathbf{f}=(f_i)_{1\leq i\leq r}$ a system of $r$ elements of $A$.
The \emph{exterior algebra complex $K_\bullet(\mathbf{f})$} corresponding to $\mathbf{f}$ is a chain complex (G, I, 2.2) defined in the following way: the graded $A$-module $K_\bullet(\mathbf{f})$ is equal to the \emph{exterior algebra $\wedge(A^r)$}, graded in the usual way, and the boundary map is the \emph{interior multiplication $i_\mathbf{f}$} by $\mathbf{f}$ considered as an element of the dual $\dual{(A^r)}$; we recall that $i_\mathbf{f}$ is an \emph{antiderivation} of degree $-1$ of $\wedge(A^r)$, and if $(\mathbf{e}_i)_{1\leq i\leq r}$ is the canonical basis of $A^r$, then we have $i_\mathbf{f}(\mathbf{e}_i)=f_i$; the verification of the condition $i_\mathbf{f}\circ i_\mathbf{f}=0$ is immediate.

An equivalent definition is the following: for each $i$, we consider a chain complex $K_\bullet(f_i)$ defined as follows: $K_0(f_i)=K_1(f_i)=A$, $K_n(f_i)=0$ for $n\neq 0,1$: the boundary map is defined by the condition that $d_1:A\to A$ is \emph{multiplication by $f_i$}.
We then take $K_\bullet(\mathbf{f})$ to be the \emph{tensor product $K_\bullet(f_1)\otimes K_\bullet(f_2)\otimes\cdots\otimes K_\bullet(f_r)$} (G, I, 2.7) with its total degree; the verification of the isomorphism from this complex to the complex defined above is immediate.
\end{env}

\begin{env}[1.1.2]
\label{3.1.1.2}
For every $A$-module $M$, we define the \emph{chain complex}
\[
  K_\bullet(\mathbf{f},M)=K_\bullet(\mathbf{f})\otimes_A M
  \tag{1.1.2.1}
\]
and the \emph{cochain complex} (G, I, 2.2)
\[
  K^\bullet(\mathbf{f},M)=\Hom_A(K_\bullet(\mathbf{f},M).
  \tag{1.1.2.2}
\]

If $g$ is a $k$-cochain of this latter complex, and if we set
\[
  g(i_1,\dots,i_k)=g(\mathbf{e}_{i_1}\wedge\cdots\wedge\mathbf{e}_{i_k}),
\]
then $g$ identifies with an \emph{alternating} map from $[1,r]^k$ to $M$, and it follows from the above definitions that we have
\[
  d^k g(i_1,i_2,\dots,i_{k+1})=\sum_{h=1}^{k+1}(-1)^{h-1}f_{i_h}g(i_1,\dots,\wh{i_h},\dots,i_{k+1}).
  \tag{1.1.2.3}
\]
\end{env}

\begin{env}[1.1.3]
\label{3.1.1.3}
\oldpage[III]{83}
From the above complexes, we deduce as usual the \emph{homology and cohomology $A$-modules} (G, I, 2.2)
\[
  \HH_\bullet(\mathbf{f},M)=\HH_\bullet(K_\bullet(\mathbf{f},M)),
  \tag{1.1.3.1}
\]
\[
  \HH^\bullet(\mathbf{f},M)=\HH^\bullet(K^\bullet(\mathbf{f},M)).
  \tag{1.1.3.2}
\]

We define an \emph{$A$-isomorphism $K_\bullet(\mathbf{f},M)\isoto K^\bullet(\mathbf{f},M)$} by sending each chain $z=\sum(\mathbf{e}_{i_1}\wedge\cdots\wedge\mathbf{e}_{i_k})\otimes z_{i_1,\dots,i_k}$ to the cochain $g_z$ such that $g_z(j_1,\dots,j_{r-k})=\varepsilon z_{i_1,\dots,i_k}$, where $(j_h)_{1\leq h\leq r-k}$ is the strictly increasing sequence complementary to the strictly increasing sequence $(i_h)_{1\leq h\leq k}$ in $[1,r]$ and $\varepsilon=(-1)^\nu$, where $\nu$ is the number of inversions of the permutation $i_1,\dots,i_k,j_1,\dots,j_{r-k}$ of $[1,r]$.
We verify that $g_{dz}=d(g_z)$, which gives an isomorphism
\[
  \HH^i(\mathbf{f},M)\isoto\HH_{r-i}(\mathbf{f},M)\text{ for }0\leq i\leq r.
  \tag{1.1.3.3}
\]

In this chapter, we will especially consider the cohomology modules $\HH^\bullet(\mathbf{f},M)$.

For a given $\mathbf{f}$, it is immediate (G, I, 2.1) that $M\mapsto\HH^\bullet(\mathbf{f},M)$ is a \emph{cohomological functor} (T, II, 2.1) from the category of $A$-modules to the category of graded $A$-modules, zero in degrees $<0$ and $>r$.
In addition, we have
\[
  \HH^0(\mathbf{f},M)=\Hom_A(A/(\mathbf{f}),M),
  \tag{1.1.3.4}
\]
denoting by $(\mathbf{f})$ the ideal of $A$ generated by $f_1,\dots,f_r$; this follows immediately from (1.1.2.3), and it is clear that $\HH^0(\mathbf{f},M)$ identifies with the submodule of $M$ \emph{killed by $(\mathbf{f})$}.
Similarly, we have by (1.1.2.3) that
\[
  \HH^r(\mathbf{f},M)=M/\bigg(\sum_{i=1}^r f_i M\bigg)=(A/(\mathbf{f}))\otimes_A M.
  \tag{1.1.3.5}
\]

We will use the following known result, which we will recall a proof of to be complete:
\end{env}

\begin{prop}[1.1.4]
\label{3.1.1.4}
Let $A$ be a ring, $\mathbf{f}=(f_i)_{1\leq i\leq r}$ a finite family of elements of $A$, and $M$ an $A$-module.
If, for $1\leq i\leq r$, the scaling $z\mapsto f_i\cdot z$ on $M_{i-1}=M/(f_1 M+\cdots+f_{i-1}M)$ is injective, then we have $\HH^i(\mathbf{f},M)=0$ for $i\neq r$.
\end{prop}

It suffices to prove that $\HH_i(\mathbf{f},M)=0$ for all $i>0$ according to (1.1.3.3).
We argue by induction on $r$, the case $r=0$ being trivial.
Set $\mathbf{f}'=(f_i)_{1\leq i\leq r-1}$; this family satisfies the conditions in the statement, so if we set $L_\bullet=K_\bullet(\mathbf{f}',M)$, then we have $\HH_i(L_\bullet)=0$ for $i>0$ by hypothesis, and $\HH_0(L_\bullet)=M_{r-1}$ by virtue of (1.1.3.3) and (1.1.3.5).
To abbreviate, set $K_\bullet=K_\bullet(f_r)=K_0\oplus K_1$, with $K_0=K_1=A$, $d_1:K_1\to K_0$ multiplication by $f_r$; we have by definition \sref{3.1.1.1} that $K_\bullet(\mathbf{f},M)=K_\bullet\otimes_A L_\bullet$.
We have the following lemma:

\begin{lem}[1.1.4.1]
\label{3.1.1.4.1}
Let $K_\bullet$ be a chain complex of free $A$-modules, zero except in dimensions $0$ and $1$.
For every chain complex $L_\bullet$ of $A$-modules, we have an exact sequence
\[
  0\to\HH_0(K_\bullet\otimes\HH_p(L_\bullet))\to\HH_p(K_\bullet\otimes L_\bullet)\to\HH_1(K_\bullet\otimes\HH_{p-1}(L_\bullet))\to 0
\]
for every index $p$.
\end{lem}

\oldpage[III]{84}
This is a particular case of an exact sequece of low-order terms of the K\"unneth spectral sequence (M, XVII, 5.2 (a) and G, I, 5.5.2); it can be proved directly as follows.
Consider $K_0$ and $K_1$ as chain complexes (zero in dimensions $\neq 0$ and $\neq 1$ respectively); we then have an exact sequence of complexes
\[
  0\to K_0\otimes L_\bullet\to K_\bullet\otimes L_\bullet\to K_1\otimes L_\bullet\to 0,
\]
to which we can apply the exact sequence in homology
\[
  \cdots\to\HH_{p+1}(K_1\otimes L_\bullet)\xrightarrow{\partial}\HH_p(K_0\otimes L_\bullet)\to\HH_p(K_\bullet\otimes L_\bullet)\to\HH_p(K_1\otimes L_\bullet)\xrightarrow{\partial}\HH_{p-1}(K_0\otimes L_\bullet)\to\cdots.
\]
But it is evident that $\HH_p(K_0\otimes L_\bullet)=K_0\otimes\HH_p(L_\bullet)$ and $\HH_p(K_1\otimes L_\bullet)=K_1\otimes\HH_{p-1}(L_\bullet)$ for all $p$; in addition, we verify immediately that the operator $\partial:K_1\otimes\HH_p(L_\bullet)\to K_0\otimes\HH_p(L_\bullet)$ is none other than $d_1\otimes 1$; the lemma thus follows from the above exact sequence and the definition of $\HH_0(K_\bullet\otimes\HH_p(L_\bullet))$ and $\HH_1(K_\bullet\otimes\HH_{p-1}(L_\bullet))$.

The lemma having been established, the end of the proof of Proposition \sref{3.1.1.4} is immediate: the induction hypothesis of Lemma \sref{3.1.1.4.1} gives $\HH_p(K_\bullet\otimes L_\bullet)=0$ for $p\geq 2$; in addition if we show that $\HH_1(K_\bullet,\HH_0(L_\bullet))=0$, then we also deduce from Lemma \sref{3.1.1.4.1} that $\HH_1(K_\bullet\otimes L_\bullet)=0$; but by definition, $\HH_1(K_\bullet,\HH_0(L_\bullet))$ is none other than the kernel of the scaling $z\mapsto f_r\cdot z$ on $M_{r-1}$, and as by hypothesis this kernel is zero, this finishes the proof.

\begin{env}[1.1.5]
\label{3.1.1.5}
Let $\mathbf{g}=(g_i)_{1\leq i\leq r}$ be a second sequence of $r$ elements of $A$, and set $\mathbf{f}\mathbf{g}=(f_i g_i)_{1\leq i\leq r}$.
We can define a canonical homomorphism of complexes
\[
  \vphi_\mathbf{g}:K_\bullet(\mathbf{f}\mathbf{g})\to K_\bullet(\mathbf{f})
  \tag{1.1.5.1}
\]
as the canonical extension to the exterior algebra $\wedge(A^r)$ of the $A$-linear map $(x_1,\dots,x_r)\mapsto(g_1 x_1,\dots,g_r x_r)$ from $A^r$ to itself.
To see that we have a homomorphism of complexes, it suffices to note, in general, that if $u:E\to F$ is an $A$-linear map, and if $\mathbf{x}\in\dual{F}$ and $\mathbf{y}={}^t u(\mathbf{x})\in\dual{E}$, then we have the formula
\[
  (\wedge u)\circ i_\mathbf{y}=i_\mathbf{x}\circ(\wedge u);
  \tag{1.1.5.2}
\]
indeed, the two elements are antiderivations of $\wedge F$, and it suffies to check that they coincide on $F$, which follows immediately from the definitions.

When we identify $K_\bullet(\mathbf{f})$ with the tensor product of the $K_\bullet(f_i)$ \sref{3.1.1.1}, $\vphi_\mathbf{g}$ is the tensor product of the $\vphi_{g_i}$, where $\vphi_{g_i}$ is the identity in degree $0$ and multiplication by $g_i$ in degree $1$.
\end{env}

\begin{env}[1.1.6]
\label{3.1.1.6}
In particular, for every pair of integers $m$ and $n$ such that $0\leq n\leq m$, we have homomorphisms of complexes
\[
  \vphi_{\mathbf{f}^{m-n}}:K_\bullet(\mathbf{f}^m)\to K_\bullet(\mathbf{f}^n)
  \tag{1.1.6.1}
\]
and as a result, homomorphisms
\[
  \vphi_{\mathbf{f}^{m-n}}:K^\bullet(\mathbf{f}^n,M)\to K^\bullet(\mathbf{f}^m,M),
  \tag{1.1.6.2}
\]
\[
  \vphi_{\mathbf{f}^{m-n}}:\HH^\bullet(\mathbf{f}^n,M)\to\HH^\bullet(\mathbf{f}^m,M).
  \tag{1.1.6.3}
\]

\oldpage[III]{85}
The latter homomorphisms evidently satisfy the transitivity condition $\vphi_{\mathbf{f}^{m-p}}=\vphi_{\mathbf{f}^{m-n}}\circ\vphi_{\mathbf{f}^{n-p}}$ for $p\leq n\lq m$; they therefore define two \emph{inductive systems} of $A$-modules; we set
\[
  C^\bullet((\mathbf{f}),M)=\varinjlim_n K^\bullet(\mathbf{f}^n,M),
  \tag{1.1.6.4}
\]
\[
  \HH^\bullet((\mathbf{f}),M)=\HH^\bullet(C^\bullet((\mathbf{f}),M))=\varinjlim_n\HH^\bullet(\mathbf{f}^n,M),
  \tag{1.1.6.5},
\]
the last equality following from the fact that passing to the inductive limit commutes with the functor $\HH^\bullet$ (G, I, 2.1).
We will later see \sref{3.1.4.3} that $\HH^\bullet((\mathbf{f}),M)$ does not depend on the \emph{ideal $(\mathbf{f})$} of $A$ (and similarly on the $(\mathbf{f})$-pre-adic topology on $A$), which justifies the notations.

It is clear that $M\mapsto C^\bullet((\mathbf{f}),M)$ is an exact $A$-linear functor, and $M\mapsto\HH^\bullet((\mathbf{f}),M)$ is a cohomological functor.
\end{env}

\begin{env}[1.1.7]
\label{3.1.1.7}
Set $\mathbf{f}=(f_i)\in A^r$ and $\mathbf{g}=(g_i)\in A^r$; denote by $e_\mathbf{g}$ the left multiplication by the vector $\mathbf{g}\in A^r$ on the exterior algebra $\wedge(A^r)$; we know that we have the \emph{homotopy formula}
\[
  i_\mathbf{f}e_\mathbf{g}+e_\mathbf{g}i_\mathbf{f}=\langle\mathbf{g},\mathbf{f}\rangle 1
  \tag{1.1.7.1}
\]
in the $A$-module $A^r$ ($1$ denotes the identity automorphism of $A^r$); this relation also implies that \emph{in the complex $K_\bullet(\mathbf{f})$} we have
\[
  de_\mathbf{g}+e_\mathbf{g}d=\langle\mathbf{g},\mathbf{f}\rangle 1.
  \tag{1.1.7.2}
\]

If the ideal $(\mathbf{f})$ is equal to $A$, then there exists a $\mathbf{g}\in A^r$ such that $\langle\mathbf{g},\mathbf{f}\rangle=\sum_{i=1}^r g_i f_i=1$.
As a result (G, I, 2.4):
\end{env}

\begin{prop}[1.1.8]
\label{3.1.1.8}
Suppose that the ideal $(\mathbf{f})$ generated by the $f_i$ is equal to $A$.
Then the complex $K_\bullet(\mathbf{f})$ is homotopically trivial, and so are the complexes $K_\bullet(\mathbf{f},M)$ and $K^\bullet(\mathbf{f},M)$ for every $A$-module $M$.
\end{prop}

\begin{cor}[1.1.9]
\label{3.1.1.9}
If $(\mathbf{f})=A$, then we have $\HH^\bullet(\mathbf{f},M)=0$ and $\HH^\bullet((\mathbf{f}),M)=0$ for every $A$-module $M$.
\end{cor}

\begin{proof}
\label{proof-3.1.1.9}
Indeed, we then have $(\mathbf{f}^n)=A$ for all $n$.
\end{proof}

\begin{rmk}[1.1.10]
\label{3.1.1.10}
With the same notations as above, set $X=\Spec(A)$ and $Y$ the closed subprescheme of $X$ defined by the ideal $(\mathbf{f})$.
We will prove in \textsection9 that $\HH^\bullet((\mathbf{f}),M)$ is isomorphic to the cohomology $\HH_Y^\bullet(X,\wt{M})$ corresponding to the antifilter $\Phi$ of closed subsets of $Y$ (T, 3.2).
We will also show that Proposition \sref{3.1.2.3} applied to $X$ and to $\sh{F}=\wt{M}$ is a particular case of an exact sequence in cohomology
\[
  \cdots\to\HH_Y^p(X,\sh{F})\to\HH^p(X,\sh{F})\to\HH^p(X\setmin Y,\sh{F})\to\HH_Y^{p+1}(X,\sh{F})\to\cdots.
\]
\end{rmk}

\subsection{\v Cech cohomology of an open cover}
\label{subsection-cech-cohomology-open-cover}

