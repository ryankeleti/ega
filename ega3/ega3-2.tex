\section{Cohomological study of projective morphisms}
\label{section:III.2}


\subsection{Explicit calculations of certain cohomology groups}
\label{subsection:III.2.1}

\begin{env}[2.1.1]
\label{III.2.1.1}
Let $X$ be a prescheme, and $\sh{L}$ an invertible $\sh{O}_X$-module;
consider the graded ring \sref[0]{0.5.4.6}
\[
\label{III.2.1.1.1}
  S = \Gamma_\bullet(X,\sh{L}) = \bigoplus_{n\in\bb{Z}}\Gamma(X,\sh{L}^{\otimes n}).
\tag{2.1.1.1}
\]

Let $(f_i)_{1\leq i\leq r}$ be a finite family of \emph{homogeneous} elements of $S$, with $f_i\in S_{d_i}$;
set $U_i=X_{f_i}$ and $U=\bigcup_i U_i$, and denote by $\mathfrak{U}$ the cover $(U_i)$ of $U$.
For every quasi-coherent $\sh{O}_X$-module $\sh{F}$, we set
\[
\label{III.2.1.1.2}
  \HH^\bullet(U,\sh{F}(\anotherbullet)) = \bigoplus_{n\in\bb{Z}}\HH^\bullet(U,\sh{F}\otimes\sh{L}^{\otimes n})
\tag{2.1.1.2}
\]
\[
\label{III.2.1.1.3}
  \HH^\bullet(\mathfrak{U},\sh{F}(\anotherbullet)) = \bigoplus_{n\in\bb{Z}}\HH^\bullet(\mathfrak{U},\sh{F}\otimes\sh{L}^{\otimes n}).
\tag{2.1.1.3}
\]

The abelian groups \sref{III.2.1.1.2} and \sref{III.2.1.1.3} are \emph{bigraded}, by taking
\[
  (\HH^\bullet(U,\sh{F}(\anotherbullet)))_{mn} = \HH^m(U,\sh{F}\otimes\sh{L}^{\otimes n})
\]
and an analogous definition for \sref{III.2.1.1.3}.
For the second grading, it is clear that these groups are graded $S$-modules, as follows, for example, from the fact that $\sh{F}\mapsto\HH^m(U,\sh{F})$ and $\sh{F}\mapsto\HH^m(\mathfrak{U},\sh{F})$ are functors.
\end{env}

\begin{env}[2.1.2]
\label{III.2.1.2}
Now consider the graded $S$-module \sref[0]{0.5.4.6}
\[
\label{III.2.1.2.1}
  M = \Gamma_\bullet(\sh{L},\sh{F}) = \HH^0(X,\sh{F}(\anotherbullet)) = \bigoplus_{n\in\bb{Z}}(X,\sh{F}\otimes\sh{L}^{\otimes n}).
\tag{2.1.2.1}
\]

\oldpage[III]{96}
If $X$ is a prescheme whose underlying space is Noetherian, or a quasi-compact scheme, then it follows from \sref[I]{I.9.3.1} that, setting $U_{i_0i_1\ldots i_p}=\bigcap_{k=0}^p U_{i_k}$ as usual, we have, up to canonical isomorphism,
\[
  \Gamma(U_{i_0i_1\ldots i_p}, \sh{F}(\anotherbullet))
  = \HH^0(U_{i_0i_1\ldots i_p}, \sh{F}(\anotherbullet))
  = M_{f_{i_0}f_{i_1}\ldots f_{i_p}}.
\]

We can again, with the notation of \sref{III.1.2.2}, identify $M_{f_{i_0}f_{i_1}\ldots f_{i_p}}$ with $\varinjlim_n M_{i_0i_1\ldots i_p}^{(n)}$.
This identification is an isomorphism of \emph{graded} $S$-modules, if we define the degree of a homogeneous element $z\in\varinjlim_n M_{i_0i_1\ldots i_p}^{(n)}$ in the following way:
$z$ is the canonical image of a homogeneous element $x\in M_{i_0i_1\ldots i_p}^{(n)}=M$ of degree~$m$;
we then take the degree of $z$ to be the integer $m-n(d_{i_0}+d_{i_1}+\ldots+d_{i_p})$.
Taking into account the definition of the homomorphisms $\vphi_{kh}:M_{i_0i_1\ldots i_p}^{(h)}\to M_{i_0i_1\ldots i_p}^{(k)}$ \sref{III.1.2.2}, we immediately see that this definition does not depend on the ``representative'' $x$ of $z$ that we have chosen.
Denote, as in \sref{III.1.2.2}, by $C_n^p(M)$ the set of alternating maps from $[1,r]^{p+1}$ to $M$ (for all $n$), we define in the same way as above the structure of a \emph{graded} $S$-module on $\varinjlim_n C_n^p(M)$;
we again have, as in \sref{III.1.2.2},
\[
\label{III.2.1.2.2}
  C^p(\mathfrak{U}, \sh{F}(\anotherbullet))
  = \varinjlim_n C_n^p(M)
\tag{2.1.2.2}
\]
with the isomorphism \emph{respecting degrees}.
We then have, as in \sref{III.1.2.2},
\[
\label{III.2.1.2.3}
  C^p(\mathfrak{U}, \sh{F}(\anotherbullet))
  = C^{p+1}((\mathbf{f}), M)
  = \varinjlim_n K^{p+1}(\mathbf{f}^n, M)
\tag{2.1.2.3}
\]
with the isomorphism \emph{preserving degrees}:
the degree of an element of $\varinjlim_n K^{p+1}(\mathbf{f}^n, M)$, given by the canonical image of a cochain $g\in K^{p+1}(\mathbf{f}^n,M)$ such that the values $g(i_0,\ldots,i_p)$ are all in the \emph{same} homogeneous component $M_m$ of $M$, is $m-n(d_{i_0}+\ldots+d_{i_p})$, and it is independent of the choice of this cochain as a representative of the element in question.

Since the above isomorphisms are compatible with the coboundary operators, we thus conclude, as in \sref{III.1.2.2}, that we have:
\end{env}

\begin{proposition}[2.1.3]
\label{III.2.1.3}
Let
\end{proposition}


% \subsection{The fundamental theorem of projective morphisms}
% \label{subsection:III.2.2}


% \subsection{Application to graded sheaves of algebras and of modules}
% \label{subsection:III.2.3}


% \subsection{A generalisation of the fundamental theorem}
% \label{subsection:III.2.4}


% \subsection{Euler-Poincar\'e characteristic and the Hilbert polynomial}
% \label{subsection:III.2.5}


% \subsection{Application: ampleness criteria}
% \label{subsection:III.2.6}
