\section{Cohomological study of projective morphisms}
\label{section:III.2}


\subsection{Explicit calculations of certain cohomology groups}
\label{subsection:III.2.1}

\begin{env}[2.1.1]
\label{III.2.1.1}
Let $X$ be a prescheme, and $\sh{L}$ an invertible $\sh{O}_X$-module;
consider the graded ring \sref[0]{0.5.4.6}
\[
\label{III.2.1.1.1}
  S = \Gamma_\bullet(X,\sh{L}) = \bigoplus_{n\in\bb{Z}}\Gamma(X,\sh{L}^{\otimes n}).
\tag{2.1.1.1}
\]

Let $(f_i)_{1\leq i\leq r}$ be a finite family of \emph{homogeneous} elements of $S$, with $f_i\in S_{d_i}$;
set $U_i=X_{f_i}$ and $U=\bigcup_i U_i$, and denote by $\mathfrak{U}$ the cover $(U_i)$ of $U$.
For every quasi-coherent $\sh{O}_X$-module $\sh{F}$, we set
\[
\label{III.2.1.1.2}
  \HH^\bullet(U,\sh{F}(\anotherbullet)) = \bigoplus_{n\in\bb{Z}}\HH^\bullet(U,\sh{F}\otimes\sh{L}^{\otimes n})
\tag{2.1.1.2}
\]
\[
\label{III.2.1.1.3}
  \HH^\bullet(\mathfrak{U},\sh{F}(\anotherbullet)) = \bigoplus_{n\in\bb{Z}}\HH^\bullet(\mathfrak{U},\sh{F}\otimes\sh{L}^{\otimes n}).
\tag{2.1.1.3}
\]

The abelian groups \sref{III.2.1.1.2} and \sref{III.2.1.1.3} are \emph{bigraded}, by taking
\[
  (\HH^\bullet(U,\sh{F}(\anotherbullet)))_{mn} = \HH^m(U,\sh{F}\otimes\sh{L}^{\otimes n})
\]
and an analogous definition for \sref{III.2.1.1.3}.
For the second grading, it is clear that these groups are graded $S$-modules, as follows, for example, from the fact that $\sh{F}\mapsto\HH^m(U,\sh{F})$ and $\sh{F}\mapsto\HH^m(\mathfrak{U},\sh{F})$ are functors.
\end{env}

\begin{env}[2.1.2]
\label{III.2.1.2}
Now consider the graded $S$-module \sref[0]{0.5.4.6}
\[
\label{III.2.1.2.1}
  M = \Gamma_\bullet(\sh{L},\sh{F}) = \HH^0(X,\sh{F}(\anotherbullet)) = \bigoplus_{n\in\bb{Z}}(X,\sh{F}\otimes\sh{L}^{\otimes n}).
\tag{2.1.2.1}
\]

\oldpage[III]{96}
If
\end{env}


% \subsection{The fundamental theorem of projective morphisms}
% \label{subsection:III.2.2}


% \subsection{Application to graded sheaves of algebras and of modules}
% \label{subsection:III.2.3}


% \subsection{A generalisation of the fundamental theorem}
% \label{subsection:III.2.4}


% \subsection{Euler-Poincar\'e characteristic and the Hilbert polynomial}
% \label{subsection:III.2.5}


% \subsection{Application: ampleness criteria}
% \label{subsection:III.2.6}
