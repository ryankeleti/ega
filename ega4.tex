\ProvidesPackage{preamble}

\usepackage[utf8]{inputenc}
\usepackage[T1]{fontenc}
\usepackage{microtype}
\usepackage[left=0.75in,right=0.75in,top=0.75in,bottom=0.75in]{geometry}
\usepackage[all]{xy}
\usepackage{enumitem}
\usepackage{color}
\usepackage{soul}
\usepackage{fancyhdr}
\usepackage{mathtools}
\usepackage{amssymb}
\usepackage{amsthm}
\usepackage[charter,
            greekfamily=didot,
            uppercase=upright,
            greeklowercase=upright]{mathdesign}
\usepackage[compact]{titlesec}
\usepackage[colorlinks=true,hyperindex,citecolor=blue,linkcolor=magenta]{hyperref}
\usepackage{bookmark}
\usepackage[asterism]{sectionbreak}


%%%%%%%%%%%%%%
% formatting %
%%%%%%%%%%%%%%

\allowdisplaybreaks[1]
\binoppenalty=9999
\relpenalty=9999
\setitemize{nosep}

% for Chapter 0, Chapter I, etc.
% credit for ZeroRoman https://tex.stackexchange.com/questions/211414/
\newcommand{\ZeroRoman}[1]{\ifcase\value{#1}\relax 0\else\Roman{#1}\fi}
\renewcommand{\thechapter}{\ZeroRoman{chapter}}

%%%%%%%%%%%%%%%%%
% math commands %
%%%%%%%%%%%%%%%%%

% for easy changes to style
\newcommand{\sh}{\mathscr}         % sheaf font
\newcommand{\bb}{\mathbf}          % bold font
\newcommand{\cat}{\mathsf}         % category font
%
\newcommand{\rad}{\mathfrak{r}}    % radical
\newcommand{\nilrad}{\mathfrak{R}} % nilradical
\newcommand{\emp}{\varnothing}     % empty set
\newcommand{\vphi}{\phi}           % font switches \phi and \varphi, change if needed
\newcommand{\HH}{\mathrm{H}}       % cohomology
\newcommand{\dual}[1]{{#1}^\vee}   % dual
\renewcommand{\k}{\bb{k}}          % residue field
\newcommand{\K}{\cat{K}}           % category
\newcommand{\OO}{\sh{O}}           % structure sheaf
\newcommand{\F}{\sh{F}}            % sheaf F
\newcommand{\G}{\sh{G}}            % sheaf G

% operators
%\newcommand*{\sheafHom}{\mathscr{H}\text{\normalfont\kern -3pt {\calligra\large om}}\,}
\def\shHom{\sh{H}\textit{om}} % sheaf Hom
\def\Hom{{\mathop{\mathrm{Hom}}\nolimits}}
\def\Supp{{\mathop{\mathrm{Supp}}\nolimits}}
\def\img{{\mathop{\mathrm{im}}\nolimits}}
\def\Spec{{\mathop{\mathrm{Spec}}\nolimits}}

% if unsure of a translation
\newcommand{\unsure}[2][]{\hl{#2}\marginpar{#1}}
\newcommand{\completelyunsure}{\unsure{[\ldots]}}

% use to mark where original page starts
\newcommand{\oldpage}[1]{\marginpar{\textbf{#1}}\ignorespaces}

% special ref
\newcommand{\sref}[2]{\hyperref[#1-\arabic{chapter}.#2]{\normalfont{(#2)}}}

% ref prelim
\newcommand{\pref}[2]{\hyperref[#1-0.#2]{\normalfont{(\textbf{0}, #2)}}}

%% ref out of chapter
%\newcommand{\cref}[4]{\hyperref[#1-#2.#3]{\normalfont{(\textbf{#3}, #4)}}}

% currently this works as \begin{env}[optional rmk]{x.y.z}
\makeatletter
\newenvironment{env}[2][\@nil]{%
    \def\tmp{#1}%
    \ifx\tmp\@nnil
        \par\medskip\noindent\indent\textbf{(#2)}\rmfamily
    \else
        \par\medskip\noindent\indent\textit{\textbf{#1}}~\textbf{(#2)}.\,---\rmfamily
    \fi}
\makeatother

% use this for definitions, propositions, corollaries, etc.
\makeatletter
\newenvironment{envs}[2][\@nil]{
  \par\medskip\noindent\indent\textit{\textbf{#1}}~\textbf{(#2)}.\,---\itshape
}
\makeatother



\begin{document}
\title{Local study of schemes and their morphisms (EGA IV)}
\maketitle

\phantomsection
\label{section:ega4}

build hack
\cite{I-1}

\tableofcontents

\section*{Summary}

\oldpage[IV-1]{222}
\begin{longtable}{ll}
  \textsection\hyperref[section:IV.1]{1}.   & Relative finiteness conditions. Constructible sets of preschemes.\\
  \textsection\hyperref[section:IV.2]{2}.   & Base change and flatness.\\
  \textsection\hyperref[section:IV.3]{3}.   & Associated prime cycles and primary decomposition.\\
  \textsection\hyperref[section:IV.4]{4}.   & Change of base field for algebraic preschemes.\\
  \textsection\hyperref[section:IV.5]{5}.   & Dimension and depth for preschemes.\\
  \textsection\hyperref[section:IV.6]{6}.   & Flat morphisms of locally Noetherian preschemes.\\
  \textsection\hyperref[section:IV.7]{7}.   & Application to the relations between a local Noetherian ring and its completion. Excellent rings.\\
  \textsection\hyperref[section:IV.8]{8}.   & Projective limits of preschemes.\\
  \textsection\hyperref[section:IV.9]{9}.   & Constructible properties.\\
  \textsection\hyperref[section:IV.10]{10}. & Jacobson preschemes.\\
  \textsection\hyperref[section:IV.11]{11}.\footnote{The order and content of \textsection\textsection11--21 are given only as an indication of what the titles will be, and will possibly be modified before their publication. \emph{[Trans.] This was indeed the case: many of \textsection\textsection11--21 ended up having entirely different titles or content. See \hyperref[section:what-ega4-sections]{here}.}} & Topological properties of finitely presented flat morphisms. Flatness criteria.\\
  \textsection\hyperref[section:IV.12]{12}. & Study of fibres of finitely presented flat morphisms.\\
  \textsection\hyperref[section:IV.13]{13}. & Equidimensional morphisms.\\
  \textsection\hyperref[section:IV.14]{14}. & Universally open morphisms.\\
  \textsection\hyperref[section:IV.15]{15}. & Study of fibres of a universally open morphism.\\
  \textsection\hyperref[section:IV.16]{16}. & Differential invariants. Differentially smooth morphisms.\\
  \textsection\hyperref[section:IV.17]{17}. & Smooth morphisms, unramified (or net) morphisms, and \'etale morphisms.\\
  \textsection\hyperref[section:IV.18]{18}. & Supplement on \'etale morphisms. Henselian local rings and strictly local rings.\\
  \textsection\hyperref[section:IV.19]{19}. & Regular immersions and normal flatness.\\
  \textsection\hyperref[section:IV.20]{20}. & Meromorphic functions and pseudo-morphisms\\
  \textsection\hyperref[section:IV.21]{21}. & Divisors.
\end{longtable}
\bigskip

\oldpage[IV-1]{223}
The subjects discussed in the chapter call for the following remarks.
\begin{enumerate}
  \item[(a)] The common property of all the subjects discussed is that they all related to \emph{local} properties of preschemes or morphisms, i.e. considered at a point, or the points of a fibre, or on a (non-specified) neighbourhood of a point or of a fibre.
    These properties are generally of a \emph{topological}, \emph{differential}, or \emph{dimensional} nature (i.e. bringing the ideas of \emph{dimension} and \emph{depth} into play), and are linked to the properties of the \emph{local rings} at the points considered.
    One type of problem is the relating, for a given morphisms $f:X\to Y$ and point $x\in X$, of the properties of $X$ at $x$ with those of $Y$ at $y=f(x)$ and those of the fibre $X_y=f^{-1}(y)$ at $x$.
    Another is the determining of the topological nature (for example, the constructibility, or the fact of being open or closed) of the set of points $x\in X$ at which $X$ has a certain property, or for which the fibre $X_{f(x)}$ passing through $x$ has a certain property at $x$.
    Similarly, we are interested in the topological nature of the set of points $y\in Y$ such that $X$ has a certain property at all the points of the fibre $X_y$, or those such that this fibre itself has a certain property.
  \item[(b)] The most important idea for the following chapters is that of \emph{flat morphisms of finite presentation}, as well as the particular cases of \emph{smooth morphisms} and \emph{\'etale morphisms}.
    Their detailed study (as well as that of connected questions) really starts in \textsection11.
  \item[(c)] Sections \textsection\textsection1--10 can be considered as being preliminary in nature, and as developing three types of techniques, used, not only in the other sections of the chapter, but also, of course, in the follow chapters:
    \begin{enumerate}
      \item[(c1)] Sections \textsection\textsection1--4 are envisaged as treating the diverse aspects of the idea of \emph{change of base}, above all in relation with the conditions of \emph{finiteness} or \emph{flatness}; we there initiate the technique of \emph{descent}, with its most elementary aspects (the questions of ``effectiveness'' linked to this technique will be studied in Chapter~V).
      \item[(c2)] Sections \textsection\textsection5--7 are focused on what we may call \emph{Noetherian} techniques, since the preschemes considered are always locally Noetherian, whereas, on the contrary, there is generally no finiteness condition imposed on the \emph{morphisms}; this is essentially due to the fact that the ideas of dimension and depth are hardly manageable except in the case of Noetherian local rings.
        Recall that \textsection7 constitutes a ``\unsure{delicate}'' theory of Noetherian local rings, not much used in what follows in the chapter.
      \item[(c3)] Sections \textsection\textsection8--10 describe, amongst other things, the means of \emph{eliminating the Noetherian hypotheses} on the preschemes considered, by substituting such hypotheses for suitable ones of \emph{finiteness} (``finite presentation'') on the \emph{morphisms} considered: the advantage of this substitution is that the latter such hypotheses (those of finiteness on the morphisms) are \emph{stable under base change}, which is not the case for the Noetherian hypotheses on the preschemes.
        The technique permitting this substitution relies, in some part, on the use of the idea of the \emph{projective limit} of preschemes, thanks to which we can reduce a question to the same question with \emph{Noetherian} hypotheses; on the other hand, it relies on the systematic use of \emph{constructible sets}, which have the double interest of being preserved under taking inverse images (of arbitrary morphisms)
\oldpage[IV-1]{224}
        and by direct images (of morphisms of finite presentation), and having manageable topological properties in locally Noetherian preschemes.
        The same techniques often even allow to restrict to the case of more specific Noetherian rings, for example the \emph{$\bb{Z}$-algebras of finite type}, and it is here that the properties of ``excellent'' rings (studied in \textsection7) intervene in a decisive manner.
        Independently of the question of elimination of Noetherian hypotheses, the techniques of \textsection\textsection8--10, elementary in nature, find constant use in nearly all applications.
    \end{enumerate}
\end{enumerate}

\section{Relative finiteness conditions. Constructible sets of preschemes}
\label{section:relative-finiteness-conditions-constructible-sets-of-preschemes}

In this section. we will resume the expos\'e of ``finiteness conditions'' for a morphism of preschemes $f:X\to Y$ given in~(\textbf{I},~\hyperref[subsection:1.6.3]{6.3}~and~\hyperref[subsection:1.6.6]{6.6}).
There are essentially two notions of ``finiteness'' of a \emph{global} nature on $X$, that of \emph{quasi-compact} morphism (defined in~\sref[I]{1.6.6.1}) and that of a \emph{quasi-separated} morphism; on the other hand, there are two notions of ``finiteness'' of a \emph{local} nature on $X$, that of a morphism \emph{locally of finite type} (defined in~\sref[I]{1.6.6.2}) and that of a morphism \emph{locally of finite presentation}.
By combining these local notions with the preceding global notions, we obtain the notion of a morphism \emph{of finite type} (defined in~\sref[I]{1.6.3.1}) and of a morphism \emph{of finite presentation}.
For the convenience of the reader, we will give again in this section the properties stated in~(\textbf{I},~\hyperref[subsection:1.6.3]{6.3}~and~\hyperref[subsection:1.6.6]{6.6}), referring to their labels in Chapter~I for their proofs.

In~n\textsuperscript{os}\hyperref[subsection:4.1.8]{1.8} and~\hyperref[subsection:4.1.9]{1.9}, we complete, in the context of preschemes, and making use of the previous notions of finiteness, the results on constructible sets given in~(\textbf{0}\textsubscript{III},~\textsection\hyperref[section:1.9]{9}).

\subsection{Quasi-compact morphisms}
\label{subsection:quasi-compact-morphisms}

\begin{defn}[1.1.1]
\label{4.1.1.1}
We say that a morphism of preschemes $f:X\to Y$ is \emph{quasi-compact} if the continous map $f$ from the topological space $X$ to the topological space $Y$ is quasi-compact~\sref[0]{0.9.1.1}, in other words, if the inverse image $f^{-1}(U)$ of every quasi-compact open subset $U$ of $Y$ is quasi-compact~(cf.~\sref[I]{1.6.6.1}).
\end{defn}

If $\mathfrak{B}$ is a basis for the topology of $Y$ consisting of affine open sets, then for $f$ to be quasi-compact, it is necessary and sufficient that for all $V\in\mathfrak{B}$, $f^{-1}(V)$ is a \emph{finite union of affine open sets}.
For example, if $Y$ is affine and $X$ is quasi-compact, \emph{every} morphism $f:X\to Y$ is quasi-compact~\sref[I]{1.6.6.1}.

If $f:X\to Y$ is a quasi-compact morphism, then it is clear that for every open subset $V$ of $Y$, the restriction of $f$ to $f^{-1}(V)$ is a quasi-compact morphism $f^{-1}(V)\to V$.
Conversely, if $(U_\alpha)$ is an open cover of $Y$ and $f:X\to Y$ is a morphism such that the restrictions $f^{-1}(U_\alpha)\to U_\alpha$ are quasi-compact, then $f$ is quasi-compact.
As
\oldpage[IV-1]{225}
a result, if $f:X\to Y$ is an $S$-morphism of $S$-preschemes, and if there exists an open cover $(S_\lambda)$ of $S$ such that the restrictions $g^{-1}(S_\lambda)\to h^{-1}(S_\lambda)$ of $f$ (where $g$ and $h$ are the structure morphisms) are quasi-compact, then $f$ is quasi-compact.




% \section{Base change and flatness}
\label{section:IV.2}


% \section{Associated prime cycles and primary decomposition}
\label{section:IV.3}


% \section{Change of base field for algebraic preschemes}
\label{section:IV.4}


% \section{Dimension, depth, and regularity in locally Noetherian preschemes}
\label{section:IV.5}


% \section{Flat morphisms of locally Noetherian preschemes}
\label{section:IV.6}


% \section{Relations between a local Notherian ring and its completion. Excellent rings}
\label{section:IV.7}


% \section{Projective limits of preschemes}
\label{section:IV.8}


% \section{Constructible properties}
\label{section:IV.9}


% \section{Jacobson preschemes}
\label{section:IV.10}


% \section{Topological properties of finitely presented flat morphisms. Flatness criteria}
\label{section:IV.11}


% \section{Study of fibres of finitely presented flat morphisms}
\label{section:IV.12}


% \section{Equidimensional morphisms}
\label{section:IV.13}


% \section{Universally open morphisms}
\label{section:IV.14}


% \section{Study of fibres of a universally open morphism}
\label{section:IV.15}


\setcounter{section}{15}
\section{Differential invariants. Differentially smooth morphisms}
\label{section:IV.16}

\oldpage[IV-4]{5}
In this paragraph we will present, in global form, some notions of differential calculus particularly useful in algebraic geometry.
We will ignore many classic developments in differential geometry (connections, infinitesimal transformations associated to vector fields, jets, etc.), although these notions are translated in a particularly natural way for schemes.
We will similarly ignore phenomena exclusive to characteristic $p>0$ (some of which are seen, in the affine case, in \hyperref[section:0.21]{(\textbf{0}, 21)}.
For certain complements to the differential formalism for preschemes the reader may consult Expos\'es~II and VII of \cite{IV-42} as well as subsequent chapters of this treatise. 

\subsection{Normal invariants of an immersion}
\label{IV.16.1}

\begin{env}[16.1.1]
\label{IV.16.1.1}

Let $(X, \sh{O}_X), (Y, \sh{O}_Y)$ be two ringed spaces and $f = (\psi, \theta): Y \to X$ a morphism of ringed spaces \sref[0]{0.4.1.1} such that the homomorphism
\[
  \theta^\#: \psi^*(\sh{O}_X) \to \sh{O}_Y
\]
is surjective, so that $\sh{O}_Y$ is identified with a sheaf of quotient rings $\psi^*(\sh{O}_X)/\sh{I}_f$. 
We can then endow $\psi^*(\sh{O}_X)$ with the $\sh{I}_f$-preadic filtration.
\end{env}

\begin{definition}[16.1.2]
\label{IV.16.1.2}
The $\sh{O}_Y$-augmented sheaf of rings $\psi^*(\sh{O}_X)/\sh{I}_f^{n+1}$ is called the $n$'th \emph{normal invariant} of $f$;
the ringed space $(Y, \psi^*(\sh{O}_X)/\sh{I}_f^{n+1})$ is called the $n$'th \emph{infinitesimal neighborhood} of $Y$ along $f$ and is denoted by $Y^{(n)}_f$ or simply $Y^{(n)}$.
The sheaf of graded rings associated to the sheaf of filtered rings $\psi^*(\sh{O}_X)$
\[
  \label{IV.16.1.2.1}
  \shGr_\bullet(f) = \bigoplus_{n \geq 0}(\sh{I}_f^{n}/\sh{I}_f^{n+1} )
  \tag{16.1.2.1}
\]
is called the sheaf of graded rings \emph{associated to} $f$. The sheaf $\shGr_1(f) = \sh{I}_f/\sh{I}_f^{2}$ is called the \emph{conormal sheaf} of $f$ (that will be denoted by $\sh{N}_{Y/X}$ when there is no risk of confusion). 
\end{definition}

It is clear that the $\sh{O}_{Y^{(n)}} = \psi^*(\sh{O}_X)/\sh{I}_f^{n+1}$ (that we also denote $\sh{O}_{Y_f^{(n)}})$ form a
\oldpage[IV-4]{6}
projective system of sheaves of rings on $Y$, the transition homomorphism $\phi_{nm}:\sh{O}_{Y^{(m)}} \to \sh{O}_{Y^{(n)}}$ for $n \leq m$ identifies $\sh{O}_{Y^{(n)}}$ with the quotient of $\sh{O}_{Y^{(m)}}$ by the power $(\sh{I}_f/\sh{I}_f^{n+1} )^m$ of the \emph{agumentation ideal} of $\sh{O}_{Y^{(n)}}$, kernel of $\phi_{0n}: \sh{O}_{Y^{(n)}} \to \sh{O}_{Y}$.
The $Y^{(n)}$ therefore form a inductive system of ringed spaces, all having underlying space $Y$, and we have canonical morphisms of ringed spaces $h_n: Y^{(n)} \to X$ equal to $(\psi, \theta_n)$, where $\theta^\#_n$ is the canonical morphism $\psi^*(\sh{O}_X) \to \psi^*(\sh{O}_X)/\sh{I}_f^{n+1}$.
It is clear that the sheaf $\shGr_\bullet(f)$ is a sheaf of graded algebras over the sheaf of rings $\sh{O}_Y = \shGr_0(f)$ and the $\shGr_k(f)$ of $\sh{O}_Y$-modules.

As with every sheaf of filtered rings, we have a \emph{canonical surjective homomorphism} of graded $\sh{O}_Y$-algebras
\[
  \label{IV.16.1.2.2}
  \bb{S}_{\shGr_1(f)}^\bullet \to \shGr_\bullet(f)
  \tag{16.1.2.2}
\]
which coincide in degrees $0$ and $1$ with the identities.

\begin{examples}[16.1.3]
\label{IV.16.1.3}
\begin{enumerate}
  \item[(i)] Suppose that $X$ is a locally ringed space, $Y$ is reduced to a single point $y$ (endowed with a ring $\sh{O}_y$) and that, if $x = \psi(y)$, $\theta^\#:\sh{O}_x \to \sh{O}_y$ is a \emph{surjective} homomorphism of rings having as kernel the maximal ideal $\mathfrak{m}_x$ of $\sh{O}_x$.
  So the $\sh{O}_{Y^{(n)}}$ are identified with the rings $\sh{O}_x/\mathfrak{m}_x^{n+1}$ and $\shGr_\bullet(f)$ with the graded ring associated with the local ring $\sh{O}_x$ endowed with the $\mathfrak{m}_x$-preadic filtration.
  \item[(ii)] Suppose that $Y$ is a closed subset of an open subspace $U$ of $X$ and that the $\sh{O}_Y$ is induced on $Y$ by a quotient sheaf $\sh{O}_U/\sh{I}$, where $\sh{I}$ is an ideal of $\sh{O}_U$ such that $\sh{I}_x = \sh{O}_x$ for every $x \not\in Y$;
  if $X$ is a locally ringed space we also suppose that $\sh{I}_x \neq \sh{O}_x$ for $y \in Y$ so that $(Y, \sh{O}_Y)$ is a locally ringed space.
  
  Let $\psi_0: Y \to U$ be the canonical injection and denote by $\theta_0: \sh{O}_U \to (\psi_0)_*(\sh{O}_Y)$ the homomorphism such that $\theta_0^\#$ is the canonical homomorphism $\psi^*_0(\sh{O}_U) = \sh{O}_U|Y \to (\sh{O}_U/\sh{I})|Y$, so that $j_0=(\psi_0, \theta_0):Y \to U$ is a morphism of ringed spaces (and of locally ringed spaces if $X$ is a locally ringed space);
  if $i:U \to X$ is the canonical injection (morphism of ringed spaces), $j = i\circ j_0$ is the morphism $(\psi, \theta)$ of $Y$ to $X$ where $\psi: Y \to X$ is the canonical injection and $\theta:\sh{O}_X \to \psi_*(\sh{O}_Y)$ is the homomorphism such that $\theta^\# = \theta_0^\#$.
  Since $\theta^\#$ is surjective we can apply the previous definitions;
  $\sh{O}_{Y^{(n)}}$ is equal to $\psi^*_0(\sh{O}_U/\sh{I}^{n+1})$, and we have $(\psi_0)_*(\sh{O}_{Y^{(n)}} ) = \sh{O}_U/\sh{I}^{n+1}$, and $\shGr_n(j) = \shGr_n(j_0) = \psi^*_0(\sh{I}^n/\sh{I}^{n+1}) = j^*_0(\sh{I}^n/\sh{I}^{n+1})$.
  %I am pretty sure it should be \psi^* instead of j^*_0 in the last line... ~solov-t
\end{enumerate}
\end{examples}

\begin{env}[16.1.4]
\label{IV.16.1.4}
The example \sref{IV.16.1.3}[ii] shows that in general the $\sh{O}_{Y^{(n)}}$ are \emph{not canonically endowed with a structure of $\sh{O}_Y$-module}, or \emph{a fortiori} with a structure of $\sh{O}_Y$-algebra.
The data of such structure is equivalent to the data of a homomorphism of sheaves of rings $\lambda_n:\sh{O}_Y \to \sh{O}_{Y^{(n)}}$, right inverse to the augmentation morphism $\phi_{0n}$;
it is also equivalent to the data of a morphism of ringed spaces $(I_Y, \lambda_n): Y^{(n)} \to Y$ right inverse to the canonical morphism $(I_Y, \phi_{0n}): Y \to Y^{(n)}$.
\end{env}

\begin{proposition}[16.1.5]
\label{IV.16.1.6}
Let $f = (\psi, \theta): Y \to X$ be an immersion of preschemes. We have:
\begin{enumerate}
  \item[(i)] $\shGr_\bullet(f)$ is a quasi-coherent graded $\sh{O}_Y$-algebra.
\oldpage[IV-4]{7}
  \item[(ii)] The $Y^{(n)}$ are preschemes, canonically isomorphic to subpreschemes of $X$.
  \item[(iii)] Every homomorphism of sheaves of rings $\lambda_n: \sh{O}_Y \to \sh{O}_{Y^{(n)}}$, right inverse to the augmentation homomorphism $\phi_{0n}$, makes the $\sh{O}_{Y^{(n)}}$ and $\sh{O}_{Y^{(k)}}$ for $k\leq n$ quasi-coherent $\sh{O}_Y$-algebras;
  the structure of $\sh{O}_Y$-module deducted from the preceding structures on the $\shGr_k(f)$ for $k \leq n$ coincide with the ones defined in \sref{IV.16.1.2}.
\end{enumerate}
\end{proposition}

\begin{proof}
(i) Since the question is local on $X$ and $Y$, we can reduce to the case where $Y$ is a closed subpreschemes of $X$ defined by an quasi-coherent ideal $\sh{I}$ of $\sh{O}_X$;
since $\sh{O}_Y$ is the restriction to $Y$ of $\sh{O}_X/\sh{I}$ the assertion (i) is evident, and $Y^{(n)}$ is the closed subprescheme of $X$ defined by the quasi-coherent ideal $\sh{I}^{n+1}$ of $\sh{O}_X$.
Finally, to prove (iii) we notice that the data of $\lambda_n$ makes the ideal $\sh{I}/\sh{I}^n$ of the augmentation $\phi_{0n}$ and their quotients $\sh{I}/\sh{I}^{k+1} (1\leq k \leq n)$ $\sh{O}_Y$-modules, and it suffices to prove by induction on $k$ that the $\sh{I}/\sh{I}^{k+1}$ are quasi-coherent $\sh{O}_Y$-modules and the structure of quotient $\sh{O}_Y$-module induced on $\sh{I}^k/\sh{I}^{k+1}$ is the same as defined on \sref{IV.16.1.2}.
The second assertion is immediate, $\sh{I}^k/\sh{I}^{k+1}$ being killed by $\sh{I}/\sh{I}^{n+1}$;
the first result, by induction on $k$, is trivial for $k=1$ and for $\sh{I}/\sh{I}^{k+1}$ being an extension of $\sh{I}/\sh{I}^{k}$ by $\sh{I}^k/\sh{I}^{k+1}$ \hyperref[section:III.1.4.17]{(\textbf{III}, 1.4.17)}.
\end{proof}

\begin{corollary}[16.1.6]
\label{IV.16.1.6}
Under the general hypothesis of \sref{IV.16.1.5}, if the immersion $f$ is locally of finite presentation then the $\shGr_n(f)$ are quasi-coherent $\sh{O}_Y$-modules of finite type.
\end{corollary}

\begin{proof}
Indeed, with the notation from the proof of \sref{IV.16.1.5}, $\sh{I}$ is an ideal of finite type of $\sh{O}_X$ \sref{IV.1.4.7}, therefore the $\sh{I}^n/\sh{I}^{n+1}$ are $\sh{O}_Y$-modules of finite type, hence the conclusion.
\end{proof}

\begin{corollary}[16.1.7]
\label{IV.16.1.7}
Under the general hypotheses of \sref{IV.16.1.5}, let $g:X \to Y$ be a morphism of preschemes, left inverse to $f$.
Therefore, for every $n$, the composite morphism $(I, \lambda_n): Y^{(n)}\xrightarrow{h_n} X \xrightarrow{g} Y$ defines a homomorphism of sheaves of rings $\lambda_n: \sh{O}_Y \to \sh{O}_{Y^{(n)}}$ right inverse to the augmentation $\phi_{0n}$, making $\sh{O}_{Y^{(n)}}$ a quasi-coherent $\sh{O}_Y$-algebra;
via these homomorphisms, the transition homomorphism $\phi_{nm}:\sh{O}_{Y^{(m)}} \to \sh{O}_{Y^{(n)}}$ ($n\leq m$) are homomorphisms of $\sh{O}_Y$-algebras. 
Also, if $g$ is locally of finite type, then the $\sh{O}_{Y^{(n)}}$ are quasi-coherent $\sh{O}_Y$-modules of finite type.
\end{corollary}

\begin{proof}
The first assertion is an immediate result from the definitions and \sref{IV.16.1.5}.
On the other hand, if $g$ is locally of finite type, then $f$ is locally of finite presentation \sref{IV.1.4.3}[(v)];
the $\shGr_n(f)$ being then quasi-coherent $\sh{O}_Y$-modules of finite type by \sref{IV.16.1.6}, the same goes for the $\sh{O}_Y$-modules $\sh{I}/\sh{I}^{n+1}$, being extensions of a finite number of the $\shGr_k(f)$ \sref[III]{III.1.4.17}.
\end{proof}

\begin{proposition}[16.1.8]
\label{IV.16.1.8}
Let $X$ be a locally Noetherian prescheme, $j:Y \to X$ an immersion;
Then the $Y^{(n)}$ are locally Noetherian preschemes, the $\shGr_n(j)$ are coherent $\sh{O}_Y$-modules and the $\shGr_\bullet(j)$ is a coherent sheaf of rings over the space $Y$.
\end{proposition}

\begin{proof}
Everything is local on $X$ and $Y$, so we reduce to the case where $X$ is affine and $j$ is a closed immersion and therefore all the assertions are evident except for the last, which results from the fact that if $A$ is a Noetherian ring and $\mathfrak{I}$ is an ideal of $A$, $\gr_\mathfrak{I}^\bullet(A)$ is a Noetherian ring, taking into account the exactness of the functor $\psi^*$ and \sref[0]{0.5.3.7}.
\end{proof}

\begin{proposition}[16.1.9]
\label{IV.16.1.9}
\oldpage[IV-4]{8}
Let $X$ be a prescheme, $j: Y \to X$ an immersion locally of finite presentation, $y$ a point of $Y$. The following conditions are equivalent:
\begin{enumerate}
  \item[(a)] There exists an open neighborhood $U$ of y in $Y$ such that $j|U$ is a homeomorphism of $U$ onto an open set of $X$.
  \item[(b)] There is an integer $n>0$ such that the canonical homomorphism
  \[
    (\phi_{n-1,n})_y: \sh{O}_{Y^{(n)},y} \to \sh{O}_{Y^{(n-1)},y}
  \]
  is bijective.
  \item[(c)] There is an integer $n>0$ such that $(\shGr_n(j))_y = 0$.
  
  In addition, if the integer $n$ satisfiess \emph{(b)} or \emph{(c)}, then there is a neighborhood $V$ of $y$ in $Y$ such that $\shGr_m(j)|V = 0$ for $m \geq n$ and that $\phi_{nm}|V: \sh{O}_{Y^{(m)}}|V \to \sh{O}_{Y^{(n)}}|V$ is bijective for $m \geq n$. 
\end{enumerate}
\end{proposition}

\begin{proof}
The question being local on $Y$, we can restrict ourselves to the case where $j$ is a closed immersion, $Y$ being defined by a quasi-coherent ideal \emph{of finite type} $\mathfrak{I}$ of $\sh{O}_X$.
The equivalence of (b) and (c), for a given $n$, is immediate;
also, since $\sh{I}^n/\sh{I}^{n+1}$ is an $\sh{O}_X$-module of finite type, there is an open neighborhood $U$ of $y$ in $X$ such that $\sh{I}^n|U = \sh{I}^{n+1}|U$ \sref[0]{0.5.2.2}, so we also have $\sh{I}^n|U = \sh{I}^m|U$ for $m \geq n$ proving the last assertions.
To prove that (a) implies (b), we can restrict ourselves to the cases where the underlying space of $Y$ is equal to the underlying space of $X$ and where $\sh{I}$ is generated by a finite number of sections over $X$:
since $\sh{I}$ is contained in the nilradical $\sh{N}$ of $\sh{O}_X$ \sref[I]{I.5.1.2}, it is now nilpotent which proves b).
Finally, to prove that (b) implies (a), we can restrict ourselves to the case where $\sh{I}^n = \sh{I}^m$; 
therefore, for every $y \in Y$, since $\sh{I}_y \subset \mathfrak{m}_y$, maximal ideal of $\sh{O}_{X,y}$, we must have $\sh{I}^n_y = 0$ because of Nakayama's lemma, since $\sh{I}_y$ is an ideal of finite type.
The set of $x \in X$ such that $\sh{I}^n_x = 0$ is an open $U$ of $X$ contained in $Y$ \sref[0]{0.5.2.2};
since on the other hand $\sh{I}_x \neq 0$ for $x \notin Y$, we must have $U = Y$.
\end{proof}

\begin{corollary}[16.1.10]
\label{IV.16.1.10}
For a restriction of the immersion $j$ to an open neighborhood of $y$ in $Y$ to be an open immersion (in other words, for $j$ to be a \emph{local isomorphism} on the point $y$), it is necessary and sufficient that $(\shGr_1(j))_y = (\sh{N}_{Y/X})_y = 0$.
\end{corollary}

\begin{proof}
The condition is clearly necessary, and the previous reasoning applied to $n=1$ proves that it is sufficient.
\end{proof}

\begin{remark}[16.1.11]
\label{IV.16.1.11}
\begin{enumerate}
  \item[(i)] Under the conditions of the definition \sref{IV.16.1.1}, the projective limit of the projective system $(\sh{O}_{Y^{(n)}}, \phi_{nm})$ of sheaves of rings over $Y$ is called the \emph{normal invariant of infinite order} of $f$, and sometimes denoted by $\sh{O}_{Y^{(\infty)}}$.
  When $X$ is a locally noetherian prescheme, $j:Y \to X$ a closed immersion, $Y$ then is a closed subprescheme of $X$ defined by a coherent ideal $\sh{I}$ and $\sh{O}_{Y^{(\infty)}}$ is exactly the \emph{formal completion} of $\sh{O}_X$ along $Y$ \sref[I]{I.10.8.4}, and $Y^{(\infty)} = (Y, \sh{O}_{Y^{(\infty)}})$ is the formal prescheme that is the \emph{completion} of $X$ along $Y$ \sref[I]{I.10.8.5}.
  In all cases, we could say that $Y^{(\infty)}$ is the \emph{formal neighborhood} of $Y$ in $X$ (via the morphism $f$).
  In the particular case we have just considered, it is the formal prescheme that is the inductive limit of the infinitesimal neighborhoods of order $n$.
  \item[(ii)] Note that for a morphism of preschemes $f=(\psi, \theta): Y \to X$, it can happen that the homomorphism $\theta^\#:\psi^*(\sh{O}_X) \to \sh{O}_Y$ is surjective without $f$ being a local 
\oldpage[IV-4]{8}
  immersion and without $f$ being injective.
  We have an example by taking $Y$ to be a sum of preschemes $Y_\lambda$ all isomorphic to $\Spec(\sh{O}_x)$, where $x \in X$, and taking $f$ to be the morphism equal to the canonical morphism in each of the $Y_\lambda$.
\end{enumerate}
\end{remark}



\section{Smooth morphisms, unramified (or net) morphisms, and \'etale morphisms.}
\label{section:IV.17}

In this paragraph, we revisit the concepts studied in (\textbf{0}\textsubscript{III},~\hyperref[section:0.9]{9}), expressed in the geometric language of schemes from a global point of view, for preschemes locally of finite presentation over a given base.

Most of the results (except~\hyperref[subsection:IV.17.7]{17.7},~\hyperref[subsection:IV.17.8]{17.8},~\hyperref[subsection:IV.17.9]{17.9},~\hyperref[subsection:IV.17.13]{17.13}, and~\hyperref[subsection:IV.17.16]{17.16}) are reduced to various properties already encountered in (\textbf{0}\textsubscript{III},~\hyperref[section:0.9]{9}).

For more specific results on \'etale morphisms, the reader should consult~\textsection\hyperref[section:IV.18]{18}.

\subsection{Formally smooth morphisms, formally unramified morphisms, formally \'etale morphisms.}
\label{subsection:IV.17.1}

\begin{definition}[17.1.1]
\label{IV.17.1.1}
Let $f:X\to Y$ be a morphism of preschemes.
We say that $f$ is \emph{formally smooth} (resp. \emph{formally unramified}, resp. \emph{formally \'etale}) if, for all affine schemes $Y'$, all closed subschemes $Y_0'$ of $Y'$ defined by a nilpotent ideal $\sh{J}$ of $\sh{O}_{Y'}$, and every morphism $Y'\to Y$, the map 
\[
\label{IV.17.1.1.1}
  \Hom_Y(Y',X)\to\Hom_Y(Y_0',X)
  \tag{17.1.1.1}
\]
induced by the canonical map $Y_0'\to Y'$, is \emph{surjective} (resp. \emph{injective}, resp. \emph{bijective}).

One also says that $X$ is \emph{formally smooth} (resp. \emph{formally unramified}, resp. \emph{formally \'etale}) over $Y$.

It is clear that for $f$ to be formally \'etale, it is necessary and sufficient for $f$ to be formally smooth and formally unramified.
\end{definition}

\begin{remark}[17.1.2]
\label{IV.17.1.2}
\medskip\noindent
\begin{enumerate}
  \item[(i)] Suppose that $Y=\Spec(A)$ and $X=\Spec(B)$ are affine, so that $f$ comes from a homomorphism of rings $\vphi:A\to B$. 
    According to \sref[0]{0.19.3.1} and \sref[0]{0.19.10.1}, saying that $f$ is formally smooth (resp. formally unramified, resp. formally \'etale) means that, via $\vphi$, $B$ is a \emph{formally smooth} (resp. \emph{formally unramified}, resp. \emph{formally \'etale}) $A$-algebra, for the \emph{discrete} topologies on $A$ and $B$.
  \item[(ii)] To verify that $f$ is formally smooth (resp. formally unramified, resp. formally \'etale), we can, in Definition~\sref{IV.17.1.1}, restrict to the case where $\sh{J}^2=0$.
  To see this, if $f$ satisfies the corresponding condition of Definition~\sref{IV.17.1.1} in the particular case $\sh{J}^2=0$, and if we have $\sh{J}^n=0$, then we consider the closed subscheme $Y_j'$ of $Y'$ defined by the sheaf of ideals $\sh{J}^{j+1}$ for $0\leq j\leq n-1$, so that $Y_j'$ is a closed subscheme of $Y_{j+1}'$ defined by a square-zero sheaf of ideals;
the hypotheses imply that each of the maps
\[
  \Hom_Y(Y_{j+1}',X)\to\Hom_Y(Y_j',X)\quad(0\leq j\leq n-1) 
\]
\oldpage[IV]{57}
is surjective (resp. injective, resp. bijective);
by composition, we conclude that the same holds for \sref{IV.17.1.1.1}.
\item[(iii)] Note that the properties of the morphism $f$ defined in \sref{IV.17.1.1} are properties of the \emph{representable functor} \sref[\textbf{0}\textsubscript{III}]{0.8.1.8}
\[
  Y'\mapsto\Hom_Y(Y',X) 
\]
from the category of $Y$-preschemes to the category of sets;
they keep a meaning for \emph{any} contravariant functor with the same domain and codomain, representable or not.
\item[(iv)] Assume that the morphism $f$ is formally unramified (resp. formally \'etale);
consider an \emph{arbitrary} $Y$-prescheme $Z$ and a closed subprescheme $Z_0$ of $Z$ defined by a \emph{locally nilpotent} sheaf of ideals $\sh{J}$ of $\sh{O}_Z$. 
Then the map
\[
\label{IV.17.1.2.1}  
  \Hom_Y(Z,X)\to\Hom_Y(Z_0,X)
  \tag{17.1.2.1}
\]
induced by the canonical injection $Z_0\to Z$, is still injective (resp. bijective).
To see this, let $(U_\alpha)$ be an affine open covering of $Z$ such that the sheaves of ideals $\sh{J}|U_\alpha$ are nilpotent, and for each $\alpha$, let $U_\alpha^0$ be the inverse image of $U_\alpha$ in $Z_0$, which is the closed subprescheme of $U_\alpha$ defined by $\sh{J}|U_\alpha$.
Let $f_0:Z_0\to X$ by a $Y$-morphism;
by hypothesis, for each $\alpha$, there is at most one (resp. one and only one) $Y$-morphism $f_\alpha:U_\alpha\to X$ whose restriction to $Z_0$ coincides with $f_0|U_\alpha$.
We immediately conclude that if $f_\alpha$ and $f_\beta$ are defined, then, for each affine open $V\subset U_\alpha\cap U_\beta$, we have $f_\alpha|V=f_\beta|V$, as the restrictions of these morphisms to the inverse image $V_0$ of $V$ in $Z_0$ coincide.
There is therefore at most one (resp. one and only one) $Y$-morphism $f:Z\to X$ whose restriction to $Z_0$ coincides with $f_0$.
\end{enumerate}
\end{remark}

\begin{proposition}[17.1.3]
\label{IV.17.1.3}
\medskip\noindent
\begin{enumerate}
  \item[{\rm(i)}] A monomorphism of preschemes is formally unramified;
    an open immersion is formally \'etale.
  \item[{\rm(ii)}] The composition of two formally smooth (resp. formally unramified, resp. formally \'etale) morphisms is formally smooth (resp. formally unramified, resp. formally \'etale).
  \item[{\rm(iii)}] If $f:X\to Y$ is a formally smooth (resp. formally unramified, resp. formally \'etale) $S$-morphism, then so is $f_{(S')}:X_{(S')}\to Y_{(S')}$ for any base extension $S'\to S$.
  \item[{\rm(iv)}] If $f:X\to X'$ and $g:Y\to Y'$ are two formally smooth (resp. formally unramified, resp. formally \'etale) $S$-morphisms, then so is $f\times_S g:X\times_S Y\to X'\times_S Y'$.
  \item[{\rm(v)}] Let $f:X\to Y$ and $g:Y\to Z$ be two morphisms;
  if $g\circ f$ is formally unramified, then so is $f$.
  \item[{\rm(vi)}] If $f:X\to Y$ is a formally unramified morphism, then so is $f_\red:X_\red\to Y_\red$.
\end{enumerate}
\end{proposition}

\begin{proof}
According to \sref[I]{I.5.5.12}, it suffices to prove (i), (ii), and (iii). 
The assertions in (i) are both trivial.
To prove (ii), consider two morphisms $f:X\to Y$, $g:Y\to Z$, an affine scheme $Z'$, a closed subscheme $Z_0'$ of $Z$ defined by a nilpotent ideal and a morphism $Z'\to Z$. 
Suppose that $f$ and $g$ formally smooth, and consider a $Z$-morphism
\oldpage[IV]{58}
$u_0:Z_0'\to X$;
the hypothesis  on $g$ implies that there exists a $Z$-morphism $v:Z'\to Y$ such that $f\circ u_0=v\circ j$ (where $j:Z_0'\to Z$ is the canonical injection);
the hypothesis on $f$ then implies that there exists a morphism $u:Z'\to X$ such that $f\circ u=v$ and $u\circ j=u_0$, therefore $(g\circ f)\circ u$ is equal to the given morphism $Z'\to Z$ and $u\circ j=u_0$, which proves that $g\circ f$ is formally smooth;
we argue the same way when we suppose that $f$ and $g$ are formally unramified.

Finally, to prove (iii), let $X'=X_{S'}$, $Y'=Y_{S'}$, $f'=f_{S'}$;
consider an affine scheme $Y''$, a closed subscheme $Y_0''$ defined by a nilpotent sheaf of ideals, and a morphism $g:Y''\to Y'$ making $Y''$ a $Y'$-prescheme;
we then know by \sref[I]{I.3.3.8} that $\Hom_{Y'}(Y'',X')$ is canonically identified with $\Hom_Y(Y'',X)$, and $\Hom_{Y'}(Y_0'',X')$ with $\Hom_Y(Y_0'',X)$, and the conclusion follows immediately from Definition~\sref{IV.17.1.1}.
\end{proof}

We note that a \emph{closed immersion} is not necessarily formally smooth.
\begin{proposition}[17.1.4]
\label{IV.17.1.4}
Let $f:X\to Y$ and $g:Y\to Z$ be two morphisms, and suppose that $g$ is formally unramified.
Then, if $g\circ f$ is formally smooth (resp. formally \'etale), so is $f$.
\end{proposition}

\begin{proof}
Let $Y'$ be an affine scheme, $Y_0'$ a closed subscheme of $Y'$ defined by a nilpotent sheaf of ideals, $h:Y'\to Y$ a morphism, $j:Y_0'\to Y'$ the canonical injection, $u_0:Y_0'\to Y$ a $Y$-morphism, such that $f\circ u_0=h\circ j$. 
Suppose that $g\circ f$ is formally smooth;
then there exists a morphism $u:Y'\to X$ such that $u\circ j=u_0$ and $(g\circ f)\circ u=g\circ h$. 
But these two relations imply that $f\circ u$ and $h$ are $Z$-morphisms from $Y'$ to $Y$ such that $(f\circ u)\circ j=h\circ j$;
by virtue of the hypothesis that $g$ is formally unramified, we get that $f\circ u=h$, in other words that $u$ is a $Y$-morphism;
thus $f$ is formally smooth.
Taking into account \sref{IV.17.1.3}[(v)], this proves the proposition.
\end{proof}

\begin{corollary}[17.1.5]
\label{IV.17.1.5}
Suppose that $g$ is formally \'etale;
then, for $g\circ f$ to be formally smooth (resp. formally unramified, resp. formally \'etale), it is necessary and sufficient that $f$ is.
\end{corollary}

\begin{proof}
This follows from \sref{IV.17.1.4} and \sref{IV.17.1.3}[(ii) and (iv)].
\end{proof}

\begin{proposition}[17.1.6]
\label{IV.17.1.6}
Let $f:X\to Y$ be a morphism of preschemes.
\begin{enumerate}
  \item[{\rm(i)}] Let $(U_\alpha)$ be an open covering of $X$ and, for each $\alpha$, let $i_\alpha:U_\alpha\to X$ be the canonical injection.
    For $f$ to be formally smooth (resp. formally unramified, resp. formally \'etale), it is necessary and sufficient that each $f\circ i_\alpha$ is.
  \item[{\rm(ii)}] Let $(V_\lambda)$ be an open covering of $Y$.
    For $f$ to be formally smooth (resp. formally unramified, resp. formally \'etale), it is necessary and sufficient that each of the restrictions $f^{-1}(V_\lambda)\to V_\lambda$ of $f$ is.
\end{enumerate}
\end{proposition}

\begin{proof}
First note that (ii) is a consequence of (i): if $j_\lambda:V_\lambda\to Y$ and $i_\lambda:f^{-1}(V_\lambda)\to X$ are the canonical injections, then the restriction $f_\lambda:f^{-1}(V_\lambda)\to V_\lambda$ of $f$ is such that $j_\lambda\circ f_\lambda=f\circ i_\lambda$;
if $f$ is formally smooth (resp. formally unramified), then so is $f\circ i_\lambda$ since $i_\lambda$ is formally \'etale \sref{IV.17.1.3};
but since $j_\lambda$ is formally \'etale, this means that $f_\lambda$ is formally smooth (resp. formally unramified), by virtue of \sref{IV.17.1.5}.
Conversely, if all the $f_\lambda$ are formally smooth (resp. formally unramified), the same applies to $j_\lambda\circ f_\lambda$ \sref{IV.17.1.3}, so also to $f$ in virtue of (i).

\oldpage[IV]{59}
If we take into account that the $i_\alpha$ are formally \'etale, everything comes down to proving that if the $f\circ i_\alpha$ are formally smooth (resp. formally unramified), then the same applies to $f$.

Therefore let $Y'$ be an affine scheme, $Y_0'$ a closed subscheme of $Y'$ defined by a nilpotent ideal $\sh{J}$, which we may assume to satisfy $\sh{J}^2=0$ \sref{IV.17.1.2}[(ii)], and finally let $g:Y'\to Y$ be a morphism. 
Suppose we are given a $Y$-morphism $u_0:Y_0'\to X$;
denote by $W_\alpha$ (resp. $W_\alpha^0$) the prescheme induced by $Y'$ (resp. $Y_0'$) on the open subset $u_0^{-1}(U_\alpha)$ (we recall that $Y'$ and $Y_0'$ share the \emph{same underlying topological space}). 
Let us first suppose that the $f\circ i_\alpha$ are \emph{formally unramified}, and show that, if $u'$ and $u''$ are two $Y$-morphisms from $Y'$ to $X$ whose restrictions to $Y_0'$ coincide, then we have $u'=u''$. 
Indeed, taking into account \sref{IV.17.1.2}[(iv)], the hypothesis that the $f\circ i_\alpha$ are formally unramified implies that for all $\alpha$, we have $u'|W_\alpha=u''|W_\alpha$, since the restrictions of both $Y$-morphisms to $W_\alpha^0$ coincide. 
Hence the conclusion follows.

Now suppose that the $f\circ i_\alpha$ are \emph{formally smooth} and prove the existence of a $Y$-morphism $u:Y'\to X$ whose restriction to $Y_0'$ is $u_0$.
Now, since $Y'$ is an \emph{affine scheme}, we can apply \sref{IV.16.5.17}, the hypotheses of which are satisfied, and the conclusion of which precisely proves the existence of $u$.
\end{proof}

We can therefore say that the notions introduced in \sref{IV.17.1.1} are \emph{local} on $X$ and $Y$, which always allows, in virtue of \sref{IV.17.1.2}[(i)], to be reduced to the study of formally smooth (resp. formally unramified, resp. formally \'etale) \emph{algebras}.

\subsection{General properties of differentials}
\label{subsection:IV.17.2}

\begin{proposition}[17.2.1]
\label{IV.17.2.1}
For a morphism $f:X\to Y$ to be formally unramified, it is necessary and sufficient that $\Omega_f^1=0$ (what we still write $\Omega_{X/Y}^1=0$ \sref{IV.16.3.1}).  
\end{proposition}

\begin{proof}
Taking into account \sref{IV.17.1.6}, we reduce to the case where $Y=\Spec(A)$ and $X=\Spec(B)$ are affine, and the conclusion then follows from \sref[0]{0.20.7.4} and the interpretation of $\Omega_{X/Y}^1$ in this case \sref{IV.16.3.7}.
\end{proof}

\begin{corollary}[17.2.2]
\label{IV.17.2.2}
Let $f:X\to Y$ and $g:Y\to Z$ be two morphisms. 
For $f$ being formally unramified, it is necessary and sufficient that the canonical morphism \sref{IV.16.4.19}
\[
  f^*(\Omega^1_{Y/Z})\to \Omega^1_{X/Z}
\] is surjective.
\end{corollary}

\begin{proof}
This is an immediate consequence of \sref{IV.17.2.1} and the exact sequence \sref{IV.16.4.19.1}.
\end{proof}

\begin{proposition}[17.2.3]
\label{IV.17.2.3}
Let $f:X\to Y$ be a formally smooth morphism.
\begin{enumerate}
  \item[{\rm(i)}] The $\sh{O}_X$-module $\Omega_{X/Y}^1$ is locally projective \sref{IV.16.10.1}.
    If $f$ is locally of finite type, then $\Omega_{X/Y}^1$ is locally free and of finite type.
  \item[{\rm(ii)}] For all morphisms $g:Y\to Z$, the sequence \sref{IV.16.4.19} of $\sh{O}_X$-modules
\[
\label{IV.17.2.3.1}
  0\to f^*(\Omega_{Y/Z}^1)\to\Omega_{X/Z}^1\to\Omega_{X/Y}^1\to 0
  \tag{17.2.3.1}
\]
is exact; moreover, for each $x\in X$, there exists an open neighborhood $U$ of $x$ such that the restrictions to $U$ of the homomorphisms in \sref{IV.17.2.3.1} form a \emph{split} exact sequence.
\end{enumerate}
\end{proposition}

\oldpage[IV]{60}
\begin{proof}
\medskip\noindent
\begin{enumerate}
  \item[(i)] We know \sref{IV.16.3.9} that if $f$ is locally of finite type, then $\Omega_f^1$ is an $\sh{O}_X$-module of finite type. 
    To prove that, in all cases, it is locally projective, we can reduce, by virtue of \sref{IV.17.1.6}, to the case where $Y=\Spec(A)$ and $X=\Spec(B)$ are affine, and the result follows from the hypothesis on $f$ and from \sref[0]{0.20.4.9} and \sref[0]{0.19.2.1}.
  \item[(ii)] Again, we can restrict to the case where $X$, $Y$, and $Z$ are affine \sref{IV.17.1.6}, and the conclusion in this case follows from the interpretation of the sheaves of modules in the sequence \sref{IV.17.2.3.1} and from \sref[0]{0.20.5.7}.
\end{enumerate}
\end{proof}

\begin{corollary}[17.2.4]
\label{IV.17.2.4}
If $f:X\to Y$ is formally \'etale, then, for all morphisms $g:Y\to Z$, the canonical homomorphism of $\sh{O}_X$-modules
\[
  f^*(\Omega_{Y/Z}^1)\to\Omega_{X/Z}^1
\]
is bijective.
\end{corollary}

\begin{proof}
This follows from the exactness of the sequence \sref{IV.17.2.3.1} and from the fact that we then have $\Omega_{X/Y}^1=0$ \sref{IV.17.2.1}.
\end{proof}

\begin{proposition}[17.2.5]
\label{IV.17.2.5}
Let $f:X\to Y$ be a morphism, $X'$ a subprescheme of $X$ such that the composite morphism $X'\xrightarrow{j}X\xrightarrow{f}Y$ (where $j$ is the canonical injection) is formally smooth.
Then the sequence of $\sh{O}_X$-modules \sref{IV.16.4.21}
\[
\label{IV.17.2.5.1}
  0\to\sh{N}_{X'/X}\to\Omega_{X/Y}^1\otimes_{\sh{O}_X}\sh{O}_{X'}\to\Omega_{X'/Y}^1\to 0
\tag{17.2.5.1}
\]
is exact; moreover, for each $x\in X$, there exists an open neighborhood $U$ of $x$ such that the restrictions to $U$ of the homomorphisms in \sref{IV.17.2.5.1} form a \emph{split} exact sequence.
\end{proposition}

\begin{proof}
By virtue of \sref{IV.17.1.6}, we reduce to the case where $Y=\Spec(A)$ and $X=\Spec(B)$ are affine, and $X'=\Spec(B/\mathfrak{J})$, where $\mathfrak{J}$ is an ideal of $B$. 
The conormal sheaf $\sh{N}_{X'/X}$ then corresponds to the $B$-module $\mathfrak{J}/\mathfrak{J}^2$ \sref{IV.16.1.3}, and the conclusion follows from \sref[0]{0.20.5.14}.
\end{proof}

\begin{proposition}[17.2.6]
\label{IV.17.2.6}
Let $X$ and $Y$ be two preschemes, $f:X\to Y$ a morphism locally of finite type. 
The following conditions are equivalent:
\begin{enumerate}
  \item[{\rm(a)}] $f$ is a monomorphism.
  \item[{\rm(b)}] $f$ is radicial and formally unramified.
  \item[{\rm(c)}] For each $y\in Y$, the fibre $f^{-1}(y)$ is empty or $\kres(y)$-isomorphic to $\Spec(\kres(y))$ (in other words, it is reduced to a single point $z$ such that $\kres(y)\to\sh{O}_z/\mathfrak{m}_y\sh{O}_z$ is an isomorphism).
\end{enumerate}
\end{proposition}

\begin{proof}
The fact that (a) implies (c) follows from \sref{IV.8.11.5.1}. 
It is clear that (c) implies that $f$ is radicial;
let us prove that it also follows from (c) that $\Omega_{X/Y}^1=0$, which will prove that (c) implies (b) \sref{IV.17.2.1}.
Note that the $\sh{O}_X$-module $\Omega_{X/Y}^1$ is quasi-coherent of finite type \sref{IV.16.3.9}.
It follows from \sref[I]{I.9.1.13.1} that, for $(\Omega_{X/Y}^1)_x=0$, it is necessary and sufficient that if we set $Y_1=\Spec(\kres(y))$, $X_1=f^{-1}(y)=X\times_Y Y_1$, then we have $(\Omega_{X_1/Y_1}^1)_x=0$;
but as the morphism $f_1:X_1\to Y_1$ induced by $f$ is formally unramified by virtue of the hypothesis (c) \sref{IV.17.1.3}, the conclusion follows from \sref{IV.17.2.1}.
Finally, let us prove that (b) implies (a);
for this, consider the diagonal morphism $g=\Delta_f:X\to X\times_Y X$;
since $f$ is radicial, $g$ is surjective \sref{IV.1.8.7.1};
on the other hand, $\Omega_{X/Y}^1$ is by definition the conormal sheaf $\shGr_1(g)$ of the immersion $g$ \sref{IV.16.3.1}, and to say that $f$ is formally unramified therefore means that
\oldpage[IV]{61}
$\shGr_1(g)=0$ \sref{IV.17.2.1}. 
In addition, $g$ is locally of finite presentation \sref{IV.1.4.3.1};
therefore the hypothesis $\shGr_1(g)=0$ implies that $g$ is an open immersion \sref{IV.16.1.10};
being surjective, this immersion is an isomorphism, hence $f$ is a monomorphism \sref[I]{I.5.3.8}.
\end{proof}

\subsection{Smooth morphisms, unramified morphisms, \'etale morphisms}
\label{subsection:IV.17.3}

\begin{definition}[17.3.1]
\label{IV.17.3.1}
We say that a morphism $f:X\to Y$ is \emph{smooth} (resp. \emph{unramified}, or \emph{net}
\footnote{The words ``net'' and ``formally net'' seem more preferable to the terminology used in ``unramified'' (resp. formally unramified'') and will be used almost exclusively in Chapter~V.
In this chapter, we have kept the old terminology so as not to conflict with \hyperref[subsection:IV.19.10]{\textbf{0}, 19.10}.}
resp. \emph{\'etale})
if it is locally of finite presentation and formally smooth (resp. formally unramified, resp. formally \'etale).
\end{definition}

We then also say that $X$ is \emph{smooth} (resp. \emph{unramified}, resp. \emph{\'etale}) \emph{over $Y$}.

We will see later \sref{IV.17.5.2} that this definition of a smooth morphism coincides with the definition already given in \sref{IV.6.8.1};
until then, we will exclusively use definition \sref{IV.17.3.1}.

It is clear that saying that $f$ is \'etale means that it is \emph{both} smooth and unramified.

\begin{remark}[17.3.2]
\label{IV.17.3.2}
\medskip\noindent
\begin{enumerate}
  \item[(i)] Note that definition \sref{IV.17.3.1} can be phrased using only the functor
    \[
      Y'\mapsto\Hom_Y(Y',X)
    \]
    considered in \sref{IV.17.1.2}[(iii)] because to say that $f$ is locally of finite presentation is equivalent to saying that the preceding functor \emph{commutes with projective limits of affine schemes} \sref{IV.8.14.2}.
  \item[(ii)] Let $A$ be a ring and $B$ an $A$-algebra. 
   We say that $B$ is a \emph{smooth} (resp. \emph{unramified}, resp. \emph{\'etale}) $A$-algebra if the corresponding morphism $\Spec(B)\to\Spec(A)$ is smooth (resp. unramified, resp. \'etale).
   It is equivalent to say that $B$ is an $A$-algebra \emph{of finite presentation} \sref{IV.1.4.6} that is furthermore formally smooth (resp. formally unramified, resp. formally \'etale) for the discrete topologies.
  \item[(iii)] It follows from \sref{IV.17.1.6} and the definition of a morphism locally of finite presentation \sref{IV.1.4.2} that the notion of a smooth (resp. unramified, resp. \'etale) morphism is \emph{local on $X$ and on $Y$}.
\end{enumerate}
\end{remark}

\begin{proposition}[17.3.3]
\label{IV.17.3.3}
\medskip\noindent
\begin{enumerate}
  \item[{\rm(i)}] An open immersion is \'etale.
    For an immersion to be unramified, it is necessary and sufficient to it be locally of finite presentation.
  \item[{\rm(ii)}] The composition of two smooth (resp. unramified, resp. \'etale) morphisms is smooth (resp. unramified, resp. \'etale).
  \item[{\rm(iii)}] If $f:X\to Y$ is a smooth (resp. unramified, resp. \'etale) $S$-morphism, then so is $f_{(S')}:X_{(S')}\to Y_{(S')}$ for any base extension $S'\to S$. 
  \item[{\rm(iv)}] If $f:X\to X'$ and $g:Y\to Y'$ are smooth (resp. unramified, resp. \'etale) $S$-morphisms, then so is $f\times_S g:X\times_S Y\to X'\times_S Y'$.
\oldpage[IV]{62}
  \item[{\rm(v)}] Let $f:X\to Y$ and $g:Y\to Z$ be two morphisms;
    if $g$ is locally of finite type and if $g\circ f$ is unramified, then $f$ is unramified.
\end{enumerate}
\end{proposition}

\begin{proof}
This follows from \sref{IV.1.4.3} and \sref{IV.17.1.3}.
\end{proof}

\begin{proposition}[17.3.4]
\label{IV.17.3.4}
Let $f:X\to Y$ and $g:Y\to Z$ be two morphisms, and suppose that $g$ is unramified.
Then, if $g\circ f$ is smooth (resp. unramified, resp. \'etale), so is $f$.
\end{proposition}

\begin{proof}
As $g$ and $g\circ f$ are locally of finite presentation, so is $f$ \sref{IV.1.4.3}[(v)];
the conclusion thus follows from \sref{IV.17.1.4} and \sref{IV.17.1.3}[(v)].
\end{proof}

\begin{corollary}[17.3.5]
\label{IV.17.3.5}
Suppose that $g$ is \'etale;
then, for $f$ to be smooth (resp. unramified, resp. \'etale) it is necessary and sufficient that $g\circ f$ is.
\end{corollary}

\begin{proof}
This follows from \sref{IV.17.3.4} and \sref{IV.17.3.3}[(ii)].
\end{proof}

\begin{proposition}[17.3.6]
\label{IV.17.3.6}
Let $g:Y\to S$ and $h:X\to S$ be two morphisms locally of finite presentation.
For an $S$-morphism $f:X\to Y$ to be unramified, it is necessary and sufficient that the canonical homomorphism \sref{IV.16.4.19}
\[
  f^*(\Omega_{Y/S}^1)\to\Omega_{X/S}^1
\]
is surjective.
\end{proposition}

\begin{proof}
As $f$ is locally of finite presentation \sref{IV.1.4.3}[(v)], the proposition follows from \sref{IV.17.2.2}.
\end{proof}

\begin{definition}[17.3.7]
\label{IV.17.3.7}
Let $f:X\to Y$ be a morphism.
We say that $f$ is \emph{smooth} (resp. \emph{unramifed}, resp. \emph{\'etale}) at a point $x\in X$, if there exists an open neighborhood $U$ of $x$ in $X$ such that the restriction $f|U$ is a smooth (resp. unramified, resp. \'etale) morphism from $U$ to $Y$.
\end{definition}

We then also say that $X$ is \emph{smooth} (resp. \emph{unramified}, resp. \emph{\'etale}) \emph{over $Y$ at the point $x$}.

Taking into account remark \sref{IV.17.3.2}[(iii)], it is equivalent to say that $f$ is smooth (resp. unramified, resp. \'etale) and to say that $f$ is smooth (resp. unramified, resp. \'etale) at all points of $X$.

It is clear that the set of points of $X$ at which the morphism $f:X\to Y$ is smooth (resp. unramified, resp. \'etale) is \emph{open} in $X$.

\begin{proposition}[17.3.8]
\label{IV.17.3.8}
For all preschemes $Y$ and all locally free $\sh{O}_Y$-modules $\sh{E}$ of finite type, the vector bundle prescheme $\bb{V}(\sh{E})$ \sref[II]{II.1.7.8} associated to $\sh{E}$ is a smooth $Y$-prescheme.
\end{proposition}

\begin{proof}
Indeed \sref{IV.17.3.2}[(iii)], we can restrict ourselves to the case where $Y=\Spec(A)$ is affine and $\bb{V}(\sh{E})=\Spec(A[T_1,\dots,T_r])$;
as $A[T_1,\dots,T_r]$ is a formally smooth $A$-algebra for the discrete topologies \sref[0]{0.19.3.2}, and of finite presentation, this proves the proposition \sref{IV.17.3.2}[(ii)] 
\end{proof}

\begin{corollary}[17.3.9]
\label{IV.17.3.9}
Under the hypotheses of \sref{IV.17.3.8}, the projective prescheme $\bb{P}(\sh{E})$ \sref[II]{II.4.1.1} is a smooth $Y$-prescheme.
\end{corollary}

\begin{proof}
We can still restrict to the case where $Y=\Spec(A)$ is affine and $\bb{P}(\sh{E})=\bb{P}_Y^r$.
We then know \sref[II]{II.2.3.14} that we have a finite open cover of $\bb{P}^r_{A}$ by the $D_+(T_i)$ ($0\leq i\leq r$) respectively equal to the spectrum of the ring $S_{(f)}$, where we wrote $S$ for $A[T_1,\dots,T_r]$ and $f$ for $T_i$;
but it follows immediately from the definition of $S_{(f)}$ \sref[II]{II.2.2.1}, that this ring, in this case, is isomorphic to $A[T_0,\dots,T_{i-1},T_{i+1},\dots,T_r]$;
hence the corollary follows by \sref{IV.17.3.8}.
\end{proof}

\subsection{Characterizations of unramified morphisms.}
\label{subection:IV.17.4} 

\oldpage[IV]{63}



% \section{Supplement on \'etale morphisms. Henselian local rings and strictly local rings}
\label{section:IV.18}


% \section{Regular immersions and normal flatness}
\label{section:IV.19}


% \section{Meromorphic functions and pseudo-morphisms}
\label{section:IV.20}


% \section{Divisors}
\label{section:IV.21}



\bibliography{the}
\bibliographystyle{amsalpha}

\end{document}

