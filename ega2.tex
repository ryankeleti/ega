\ProvidesPackage{preamble}

\usepackage[utf8]{inputenc}
\usepackage[T1]{fontenc}
\usepackage{microtype}
\usepackage[left=0.75in,right=0.75in,top=0.75in,bottom=0.75in]{geometry}
\usepackage[all]{xy}
\usepackage{enumitem}
\usepackage{color}
\usepackage{soul}
\usepackage{fancyhdr}
\usepackage{mathtools}
\usepackage{amssymb}
\usepackage{amsthm}
\usepackage[charter,
            greekfamily=didot,
            uppercase=upright,
            greeklowercase=upright]{mathdesign}
\usepackage[compact]{titlesec}
\usepackage[colorlinks=true,hyperindex,citecolor=blue,linkcolor=magenta]{hyperref}
\usepackage{bookmark}
\usepackage[asterism]{sectionbreak}


%%%%%%%%%%%%%%
% formatting %
%%%%%%%%%%%%%%

\allowdisplaybreaks[1]
\binoppenalty=9999
\relpenalty=9999
\setitemize{nosep}

% for Chapter 0, Chapter I, etc.
% credit for ZeroRoman https://tex.stackexchange.com/questions/211414/
\newcommand{\ZeroRoman}[1]{\ifcase\value{#1}\relax 0\else\Roman{#1}\fi}
\renewcommand{\thechapter}{\ZeroRoman{chapter}}

%%%%%%%%%%%%%%%%%
% math commands %
%%%%%%%%%%%%%%%%%

% for easy changes to style
\newcommand{\sh}{\mathscr}         % sheaf font
\newcommand{\bb}{\mathbf}          % bold font
\newcommand{\cat}{\mathsf}         % category font
%
\newcommand{\rad}{\mathfrak{r}}    % radical
\newcommand{\nilrad}{\mathfrak{R}} % nilradical
\newcommand{\emp}{\varnothing}     % empty set
\newcommand{\vphi}{\phi}           % font switches \phi and \varphi, change if needed
\newcommand{\HH}{\mathrm{H}}       % cohomology
\newcommand{\dual}[1]{{#1}^\vee}   % dual
\renewcommand{\k}{\bb{k}}          % residue field
\newcommand{\K}{\cat{K}}           % category
\newcommand{\OO}{\sh{O}}           % structure sheaf
\newcommand{\F}{\sh{F}}            % sheaf F
\newcommand{\G}{\sh{G}}            % sheaf G

% operators
%\newcommand*{\sheafHom}{\mathscr{H}\text{\normalfont\kern -3pt {\calligra\large om}}\,}
\def\shHom{\sh{H}\textit{om}} % sheaf Hom
\def\Hom{{\mathop{\mathrm{Hom}}\nolimits}}
\def\Supp{{\mathop{\mathrm{Supp}}\nolimits}}
\def\img{{\mathop{\mathrm{im}}\nolimits}}
\def\Spec{{\mathop{\mathrm{Spec}}\nolimits}}

% if unsure of a translation
\newcommand{\unsure}[2][]{\hl{#2}\marginpar{#1}}
\newcommand{\completelyunsure}{\unsure{[\ldots]}}

% use to mark where original page starts
\newcommand{\oldpage}[1]{\marginpar{\textbf{#1}}\ignorespaces}

% special ref
\newcommand{\sref}[2]{\hyperref[#1-\arabic{chapter}.#2]{\normalfont{(#2)}}}

% ref prelim
\newcommand{\pref}[2]{\hyperref[#1-0.#2]{\normalfont{(\textbf{0}, #2)}}}

%% ref out of chapter
%\newcommand{\cref}[4]{\hyperref[#1-#2.#3]{\normalfont{(\textbf{#3}, #4)}}}

% currently this works as \begin{env}[optional rmk]{x.y.z}
\makeatletter
\newenvironment{env}[2][\@nil]{%
    \def\tmp{#1}%
    \ifx\tmp\@nnil
        \par\medskip\noindent\indent\textbf{(#2)}\rmfamily
    \else
        \par\medskip\noindent\indent\textit{\textbf{#1}}~\textbf{(#2)}.\,---\rmfamily
    \fi}
\makeatother

% use this for definitions, propositions, corollaries, etc.
\makeatletter
\newenvironment{envs}[2][\@nil]{
  \par\medskip\noindent\indent\textit{\textbf{#1}}~\textbf{(#2)}.\,---\itshape
}
\makeatother



\begin{document}
\title{Elementary global study of some classes of morphisms (EGA II)}
\maketitle

\phantomsection
\label{section:ega2}

\tableofcontents

\section*{Summary}

\begin{longtable}{ll}
  \hyperref[section:II.1]{\textsection1}. & Affine morphisms.\\
  \hyperref[section:II.2]{\textsection2}. & Homogeneous prime spectra.\\
  \hyperref[section:II.3]{\textsection3}. & Homogeneous prime spectrum of a sheaf of graded algebras.\\
  \hyperref[section:II.4]{\textsection4}. & Projective bundles; ample sheaves.\\
  \hyperref[section:II.5]{\textsection5}. & Quasi-affine morphisms; quasi-projective morphisms; proper morphisms; projective morphisms.\\
  \hyperref[section:II.6]{\textsection6}. & Integral morphisms and finite morphisms.\\
  \hyperref[section:II.7]{\textsection7}. & Valuative criteria.\\
  \hyperref[section:II.8]{\textsection8}. & Blowup schemes; projective cones; projective closure.\\
\end{longtable}
\bigskip

\oldpage[II]{5}
The various classes of morphisms studied in this chapter are used extensively in cohomological methods; further study using these methods will be done in Chapter~III, where we make particular use of \textsection\textsection2, 4, and 5 of Chapter~II.
On a first reading, \textsection8 can be omitted: it supplements the formalism developed in \textsection\textsection1 and 3, reducing to easy applications of this formalism, and we will use it less consistently than the other results of this chapter.
\bigskip

\section{Affine morphisms}
\label{section:affine-morphisms}

\subsection{$S$-preschemes and $\mathcal{O}_S$-algebras}
\label{subsection:s-preschemes-algebras}

\begin{env}[1.1.1]
\label{2.1.1.1}
Let $S$ be a prescheme, $X$ an $S$-prescheme, and $f:X\to S$ its structure morphism.
We know \sref[0]{0.4.2.4} that the direct image $f_*(\OO_X)$ is an $\OO_S$-algebra, which we
\oldpage[II]{6}
denote $\sh{A}(X)$ when there is little chance of confusion; if $U$ is an open subset of $S$, then we have
\[
  \sh{A}(f^{-1}(U))=\sh{A}(X)|U.
\]
Similarly, for every $\OO_X$-module $\sh{F}$ (resp. every $\OO_X$-algebra $\sh{B}$), we write $\sh{A}(\sh{F})$ (resp. $\sh{A}(\sh{B})$) for the direct image $f_*(\sh{F})$ (resp. $f_*(\sh{B})$) which is an $\sh{A}(X)$-module (resp. an $\sh{A}(X)$-algebra) and not only an $\OO_S$-module (resp. an $\OO_S$-algebra).
\end{env}

\begin{env}[1.1.2]
\label{2.1.1.2}
Let $Y$ be a second $S$-prescheme, $g:Y\to S$ its structure morphism, and $h:X\to Y$ an $S$-morphism; we then have the commutative diagram
\[
  \xymatrix{
    X\ar[rr]^h\ar[rd]_f & &
    Y\ar[ld]^g\\
    & S.
  }
  \tag{1.1.2.1}
\]

We have by definition $h=(\psi,\theta)$, where $\theta:\OO_Y\to h_*(\OO_X)=\psi_*(\OO_X)$ is a homomorphism of sheaves of rings; we induce \sref[0]{0.4.2.2} a homomorphism of $\OO_S$-algebras $g_*(\theta):g_*(\OO_Y)\to g_*(h_*(\OO_X))=f_*(\OO_X)$, in other words, a homomorphism of $\OO_S$-algebras $\sh{A}(Y)\to\sh{A}(X)$, which we denote by $\sh{A}(h)$.
If $h':Y\to Z$ is a second $S$-morphism, then it is immediate that $\sh{A}(h'\circ h)=\sh{A}(h)\circ\sh{A}(h')$.
We havve thus define a \emph{contravariant functor $\sh{A}(X)$} from the category of $S$-preschemes to the category of $\OO_S$-algebras.

Now let $\sh{F}$ be an $\OO_X$-module, $\sh{G}$ an $\OO_Y$-module, and $u:\sh{G}\to\sh{F}$ be an $h$-morphism, that is \sref[0]{0.4.4.1} a homomorphism of $\OO_Y$-modules $\sh{G}\to h_*(\sh{F})$.
Then $g_*(u):g_*(\sh{G})\to g_*(h_*(\sh{F}))=f_*(\sh{F})$ is a homomorphism $\sh{A}(\sh{G})\to\sh{A}(\sh{F})$ of $\OO_S$-modules, which we denote by $\sh{A}(u)$; in addition, the pair $(\sh{A}(h),\sh{A}(u))$ form a \emph{di-homomorphism} from the $\sh{A}(Y)$-module $\sh{A}(\sh{G})$ to the $\sh{A}(X)$-module $\sh{A}(\sh{F})$.
\end{env}

\begin{env}[1.1.3]
\label{2.1.1.3}
If we fix the prescheme $S$, then we can consider the pairs $(X,\sh{F})$, where $X$ is an $S$-prescheme and $\sh{F}$ is an $\OO_X$-module, as forming a \emph{category}, by defining a \emph{morphism} $(X,\sh{F})\to(Y,\sh{G})$ as a pair $(h,u)$, where $h:X\to Y$ is an $S$-morphism and $u:\sh{G}\to\sh{F}$ is an $h$-morphism.
We can theen say that $(\sh{A}(X),\sh{A}(\sh{F}))$ is a \emph{contravariant functor} with values in the category whose objects are pairs consisting of an $\OO_S$-algebra and a module over that algebra, and the morphisms are the di-homomorphisms.
\end{env}

\subsection{Affine preschemes over a prescheme}
\label{subsection:affine-preschemes-over-a-prescheme}


\section{Homogeneous prime spectra}
\label{section:II.2}

\subsection{Generalities on graded rings and modules}
\label{subsection:II.2.1}

\begin{notation}[2.1.1]
\label{II.2.1.1}
Given a \emph{positively-graded} ring $S$, we denote by $S_n$ the subset of $S$ consisting of homogeneous elements of degree $n$ ($n\geq 0$), by $S_+$ the (direct) sum of the $S_n$ for $n>0$;
we have $1\in S_0$, $S_0$ is a subring of $S$, $S_+$ is a graded ideal of $S$, and $S$ is the direct sum of $S_0$ and $S_+$.
If $M$ is a \emph{graded} module over $S$ (with positive or negative degrees), we similarly denote by $M_n$ the $S_0$-module consisting of homogeneous elements of $M$ of degree $n$ (with $n\in\bb{Z}$).

For every integer $d>0$, we denote by $S^{(d)}$ the direct sum of the $S_{nd}$;
by considering the elements of $S_{nd}$ as homogeneous of degree $n$, the $S_{nd}$ define on $S^{(d)}$ a graded ring structure.

For every integer $k$ such that $0\leq k\leq d-1$, we denote by $M^{(d,k)}$ the direct sum
\oldpage[II]{20}
of the $M_{nd+k}$ ($n\in\bb{Z}$);
this is a graded $S^{(d)}$-module when we consider the elements of $M_{nd+k}$ as homogeneous of degree $n$.
We write $M^{(d)}$ in place of $M^{(d,0)}$.

With the above notation, for every integer $n$ (positive or negative), we denote by $M(n)$ the graded $S$-module defined by $(M(n))_k=M_{n+k}$ for every $k\in\bb{Z}$.
In particular, $S(n)$ will be a graded $S$-module such that $(S(n))_k=S_{n+k}$, by agreeing to set $S_n=0$ for $n<0$.
We say that a graded $S$-module $M$ is \emph{free} if it is isomorphic, considered as a \emph{graded} module, to a direct sum of modules of the form $S(n)$;
as $S(n)$ is a monogeneous $S$-module, generated by the element $1$ of $S$ considered as an element of degree $-n$, it is equivalent to say that $M$ admits a \emph{basis} over $S$ consisting of \emph{homogeneous} elements.

We say that a graded $S$-module $M$ \emph{admits a finite presentation} if there exists an exact sequence $P\to Q\to M\to 0$, where $P$ and $Q$ are finite direct sums of modules of the form $S(n)$ and the homomorphisms are of degree $0$ (cf.~\sref{II.2.1.2}).
\end{notation}

\begin{env}[2.1.2]
\label{II.2.1.2}
Let $M$ and $N$ be two graded $S$-modules;
we define on $M\otimes_S N$ a \emph{graded} $S$-module structure in the following way.
On the tensor product $M\otimes_\bb{Z}N$, we can define a graded $\bb{Z}$-module structure (where $\bb{Z}$ is graded by $\bb{Z}_0=\bb{Z}$, $\bb{Z}_n=0$ for $n\neq 0$) by setting $(M\otimes_\bb{Z}N)_q=\bigoplus_{m+n=q}M_m\otimes_\bb{Z}N_n$ (as $M$ and $N$ are respectively direct sums of the $M_m$ and the $N_n$, we know that we can canonically identify $M\otimes_\bb{Z}N$ with the direct sum of all the $M_m\otimes_\bb{Z}N_n$).
This being so, we have $M\otimes_S N=(M\otimes_\bb{Z}N)/P$, where $P$ is the $\bb{Z}$-submodule of $M\otimes_\bb{Z}N$ generated by the elements $(xs)\otimes y-x\otimes(sy)$ for $x\in M$, $y\in N$, $s\in S$;
it is clear that $P$ is a \emph{graded} $\bb{Z}$-submodule of $M\otimes_\bb{Z}N$, and we see immediately that we obtain a graded $S$-module structure on $M\otimes_S N$ by passing to the quotient.

For two graded $S$-modules $M$ and $N$, recall that a homomorphism $u:M\to N$ of $S$-modules is said to be \emph{of degree $k$} if $u(M_j)\subset N_{j+k}$ for all $j\in\bb{Z}$.
If $H_n$ denotes the set of all the homomorphisms of degree $n$ from $M$ to $N$, then we denote by $\Hom_S(M,N)$ the (direct) \emph{sum} of the $H_n$ ($n\in\bb{Z}$) in the $S$-module $H$ of all the homomorphisms (of $S$-modules) from $M$ to $N$;
in general, $\Hom_S(M,N)$ is not equal to the later.
However, we have $H=\Hom_S(M,N)$ when $M$ is \emph{of finite type};
indeed, we can then suppose that $M$ is generated by a finite number of homogeneous elements $x_i$ ($1\leq i\leq n$), and every homomorphism $u\in H$ can be written in a unique way as $\sum_{k\in\bb{Z}}u_k$, where for each $k$, $u_k(x_i)$ is equal to the homogeneous component of degree $k+\deg(x_i)$ of $u(x_i)$ ($1\leq i\leq n$), which implies that $u_k=0$ except for a finite number of indices;
we have by definition that $u_k\in H_k$, hence the conclusion.

We say that the elements of degree $0$ of $\Hom_S(M,N)$ are the \emph{homomorphisms of graded $S$-modules}.
It is clear that $S_m H_n\subset H_{m+n}$, so the $H_n$ define on $\Hom_S(M,N)$ a graded $S$-module structure.

It follows immediately from these definitions that we have
\[
\label{II.2.1.2.1}
  M(m)\otimes_S N(n)=(M\otimes_S N)(m+n),
\tag{2.1.2.1}
\]
\[
\label{II.2.1.2.2}
  \Hom_S(M(m),N(n))=(\Hom_S(M,N))(n-m),
\tag{2.1.2.2}
\]
for two graded $S$-modules $M$ and $N$.

\oldpage[II]{21}
Let $S$ and $S'$ be two graded rings;
a homomorphism of \emph{graded rings $\vphi:S\to S'$} is a homomorphism of rings such that $\vphi(S_n)\subset S_n'$ for all $n\in\bb{Z}$ (in other words, $\vphi$ must be a homomorphism \emph{of degree $0$} of graded $\bb{Z}$-modules).
The data of such a homomorphism defines on $S'$ a \emph{graded} $S'$-module structure;
equipped with this structure and its graded ring structure, we say that $S'$ is a \emph{graded $S'$-algebra}.

If $M$ is also a graded $S$-module, then the tensor product $M\otimes_S S'$ of \emph{graded} $S$-modules is equipped in a natural way with a \emph{graded} $S'$-module structure, the grading being defined as above.
\end{env}

\begin{lemma}[2.1.3]
\label{II.2.1.3}
Let $S$ be a ring graded in positive degrees.
For a subset $E$ of $S_+$ consisting of homogeneous elements to generate $S_+$ as an $S$-module, it is necessary and sufficient for $E$ to generate $S$ an an $S_0$-algebra.
\end{lemma}

\begin{proof}
The condition is evidently sufficient; we show that it is necessary.
Let $E_n$ (resp. $E^n$) be the set of elements of $E$ equal to $n$ (resp. $\leq n$);
it suffices to show, by induction on $n>0$, that $S_n$ is the $S_0$-module generated by the elements of degree $n$ which are products of elements of $E^n$.
This is evident for $n=1$ by virtue of the hypothesis;
the latter also shows that $S_n=\sum_{p=0}^{n-1}S_p E_{n-p}$, and the induction argument is then immediate.
\end{proof}

\begin{corollary}[2.1.4]
\label{II.2.1.4}
For $S_+$ to be an ideal of finite type, it is necessary and sufficient for $S$ to be an $S_0$-algebra of finite type.
\end{corollary}

\begin{proof}
We can always assume that a finite system of generators of the $S_0$-algebra $S$ (resp. of the $S$-ideal $S_+$) consists of homogeneous elements, by replacing each of the generators considered by its homogeneous components.
\end{proof}

\begin{corollary}[2.1.5]
\label{II.2.1.5}
For $S$ to be Noetherian, it is necessary and sufficient for $S_0$ to be Noetherian and for $S$ to be an $S_0$-algebra of finite type.
\end{corollary}

\begin{proof}
The condition is evidently sufficient;
it is necessary, since $S_0$ is isomorphic to $S/S_+$ and $S_+$ must be an ideal of finite type \sref{II.2.1.4}.
\end{proof}

\begin{lemma}[2.1.6]
\label{II.2.1.6}
Let $S$ be a ring graded in positive degrees, which is an $S_0$-algebra of finite type.
Let $M$ be a graded $S$-module of finite type.
Then:
\begin{enumerate}
  \item[{\rm(i)}] The $M_n$ are $S_0$-modules of finite type, and there exists an integer $n_0$ such that $M_n=0$ for $n\leq n_0$.
  \item[{\rm(ii)}] There exists an integer $n_1$ and an integer $h>0$ such that, for every integer $n\geq n_1$, we have $M_{n+h}=S_h M_n$.
  \item[{\rm(iii)}] For every pair of integers $(d,k)$ such that $d>0$, $0\leq k\leq d-1$, $M^{(d,k)}$ is an $S^{(d)}$-module of finite type.
  \item[{\rm(iv)}] For every integer $d>0$, $S^{(d)}$ is an $S_0$-algebra of finite type.
  \item[{\rm(v)}] There exists an integer $h>0$ such that $S_{mh}=(S_h)^m$ for all $m>0$.
  \item[{\rm(vi)}] For every integer $n>0$, there exists an integer $m_0$ such that $S_m\subset S_+^n$ for all $m\geq m_0$.
\end{enumerate}
\end{lemma}

\begin{proof}
We can assume that $S$ is generated (as an $S_0$-algebra) by homogeneous elements $f_i$, of degrees $h_i$ ($1\leq i\leq r$), and $M$ is generated (as an $S$-module) by homogeneous elements $x_j$ of degrees $k_j$ ($1\leq j\leq s$).
It is clear that $M_n$ is formed by linear combinations,
\oldpage[II]{22}
with coefficients in $S_0$, of elements $f_1^{\alpha_1}\cdots f_r^{\alpha_r}x_j$ such that the $\alpha_i$ are integers $\geq 0$ satisfying $k_j+\sum_i\alpha_i h_i=n$;
for each $j$, there are only finitly many systems $(\alpha_i)$ satisfying this equation, since the $h_i$ are $>0$, hence the first assertion of (i);
the second is evident.
On the other hand, let $h$ be the l.c.m. of the $h_i$ and set $g_i=f_i^{h/h_i}$ ($1\leq i\leq r$) such that all the $g_i$ are of degree $h$;
let $z_\mu$ be the elements of $M$ of the form $f_1^{\alpha_1}\cdots f_r^{\alpha_r}x_j$ with $0\leq\alpha_i<h/h_i$ for $1\leq i\leq r$;
there are finitely many of these elements, so let $n_1$ be the largest of their degrees.
It is clear that for $n\geq n_1$, every element of $M_{n+h}$ is a linear combination of the $z_\mu$ whose cofficients are monomials of degree $>0$ with respect to the $g_i$, so we have $M_{n+h}=S_h M_n$, which establishes (ii).
In a similar way, we see (for all $d>0$) that an element of $M^{(d,k)}$ is a linear combinations, with coeffients in $S_0$, of elements of the form $g^d f_1^{\alpha_1}\cdots f_r^{\alpha_r}x_j$ with $0\leq\alpha_i<d$, $g$ being a homogeneous element of $S$;
hence (iii);
(iv) then follows from (iii) and from Lemma~\sref{II.2.1.3}, by taking $M=S_+$, since $(S_+)^{(d)}=(S^{(d)})_+$.
The assertion of (v) is deduced from (ii) by taking $M=S$.
Finally, for a given $n$, there are finitely many systems $(\alpha_i)$ such that $\alpha_i\geq 0$ and $\sum_i\alpha_i<n$, so if $m_0$ is the largest value of the sum $\sum_i\alpha_i h_i$ of these systems, then we have $S_m\subset S_+^n$ for $m>m_0$, which proves (vi).
\end{proof}

\begin{corollary}[2.1.7]
\label{II.2.1.7}
If $S$ is Noetherian, then so is $S^{(d)}$ for every integer $d>0$.
\end{corollary}

\begin{proof}
This follows from \sref{II.2.1.5} and \sref{II.2.1.6}[iv].
\end{proof}

\begin{env}[2.1.8]
\label{II.2.1.8}
Let $\mathfrak{p}$ be a \emph{graded} prime ideal of the graded ring $S$;
$\mathfrak{p}$ is thus a direct sum of the subgroups $\mathfrak{p}_n=\mathfrak{p}\cap S_n$.
Suppose that \emph{$\mathfrak{p}$ does not contain $S_+$}.
Then if $f\in S_+$ is not in $\mathfrak{p}$, the relation $f^n x\in\mathfrak{p}$ is equivalent to $x\in\mathfrak{p}$;
in particular, if $f\in S_d$ ($d>0$), for all $x\in S_{m-nd}$, then the relation $f^n x\in\mathfrak{p}_m$ is equivalent to $x\in\mathfrak{p}_{m-nd}$.
\end{env}

\begin{proposition}[2.1.9]
\label{II.2.1.9}
Let $n_0$ be an integer $>0$;
for all $n\geq n_0$, let $\mathfrak{p}_n$ be a subgroup of $S_n$.
For there to exist a graded prime ideal $\mathfrak{p}$ of $S$ not containing $S_+$ and such that $\mathfrak{p}\cap S_n=\mathfrak{p}_n$ for all $n\geq n_0$, it is necessary and sufficient for the following coniditions to be satisfied:
\begin{enumerate}
  \item[{\rm(1st)}] $S_m\mathfrak{p}_n\subset\mathfrak{p}_{m+n}$ for all $m\geq 0$ and all $n\geq n_0$.
  \item[{\rm(2nd)}] For $m\geq n_0$, $n\geq n_0$, $f\in S_m$, $g\in S_n$, the relation $fg\in\mathfrak{p}_{m+n}$ implies $f\in\mathfrak{p}_m$ or $g\in\mathfrak{p}_n$.
  \item[{\rm(3rd)}] $\mathfrak{p}_n\neq S_n$ for at least one $n\geq n_0$.
\end{enumerate}
In addition, the graded prime ideal $\mathfrak{p}$ is then unique.
\end{proposition}

\begin{proof}
It is evident that the conditions (1st) and (2nd) are necessary.
In addition, if $\mathfrak{p}\not\supset S_+$, then there exists at least one $k>0$ such that $\mathfrak{p}\cap S_k\neq S_k$;
if $f\in S_k$ is not in $\mathfrak{p}$, the relation $\mathfrak{p}\cap S_n=S_n$ implies $\mathfrak{p}\cap S_{n-mk}=S_{n-mk}$ according to \sref{2.2.1.8};
therefore, if $\mathfrak{p}\cap S_n=S_n$ for a certain value of $n$, we would have $\mathfrak{p}\supset S_+$ contrary to the hypothesis, which proves that (3rd) is necessary.
Conversely, suppose that the conditions (1st), (2nd), and (3rd) are satisfied.
Note that if for an integer $d\geq n_0$, $f\in S_d$ is not in $\mathfrak{p}_d$, then, if $\mathfrak{p}$ exists, $\mathfrak{p}_m$, for $m<n_0$, is necessarily equal to the set of the $x\in S_m$ such that $f^r x\in\mathfrak{p}_{m+rd}$, except for a finite number of values of $r$.
This already proves that if $\mathfrak{p}$ exists, then it is unique.
It remains to show that if we define the $\mathfrak{p}_m$ for $m<n_0$ by the previous condition, then $\mathfrak{p}=\sum_{n=0}^\infty\mathfrak{p}_n$ is a prime ideal.
First, note that by virtue of (2nd), for $m\geq n_0$, $\mathfrak{p}_m$ is also defined as the set of the $x\in S_m$ such that $f^r x\in\mathfrak{p}_{m+rd}$ except for a finite number of values of $r$.
This
\oldpage[II]{23}
being so, if $g\in S_m$, $x\in\mathfrak{p}_n$, then we have $f^r gx\in\mathfrak{p}_{m+n+rd}$ except for a finite number of values of $r$, so $gx\in\mathfrak{p}_{m+n}$, which proves that $\mathfrak{p}$ is an ideal of $S$.
To establish that this ideal is prime, in other words that the ring $S/\mathfrak{p}$, graded by the subgroups $S_n/\mathfrak{p}_n$, is an integral domain, it suffices (by considering the components of higher degree of two elements of $S/\mathfrak{p}$) to prove that if $x\in S_m$ and $y\in S_n$ are such that $x\not\in\mathfrak{p}_m$ and $y\not\in\mathfrak{p}_n$, then $xy\not\in\mathfrak{p}_{m+n}$.
If not, for $r$ large enough, we would have $f^{2r}xy\in\mathfrak{p}_{m+n+2rd}$;
but we have $f^r y\not\in\mathfrak{p}_{n+rd}$ for all $r>0$;
it then follows from (2nd) that, except for a finite number of values of $r$, we have $f^r x\in\mathfrak{p}_{m+rd}$, and we conclude that $x\in\mathfrak{p}_m$ contrary to the hypothesis.
\end{proof}

\begin{env}[2.1.10]
\label{II.2.1.10}
We say that a subset $\mathfrak{J}$ of $S_+$ is an \emph{ideal of $S_+$} if it is an ideal of $S$, and $\mathfrak{J}$ is a \emph{graded prime ideal of $S_+$} if it is the intersection of $S_+$ and a graded prime ideal of $S$ \emph{not containing $S_+$} (this prime ideal is also unique according to Proposition~\sref{II.2.1.9}).
If $\mathfrak{J}$ is an ideal of $S_+$, the \emph{radical of $\mathfrak{J}$ in $S_+$} is the set of elements of $S_+$ which have a power in $\mathfrak{J}$, in other words the set $\rad_+(\mathfrak{J})=\rad(\mathfrak{J})\cap S_+$;
in particular, the radical of $0$ in $S_+$ is then called the \emph{nilradical} of $S_+$ and denoted by $\nilrad_+$: this is the set of nilpotent elements of $S_+$.
If $\mathfrak{J}$ is an \emph{graded} ideal of $S_+$, then its radical $\rad_+(\mathfrak{J})$ is a \emph{graded} ideal: by passing to the quotient ring $S/\mathfrak{J}$, we can reduce to the case $\mathfrak{J}=0$, and it remains to see that if $x=x_h+x_{h+1}+\cdots+x_k$ is nilpotent, then so are the $x_i\in S_i$ ($1\leq h\leq i\leq k$);
we can assume $x_k\neq 0$ and the component of highest degree of $x^n$ is then $x_k^n$, hence $x_k$ is nilpotent, and we then argue by induction on $k$.
We say that the graded ring $S$ is \emph{essentially reduced} if $\nilrad_+=0$, in other words, if $S_+$ does not contain nilpotent elements $\neq 0$.
\end{env}

\begin{env}[2.1.11]
\label{II.2.1.11}
We note that if, in the graded ring $S$, an element $x$ is a zero-divisor, then so is its component of highest degree.
We say that a ring $S$ is \emph{essentially integral} if the ring $S_+$ (\emph{without the unit element}) does not contain a zero-divisor and is $\neq 0$;
it suffices that a homogeneous element $\neq 0$ in $S_+$ is not a zero-divisor in this ring.
It is clear that if $\mathfrak{p}$ is a graded prime ideal of $S_+$, then $S/\mathfrak{p}$ is essentially integral.

Let $S$ be an essentially integral graded ring, and let $x_0\in S_0$:
if there then exists \emph{a} homogeneous element $f\neq 0$ of $S_+$ such that $x_0 f=0$, then we have $x_0 S_+=0$, since we have $(x_0 g)f=(x_0 f)g=0$ for all $g\in S_+$, and the hypothesis thus implies $x_0 g=0$.
For $S$ to be integral, it is necessary and sufficient for $S_0$ to be integral and the annihilator of $S_+$ in $S_0$ to be $0$.
\end{env}

\subsection{Rings of fractions of a graded ring}
\label{subsection:II.2.2}

\begin{env}[2.2.1]
\label{II.2.2.1}
Let $S$ be a graded ring, in positive degrees, $f$ a \emph{homogeneous} element of $S$, of degree $d>0$;
then the ring of fractions $S'=S_f$ is graded, taking for $S_n'$ the set of the $x/f^k$, where $x\in S_{n+kd}$ with $k\geq 0$ (we observe here that $n$ can take arbitrary negative values);
we denote the subring $S_0'=(S_f)_0$ of $S'$ consisting of elements \emph{of degree $0$} by the notation $S_{(f)}$.

If $f\in S_d$, then the monomials $(f/1)^h$ in $S_f$ ($h$ a positive or negative integer) form a \emph{free system} over the ring $S_{(f)}$, and the set of their linear combinations is none other than
\oldpage[II]{24}
the ring $(S^{(d)})_f$, which is thus \emph{isomorphic to $S_{(f)}[T,T^{-1}]=S_{(f)}\otimes_\bb{Z}\bb{Z}[T,T^{-1}]$} (where $T$ is an indeterminate).
Indeed, if we have a relation $\sum_{h=-a}^b z_h(f/1)^h=0$ with $z_h=x_h/f^m$, where the $x_h$ are in $S_{md}$, then this relation is equivalent by definition to the existence of a $k>-a$ such that $\sum_{h=-a}^b f^{h+k}x_h=0$, and as the degrees of the terms of this sum are distinct, we have $f^{h+k}x_h=0$ for all $h$, hence $z_h=0$ for all $h$.

If $M$ is a graded $S$-module, then $M'=M_f$ is a graded $S_f$-module, $M_n'$ being the set of the $z/f^k$ with $z\in M_{n+kd}$ ($k\geq 0$);
we denote by $M_{(f)}$ the set of the homomogenous elements of degree $0$ of $M'$;
it is immediate that $M_{(f)}$ is an $S_{(f)}$-module and that we have $(M^{(d)})_f=M_{(f)}\otimes_{S_{(f)}}(S^{(d)})_f$.
\end{env}

\begin{lemma}[2.2.2]
\label{II.2.2.2}
Let $d$ and $e$ be integers $>0$, $f\in S_d$, $g\in S_e$.
There exists a canonical ring isomorphism
\[
  S_{(fg)}\isoto(S_{(f)})_{g^d/f^e};
\]
if we canonically identify these two rings, then there exists a canonical module isomorphism
\[
  M_{(fg)}\isoto(M_{(f)})_{g^d/f^e}.
\]
\end{lemma}

\begin{proof}
Indeed, $fg$ divides $f^e g^d$, and this latter element divides $(fg)^{de}$, so the graded rings $S_{fg}$ and $S_{f^e g^d}$ are canonically identified;
on the other hand, $S_{f^e g^d}$ also identifies with $(S_{f^e})_{g^d/1}$ \sref[0]{0.1.4.6}, and as $f^e/1$ is invertible in $S_{f^e}$, $S_{f^e g^d}$ also identifies with $(S_{f^e})_{g^d/f^e}$.
The element $g^d/f^e$ is of degree $0$ in $S_{f^e}$;
we immediately conclude that the subring of $(S_{f^e})_{g^d/f^e}$ consisting of elements of degree $0$ is $(S_{(f^e)})_{g^d/f^e}$, and as we evidently have $S_{(f^e)}=S_{(f)}$, this proves the first part of the proposition;
the second is established in a similar way.
\end{proof}

\begin{env}[2.2.3]
\label{II.2.2.3}
Under the hypotheses of \sref{II.2.2.2}, it is clear that the canonical homomorphism $S_f\to S_{fg}$ \sref[0]{0.1.4.1}, which sends $x/f^k$ to $g^k x/(fg)^k$, is of degree $0$, thus gives by restriction a \emph{canonical homomorphism $S_{(f)}\to S_{(fg)}$}, such that the diagram
\[
  \xymatrix{
    & S_{(f)}\ar[dl]\ar[dr]\\
    S_{(fg)}\ar[rr]^-{\sim} & &
    (S_{(f)})_{g^d/f^e}
  }
\]
is commutative.
We similarly define a canonical homomorphism $M_{(f)}\to M_{(fg)}$.
\end{env}

\begin{lemma}[2.2.4]
\label{II.2.2.4}
If $f$ and $g$ are two homogeneous elements of $S_+$, then the ring $S_{(fg)}$ is generated by the union of the canonical images of $S_{(f)}$ and $S_{(g)}$.
\end{lemma}

\begin{proof}
By virtue of Lemma~\sref{II.2.2.2}, it suffices to see that $1/(g^d/f^e)=f^{d+e}/(fg)^d$ belongs to the canonical image of $S_{(g)}$ in $S_{(fg)}$, which is evident by definition.
\end{proof}

\begin{proposition}[2.2.5]
\label{II.2.2.5}
Let $d$ be an integer $>0$ and let $f\in S_d$.
Then there exists a canonical ring isomorphisms $S_{(f)}\isoto S^{(d)}/(f-1)S^{(d)}$;
if we identify these two rings by this isomorphism, then there exists a canonical module isomorphism $M_{(f)}\isoto M^{(d)}/(f-1)M^{(d)}$.
\end{proposition}

\begin{proof}
The first of these isomorphisms is defined by sending $x/f^n$, where $x\in S_{nd}$, to the element $\overline{x}$, the class of $x\text{ mod. }(f-1)S^{(d)}$;
this map is well-defined, because we have the congruence $f^h x\equiv x\,(\text{mod.}\,(f-1)S^{(d)})$ for all $x\in S^{(d)}$, so if $f^h x=0$ for an $h>0$,
\oldpage[II]{25}
then we have $\overline{x}=0$.
On the other hand, if $x\in S_{nd}$ is such that $x=(f-1)y$ with $y=y_{hd}+y_{(h+1)d}+\cdots+y_{kd}$ with $y_{jd}\in S_{jd}$ and $y_{hd}\neq 0$, then we necessarily have $h=n$ and $x=-y_{hd}$, as well as the relations $y_{(j+1)d}=fy_{jd}$ for $h\leq j\leq k-1$, $fy_{kd}=0$, which ultimately gives $f^{k-n}x=0$;
we send every class $\overline{x}\text{ mod. }(f-1)S^{(d)}$ of an element $x\in S_{nd}$ to the element $x/f^n$ of $S_{(f)}$, since the preceding remark shows that this map is well-defined.
It is immediate that these two maps thus defined are ring homomorphisms, each the reciprocal of the other.
We proceed exactly the same way for $M$.
\end{proof}

\begin{corollary}[2.2.6]
\label{II.2.2.6}
If $S$ is Noetherian, then so is $S_{(f)}$ for $f$ homogeneous of degree $>0$.
\end{corollary}

\begin{proof}
This follows immediately from Corollary~\sref{II.2.1.7} and Proposition~\sref{II.2.2.5}.
\end{proof}

\begin{env}[2.2.7]
\label{II.2.2.7}
Let $T$ be a multiplicative subset of $S_+$ consisting of \emph{homogeneous} elements;
$T_0=T\cup\{1\}$ is then a multiplicative subset of $S$;
as the elements of $T_0$ are homogeneous, the ring $T_0^{-1}S$ is still graded in the evident way;
we denote by $S_{(T)}$ the subring of $T_0^{-1}S$ consisting of elements of order $0$, that is to say, the elements of the form $x/h$, where $h\in T$ and $x$ is homogeneous of degree equal to that of $h$.
We know \sref[0]{0.1.4.5} that $T_0^{-1}S$ is canonically identified with the inductive limit of the rings $S_f$, where $f$ varies over $T$ (with respect to the canonical homomorphisms $S_f\to S_{fg}$);
as this identification respects the degrees, it identifies $S_{(T)}$ with the \emph{inductive limit} of the $S_{(f)}$ for $f\in T$.
For every graded $S$-module $M$, we similarly define the module $M_{(T)}$ (over the ring $S_{(T)}$) consisting of elements of degree $0$ of $T_0^{-1}M$, and we see that this module is the inductive limit of the $M_{(f)}$ for $f\in T$.

If $\mathfrak{p}$ is a graded prime ideal of $S_+$, then we denote by $S_{(\mathfrak{p})}$ and $M_{(\mathfrak{p})}$ the ring $S_{(T)}$ and the module $M_{(T)}$ respectively, where $T$ is the set of \emph{homogeneous} elements of $S_+$ which do not belong to $\mathfrak{p}$.\end{env}

\subsection{Homogeneous prime spectrum of a graded ring}
\label{subsection:II.2.3}

\begin{env}[2.3.1]
\label{II.2.3.1}
Given a graded ring $S$, in positive degrees, we call the \emph{homogeneous prime spectrum} of $S$ and denote it by $\Proj(S)$ the set of graded prime ideals of $S_+$ \sref{II.2.1.10}, or equivalently the set of graded prime ideals of $S$ \emph{not containing $S_+$};
we will define a \emph{scheme} structure having $\Proj(S)$ as the underlying set.
\end{env}

\begin{env}[2.3.2]
\label{II.2.3.2}
For every subset $E$ of $S$, let $V_+(E)$ be the set of graded prime ideals of $S$ containing $S$ and not containing $S_+$;
this is thus the subset $V(E)\cap\Proj(S)$ of $\Spec(S)$.
From \sref[I]{I.1.1.2} we deduce:
\[
\label{II.2.3.2.1}
  V_+(0)=\Proj(S),\ V_+(S)=V_+(S_+)=\emp,
\tag{2.3.2.1}
\]
\[
\label{II.2.3.2.2}
  V_+\big(\textstyle\bigcup_\lambda E_\lambda\big)=\textstyle\bigcap_\lambda V_+(E_\lambda),
\tag{2.3.2.2}
\]
\[
\label{II.2.3.2.3}
  V_+(EE')=V_+(E)\cup V_+(E').
\tag{2.3.2.3}
\]

We do not change $V_+(E)$ by replacing $E$ with the graded ideal generated by $E$;
in addition, if $\mathfrak{J}$ is a graded ideal of $S$, then we have
\[
  V_+(\mathfrak{J})=V_+\big(\textstyle\bigcup_{q\geq n}(\mathfrak{J}\cap S_q)\big)
\tag{2.3.2.4}
\]
\oldpage[II]{26}
for all $n>0$: indeed, if $\mathfrak{p}\in\Proj(S)$ contains the homogeneous elements of $\mathfrak{J}$ of degree $\geq n$, then as by hypothesis there exists a homogeneous element $f\in S_d$ not contained in $\mathfrak{p}$, for every $m\geq 0$ and every $x\in S_m\cap\mathfrak{J}$, we have $f^r x\in\mathfrak{J}\cap S_{m+rd}$ for all but finitely many values of $r$, so $f^r x\in\mathfrak{p}\cap S_{m+rd}$, which implies that $x\in\mathfrak{p}\cap S_m$ \sref{II.2.1.9}.

Finally, we have, for every graded ideal $\mathfrak{J}$ of $S$,
\[
  V_+(\mathfrak{J})=V_+(\rad_+(\mathfrak{J})).
\tag{2.3.2.5}
\]
\end{env}

\begin{env}[2.3.3]
\label{II.2.3.3}
By definition, the $V_+(E)$ are the closed subsets of $X=\Proj(S)$ for the topology induced by the spectral topology of $\Spec(S)$, which we also call the \emph{spectral topology} on $X$.
For all $f\in S$, we set
\[
\label{II.2.3.3.1}
  D_+(f) = D(f)\cap\Proj(S) = \Proj(S)\setmin V_+(f)
\tag{2.3.3.1}
\]
and so, for any two elements $f$ and $g$ of $S$ \sref[I]{I.1.1.9.1},
\[
\label{II.2.3.3.2}
  D_+(fg) = D_+(f)\cap D_+(g).
\tag{2.3.3.2}
\]
\end{env}

\begin{proposition}[2.3.4]
\label{II.2.3.4}
The $D_+(f)$, as $f$ runs over the set of homogeneous elements of $S_+$, form a base for the topology of $X=\Proj(S)$.
\end{proposition}

\begin{proof}
It follows from \sref{II.2.3.2.2} and \sref{II.2.3.2.4} that every closed subset of $X$ is the intersection of sets of the form $V_+(f)$, where $f$ is homogeneous of degree $>0$.
\end{proof}

\begin{env}[2.3.5]
\label{II.2.3.5}
Let $f$ be a \emph{homogeneous} element of $S_+$, of degree $d>0$;
\end{env}

\section{Homogeneous spectrum of a sheaf of graded algebras}
\label{section:II.3}


\subsection{Homogeneous spectrum of a quasi-coherent graded $\mathcal{O}_Y$-algebra}
\label{subsection:II.3.1}

\begin{env}[3.1.1]
\label{II.3.1.1}
Let $Y$ be a prescheme, $\sh{S}$ a graded $\sh{O}_Y$-algebra, and $\sh{M}$ a graded $\sh{S}$-module.
If $\sh{S}$ is \emph{quasi-coherent}, then each of its homogenous components $\sh{S}_n$ is a \emph{quasi-coherent} $\sh{O}_Y$-module, since they are the images of $\sh{S}$ under a homomorphism from $\sh{S}$ to itself (\sref[I]{I.1.3.8} and \sref[I]{I.1.3.9});
similarly, if $\sh{M}$ is quasi-coherent as an $\sh{O}_Y$-module, then its homogenous components $\sh{M}_n$ are also quasi-coherent, and the converse is also true.
For an integer $d>0$, we denote by $\sh{S}^{(d)}$ the direct sum of the $\sh{O}_Y$-modules $\sh{S}_{nd}$ (for $n\in\bb{Z}$), which is quasi-coherent if $\sh{S}$ is \sref[I]{I.1.3.9};
for every integer $k$ such that $0\leq k\leq d-1$, we denote by $\sh{M}^{(d,k)}$ (or $\sh{M}^{(d)}$, for $k=0$) the direct sum of the $\sh{M}_{nd+k}$ (for $n\in\bb{Z}$), which is a graded $\sh{S}^{(d)}$-module, and quasi-coherent if both $\sh{S}$ and $\sh{M}$ are quasi-coherent \sref[I]{I.9.6.1}.
We denote by $\sh{M}(n)$ the graded $\sh{S}$-module such that $(\sh{M}(n))_k=\sh{M}_{n+k}$ for all $k\in\bb{Z}$;
if $\sh{S}$ and $\sh{M}$ are quasi-coherent, then $\sh{M}(n)$ is a quasi-coherent graded $\sh{S}$-module \sref[I]{I.9.6.1}.

We say that $\sh{M}$ is a graded $\sh{S}$-module \emph{of finite type} (resp. admitting a \emph{finite presentation})  if, for all $y\in Y$, there exists an open neighbourhood $U$ of $y$, along with integers $n_i$ (resp. integers $m_i$ and $n_j$) such that there is a surjective degree~$0$ homomorphism $\bigoplus_{i=1}^r(\sh{S}(n_i)|U)\to\sh{M}|U$ (resp. such that $\sh{M}|U$ is isomorphic to the cokernel of a degree~$0$ homomorphism $\bigoplus_{i=1}^r(\sh{S}(m_i)|U)\to\bigoplus_{j 1}^s(\sh{S}(n_J)|U)$).

Let $U$ be an open affine of $Y$, with ring $A=\Gamma(U,\sh{O}_Y)$;
by hypothesis, the graded $(\sh{O}_Y|U)$-algebra $\sh{S}|U$ is isomorphic to $\widetilde{S}$, where $S=\Gamma(U,\sh{S})$ is a graded $A$-algebra \sref[I]{I.1.4.3};
\oldpage[II]{50}
set $X_U=\Proj(\Gamma(U,\sh{S}))$.
Let $U'\subset U$ be another open affine of $Y$, with ring $A'$, and let $j:U'\to U$ be the canonical injection, which corresponds to the restriction homomorphism $A\to A'$;
we have that $\sh{S}|U'=j^*(\sh{S}|U)$, and so $S'=\Gamma(U',\sh{S})$ is canonically identified with $X_U\times_U U'$, and thus also with $f_U^{-1}(U')$, where we denote by $f_U$ the structure morphism $X_U\to U$ \sref[I]{I.4.4.1}.
We denote by $\sigma_{U',U}$ the canonical isomorphism $f_U^{-1}(U')\xrightarrow{\sim}X_{U'}$ thus defined, and by $\rho_{U',U}$ the open immersion $X_{U'}\to X_U$ obtained by composing $\sigma_{U',U}^{-1}$ with the canonical injection $f_U^{-1}(U')\to X_U$.
It is immediate that, if $U''\subset U'$ is another open affine of $Y$, then $\rho_{U'',U}=\rho_{U'',U'}\circ\rho_{U',U}$.
\end{env}

\begin{proposition}[3.1.2]
\label{II.3.1.2}
Let $Y$ be a prescheme.
For every quasi-coherent positively graded $\sh{O}_Y$-algebra, there exists exactly one (up to $Y$-isomorphism) prescheme $X$ over $Y$ satisfying the following property:
if $f:X\to Y$ is the structure morphism, then, for every open affine $U$ of $Y$, there exists an \emph{isomorphism} $\eta_U$ from the induced prescheme $f^{-1}(U)$ to $X_U=\Proj(\Gamma(U,\sh{S}))$ such that, if $V$ is another open affine of $Y$ that is contained in $U$, then the diagram
\[
\label{II.3.1.2.1}
  \xymatrix{
    f^{-1}(V) \ar[r]^{\eta_V} \ar[d]
    & X_V \ar[d]^{\rho_{V,U}}
  \\f^{-1}(U) \ar[r]_{\eta_V}
    & X_U
  }
\tag{3.1.2.1}
\]
commutes.
\end{proposition}

\begin{proof}
Given affine opens $U$ and $V$ of $Y$, let $X_{U,V}$ be the prescheme induced on $f_U^{-1}(U\cap V)$ by $X_U$;
we are going to define a $Y$-isomorphism $\theta_{U,V}:X_{V,U}\xrightarrow{\sim}X_{U,V}$.
For this, we consider an open affine $W\subset U\cap V$:
by composing the isomorphisms
\[
  f_U^{-1}(W)
  \xrightarrow{\sigma_{W,U}} X_W
  \xrightarrow{\sigma_{W,V}^{-1}} f_V^{-1}(W),
\]
we obtain an isomorphism $\tau_W$, and we immediately see that, if $W'\subset W$ is an open affine, then $\tau_{W'}$ is the restriction of $\tau_W$ to $f_U^{-1}(W')$;
the $\tau_W$ are thus indeed the restrictions of a $Y$-isomorphism $\theta_{V,U}$.
Further, if $U$, $V$, and $W$ are open affines of $Y$, and $\theta'_{U,V}$, $\theta'_{V,W}$, and $\theta'_{U,W}$ the restrictions of $\theta_{U,V}$, $\theta_{V,W}$, and $\theta_{U,W}$ (respectively) to the inverse images of $U\cap V\cap W$ in $X_V$, $X_W$, and $X_W$ (respectively), then it follows from the above definitions that we have $\theta'_{U,V}\circ\theta'_{V,W}=\theta'_{U,W}$.
The existence of some $X$ satisfying the properties in the statement thus follows from \sref[I]{I.2.3.1};
its uniqueness up to $Y$-isomorphism is trivial, taking \sref{I.3.1.2.1} into account.
\end{proof}

\begin{env}[3.1.3]
\label{II.3.1.3}
We say that the prescheme $X$ defined in \sref{II.3.1.2} is the \emph{homogeneous spectrum} of the quasi-coherent graded $\sh{O}_Y$-algebra $\sh{S}$, and we denote it by $\Proj(\sh{S})$.
It is immediate that $\Proj(\sh{S})$ is \emph{separated over $Y$} (\sref{II.2.4.2} and \sref[I]{I.5.5.5});
if $\sh{S}$ is an $\sh{O}_Y$-algebra \emph{of finite type} \sref[I]{I.9.6.2}, then $\Proj(\sh{S})$ is \emph{of finite type} over $Y$ (\sref{II.2.7.1}[(ii)] and \sref[I]{I.6.3.1}).

If $f$ is the structure morphism $X\to Y$, then it is immediate that, for every prescheme induced by $Y$ on an open subset $U$ of $Y$, $f^{-1}(U)$ can be identified with the homogeneous spectrum $\Proj(\sh{S}|U)$.
\end{env}

\begin{proposition}[3.1.4]
\label{II.3.1.4}
Let $f\in\Gamma(Y,\sh{S}_d)$ for $d>0$.
Then there exists an open subset $X_f$ of the underlying space of $X=\Proj(\sh{S})$ that satisfies the following property:
for every open affine $U$ of $Y$, we have $X_f\cap\varphi^{-1}(U)=D_+(f|U)$ in $\varphi^{-1}(U)$ identified with $X_U=\Proj(\Gamma(U,\sh{S}))$, where $\varphi$ denotes the structure morphism $X\to Y$.
\oldpage[II]{51}
Furthermore, the prescheme induced on $X_f$ is affine over $Y$, and is canonically isomorphic to $\Spec(\sh{S}^{(d)}/(f-1)\sh{S}^{[d]})$ \sref{II.1.3.1}.
\end{proposition}

\begin{proof}
We have $f|U\in\Gamma(U,\sh{S}_d)=(\Gamma(U,\sh{S}))_d$.
If $U$ and $U'$ are open affines of $Y$ such that $U'\subset U$, then $f|U'$ is the image of $f|U$ under the restriction homomorphism
\[
  \Gamma(U,\sh{S}) \to \Gamma(U',\sh{S})
\]
and so $D_+(f|U')$ is equal (with the notation of \sref{II.3.1.1}) to the prescheme induced on the inverse image $\rho_{U',U}^{-1}(D_+(f|U))$ in $X_{U'}$ \sref{II.2.8.1};
whence the first claim.
Furthermore, the prescheme induced on $D_+(f|U)$ by $X_U$ is canonically identified with $\Spec((\Gamma(U,\sh{S}))_{f|U})$, with these identifications being compatible with the restriction homomorphisms \sref{II.2.8.1};
the second claim then follows from \sref{II.2.2.5} and from the commutativity of the diagram \sref{II.2.8.2.1}.
\end{proof}

We also say that $X_f$ (as an open subset of the underlying space $X$) is the set of $x\in X$ where $f$ \emph{does not vanish}.

\begin{corollary}[3.1.5]
\label{II.3.1.5}
If $f\in\Gamma(Y,\sh{S}_d)$ and $g\in\Gamma(Y,\sh{S}_e)$, then
\[
\label{II.3.1.5.1}
  X_{fg} = X_f\cap X_g.
\tag{3.1.5.1}
\]
\end{corollary}

\begin{proof}
It suffices to consider the intersection of the two sets with a set $\varphi^{-1}(U)$, where $U$ is an open affine in $Y$, and to then apply formula \sref{II.2.3.3.2}.
\end{proof}

\begin{corollary}[3.1.6]
\label{II.3.1.6}
Let $(f_\alpha)$ be a family of sections of $\sh{S}$ over $Y$ such that $f_\alpha\in\Gamma(Y,\sh{S}_{d_\alpha})$;
if the sheaf of ideals of $\sh{S}$ generated by this family \sref[0]{0.5.1.1} contains all the $\sh{S}_n$ starting from a certain rank, then the underlying space $X$ is the union of the $X_{f_\alpha}$.
\end{corollary}

\begin{proof}
  For every open affine $U$ of $Y$, $\varphi^{-1}(U)$ is the union of the $X_{f_\alpha}\cap\varphi^{-1}(U)$ \sref{II.2.3.14.
\end{proof}

\begin{corollary}[3.1.7]
\label{II.3.1.7}
Let $\sh{A}$ be a quasi-coherent $\sh{O}_Y$-algebra;
set
\[
  \sh{S} = \sh{A}[T] = \sh{A}\otimes_{\bb{Z}}\bb{Z}[T]
\]
where $T$ is an indeterminate (and $\bb{Z}$ and $\bb{Z}[T]$ are considered as simple sheaves over $Y$).
Then $X=\Proj(\sh{S})$ is canonically identified with $\Spec(\sh{A})$.
In particular, $\Proj(\sh{O}_Y[T])$ is identified with $Y$.
\end{corollary}

\begin{proof}
By applying \sref{II.3.1.6} to the unique section $f\in\Gamma(Y,\sh{S})$ that is equal to $T$ at each point of $Y$< we see that $X_f=X$.
Further, here we have $d=1$, and $\sh{S}^{(1)}/(f-1)\sh{S}^{(1)}=\sh{S}/(f-1)\sh{S}$ is canonically isomorphic to $\sh{A}$, whence the corollary \sref{II.1.2.2}.
\end{proof}

Let $g\in\Gamma(Y,\sh{O}_Y)$;
if we take $\sh{S}=\sh{O}_Y[T]$, then $g\in\Gamma(Y,\sh{S}_0)$;
let
\[
  h = gT\in\Gamma(Y,\sh{S}_1).
\]
If $X=\Proj(\sh{S})$, then the canonical identification defined in \sref{II.3.1.7} identifies $X_h$ with the open subset $Y_g$ of $Y$ (in the sense of \sref[0]{0.5.5.2}):
indeed, we can restrict to the case where $Y=\Spec(A)$ is affine, and everything then reduces (taking \sref{II.2.2.5} into account) to the fact that the ring of fractions $A_g$ is canonically identified with $A[T]/(gT-1)A[T]$ \sref[0]{0.1.2.3}.

\begin{proposition}[3.1.8]
\label{II.3.1.8}
Let
\end{proposition}


% \subsection{Sheaf on $\operatorname{Proj}(\mathcal{S})$ associated to a graded $\mathcal{S}$-module}
% \label{subsection:II.3.2}


% \subsection{Graded $\mathcal{S}$-module associated to a sheaf on $\operatorname{Proj}(\mathcal{S})$}
% \label{subsection:II.3.3}


% \subsection{Finiteness conditions}
% \label{subsection:II.3.4}


% \subsection{Functorial behaviour}
% \label{subsection:II.3.5}


% \subsection{Closed subpreschemes of $\operatorname{Proj}(\mathcal{S})$}
% \label{subsection:II.3.6}


% \subsection{Morphisms from a prescheme to a homogeneous spectrum}
% \label{subsection:II.3.7}


% \subsection{Criteria for immersion into a homogeneous spectrum}
% \label{subsection:II.3.8}

\section{Projective bundles; ample sheaves}
\label{section:II.4}


\subsection{Definition of projective bundles}
\label{subsection:II.4.1}

\begin{definition}[4.1.1]
\label{II.4.1.1}
Let $Y$ be a prescheme, $\sh{E}$ a quasi-coherent $\sh{O}_Y$-module, and $\bb{S}_{\sh{O}_Y}(\sh{E})$ the symmetric $\sh{O}_Y$-algebra of $\sh{E}$ \sref{II.1.7.4}, which is quasi-coherent \sref{II.1.7.7}.
We define the \emph{projective bundle on $Y$ defined by $\sh{E}$}, denoted $\bb{P}(\sh{E})$, to be the $Y$-scheme $P=\Proj(\bb{S}_{\sh{O}_Y}(\sh{E}))$.
The $\sh{O}_P$-module $\sh{O}_P(1)$ is called the \emph{fundamental sheaf on $P$}.
\end{definition}

When $Y$ is affine of ring $A$, then we have $\sh{E}=\widetilde{E}$ for some $A$-module $E$, and we then write $\bb{P}(E)$ instead of $\bb{P}(\widetilde{E})$.

When we take $\sh{E}=\sh{O}_Y^n$, we write $\bb{P}_Y^{n-1}$ instead of $\bb{P}(\sh{E})$;
if, further, $Y$ is affine of ring $A$, then we also write $\bb{P}_A^{n-1}$ instead of $\bb{P}_Y^{n-1}$.
Since $\bb{S}_{\sh{O}_Y}(\sh{O}_Y)$ is canonically identified with $\sh{O}_Y[T]$ \sref{II.1.7.4}, $\bb{P}_Y^0$ is canonically identified with $Y$ \sref{II.3.1.7};
Example~\sref{II.2.4.3} is then exactly $\bb{P}_K^1$.

\begin{env}[4.1.2]
\label{II.4.1.2}
Let $\sh{E}$ and $\sh{F}$ be quasi-coherent $\sh{O}_Y$-modules;
let $u:\sh{E}\to\sh{F}$ be an $\sh{O}_Y$-homomorphism;
there is a canonically corresponding homomorphism $\bb{S}(u):\bb{S}_{\sh{O}_Y}(\sh{E})\to\bb{S}_{\sh{O}_Y}(\sh{F})$ of graded $\sh{O}_Y$-algebras \sref{II.1.7.4}.
If $u$ is \emph{surjective}, then so too is $\bb{S}(u)$, and thus \sref{II.3.6.2}[(i)] $\Proj(\bb{S}(u))$ is a \emph{closed immersion} $\bb{P}(\sh{F})\to\bb{P}(\sh{E})$, which we denote by $\bb{P}(u)$.
We can thus say that $\bb{P}(\sh{E})$ is a \emph{contravariant functor} in $\sh{E}$, with the condition that we only consider \emph{surjective} morphisms of quasi-coherent $\sh{O}_Y$_modules.

Still supposing that $u$ is surjective, and letting $P=\bb{P}(\sh{E})$, $Q=\bb{P}(\sh{F})$, and $j=\bb{P}(u)$, we have, up to isomorphism, that
\[
\label{II.4.1.2.1}
  j^*(\sh{O}_P(n)) = \sh{O}_Q(n)
  \qquad\mbox{for all $n\in\bb{Z}$}
  \tag{4.1.2.1}
\]
by \sref{II.3.6.3}.
\end{env}

\begin{env}[4.1.3]
\label{II.4.1.3}
Now let $\psi:Y'\to Y$ be a morphism, and let $\sh{E}'=\psi^*(\sh{E})$;
then $\bb{S}_{\sh{O}_{Y'}}(\sh{E}') = \psi^*(\bb{S}_{\sh{O}_Y}(\sh{E}))$ \sref{II.1.7.5};
thus \sref{II.3.5.3}
\[
\label{II.4.1.3.1}
  \bb{P}(\psi^*(\sh{E})) = \bb{P}(\sh{E})\times_Y Y'
  \tag{4.1.3.1}
\]
up to canonical isomorphism;
furthermore, we clearly have that
\[
  \psi^*((\bb{S}_{\sh{O}_Y}(\sh{E}))(n)) = (\bb{S}_{\sh{O}_{Y'}}(\sh{E}'))(n)
\]
for all $n\in\bb{Z}$, whence, letting $P=\bb{P}(\sh{E})$ and $P'=\bb{P}(\sh{E}')$, we have \sref{II.3.5.4}, up to isomorphism, that
\[
\label{II.4.1.3.2}
  \sh{O}_{P'}(n) = \sh{O}_p(n)\otimes_Y\sh{O}_{Y'}
  \qquad\mbox{for all $n\in\bb{Z}$.}
  \tag{4.1.3.2}
\]
\end{env}


% \subsection{Morphisms from a prescheme to a projective bundle}
% \label{subsection:II.4.2}


% \subsection{The Segre morphism}
% \label{subsection:II.4.3}


% \subsection{Immersions in projective bundles; very ample sheaves}
% \label{subsection:II.4.4}


% \subsection{Ample sheaves}
% \label{subsection:II.4.5}


% \subsection{Relatively ample sheaves}
% \label{subsection:II.4.6}

\section{Quasi-affine morphisms; quasi-projective morphisms; proper morphisms; projective morphisms}
\label{section-quasi-affine-projective-proper-morphisms}

\begin{defn}[5.1.1]
\label{2.5.1.1}
We define a quasi-affine scheme to be a scheme isomorphic to some subscheme induced on some quasi-compact open subset of an affine scheme.
We say that a morphism $f:X\to Y$ is quasi-affine, or that $X$ (considered as a $Y$-prescheme via $f$) is a quasi-affine $Y$-scheme, if there exists a cover $(U_\alpha)$ of $Y$ by affine open subsets such that the $f^{-1}(U_\alpha)$ are quasi-affine schemes.
\end{defn}

It is clear that a quasi-affine morphism is \emph{separated} (\sref[I]{1.5.5.5} and \sref[I]{1.5.5.8}) and \emph{quasi-compact} \sref[I]{1.6.6.1};
every affine morphisms is evidently quasi-affine.

Recall that, for any prescheme $X$, setting $A=\Gamma(X,\OO_X)$, the identity homomorphism $A\to A=\Gamma(X,\OO_X)$ defines a morphism $X\to\Spec(A)$, said to be \emph{canonical} \sref[I]{1.2.2.4};
this is nothing but the canonical morphism defined in \sref{2.4.5.1} for the specific case where $\sh{L}=\OO_X$, if we remember that $\Proj(A[T])$ is canonically identified with $\Spec(A)$ \sref{2.3.1.7}.

\begin{prop}[5.1.2]
\label{2.5.1.2}
Let $X$ be a quasi-compact scheme or a prescheme whose underlying space is Noetherian, and $A$ the ring $\Gamma(X,\OO_X)$.
The following conditions are equivalent.
\begin{enumerate}[label=\emph{(\alph*)}]
    \item $X$ is a quasi-affine scheme.
    \item The canonical morphism $u:X\to\Spec(A)$ is an open immersion.
    \item[\emph{(b')}] The canonical morphism $u:X\to\Spec(A)$ is a homeomorphism from $X$ to some subspace of the underlying space of $\Spec(A)$.
    \item The $\OO_X$-module $\OO_X$ is very ample relative to $u$ \sref{2.4.4.2}.
    \item[\emph{(c')}] The $\OO_X$-module $\OO_X$ is ample \sref{2.4.5.1}.
    \item When $f$ ranges over $A$, the $X_f$ form a base for the topology of $X$.
    \item[\emph{(d')}] When $f$ ranges over $A$, the $X_f$ that are affine form a cover of $X$.
\oldpage[II]{95}
    \item Every quasi-coherent $\OO_X$-module is generated by its sections over $X$.
    \item[\emph{(e')}] Every quasi-coherent sheaf of ideals of $\OO_X$ of finite type is generated by its sections over $X$.
\end{enumerate}
\end{prop}

\begin{proof}
\label{proof-2.5.1.2}
It is clear that \emph{(b)} implies \emph{(a)}, and \emph{(a)} implies \emph{(c)} by \sref{2.4.4.4}[b] applied to the identity morphism (taking into account the remark preceding this proposition);
Furthermore, \emph{(c)} implies \emph{(c')} \sref{2.4.5.10}[i], and \emph{(c')}, \emph{(b)}, and \emph{(b')} are all equivalent by \sref{2.4.5.2}[b] and \sref{2.4.5.2}[b'].
Finally, \emph{(c')} is the same as each of \emph{(d)}, \emph{(d')}, \emph{(e)}, and \emph{(e')} by \sref{2.4.5.2}[a], \sref{2.4.5.2}[a'], \sref{2.4.5.2}[c], and \sref{2.4.5.5}[d''].
\end{proof}

We further observe that, with the previous notation, the $X_f$ that are affine form a \emph{base} for the topology of $X$, and that the canonical morphism $u$ is \emph{dominant} \sref{2.4.5.2}.

\begin{cor}[5.1.3]
\label{1.5.1.3}
Let $X$ be a quasi-compact prescheme.
If there exists a morphism $v:X\to Y$ from $X$ to some affine scheme $Y$ (which would be a homeomorphism from $X$ to some open subspace of $Y$), then $X$ is quasi-affine.
\end{cor}

\begin{proof}
\label{proof-2.5.1.3}
There exists a family $(g_\alpha)$ of sections of $\OO_Y$ over $Y$ such that the $D(g_\alpha)$ form a base for the topology of $v(X)$;
if $v=(\psi,\theta)$ and we set $f_\alpha=\theta(g_\alpha)$, then we have $X_{f_\alpha}=\psi^{-1}(D(g_\alpha))$ \sref[I]{1.2.2.4.1}, so the $X_{f_\alpha}$ form a base for the topology of $X$, and the criterion \sref{2.5.1.2}[d] is satisfied.
\end{proof}

\begin{cor}[5.1.4]
\label{2.5.1.4}
If $X$ is a quasi-affine scheme, then \emph{every} invertible $\OO_X$-module is very ample (relative to the canonical morphism), and \emph{a fortiori} ample.
\end{cor}

\begin{proof}
\label{proof-2.5.1.4}
Such a module $\sh{L}$ is generated by its sections over $X$ \sref{2.5.1.2}[e], so $\sh{L}\otimes\OO_X=\sh{L}$ is very ample \sref{2.4.4.8}.
\end{proof}

\begin{cor}[5.1.5]
\label{2.5.1.5}
Let $X$ be a quasi-compact prescheme.
If there exists an invertible $\OO_X$-module $\sh{L}$ such that $\sh{L}$ and $\sh{L}^{-1}$ are ample, then $X$ is a quasi-affine scheme.
\end{cor}

\begin{proof}
\label{proof-2.5.1.5}
Indeed, $\OO_X=\sh{L}\otimes\sh{L}^{-1}$ is then ample \sref{2.4.5.7}.
\end{proof}

\begin{prop}[5.1.6]
\label{2.5.1.6}
Let $f:X\to Y$ be a quasi-compact morphism.
Then the following conditions are equivalent.
\begin{enumerate}[label=\emph{(\alph*)}]
    \item The morphism $f$ is quasi-affine.
    \item The $\OO_Y$-algebra $f_*(\OO_X)=\sh{A}(X)$ is quasi-coherent, and the canonical morphism $X\to\Spec(\sh{A}(X))$ corresponding to the identity morphism $\sh{A}(X)\to\sh{A}(X)$ \sref{2.1.2.7} is an open immersion.
    \item[\emph{(b')}] The $\OO_Y$-algebra $\sh{A}(X)$ is quasi-coherent, and the canonical morphism $X\to\Spec(\sh{A}(X))$ is a homeomorphism from $X$ to some subspace of $\Spec(\sh{A}(X))$.
    \item The $\OO_X$-module $\OO_X$ is very ample for $f$.
    \item[\emph{(c')}] The $\OO_X$-module $\OO_X$ is ample for $f$.
    \item The morphism $f$ is separated, and, for every quasi-coherent $\OO_X$-module $\sh{F}$, the canonical homomorphism $\sigma:f^*(f_*(\sh{F}))\to\sh{F}$ \sref[0]{0.4.4.3} is surjective.
\end{enumerate}

Further, whenever $f$ is quasi-affine, every invertible $\OO_X$-module $\sh{L}$ is very ample relative to $f$.
\end{prop}

\begin{proof}
\label{proof-2.5.1.6}
The equivalence between \emph{(a)} and \emph{(c')} follows from the local (on $Y$) character of the $f$-\unsure{amplitude} \sref{2.4.6.4}, Definition~\sref{2.5.1.1}, and the criterion \sref{2.5.1.2}[c'].
The other properties are local on $Y$
\oldpage[II]{96}
and thus follow immediately from \sref{2.5.1.2} and \sref{2.5.1.4}, taking into account the fact that $f_*(\sh{F})$ is quasi-coherent whenever $f$ is separated \sref[I]{1.9.2.2}[a].
\end{proof}

\section{Integral morphisms and finite morphisms}
\label{section:II.6}


\subsection{Preschemes integral over another prescheme}
\label{subsection:II.6.1}

\begin{definition}[6.1.1]
\label{II.6.1.1}
Let $X$ be an $S$-prescheme, with structure morphism $f:X\to S$.
We say that $X$ is \emph{integral over $S$}, or that $f$ is an \emph{integral morphism}, if there exists a cover $(S_\alpha)$ of $S$ by affine opens such that, for all $\alpha$, the induced prescheme $f^{-1}(S_\alpha)$ is an affine scheme whose ring $B_\alpha$ is an integral algebra over the ring $A_\alpha$ of $S_\alpha$.
We say that $X$ is \emph{finite over $S$}, or that $f$ is a \emph{finite morphism} if $X$ is integral and of finite type over $S$.
\end{definition}

If $S$ is affine of ring $A$, then we also say ``integral (resp. finite) over $A$'' instead of ``integral (resp. finite) over $S$''.

\begin{env}[6.1.2]
\label{II.6.1.2}
It is clear that, if $X$ is integral over $S$, then it is \emph{affine} over $S$.
For an affine prescheme $X$ over $S$ to be integral (resp. finite) over $S$ it is necessary and sufficient that the associated quasi-coherent $\sh{O}_S$-algebra $\sh{A}(X)$ be such that there exist a cover $(S_\alpha)$ of $S$ by affine opens having the property that, for all $\alpha$, $\Gamma(S_\alpha,\sh{A}(X))$ is an integral (resp. integral and of finite type) algebra over $\Gamma(S_\alpha,\sh{O}_S)$.
A quasi-coherent $\sh{O}_S$-algebra with this property is said to be \emph{integral} (resp. \emph{finite}) over $\sh{O}_S$.
Giving an integral (resp. finite) prescheme over $S$ is thus \sref{II.1.3.1} the same as giving a quasi-coherent $\sh{O}_S$-algebra that is integral (resp. finite) over $\sh{O}_S$.
Note that a quasi-coherent $\sh{O}_S$-algebra $\sh{B}$ is finite if and only if it is an \emph{$\sh{O}_S$-module of finite type} \sref[I]{I.1.3.9};
it is equivalent to say that $\sh{B}$ is an \emph{integral} $\sh{O}_S$-algebra \emph{of finite type}, since an algebra that is integral and of finite type over a ring $A$ is an $A$-module of finite type.
\end{env}

\begin{proposition}[6.1.3]
\label{II.6.1.3}
Let $S$ be a locally Noetherian prescheme.
For an affine prescheme $X$ over $S$ to be finite over $S$, it is necessary and sufficient that the $\sh{O}_S$-algebra $\sh{A}(X)$ be coherent.
\end{proposition}

\begin{proof}
Taking the preceding remark into account, this reduces to noting that, if $S$ is locally Noetherian, then the quasi-coherent $\sh{O}_S$-modules of finite type are exactly the coherent $\sh{O}_S$-modules \sref[I]{I.1.5.1}.
\end{proof}

\begin{proposition}[6.1.4]
\label{II.6.1.4}
Let $X$ be an integral (resp. finite) prescheme over $S$, with structure morphism $f:X\to S$.
Then, for every affine open $U\subset S$ of ring $A$, $f^{-1}(U)$ is an affine scheme whose ring $B$ is an integral (resp. finite) algebra over $A$.
\end{proposition}

\oldpage[II]{111}
\begin{proof}
We first prove the following lemma:

  \begin{lemma}[6.1.4.1]
  \label{II.6.1.4.1}
  Let $A$ be a ring, $M$ an $A$-module, and $(g_i)_{1\leq i\leq m}$ a finite system of elements of $A$ such that the $D(g_i)$ (for $1\leq i\leq m$) cover $\Spec(A)$.
  If, for all $i$, $M_{g_i}$ is an $A_{g_i}$-module of finite type, then $M$ is an $A$-module of finite type.
  \end{lemma}

  \begin{proof}
  We can assume that $M_{g_i}$ admits a finite system of generators $(m_{ij}/g_i^n)$ with $m_{ij}\in M$, with $n$ the same for all indices $i$.
  We will show that the $m_{ij}$ for a system of generators of $M$.
  By hypothesis, for each $i$, there exist $a_{ij}\in A$ and some integer $p$ (independent of $i$) such that, in $M_{g_i}$, $m/1=(\sum_i a_{ij}m_{ij})/g_i^p$;
  this implies that there exists an integer $r\geq p$ such that, for all $i$, we have $g_i^rm\in M'$.
  But, since the $D(g_i^r)=D(g_i)$ cover $\Spec(A)$, the ideal of $A$ generated by the $g_i^r$ is equal to $A$, or, in other words, there exist elements $a_i\in A$ such that $\sum_i a_ig_i^r$;
  then $m=(\sum_i a_i g_i^r)m\in M'$, whence the lemma.
  \end{proof}

Now we already know \sref{II.1.3.2} that $f^{-1}(U)$ is affine.
If $\vphi$ is the homomorphism $A\to B$ corresponding to $f$, then there exists a finite cover of $U$ by opens $D(g_i)$ (where $g_i\in A$) such that, if $h_i=\vphi(g_i)$, $B_{h_i}$ is an integral (resp. integral and finite) algebra over $A_{g_i}$.
Indeed, there exists a cover of $U$ by affine opens $V_\alpha\subset U$ such that, if $A_\alpha=A(V_\alpha)$, $B_\alpha=A(f^{-1}(V_\alpha))$ is an integral (resp. finite) algebra over $A_\alpha$.
Every $x\in U$ belongs to some $V_\alpha$, so there exists $g\in A$ such that $x\in D(g)\subset V_\alpha$;
if $g_\alpha$ is the image of $g$ in $A_\alpha$, then $A(D(g))=A_g=(A_\alpha)_{g_\alpha}$;
let $h=\vphi(g)$, and let $h_\alpha$ be the image of $g_\alpha$ in $B_\alpha$;
we have
\[
  A(D(h)) = B_h = (B_\alpha)_{h_\alpha}
\]
and, since $B_\alpha$ is integral (resp. finite) over $A_\alpha$, $(B_\alpha)_{h_\alpha}$ is integral (resp. finite) over $(A_\alpha)_{g_\alpha}$.
It now suffices to use the fact that $U$ is quasi-compact to obtain the desired cover.

If we suppose first of all that the $B_{h_i}$ are integral and finite over the $A_{g_i}$, then since $B_{h_i}$ can also be written as $B_{g_i}$ as an $A_{g_i}$-module, Lemma~\sref{II.6.1.4.1} shows that, in this case, $B$ is an $A$-module of finite type.

Now suppose only that each $B_{h_i}$ is integral over $A_{g_i}$;
let $b\in B$, and let $C$ be the sub-$A$-algebra of $B$ generated by $b$.
For all $i$, $C_{h_i}$ is the algebra over $A_{g_i}$ generated by $b/1$ in $B_{h_i}$;
it follows from the hypothesis that each $C_{h_i}$ is an $A_{g_i}$-module of finite type, and so \sref{II.6.1.4.1} $C$ is an $A$-module of finite type, which proves that $B$ is integral over $A$.
\end{proof}

\begin{proposition}[6.1.5]
\label{II.6.1.5}
\medskip\noindent
\begin{enumerate}
  \item[(i)] A closed immersion is finite (and \emph{a fortiori} integral).
  \item[(ii)] The composition of two finite (resp. integral) morphisms is finite (resp. integral).
  \item[(iii)] If $f:X\to Y$ is a finite (resp. integral) $S$-morphism, then $f_{(S')}:X_{(S')}\to Y_{(S')}$ is finite (resp. integral) for any base extension $S'\to S$.
  \item[(iv)] If $f:X\to Y$ and $g:X'\to Y'$ are finite (resp. integral) $S$-morphisms, then $f\times_S g:X\times_S Y\to X'\times_S Y'$ is finite (resp. integral).
  \item[(v)] If $f:X\to Y$ and $g:Y\to Z$ are morphisms such that $g\circ f$ is finite (resp. integral), if $g$ is separated, then $f$ is finite (resp. integral).
  \item[(vi)] If $f:X\to Y$ is a finite (resp. integral) morphism, then $f_\red$ is finite (resp. integral).
\end{enumerate}
\end{proposition}

\oldpage[II]{112}
\begin{proof}
By \sref[I]{I.5.5.12}, it suffices to prove (i), (ii), and (iii).
To prove that a closed immersion $X\to S$ is finite, we can restrict to the case where $S=\Spec(A)$, and everything then follows from noting that a quotient ring $A/\mathfrak{J}$ is a monogeneous $A$-module.
To prove that the composition of two finite (resp. integral) morphism $X\to Y$, $Y\to Z$ is finite (resp. integral), we can again assume that $Z$ (and thus $X$ and $Y$ \sref{II.1.3.4}) is affine, and then the claim is equivalent to saying that, if $B$ is a finite (resp. integral) $A$-algebra and $C$ a finite (resp. integral) $B$-algebra, then $C$ is a finite (resp. integral) $A$-algebra, which is immediate.
Finally, to prove (iii), we can restrict to the case where $S=Y$, since $X_{(S')}$ can be identified with $X\times_Y Y_{(S')}$ \sref[I]{I.3.3.11};
we can further suppose that $S=\Spec(A)$ and $S'=\Spec(A')$;
then $X$ is affine of ring $B$ \sref{II.1.3.4}, and $X_{(S')}$ affine of ring $A'\otimes_A B$, and it suffices to note that, if $B$ is a finite (resp. integral) $A$-algebra, then $A'\otimes_A B$ is a finite (resp. integral) $A'$-algebra.
\end{proof}

We also note that, if $X$ and $Y$ are $S$-preschemes that are finite (resp. integral) over $S$, then their \emph{sum} $X\sqcup Y$ is a finite (resp. integral) prescheme over $S$, since this reduces to showing that, if $B$ and $C$ are finite (resp. integral) $A$-algebras over $A$, then so too is $B\times C$.

\begin{corollary}[6.1.6]
\label{II.6.1.6}
If $X$ is an integral (resp. finite) prescheme over $S$, then, for every open $U\subset S$, $f^{-1}(U)$ is integral (resp. finite) over $U$.
\end{corollary}

\begin{proof}
This is a particular case of \sref{II.6.1.5}[(iii)].
\end{proof}

\begin{corollary}[6.1.7]
\label{II.6.1.7}
Let $f:X\to Y$ be a finite morphism.
Then, for all $y\in Y$, the fibre $f^{-1}(y)$ is a finite algebraic scheme over $\kres(y)$, and \emph{a fortiori} its underlying space is discrete and finite.
\end{corollary}

\begin{proof}
Indeed, as a $\kres(y)$-prescheme, $f^{-1}(y)$ can be identified with $X\times_Y\Spec(\kres(y))$ \sref[I]{I.3.6.1}, which is finite over $\Spec(\kres(y))$ \sref{II.6.1.5}[(iii)];
it is thus an affine scheme whose ring is an algebra of finite rank over $\kres(y)$ \sref{II.6.1.4}.
The corollary then follows from \sref[I]{I.6.4.4}.
\end{proof}

\begin{corollary}[6.1.8]
\label{II.6.1.8}
Let $X$ and $S$ be integral preschemes, and $f:X\to S$ a \emph{dominant} morphism.
If $f$ is integral (resp. finite) then the field $R(X)$ of rational functions on $X$ is algebraic (resp. algebraic of finite degree) over the field $R(S)$ of rational functions on $S$.
\end{corollary}

\begin{proof}
Let $s$ be the generic point of $S$;
the $\kres(s)$-prescheme $f^{-1}(s)$ is integral (resp. finite) over $\Spec(\kres(s))$ \sref{II.6.1.5}[(iii)] and contains, by hypothesis, the generic point $x$ of $X$;
since the local ring of $x$ in $f^{-1}(s)$ is equal to $\kres(x)$ \sref[I]{I.3.6.5}, and is thus a local ring of an integral (resp. finite) algebra over $\kres(s)$ \sref{II.6.1.4}, whence the corollary.
\end{proof}

\begin{remark}[6.1.9]
\label{II.6.1.9}
The hypothesis that $g$ be \emph{separated} is essential for the validity of \sref{II.6.1.5}[(v)]: if $Y$ is not separated over $S$, then the identity $1_Y$ is the composite morphism $Y\xrightarrow{\Delta_Y}Y\times_Z Y\xrightarrow{p_1}Y$, but $\Delta_Y$ is not an integral morphism, as follows from \sref{II.6.1.10}:
\end{remark}

\begin{proposition}[6.1.10]
\label{II.6.1.10}
Every integral morphism is universally closed.
\end{proposition}

\begin{proof}
Let $f:X\to Y$ be an integral morphism;
by \sref{II.6.1.5}[(iii)], it suffices to show that $f$ is \emph{closed}.
Let $Z$ be a closed subset of $X$;
\oldpage[II]{113}
then there exists a subprescheme of $X$ whose underlying space is $Z$ \sref[I]{I.5.4.1}, and it thus follows from \sref{II.6.1.5}[(i) and (ii)] that it suffices to prove that $f(X)$ is \emph{closed} in $Y$.
By \sref{II.6.1.5}[(vi)], we can suppose that $X$ and $Y$ are \emph{reduced};
further, if $T$ is the closed reduced subprescheme of $Y$ whose underlying space is $\overline{f(X)}$ \sref[I]{I.5.2.1} then we know that $f$ factors as $X\to T\xrightarrow{j}Y$, where $j$ is the injection morphism \sref[I]{I.5.2.2}, and since $j$ is separated \sref[I]{I.5.5.1}[(i)], it follows from \sref{II.6.1.5}[(v)] that $g$ is an integral morphism.
We can thus suppose that $f(X)$ is \emph{dense} in $Y$.
Finally, since the question is local on $Y$, we can restrict to the case where $Y=\Spec(A)$.
Then $X=\Spec(B)$, where $B$ is an $A$-algebra that is integral over $A$ \sref{II.6.1.4};
furthermore, $A$ is reduced \sref[I]{I.5.1.4}, and the hypothesis that $f(X)$ be dense in $Y$ implies that the homomorphism $\vphi:A\to B$ corresponding to $f$ is \emph{injective} \sref[I]{I.1.2.7}.
Under these conditions, saying that $f(X)=Y$ implies that every prime ideal of $A$ is the intersection with $A$ of a prime ideal of $B$, which is exactly the first theorem of Cohen--Seidenberg (\cite[t.~I, p.~257, th.~3]{II-13}).
\end{proof}

\begin{corollary}[6.1.11]
\label{II.6.1.11}
Every finite morphism $f:X\to Y$ is projective.
\end{corollary}

\begin{proof}
Since $f$ is affine, $\sh{O}_X$ is a \emph{very ample} $\sh{O}_X$-module with respect to $f$ \sref{II.5.1.2};
furthermore, $f_*(\sh{O}_X)$ is a quasi-coherent $\sh{O}_Y$-module \emph{of finite type} \sref{II.6.1.2};
finally, $f$ is separated, of finite type, and universally closed \sref{II.6.1.10}, and thus satisfies the conditions of criterion~\sref{II.5.5.4}[(i)].
\end{proof}

\begin{proposition}[6.1.12]
\label{II.6.1.12}
Let $f:X'\to X$ be a finite morphism, and let $\sh{B}=f_*(\sh{O}_{X'})$ (which is a quasi-coherent $\sh{O}_X$-algebra, and a $\sh{O}_X$-module of finite type).
Let $\sh{F}'$ be a quasi-coherent $\sh{O}_{X'}$-module;
for $\sh{F}'$ to be locally free of rank~$r$, it is necessary and sufficient that $f_*(\sh{F}')$ be a locally free $\sh{B}$-module of rank~$r$.
\end{proposition}

\begin{proof}
It is clear that, if $f_*(\sh{F}')|U$ is isomorphic to $\sh{B}^r|U$ (where $U\subset X$ is open), then $\sh{F}'|f^{-1}(U)$ is isomorphic to $\sh{O}_{X'}^r|f^{-1}(U)$ \sref{II.1.4.2}.
Conversely, suppose that $\sh{F}'$ is locally free of rank~$r$; we will show that $f_*(\sh{F}')$ is locally isomorphic to $\sh{B}^r$ as a $\sh{B}$-module.
Let $x$ be a point of $X$;
as $U$ runs over a fundamental system of affine neighbourhoods of $x$, $f^{-1}(U)$ runs over a fundamental system of affine neighbourhoods \sref{II.1.2.5} of the finite set $f^{-1}(x)$, since $f$ is closed \sref{II.6.1.10}.
The proposition then follows from the following lemma:
\end{proof}

\begin{lemma}[6.1.12.1]
\label{II.6.1.12.1}
Let $Y$ be a prescheme, $\sh{E}$ a locally free $\sh{O}_Y$-module of rank~$r$, and $Z$ a finite subset of $Y$ contained inside some affine open $V$.
Then there exists a neighbourhood $U\subset V$ of $Z$ such that $\sh{E}|U$ is isomorphic to $\sh{O}_Y^r|U$.
\end{lemma}

\begin{proof}
We can evidently assume that $Y$ is affine;
for all $z_i\in Z$, there exists in the closure $\overline{\{z_i\}}$ at least one closed point $z'_i$ \sref[0]{0.2.1.3};
if $Z'$ is the set of the $z'_i$ then every neighbourhood of $Z'$ is a neighbourhood of $Z$, and we can thus assume that $Z$ is discrete and closed in $Y$.
Consider the closed reduced subprescheme of $Y$ that has $Z$ has its underlying space \sref[I]{I.5.2.1} and let $j:Z\to Y$ be the canonical injection;
$j^*(\sh{E})=\sh{E}\otimes_Y\sh{O}_Z$ is locally free of rank~$r$ on the discrete scheme $Z$, and is thus isomorphic to $\sh{O}_Z^r$;
in other words, there exist $r$ sections $s_i$ (for $1\leq i\leq r$) of $\sh{E}\otimes_Y\sh{O}_Z$ over $Z$ such that the homomorphism $\sh{O}_Z^r\to\sh{E}\otimes_Y\sh{O}_Z$ defined by these sections is bijective.
But $Y=\Spec(A)$ is affine, $Z$ is defined by an ideal $\mathfrak{J}$ of $A$, and we have $\sh{E}=\widetilde{M}$, where $M$ is an $A$-module;
the $s_i$ are elements of $M\otimes_A(A/\mathfrak{J})$ and are thus images of $r$ elements $t_i\in M=\Gamma(Y,\sh{E})$.
\oldpage[II]{114}
For all $z_j\in Z$ there is thus a neighbourhood $V_j$ of $z_j$ such that the restrictions of the $t_i$ to $V_j$ defined an isomorphism $\sh{O}_Y^r|V_j\to\sh{E}|V_j$ \sref[0]{0.5.5.4};
the neighbourhood $U$ given by the union of the $V_j$ is the desired neighbourhood.
\end{proof}

\begin{proposition}[6.1.13]
\label{II.6.1.13}
Let $g:X'\to X$ be an integral morphism of preschemes, $Y$ a normal locally integral prescheme, and $f$ a rational map from $Y$ to $X'$ such that $g\circ f$ is an everywhere defined rational map \sref[I]{I.7.2.1}.
Then $f$ is everywhere defined.
\end{proposition}

\begin{proof}
If $f_1$ and $f_2$ are morphisms (from dense open subsets of $Y$ to $X'$) in the class $f$, then it is clear that $g\circ f_1$ and $g\circ f_2$ are equivalent morphisms, which justifies the notation $g\circ f$ for their equivalence class.
We recall also that, if we further suppose $Y$ to be \emph{locally Noetherian}, then the hypothesis that $Y$ is normal already implies that $Y$ is locally integral \sref[I]{I.6.1.13}.

To prove the proposition, note first of all that the question is local on $Y$, and so we can suppose that there exists in the class $g\circ f$ a \emph{morphism} $h:Y\to X$.
Consider the inverse image $Y'=X'_{(h)}=X'_{(Y)}$, and note that the morphism $g'=g_{(Y)}:Y'\to Y$ is \emph{integral} \sref{II.6.1.5}[(iii)].
Given the correspondence between rational maps from $Y$ to $X'$ and rational $Y$-sections of $Y'$ \sref[I]{I.7.1.2}, we see that it suffices to prove the specific case of \sref{II.6.1.13} where $X=Y$; in other words, the following:
\end{proof}

\begin{corollary}[6.1.14]
\label{II.6.1.14}
Let $X$ be a normal locally integral prescheme, $g:X'\to X$ an \emph{integral} morphism, and $f$ a rational $X$-section of $X'$.
Then $f$ is everywhere defined.
\end{corollary}

\begin{proof}
Since the question is local on $X$, we can assume that $X$ is integral, and then $f$ is identified with a morphism from an open $U$ of $X$ to $X'$ \sref[I]{I.7.2.2} that is a $U$-section of $g^{-1}(U)$.
Since $g$ is separated, $f$ is a closed immersion from $U$ into $g^{-1}(U)$ \sref[I]{I.5.4.6};
let $Z$ be the closed subprescheme of $g^{-1}(U)$ associated to $f$ \sref[I]{I.4.2.1}, which is isomorphic to $U$, and thus integral;
let $X_1$ be the reduced subprescheme of $X'$ whose underlying space is the closure $\overline{Z}$ of $Z$ in $X'$ \sref[I]{I.5.2.1};
then $Z$ is an induced subprescheme on an open of $X_1$ \sref[I]{I.5.2.3}, and, since it is irreducible, so too is $X_1$, which is thus integral.
The morphism $f$ can then be considered as a rational $X$-section of $X_1$;
since the restriction of $g$ to $X_1$ is an integral morphism \sref{II.6.1.5}[(i) and (ii)], we can finally reduce to proving \sref{II.6.1.14} in the specific case where $X'=X_1$;
in other words, the following:
\end{proof}

\begin{corollary}[6.1.15]
\label{II.6.1.15}
Let $X$ be a normal integral prescheme, $X'$ an integral prescheme, and $g:X'\to X$ an \emph{integral} morphism.
If there exists a rational $X$-section $f$ of $X'$, then $g$ is an isomorphism.
\end{corollary}

\begin{proof}
Since the question is local on $X$, we can assume that $X$ is affine of integral ring $A$, and then $X'$ is affine of ring $A'$ with $A'$ integral over $A$ \sref{II.6.1.4} and integral;
furthermore, the argument of \sref{II.6.1.14} shows that there exists a dense open of $X$ that is isomorphic to a dense open of $X'$, and so $A$ and $A'$ have the same field of fractions.
Also, by \sref[I]{I.8.2.1.1}, and the hypothesis that the $\sh{O}_x$ are integrally closed, the ring $A$ is integrally closed, and so $A'=A$, which finishes the proof of \sref{II.6.1.13}
\end{proof}


\subsection{Quasi-finite morphisms}
\label{subsection:II.6.2}

\begin{proposition}[6.2.1]
\label{II.6.2.1}
Let $f:X\to Y$ be morphism locally of finite type, and $x$ a point of $X$.
Then the following conditions are equivalent:
\oldpage[II]{115}
\begin{enumerate}
  \item[a)] The point $x$ is isolated in its fibre $f^{-1}(f(x))$.
  \item[b)] The ring $\sh{O}_x$ is a quasi-finite $\sh{O}_{f(x)}$-module \sref[0]{0.7.4.1}.
\end{enumerate}
\end{proposition}

\begin{proof}
The
\end{proof}


% \subsection{Integral closure of a prescheme}
% \label{subsection:II.6.3}


% \subsection{Determinant of an endomorphism of $\mathcal{O}_X$-modules}
% \label{subsection:II.6.4}


% \subsection{Norm of an invertible sheaf}
% \label{subsection:II.6.5}


% \subsection{Application: criteria for ampleness}
% \label{subsection:II.6.6}


% \subsection{Chevalley's theorem}
% \label{subsection:II.6.7}

\section{Valuative criteria}
\label{section:II.7}

In this section, we give valuative criteria for separation and properness for a given morphism, that is, criteria which introduce a variable auxiliary scheme of the form $\Spec(A)$, where $A$ is a valuation ring.
Under certain suitable ``Noetherian'' hypotheses, we can refine our criteria and restrict to the case where $A$ is a \emph{discrete} valuation ring.
This will be the only case that we need to concern ourselves with in all that follows, and we introduce arbitrary valuation rings, in the general case, only to discuss the links with the classical study of such objects.


\subsection{Reminder on valuation rings}
\label{subsection:II.7.1}

\begin{env}[7.1.1]
\label{II.7.1.1}
Amongst the many diverse equivalent properties that characterise valuation rings, we will use the following: a ring $A$ is said to be a \emph{valuation ring} if it is an integral ring which is not a field, and $A$ is \emph{maximal} in the set of local rings strictly contained in the field of fractions $K$ of $A$ under the domination relation \sref[I]{I.8.1.1}.
Recall that a valuation ring is \emph{integrally closed}.
If $A$ is a valuation ring, then so too is $A_\mathfrak{p}$ for any prime ideal $\mathfrak{p}\neq0$ of $A$.
\end{env}

\begin{env}[7.1.2]
\label{II.7.1.2}
Let $K$ be a field, and $A$ a local subring of $K$ that is not a field;
\oldpage[II]{139}
then there exists a valuation ring that both dominates $A$ and has $K$ as its field of fractions (\cite[p.~1-07, lemma~2]{I-1}).

Now let $B$ be a valuation ring, $k$ its residue field, $K$ its field of fractions, and $L$ an extension of $k$.
Then there exists a \emph{complete} valuation ring $C$ that dominates $B$ and whose residue field is $L$.
Indeed, $L$ is the algebraic extension of a pure transcendental extension $L'=k(T_\mu)_{\mu\in M}$;
we know that we can extend the valuation of $B$ corresponding to $B$ to a valuation of $K'=K(T_\mu)_{\mu\in M}$ in such a way that $L'$ is the residue field of this valuation (\cite[p.~98]{II-24});
replacing $B$ by the completion of the ring of this extended valuation, we see that that we can restrict to the case where $B$ is complete and $L$ is an algebraic closure of $k$.
If $\overline{K}$ is an algebraic closure of $K$, we can then extend the valuation that defines $B$ to $\overline{K}$, and the corresponding residue field is an algebraic closure of $k$, as we can see by lifting to $\overline{K}$ the coefficients of a unitary polynomial of $k[T]$.
We are thus finally led to the case where $L=k$, and it then suffices to take $C$ to be the completion of $B$ in order to satisfy our claim.
\end{env}

\begin{env}[7.1.3]
\label{II.7.1.3}
Let $K$ be a field, and $A$ a subring of $K$;
the integral closure $A'$ of $A$ in $K$ is the intersection of the valuation rings that contain $A$ and have $K$ as their field of fractions (\cite[p.~51, th.~2]{I-11}).
Proposition~\sref{II.7.1.2} can then be expressed geometrically in an equivalent form:
\end{env}

\begin{proposition}[7.1.4]
\label{II.7.1.4}
Let $Y$ be a prescheme, $p:X\to Y$ a morphism, $x$ a point of $X$, $y=p(x)$, and $y'\neq y$ a specialisation \sref[0]{0.2.1.2} of $y$.
Then there exists a local scheme $Y'$ which is the spectrum of some valuation ring, and a separated morphism $f:Y'\to Y$ such that, denoting the unique closed point of $Y'$ by $a$ and the generic point of $Y'$ by $b$, we have $f(a)=y'$ and $f(b)=y$.
We can furthermore suppose that one of the two additional following properties are satisfied:
\begin{enumerate}
    \item[\rm{(i)}] $Y'$ is the spectrum of a complete valuation ring whose residue field is algebraically closed, and there exists a $\kres(y)$-homomorphism $\kres(x)\to\kres(b)$.
    \item[\rm{(ii)}] There exists a $\kres(y)$-isomorphism $\kres(x)\xrightarrow{\sim}\kres(b)$.
\end{enumerate}
\end{proposition}

\begin{proof}
\label{proof-II.7.1.4}
Let $Y_1$ be the reduced closed subprescheme of $Y$ that has $\overline\{y\}$ as its underlying space \sref[I]{I.5.2.1}, and let $X_1$ be the closed subprescheme given by the inverse image $p^{-1}(Y_1)$;
since $y'\in\overline{\{y\}}$ by hypothesis, and since $\kres(x)$ is the same in $X$ and in $X_1$, we can assume that $Y$ is \emph{integral}, with generic point $y$;
$\sh{O}_{y'}$ is then an integral local ring that is not a field, and whose field of fractions is $\sh{O}_y=\kres(y)$, and $\kres(x)$ is then an extension of $\kres(y)$.
To satisfy the conditions $f(a)=y'$ and $f(b)=y$ as well as the additional condition (i) (resp. (ii)), we take $Y'=\Spec(A')$, where $A'$ is a valuation ring that dominates $\sh{O}_{y'}$ (resp. a valuation ring that dominates $\sh{O}_{y'}$ and whose field of fractions is $\kres(x)$);
the existence such an of $A'$ is guaranteed by \sref{II.7.1.2}.
\end{proof}

\begin{env}[7.1.5]
\label{II.7.1.5}
Recall that a local ring $A$ is said to be \emph{of dimension 1} if there exists a prime ideal distinct from the maximal ideal $\mathfrak{m}$, and if every prime ideal of $A$ distinct from $\mathfrak{m}$ is a \emph{minimal} prime ideal;
when $A$ is \emph{integral}, it is equivalent to ask that $\mathfrak{m}$ and $(0)$ be the only prime ideals, with $\mathfrak{m}\neq(0)$;
in other words, $Y=\Spec(A)$ consists of two
\oldpage[II]{140}
points $a$ and $b$: $a$ is the unique \emph{closed} point, we have $\mathfrak{j}_a=\mathfrak{m}$, and $\kres(a)=k$ is the \emph{residue field} $k=A/\mathfrak{m}$;
$b$ is the \emph{generic point} of $Y$, $\mathfrak{j}_b=(0)$, with the set $\{b\}$ being the unique open subset of $Y$ distinct from both $\emp$ and $Y$ (an open subset which is thus \emph{everywhere dense}), and $\kres(b)=K$ is the \emph{field of fractions} of $A$.
\end{env}

\begin{env}[7.1.6]
\label{II.7.1.6}
For a local ring $A$, Noetherian and of dimension 1, we know (\cite[pp.~2-08 and 17-01]{I-1}) that the following conditions are equivalent:
\begin{enumerate}
    \item[(a)] $A$ is normal;
    \item[(b)] $A$ is regular;
    \item[(c)] $A$ is a valuation ring;
\end{enumerate}
furthermore, $A$ is then a \emph{discrete valuation ring}.
Propositions~\sref{II.7.1.2} and \sref{II.7.1.3} then have the following analogues for discrete valuation rings:
\end{env}

\begin{proposition}[7.1.7]
\label{II.7.1.7}
Let $A$ be an integral local Noetherian ring that is not a field, $K$ its field of fractions, and $L$ an extension of finite type of $K$;
then there exists a discrete valuation ring that dominates $A$ and has $L$ as its field of fractions.
\end{proposition}

\begin{proof}
\label{proof-II.7.1.7}
Suppose first of all that $L=K$.
Let $\mathfrak{m}$ be the maximal ideal of $A$, $(x_1,\ldots,x_n)$ a system of non-null generators of $\mathfrak{m}$, and $B$ the subring $A[x_2/x_1,\ldots,x_n/x_1]$ of $K$, which is Noetherian.
It is immediate that the ideal $\mathfrak{m}B$ of $B$ is identical to the principal ideal $x_1B$;
if $\mathfrak{p}$ is a minimal prime ideal of $x_1B$, then $\mathfrak{p}$ is of rank 1 (\cite[t.~I, p.~277]{I-13});
in other words, $B_\mathfrak{p}$ is a local Noetherian ring \emph{of dimension 1};
it is clear that $\mathfrak{p}B_\mathfrak{p}\cap A$ is an ideal of $A$ that contains $\mathfrak{m}$ and that does not contain $1$, and is thus equal to $\mathfrak{m}$, and so $B_\mathfrak{p}$ \emph{dominates} $A$ \sref[I]{I.8.1.1}.
It follows from the Krull-Akizuki Theorem (\cite[p.~293]{II-25}) that the integral closure $C$ of $B_\mathfrak{p}$ is a Noetherian ring (even though $C$ is not necessarily a $B_\mathfrak{p}$-module of finite type);
if $\mathfrak{n}$ is a maximal ideal of $C$, then $C_\mathfrak{n}$ is a normal local Noetherian ring of dimension 1 (\cite[p.~295]{II-25}), and thus a discrete valuation ring that dominates $B_\mathfrak{p}$ and \emph{a fortiori} $A$.

Now, if $L$ is an extension of finite type of $K$, we can, by the above, restrict to the case where $A$ is already a discrete valuation ring.
Let $w$ be a valuation of $K$ associated to $A$;
there exists a discrete valuation $w'$ of $L$ that \emph{extends} $w$: we can restrict, by induction on the number of generators of $L$, to the case where $L=K(\alpha)$, and then the proposition is classical (\cite[p.~106]{II-24}).
\end{proof}

\begin{corollary}[7.1.8]
\label{II.7.1.8}
Let $A$ be a Noetherian integral ring, $K$ its field of fractions, and $L$ an extension of finite type of $K$.
Then the integral closure of $A$ in $L$ is the intersection of the discrete valuation rings that have $L$ as their field of fractions and that contain $A$.
\end{corollary}

\begin{proof}
\label{proof-II.7.1.8}
Indeed, such a discrete valuation ring, being normal, contains \emph{a fortiori} every element of $L$ that is integral over $A$.
It thus suffices to prove that, if $x\in L$ is not integral over $A$, then there exists a discrete valuation ring $C$ that has $L$ as its field of fractions, contains $A$, and does not contain $x$.
The hypothesis on $x$ implies that $x\not\in B=A[1/x]$, or, in other words, that $1/x$ is not invertible in the Noetherian ring $B$.
There is thus a prime ideal $\mathfrak{p}$ of $B$ that contains $1/x$.
The integral local ring $B_\mathfrak{p}$ is Noetherian and contained in $L$, which is an extension  of finite type of the field of fractions of $B_\mathfrak{p}$ (with the latter containing $K$).
By \sref{II.7.1.7}, there thus exists a discrete valuation ring $C$ that dominates $B_\mathfrak{p}$ and has $L$ as its field of fractions;
since $1/x\in\mathfrak{p}B_\mathfrak{p}$ belongs to the maximal ideal of $C$, we have that $x\not\in C$, which concludes the proof.
\end{proof}

The geometric form of \sref{II.7.1.7} is the following:

\oldpage[II]{141}
\begin{proposition}[7.1.9]
\label{II.7.1.9}
Let $Y$ be a locally Noetherian prescheme, $p:X\to Y$ a morphism of locally finite type, $x$ a point of $X$, $y=p(x)$, and $y'\neq y$ a specialisation of $y$.
Then there exists a local scheme $Y'$, spectrum of a discrete valuation ring, a separated morphism $f:Y'\to Y$, and a rational $Y$-map $g$ from $Y'$ to $X$, such that, denoting the closed point of $Y'$ by $a$, and the generic point of $Y'$ by $b$, we have $f(a)=y'$, $f(b)=y$, $g(b)=x$, and such that, in the commutative diagram
\[
    \xymatrix{
        & \kres(x) \ar[dl]_{\gamma}
    \\  \kres(b)
        & \kres(y) \ar[u]_{\pi} \ar[l]^{\varphi}
    }
\]
(where $\pi$, $\varphi$, and $\gamma$ are the homomorphisms corresponding to $p$, $f$, and $g$, respectively) the morphism $\gamma$ is a bijection.
\end{proposition}


% \subsection{Valuative criterion for separatedness}
% \label{subsection:II.7.2}


% \subsection{Valuative criterion for properness}
% \label{subsection:II.7.3}


% \subsection{Algebraic curves and function fields of dimension 1}
% \label{subsection:II.7.4}

\section{Blowup schemes; projective cones; projective closure}
\label{section:2.8}



\bibliography{the}
\bibliographystyle{amsalpha}

\end{document}

