\ProvidesPackage{preamble}

\usepackage[utf8]{inputenc}
\usepackage[T1]{fontenc}
\usepackage{microtype}
\usepackage[left=0.75in,right=0.75in,top=0.75in,bottom=0.75in]{geometry}
\usepackage[all]{xy}
\usepackage{enumitem}
\usepackage{color}
\usepackage{soul}
\usepackage{fancyhdr}
\usepackage{mathtools}
\usepackage{amssymb}
\usepackage{amsthm}
\usepackage[charter,
            greekfamily=didot,
            uppercase=upright,
            greeklowercase=upright]{mathdesign}
\usepackage[compact]{titlesec}
\usepackage[colorlinks=true,hyperindex,citecolor=blue,linkcolor=magenta]{hyperref}
\usepackage{bookmark}
\usepackage[asterism]{sectionbreak}


%%%%%%%%%%%%%%
% formatting %
%%%%%%%%%%%%%%

\allowdisplaybreaks[1]
\binoppenalty=9999
\relpenalty=9999
\setitemize{nosep}

% for Chapter 0, Chapter I, etc.
% credit for ZeroRoman https://tex.stackexchange.com/questions/211414/
\newcommand{\ZeroRoman}[1]{\ifcase\value{#1}\relax 0\else\Roman{#1}\fi}
\renewcommand{\thechapter}{\ZeroRoman{chapter}}

%%%%%%%%%%%%%%%%%
% math commands %
%%%%%%%%%%%%%%%%%

% for easy changes to style
\newcommand{\sh}{\mathscr}         % sheaf font
\newcommand{\bb}{\mathbf}          % bold font
\newcommand{\cat}{\mathsf}         % category font
%
\newcommand{\rad}{\mathfrak{r}}    % radical
\newcommand{\nilrad}{\mathfrak{R}} % nilradical
\newcommand{\emp}{\varnothing}     % empty set
\newcommand{\vphi}{\phi}           % font switches \phi and \varphi, change if needed
\newcommand{\HH}{\mathrm{H}}       % cohomology
\newcommand{\dual}[1]{{#1}^\vee}   % dual
\renewcommand{\k}{\bb{k}}          % residue field
\newcommand{\K}{\cat{K}}           % category
\newcommand{\OO}{\sh{O}}           % structure sheaf
\newcommand{\F}{\sh{F}}            % sheaf F
\newcommand{\G}{\sh{G}}            % sheaf G

% operators
%\newcommand*{\sheafHom}{\mathscr{H}\text{\normalfont\kern -3pt {\calligra\large om}}\,}
\def\shHom{\sh{H}\textit{om}} % sheaf Hom
\def\Hom{{\mathop{\mathrm{Hom}}\nolimits}}
\def\Supp{{\mathop{\mathrm{Supp}}\nolimits}}
\def\img{{\mathop{\mathrm{im}}\nolimits}}
\def\Spec{{\mathop{\mathrm{Spec}}\nolimits}}

% if unsure of a translation
\newcommand{\unsure}[2][]{\hl{#2}\marginpar{#1}}
\newcommand{\completelyunsure}{\unsure{[\ldots]}}

% use to mark where original page starts
\newcommand{\oldpage}[1]{\marginpar{\textbf{#1}}\ignorespaces}

% special ref
\newcommand{\sref}[2]{\hyperref[#1-\arabic{chapter}.#2]{\normalfont{(#2)}}}

% ref prelim
\newcommand{\pref}[2]{\hyperref[#1-0.#2]{\normalfont{(\textbf{0}, #2)}}}

%% ref out of chapter
%\newcommand{\cref}[4]{\hyperref[#1-#2.#3]{\normalfont{(\textbf{#3}, #4)}}}

% currently this works as \begin{env}[optional rmk]{x.y.z}
\makeatletter
\newenvironment{env}[2][\@nil]{%
    \def\tmp{#1}%
    \ifx\tmp\@nnil
        \par\medskip\noindent\indent\textbf{(#2)}\rmfamily
    \else
        \par\medskip\noindent\indent\textit{\textbf{#1}}~\textbf{(#2)}.\,---\rmfamily
    \fi}
\makeatother

% use this for definitions, propositions, corollaries, etc.
\makeatletter
\newenvironment{envs}[2][\@nil]{
  \par\medskip\noindent\indent\textit{\textbf{#1}}~\textbf{(#2)}.\,---\itshape
}
\makeatother



\begin{document}
\title{Cohomological study of coherent sheaves (EGA III)}
\maketitle

\phantomsection
\label{section:ega3}

build hack
\cite{I-1}

\tableofcontents

\section*{Summary}

\begin{longtable}{ll}
  \textsection\hyperref[section:III.1]{1}. & Cohomology of affine schemes.\\
  \textsection\hyperref[section:III.2]{2}. & Cohomological study of projective morphisms.\\
  \textsection\hyperref[section:III.3]{3}. & Finiteness theorem for proper morphisms.\\
  \textsection\hyperref[section:III.4]{4}. & The fundamental theorem of proper morphisms. Applications.\\
  \textsection\hyperref[section:III.5]{5}. & An existence theorem for coherent algebraic sheaves.\\
  \textsection\hyperref[section:III.6]{6}. & Local and global Tor functors; K\"unneth formula.\\
  \textsection\hyperref[section:III.7]{7}. & Base change for homological functors of sheaves of modules.\\

  \textsection8. & The duality theorem for projective bundles\\
  \textsection9. & Relative cohomology and local cohomology; local duality\\
  \textsection10. & Relations between projective cohomology and local cohomology. Formal completion technique along a divisor\\
  \textsection11. & Global and local Picard groups\footnote{EGA IV does not depend on \textsection\textsection8--11, and will probably be published before these chapters. \emph{[Trans.] These last four chapters were never published.}}
\end{longtable}
\bigskip

This chapter gives the fundamental theorems concerning the cohomology of coherent algebraic sheaves, with the exception of theorems explaining the theory of residues (duality theorems), which will be the subject of a later chapter.
Amongst all those included here, there are essentially six fundamental theorems, and each one is the subject of one of the first six chapters.
These results will prove to be essential tools in all that follows, even in questions which are not truly cohomological in their nature;
the reader will see the first such examples starting from \textsection4.
\textsection7 gives some more technical results, but ones which are constantly used in applications.
Finally, in \textsection\textsection8--11, we will develop certain results, related to the duality of coherent sheaves, that are particularly important for applications, and which can be explained even before the introduction of the full general theory of residues.

The content of \textsection\textsection1 and 2 is due to J.-P.~Serre, and the reader will observe that we have had only to follow (FAC).
\textsection8 and 9 are equally inspired by (FAC) (the changes necessitated by the different contexts, however, being less evident).
Finally, as we said in the Introduction, \textsection4 should be considered as the formalisation, in modern language, of the fundamental ``invariance theorem'' of Zariski's ``theory of holomorphic functions''.

We draw attention to the fact that the results of n\textsuperscript{o}3.4 (and the preliminary propositions of (\textbf{0},~13.4 to 13.7)) will not be used in what follows Chapter~III, and can thus be skipped in a first reading.
\bigskip

\section{Cohomology of affine schemes}
\label{section-cohomology-of-affine-schemes}

\subsection{Review of the exterior algebra complex}
\label{subsection-review-exterior-algebra-complex}

\begin{env}[1.1.1]
\label{3.1.1.1}
Let $A$ be a ring, $\mathbf{f}=(f_i)_{1\leq i\leq r}$ a system of $r$ elements of $A$.
The \emph{exterior algebra complex $K_\bullet(\mathbf{f})$} corresponding to $\mathbf{f}$ is a chain complex (G, I, 2.2) defined in the following way: the graded $A$-module $K_\bullet(\mathbf{f})$ is equal to the \emph{exterior algebra $\wedge(A^r)$}, graded in the usual way, and the boundary map is the \emph{interior multiplication $i_\mathbf{f}$} by $\mathbf{f}$ considered as an element of the dual $\dual{(A^r)}$; we recall that $i_\mathbf{f}$ is an \emph{antiderivation} of degree $-1$ of $\wedge(A^r)$, and if $(\mathbf{e}_i)_{1\leq i\leq r}$ is the canonical basis of $A^r$, then we have $i_\mathbf{f}(\mathbf{e}_i)=f_i$; the verification of the condition $i_\mathbf{f}\circ i_\mathbf{f}=0$ is immediate.

An equivalent definition is the following: for each $i$, we consider a chain complex $K_\bullet(f_i)$ defined as follows: $K_0(f_i)=K_1(f_i)=A$, $K_n(f_i)=0$ for $n\neq 0,1$: the boundary map is defined by the condition that $d_1:A\to A$ is \emph{multiplication by $f_i$}.
We then take $K_\bullet(\mathbf{f})$ to be the \emph{tensor product $K_\bullet(f_1)\otimes K_\bullet(f_2)\otimes\cdots\otimes K_\bullet(f_r)$} (G, I, 2.7) with its total degree; the verification of the isomorphism from this complex to the complex defined above is immediate.
\end{env}

\begin{env}[1.1.2]
\label{3.1.1.2}
For every $A$-module $M$, we define the \emph{chain complex}
\[
  K_\bullet(\mathbf{f},M)=K_\bullet(\mathbf{f})\otimes_A M
  \tag{1.1.2.1}
\]
and the \emph{cochain complex} (G, I, 2.2)
\[
  K^\bullet(\mathbf{f},M)=\Hom_A(K_\bullet(\mathbf{f},M).
  \tag{1.1.2.2}
\]

If $g$ is a $k$-cochain of this latter complex, and if we set
\[
  g(i_1,\dots,i_k)=g(\mathbf{e}_{i_1}\wedge\cdots\wedge\mathbf{e}_{i_k}),
\]
then $g$ identifies with an \emph{alternating} map from $[1,r]^k$ to $M$, and it follows from the above definitions that we have
\[
  d^k g(i_1,i_2,\dots,i_{k+1})=\sum_{h=1}^{k+1}(-1)^{h-1}f_{i_h}g(i_1,\dots,\wh{i_h},\dots,i_{k+1}).
  \tag{1.1.2.3}
\]
\end{env}

\begin{env}[1.1.3]
\label{3.1.1.3}
From the above complexes, we deduce as usual the \emph{homology and cohomology $A$-modules} (G, I, 2.2)
\[
  \HH_\bullet(\mathbf{f},M)=\HH_\bullet(K_\bullet(\mathbf{f},M)),
  \tag{1.1.3.1}
\]
\[
  \HH^\bullet(\mathbf{f},M)=\HH^\bullet(K^\bullet(\mathbf{f},M)).
  \tag{1.1.3.2}
\]

We define an \emph{$A$-isomorphism $K_\bullet(\mathbf{f},M)\isoto K^\bullet(\mathbf{f},M)$} by sending each chain $z=\sum(\mathbf{e}_{i_1}\wedge\cdots\wedge\mathbf{e}_{i_k})\otimes z_{i_1,\dots,i_k}$ to the cochain $g_z$ such that $g_z(j_1,\dots,j_{r-k})=\varepsilon z_{i_1,\dots,i_k}$, where $(j_h)_{1\leq h\leq r-k}$ is the strictly increasing sequence complementary to the strictly increasing sequence $(i_h)_{1\leq h\leq k}$ in $[1,r]$ and $\varepsilon=(-1)^\nu$, where $\nu$ is the number of inversions of the permutation $i_1,\dots,i_k,j_1,\dots,j_{r-k}$ of $[1,r]$.
We verify that $g_{dz}=d(g_z)$, which gives an isomorphism
\[
  \HH^i(\mathbf{f},M)\isoto\HH_{r-i}(\mathbf{f},M)\text{ for }0\leq i\leq r.
  \tag{1.1.3.3}
\]

In this chapter, we will especially consider the cohomology modules $\HH^\bullet(\mathbf{f},M)$.

For a given $\mathbf{f}$, it is immediate (G, I, 2.1) that $M\mapsto\HH^\bullet(\mathbf{f},M)$ is a \emph{cohomological functor} (T, II, 2.1) from the category of $A$-modules to the category of graded $A$-modules, zero in degrees $<0$ and $>r$.
In addition, we have
\[
  \HH^0(\mathbf{f},M)=\Hom_A(A/(\mathbf{f}),M),
  \tag{1.1.3.4}
\]
denoting by $(\mathbf{f})$ the ideal of $A$ generated by $f_1,\dots,f_r$; this follows immediately from (1.1.2.3), and it is clear that $\HH^0(\mathbf{f},M)$ identifies with the submodule of $M$ \emph{killed by $(\mathbf{f})$}.
Similarly, we have by (1.1.2.3) that
\[
  \HH^r(\mathbf{f},M)=M/\bigg(\sum_{i=1}^r f_i M\bigg)=(A/(\mathbf{f}))\otimes_A M.
  \tag{1.1.3.5}
\]

We will use the following known result, which we will recall a proof of to be complete:
\end{env}

\begin{prop}[1.1.4]
\label{3.1.1.4}
Let $A$ be a ring, $\mathbf{f}=(f_i)_{1\leq i\leq r}$ a finite family of elements of $A$, and $M$ an $A$-module.
If, for $1\leq i\leq r$, the scaling $z\mapsto f_i\cdot z$ on $M_{i-1}=M/(f_1 M+\cdots+f_{i-1}M)$ is injective, then we have $\HH^i(\mathbf{f},M)=0$ for $i\neq r$.
\end{prop}

It suffices to prove that $\HH_i(\mathbf{f},M)=0$ for all $i>0$ according to (1.1.3.3).
We argue by induction on $r$, the case $r=0$ being trivial.
Set $\mathbf{f}'=(f_i)_{1\leq i\leq r-1}$; this family satisfies the conditions in the statement, so if we set $L_\bullet=K_\bullet(\mathbf{f}',M)$, then we have $\HH_i(L_\bullet)=0$ for $i>0$ by hypothesis, and $\HH_0(L_\bullet)=M_{r-1}$ by virtue of (1.1.3.3) and (1.1.3.5).
To abbreviate, set $K_\bullet=K_\bullet(f_r)=K_0\oplus K_1$, with $K_0=K_1=A$, $d_1:K_1\to K_0$ multiplication by $f_r$; we have by definition \sref{3.1.1.1} that $K_\bullet(\mathbf{f},M)=K_\bullet\otimes_A L_\bullet$.
We have the following lemma:

\begin{lem}[1.1.4.1]
\label{3.1.1.4.1}
Let $K_\bullet$ be a chain complex of free $A$-modules, zero except in dimensions $0$ and $1$.
For every chain complex $L_\bullet$ of $A$-modules, we have an exact sequence
\[
  0\to\HH_0(K_\bullet\otimes\HH_p(L_\bullet))\to\HH_p(K_\bullet\otimes L_\bullet)\to\HH_1(K_\bullet\otimes\HH_{p-1}(L_\bullet))\to 0
\]
for every index $p$.
\end{lem}

This is a particular case of an exact sequece of low-order terms of the K\"unneth spectral sequence (M, XVII, 5.2 (a) and G, I, 5.5.2); it can be proved directly as follows.
Consider $K_0$ and $K_1$ as chain complexes (zero in dimensions $\neq 0$ and $\neq 1$ respectively); we then have an exact sequence of complexes
\[
  0\to K_0\otimes L_\bullet\to K_\bullet\otimes L_\bullet\to K_1\otimes L_\bullet\to 0,
\]
to which we can apply the exact sequence in homology
\[
  \cdots\to\HH_{p+1}(K_1\otimes L_\bullet)\xrightarrow{\partial}\HH_p(K_0\otimes L_\bullet)\to\HH_p(K_\bullet\otimes L_\bullet)\to\HH_p(K_1\otimes L_\bullet)\xrightarrow{\partial}\HH_{p-1}(K_0\otimes L_\bullet)\to\cdots.
\]
But it is evident that $\HH_p(K_0\otimes L_\bullet)=K_0\otimes\HH_p(L_\bullet)$ and $\HH_p(K_1\otimes L_\bullet)=K_1\otimes\HH_{p-1}(L_\bullet)$ for all $p$; in addition, we verify immediately that the operator $\partial:K_1\otimes\HH_p(L_\bullet)\to K_0\otimes\HH_p(L_\bullet)$ is none other than $d_1\otimes 1$; the lemma thus follows from the above exact sequence and the definition of $\HH_0(K_\bullet\otimes\HH_p(L_\bullet))$ and $\HH_1(K_\bullet\otimes\HH_{p-1}(L_\bullet))$.

The lemma having been established, the end of the proof of Proposition \sref{3.1.1.4} is immediate: the induction hypothesis of Lemma \sref{3.1.1.4.1} gives $\HH_p(K_\bullet\otimes L_\bullet)=0$ for $p\geq 2$; in addition if we show that $\HH_1(K_\bullet,\HH_0(L_\bullet))=0$, then we also deduce from Lemma \sref{3.1.1.4.1} that $\HH_1(K_\bullet\otimes L_\bullet)=0$; but by definition, $\HH_1(K_\bullet,\HH_0(L_\bullet))$ is none other than the kernel of the scaling $z\mapsto f_r\cdot z$ on $M_{r-1}$, and as by hypothesis this kernel is zero, this finishes the proof.


\section{Cohomological study of projective morphisms}
\label{section:III.2}


\subsection{Explicit calculations of certain cohomology groups}
\label{subsection:III.2.1}

\begin{env}[2.1.1]
\label{III.2.1.1}
Let $X$ be a prescheme, and $\sh{L}$ an invertible $\sh{O}_X$-module;
consider the graded ring \sref[0]{0.5.4.6}
\[
\label{III.2.1.1.1}
  S = \Gamma_\bullet(X,\sh{L}) = \bigoplus_{n\in\bb{Z}}\Gamma(X,\sh{L}^{\otimes n}).
\tag{2.1.1.1}
\]

Let $(f_i)_{1\leq i\leq r}$ be a finite family of \emph{homogeneous} elements of $S$, with $f_i\in S_{d_i}$;
set $U_i=X_{f_i}$ and $U=\bigcup_i U_i$, and denote by $\mathfrak{U}$ the cover $(U_i)$ of $U$.
For every quasi-coherent $\sh{O}_X$-module $\sh{F}$, we set
\[
\label{III.2.1.1.2}
  \HH^\bullet(U,\sh{F}(\anotherbullet)) = \bigoplus_{n\in\bb{Z}}\HH^\bullet(U,\sh{F}\otimes\sh{L}^{\otimes n})
\tag{2.1.1.2}
\]
\[
\label{III.2.1.1.3}
  \HH^\bullet(\mathfrak{U},\sh{F}(\anotherbullet)) = \bigoplus_{n\in\bb{Z}}\HH^\bullet(\mathfrak{U},\sh{F}\otimes\sh{L}^{\otimes n}).
\tag{2.1.1.3}
\]

The abelian groups \sref{III.2.1.1.2} and \sref{III.2.1.1.3} are \emph{bigraded}, by taking
\[
  (\HH^\bullet(U,\sh{F}(\anotherbullet)))_{mn} = \HH^m(U,\sh{F}\otimes\sh{L}^{\otimes n})
\]
and an analogous definition for \sref{III.2.1.1.3}.
For the second grading, it is clear that these groups are graded $S$-modules, as follows, for example, from the fact that $\sh{F}\mapsto\HH^m(U,\sh{F})$ and $\sh{F}\mapsto\HH^m(\mathfrak{U},\sh{F})$ are functors.
\end{env}

\begin{env}[2.1.2]
\label{III.2.1.2}
Now consider the graded $S$-module \sref[0]{0.5.4.6}
\[
\label{III.2.1.2.1}
  M = \Gamma_\bullet(\sh{L},\sh{F}) = \HH^0(X,\sh{F}(\anotherbullet)) = \bigoplus_{n\in\bb{Z}}(X,\sh{F}\otimes\sh{L}^{\otimes n}).
\tag{2.1.2.1}
\]

\oldpage[III]{96}
If $X$ is a prescheme whose underlying space is Noetherian, or a quasi-compact scheme, then it follows from \sref[I]{I.9.3.1} that, setting $U_{i_0i_1\ldots i_p}=\bigcap_{k=0}^p U_{i_k}$ as usual, we have, up to canonical isomorphism,
\[
  \Gamma(U_{i_0i_1\ldots i_p}, \sh{F}(\anotherbullet))
  = \HH^0(U_{i_0i_1\ldots i_p}, \sh{F}(\anotherbullet))
  = M_{f_{i_0}f_{i_1}\ldots f_{i_p}}.
\]

We can again, with the notation of \sref{III.1.2.2}, identify $M_{f_{i_0}f_{i_1}\ldots f_{i_p}}$ with $\varinjlim_n M_{i_0i_1\ldots i_p}^{(n)}$.
This identification is an isomorphism of \emph{graded} $S$-modules, if we define the degree of a homogeneous element $z\in\varinjlim_n M_{i_0i_1\ldots i_p}^{(n)}$ in the following way:
$z$ is the canonical image of a homogeneous element $x\in M_{i_0i_1\ldots i_p}^{(n)}=M$ of degree~$m$;
we then take the degree of $z$ to be the integer $m-n(d_{i_0}+d_{i_1}+\ldots+d_{i_p})$.
Taking into account the definition of the homomorphisms $\vphi_{kh}:M_{i_0i_1\ldots i_p}^{(h)}\to M_{i_0i_1\ldots i_p}^{(k)}$ \sref{III.1.2.2}, we immediately see that this definition does not depend on the ``representative'' $x$ of $z$ that we have chosen.
Denote, as in \sref{III.1.2.2}, by $C_n^p(M)$ the set of alternating maps from $[1,r]^{p+1}$ to $M$ (for all $n$), we define in the same way as above the structure of a \emph{graded} $S$-module on $\varinjlim_n C_n^p(M)$;
we again have, as in \sref{III.1.2.2},
\[
\label{III.2.1.2.2}
  C^p(\mathfrak{U}, \sh{F}(\anotherbullet))
  = \varinjlim_n C_n^p(M)
\tag{2.1.2.2}
\]
with the isomorphism \emph{respecting degrees}.
We then have, as in \sref{III.1.2.2},
\[
\label{III.2.1.2.3}
  C^p(\mathfrak{U}, \sh{F}(\anotherbullet))
  = C^{p+1}((\mathbf{f}), M)
  = \varinjlim_n K^{p+1}(\mathbf{f}^n, M)
\tag{2.1.2.3}
\]
with the isomorphism \emph{preserving degrees}:
the degree of an element of $\varinjlim_n K^{p+1}(\mathbf{f}^n, M)$, given by the canonical image of a cochain $g\in K^{p+1}(\mathbf{f}^n,M)$ such that the values $g(i_0,\ldots,i_p)$ are all in the \emph{same} homogeneous component $M_m$ of $M$, is $m-n(d_{i_0}+\ldots+d_{i_p})$, and it is independent of the choice of this cochain as a representative of the element in question.

Since the above isomorphisms are compatible with the coboundary operators, we thus conclude, as in \sref{III.1.2.2}, that we have:
\end{env}

\begin{proposition}[2.1.3]
\label{III.2.1.3}
Let
\end{proposition}


% \subsection{The fundamental theorem of projective morphisms}
% \label{subsection:III.2.2}


% \subsection{Application to graded sheaves of algebras and of modules}
% \label{subsection:III.2.3}


% \subsection{A generalisation of the fundamental theorem}
% \label{subsection:III.2.4}


% \subsection{Euler-Poincar\'e characteristic and the Hilbert polynomial}
% \label{subsection:III.2.5}


% \subsection{Application: ampleness criteria}
% \label{subsection:III.2.6}

\section{Finiteness theorem for proper morphisms}
\label{section:finiteness-theorem-for-proper-morphisms}

\subsection{The d\'evissage lemma}
\label{subsection:devissage-lemma}

\begin{defn}[3.1.1]
\label{3.3.1.1}
Let $\C$ be an abelian category.
We say that a subset $\C'$ of the set of objects of $\C$ is \emph{exact} if $0\in\C'$ and if, for every exact sequence $0\to A'\to A\to A''\to 0$ in $\C$ such that two of the objects $A$, $A'$, $A''$ are in $\C'$, then the third is also in $\C'$.
\end{defn}

\begin{thm}[3.1.2]
\label{3.3.1.2}
Let $X$ be a Noetherian prescheme; we denote by $\C$ the abelian category of coherent $\OO_X$-modules.
Let $\C'$ be an exact subset of $\C$, $X'$ a closed subset of the underlying space of $X$.
Suppose that for every closed irreducible subset $Y$ of $X'$, with generic point $y$, there exists a $\OO_X$-module $\sh{G}\in\C'$ such that $\sh{G}_y$ is a $\kres(y)$-vector space of dimension~$1$.
Then every coheren $\OO_X$-module with support contained in $X'$ is in $\C'$ (and in particular, if $X'=X$, then we have $\C'=\C$).
\end{thm}

\begin{proof}
\label{proof-3.3.1.2}
Consider the following property $\textbf{P}(Y)$ of a closed subset $Y$ of $X'$: every coherent $\OO_X$-module with support contained in $Y$ is in $\C'$.
By virtue of the principle of Noetherian induction \sref[0]{0.2.2.2}, we see that we can reduce to showing that \emph{if $Y$ is a closed subset of $X'$ such that the property $\textbf{P}(Y')$ is true for every closed subset $Y'$ of $Y$, distinct from $Y$, then $\textbf{P}(Y)$ is true}.

Therefore, let $\sh{F}\in\C$ have support contained in $Y$, and we show that $\sh{F}\in\C'$.
Denote also by $Y$ the reduced closed subprescheme of $X$ having $Y$ for its underlying space \sref[I]{1.5.2.1}; it is defined by a coherent sheaf of ideals $\sh{J}$ of $\OO_X$.
We know \sref[I]{1.9.3.4} that there exists an integer $n>0$ such that $\sh{J}^n\sh{F}=0$; for $1\leq k\leq n$, we thus have an exact sequence
\[
  0\to\sh{J}^{k-1}\sh{F}/\sh{J}^k\sh{F}\to\sh{F}/\sh{J}^k\sh{F}\to\sh{F}/\sh{J}^{k-1}\sh{F}\to 0
\]
 of coherent $\OO_X$-modules (\sref[I]{1.5.3.6} and \sref[I]{1.5.3.3}); as $\C'$ is exact, we see, by induction on $k$, that it suffices to show that each of the $\sh{F}_k=\sh{J}^{k-1}\sh{F}/\sh{J}^k\sh{F}$ is in $\C'$.
We thus reduce to proving that $\sh{F}\in\C'$ under the additional hypothesis that $\sh{J}\sh{F}=0$; it is equivalent to say that $\sh{F}=j_*(j^*(\sh{F}))$, where $j$ is the canonical injection $Y\to X$.
Let us now consider two cases:
\begin{enumerate}[label=(\alph*)]
  \item $Y$ is \emph{reducible}.
    Let $Y=Y'\cap Y''$, where $Y'$ and $Y''$ are closed subsets of $Y$, distinct from $Y$; denote also by $Y'$ and $Y''$ the reduced closed subpreschemes of $X$ having $Y$ and $Y''$ for their respective underlying spaces, which are defined respectively by sheaves of ideals $\sh{J}'$ and $\sh{J}''$ of $\OO_X$.
    Set $\sh{F}'=\sh{F}\otimes_{\OO_X}(\OO_X/\sh{J}')$ and $\sh{F}''=\sh{F}\otimes_{\OO_X}(\OO_X/\sh{J}'')$.
    The canonical homomorphisms $\sh{F}\to\sh{F}'$ and $\sh{F}\to\sh{F}''$ thus define a homomorphism $u:\sh{F}\to\sh{F}'\oplus\sh{F}''$.
    We show that for every $z\not\in Y'\cap Y''$, the homomorphism $u_z:\sh{F}_z\to\sh{F}_z'\oplus\sh{F}_z''$ is \emph{bijective}.
    Indeed, we have $\sh{J}'\cap\sh{J}''=\sh{J}$, since the question is local and
\oldpage[III]{116}
    the above equality follows from (\sref[I]{1.5.2.1} and \sref[I]{1.1.1.5}); if $z\not\in Y''$, then we have $\sh{J}_z'=\sh{J}_z$, hence $\sh{F}_z'=\sh{F}_z$ and $\sh{F}_z''=0$, which establishes our assertion in this case; we reason similarly for $z\not\in Y'$.
    As a result, the kernel and cokernel of $u$, which are in $\C$ \sref[0]{0.5.3.4}, have their support in $Y'\cap Y''$, and thus is in $\C'$ by hypothesis; for the same reason, $\sh{F}'$ and $\sh{F}''$ are in $\C'$, hence also $\sh{F}'\oplus\sh{F}''$, as $\C'$ is exact.
    The conclusion then follows from the consideration of the two exact sequences
    \[
      0\to\Im u\to\sh{F}'\oplus\sh{F}''\to\Coker u\to 0,
    \]
    \[
      0\to\Ker u\to\sh{F}\to\Im u\to 0,
    \]
    and the hypothesis that $\C'$ is exact.
  \item $Y$ is irreducible, and as a result, the subprescheme $Y$ of $X$ is \emph{integral}.
    If $y$ is its generic point, then we have $(\OO_Y)_y=\kres(y)$, and as $j^*(\sh{F})$ is a coherent $\OO_Y$-module, $\sh{F}_y=(j^*(\sh{F}))_y$ is a $\kres(y)$-vector space of finite dimension~$m$.
    By hypothesis, there is a coherent $\OO_X$-module $\sh{G}\in\C'$ (necessarily of support $Y$) such that $\sh{G}_y$ is a $\kres(y)$-vector space of dimension~$1$.
    As a result, there is a $\kres(y)$-isomorphism $(\sh{G}_y)^m\isoto\sh{F}_y$, which is also an $\OO_Y$-isomorphism, and as $\sh{G}^m$ and $\sh{F}$ are coherent, there exists an open neighborhood $W$ of $y$ in $X$ and an isomorphism $\sh{G}^m|W\isoto\sh{F}|W$ \sref[0]{0.5.2.7}.
    Let $\sh{H}$ be the graph of this isomorphism, which is a coherent $(\OO_X|W)$-submodule of $(\sh{G}^m\oplus\sh{F})|W$, canonically isomorphic to $\sh{G}^m|W$ and to $\sh{F}|W$; there thus exists a coherent $\OO_X$-submodule $\sh{H}_0$ of $\sh{G}^m\oplus\sh{F}$, inducing $\sh{H}$ on $W$ and $0$ on $X\setmin Y$, since $\sh{G}^m$ and $\sh{F}$ have $Y$ for their support~\sref[I]{1.9.4.7}.
    The restrictions $v:\sh{H}_0\to\sh{G}^m$ and $w:\sh{H}_0\to\sh{F}$ of the canonical projections of $\sh{G}^m\oplus\sh{F}$ are then homomorphisms of coherent $\OO_X$-modules, which, on $W$ and on $X\setmin Y$, reduce to isomorphisms; in other words, the kernels and cokernels of $v$ and $w$ have their suppoer in the closed set $Y\setmin(Y\cap W)$, distinct from $Y$.
    They are in $\C'$; on the other hand, we have $\sh{G}^m\in\C'$ since $\sh{G}\in\C'$ and since $\C'$ is exact.
    We conclude successively, by the exactness of $\C'$, that $\sh{H}_0\in\C'$, then $\sh{F}\in\C'$.
Q.E.D.
\end{enumerate}
\end{proof}

\begin{cor}[3.1.3]
\label{3.3.1.3}
Suppose that the exact subset $\C'$ of $\C$ has in addition the property that any coherent direct factor of a coherent $\OO_X$-module $\sh{M}\in\C'$ is also in $\C'$.
In this case, the conclusion of Theorem~\sref{3.3.1.2} is still valid when the condition ``$\sh{G}_y$ is a $\kres(y)$-vector space of dimension~$1$'' is replaced by $\sh{G}_y\neq 0$ (this is equivalent to $\Supp(\sh{G})=Y$).
\end{cor}

\begin{proof}
\label{proof-3.3.1.3}
The reasoning of Theorem~\sref{3.3.1.2} must be modified only in the case~(b); now $\sh{G}_y$ is a $\kres(y)$-vector space of dimension $q>0$, and as a result, we have an $\OO_Y$-isomorphism $(\sh{G}_y)^m\isoto(\sh{F}_y)^q$; the end of the reasoning in Theorem~\sref{3.3.1.2} then proves that $\sh{F}^q\in\C'$, and the additional hypothesis on $\C'$ implies that $\sh{F}\in\C'$.
\end{proof}

\subsection{The finiteness theorem: the case of usual schemes}
\label{subsection:finiteness-theorem-usual-case}

\begin{thm}[3.2.1]
\label{3.3.2.1}
Let $Y$ be a locally Noetherian prescheme, $f:X\to Y$ a proper morphism.
For every coherent $\OO_X$-module $\sh{F}$, the $\OO_Y$-modules $\RR^q f_*(\sh{F})$ are coherent for $q\geq 0$.
\end{thm}

\begin{proof}
\label{proof-3.3.2.1}
The question being local on $Y$, we can suppose $Y$ Noetherian, thus $X$ Noetherian~\sref[I]{1.6.3.7}.
The coherent $\OO_X$-modules $\sh{F}$ for which the conclusion of Theorem~\sref{3.3.2.1} is true forms an \emph{exact} subset $\C'$ of the category $\C$ of coherent $\OO_X$-modules.
\oldpage[III]{117}
Indeed, let $0\to\sh{F}'\to\sh{F}\to\sh{F}''\to 0$ is an exact sequence of coherent $\OO_X$-modules; suppose for example that $\sh{F}'$ and $\sh{F}''$ belong to $\C'$; we have the long exact sequence in cohomology
\[
  \RR^{q-1}f_*(\sh{F}'')\xrightarrow{\partial}\RR^q f_*(\sh{F}')\to\RR^q f_*(\sh{F})\to\RR^q f_*(\sh{F}'')\xrightarrow{\partial}\RR^{q+1}f_*(\sh{F}'),
\]
in which by hypothesis the outer four terms are coherent; it is the same for the middle term $\RR^q f_*(\sh{F})$ by (\sref[0]{0.5.3.4}~and~\sref[0]{0.5.3.3}).
We show in the same way that when $\sh{F}$ and $\sh{F}'$ (resp.~$\sh{F}$ and $\sh{F}''$) are in $\C'$, then so is $\sh{F}''$ (resp.~$\sh{F}'$).
In addition, every coherent \emph{direct factor} $\sh{F}'$ of an $\OO_X$-module $\sh{F}\in\C'$ belongs to $\C'$: indeed, $\RR^q f_*(\sh{F}')$ is then a direct factor of $\RR^q f_*(\sh{F})$~(G,~II,~4.4.4), therefore it is of finite type, and as it is quasi-coherent~\sref{3.1.4.10}, it is coherent, as $Y$ is Noetherian.
By virtue of Corollary~\sref{3.3.1.3}, we reduce to proving that when $X$ is \emph{irreducible} with generic point $x$, there exists \emph{one} coherent $\OO_X$-module $\sh{F}$ belonging to $\C'$, such that $\sh{F}_x\neq 0$: indeed, if this point is established, then it can be applied to any irreducible closed subprescheme $Y$ of $X$, since if $j:Y\to X$ is the canonical injection, then $f\circ j$ is proper~\sref[II]{2.5.4.2}, and if $\sh{G}$ is a coherent $\OO_Y$-module with support $Y$, then $j_*(\sh{G})$ is a coherent $\OO_X$-module such that $\RR^q(f\circ j)_*(\sh{G})=\RR^q f_*(j_*(\sh{G}))$~(G,~II,~4.9.1), therefore we can apply COrollary~\sref{3.3.1.3}.

By virtue of Chow's lemma~\sref[II]{2.5.6.2}, there exists an irreducible prescheme $X'$ an a \emph{projective} and surjective morphism $g:X'\to X$ such that $f\circ g:X'\to Y$ is \emph{projective}.
There exists an ample $\OO_X$-module $\sh{L}$ for $g$~\sref[II]{2.5.3.1}; we apply the fundamental theorem of projective morphisms~\sref{3.2.2.1} to $g:X'\to X$ and with $\sh{L}$: there thus exixts an integer $n$ such that $\sh{F}=g_*(\OO_{X'}(n))$ is a coherent $\OO_X$-module and $\RR^q g_*(\OO_{X'}(n))=0$ for all $q>0$; in addition, as $g^*(g_*(\OO_{X'}(n)))\to\OO_{X'}(n)$ is surjective for $n$ large enough~\sref{3.2.2.1}, we see that we can suppose, at the generic point $x$ of $X$, that we have $\sh{F}_x\neq 0$~\sref[II]{2.3.4.7}.
On the other hand, as $f\circ g$ is projective as $Y$ is Noetherian, the $\RR^q(f\circ g)_*(\OO_{X'}(n))$ are \emph{coherent}~\sref{3.2.2.1}.
This being so, $\RR^\bullet(f\circ g)_*(\OO_{X'}(n))$ is the abutment of a Leray spectral sequence, whose $E_2$-term is given by $E_2^{pq}=\RR^p f_*(\RR^q g_*(\OO_{X'}(n)))$; the above shows that this spectral sequence degenerates, and we then know~\sref[0]{0.11.1.6} that $E_2^{p0}=\RR^p f_*(\sh{F})$ is isomorphic to $\RR^p(f\circ g)_*(\OO_{X'}(n))$, which finishes the proof.
\end{proof}

\begin{cor}[3.2.2]
\label{3.3.2.2}
Let $Y$ be a locally Noetherian prescheme.
For every proper morphism $f:X\to Y$, the direct image under $f$ of any coherent $\OO_X$-module is a coherent $\OO_Y$-module.
\end{cor}

\begin{cor}[3.2.3]
\label{3.3.2.3}
Let $A$ be a Noetherian ring, $X$ a proper scheme over $A$; for every coherent $\OO_X$-module $\sh{F}$, the $\HH^p(X,\sh{F})$ are $A$-modules of finite type, and there exists an integer $r>0$ such that for every coherent $\OO_X$-module $\sh{F}$ and all $p>r$, $\HH^p(X,\sh{F})=0$.
\end{cor}

\begin{proof}
\label{proof-3.3.2.3}
The second assertion has already been proved~\sref{3.1.4.12}; the first follows from the finiteness theorem~\sref{3.3.2.1}, taking into account Corollary~\sref{3.1.4.11}.
\end{proof}

In particular, if $X$ is a \emph{proper algebraic scheme} over a field $k$, then, for every coherent $\OO_X$-module $\sh{F}$, the $\HH^p(X,\sh{F})$ are \emph{finite-dimensional} $k$-vector spaces.

\begin{cor}[3.2.4]
\label{3.3.2.4}
Let $Y$ be a locally Noetherian prescheme, $f:X\to Y$ a morphism of finite type.
For every coherent $\OO_X$-module $\sh{F}$ whose support in proper over $Y$~\sref[II]{2.5.4.10}, the $\OO_Y$-modules $\RR^q f_*(\sh{F})$ are coherent.
\end{cor}

\begin{proof}
\label{proof-3.3.2.4}
\oldpage[III]{118}
The question being local on $Y$, we can suppose $Y$ Noetherian, and it is the same for $X$~\sref[I]{1.6.3.7}.
By hypothesis, every closed subprescheme $Z$ of $X$ whose underlying space is $\Supp(\sh{F})$ is proper over $Y$, in other words, if $j:Z\to X$ is the canonical injection, then $f\circ j:Z\to Y$ is proper.
We can suppose that $Z$ is such that $\sh{F}=j_*(\sh{G})$, where $\sh{G}=j^*(\sh{F})$ is a coherent $\OO_Z$-module~\sref[I]{1.9.3.5}; as we have $\RR^q f_*(\sh{F})=\RR^q(f\circ j)_*(\sh{G})$ by Corollary~\sref{3.1.3.4}, the conclusion follows immediately from Theorem~\sref{3.3.2.1}.
\end{proof}

\subsection{Generalization of the finiteness theorem (usual schemes)}
\label{subsection:finiteness-theorem-generalization-usual}



\section{The fundamental theorem of proper morphisms. Applications}
\label{section:3.4}


\subsection{The fundamental theorem}
\label{subsection:3.4.1}


% \subsection{Particular cases and variants}
% \label{subsection:3.4.2}


% \subsection{Zariski's connection theorem}
% \label{subsection:3.4.3}


% \subsection{Zariski's ``main theorem''}
% \label{subsection:3.4.4}


% \subsection{Completions of modules of homomorphisms}
% \label{subsection:3.4.5}


% \subsection{Relations between formal morphisms and usual morphisms}
% \label{subsection:3.4.6}


% \subsection{An ampleness criterion}
% \label{subsection:3.4.7}


% \subsection{Finite morphisms of formal preschemes}
% \label{subsection:3.4.8}

\section{An existence theorem for coherent algebraic sheaves}
\label{section:an-existence-theorem-for-coherent-algebraic-sheaves}


\section{Local and global Tor functors; K\"unneth formula}
\label{section:3.6}


\subsection{Introduction}
\label{subsection:3.6.1}


% \subsection{Hypercohomology of complexes of modules on a prescheme}
% \label{subsection:3.6.2}


% \subsection{Hypertor of two complexes of modules}
% \label{subsection:3.6.3}


% \subsection{Local hypertor functors of complexes of quasi-coherent modules: the affine scheme case}
% \label{subsection:3.6.4}


% \subsection{Local hypertor functors of complexes of quasi-coherent modules: the general case}
% \label{subsection:3.6.5}


% \subsection{Global hypertor functors of complexes of quasi-coherent modules and K\"unneth spectral sequences: the affine base case}
% \label{subsection:3.6.6}


% \subsection{Global hypertor functors of complexes of quasi-coherent modules and K\"unneth spectral sequences: the general case}
% \label{subsection:3.6.7}


% \subsection{The associativity spectral sequences of global hypertor}
% \label{subsection:3.6.8}


% \subsection{The base change spectral sequences of global hypertor}
% \label{subsection:3.6.9}


% \subsection{Local structure of certain cohomological functors}
% \label{subsection:3.6.10}

\section{Base change for homological functors of sheaves of modules}
\label{section:III.7}


\subsection{Functors of $A$-modules}
\label{subsection:III.7.1}


% \subsection{Characterisation of the tensor product functor}
% \label{subsection:III.7.2}


% \subsection{Exactness criteria for homological functors of modules}
% \label{subsection:III.7.3}


% \subsection{Exactness criteria for the functors $H_\bullet(P_\bullet\otimes_A M)$}
% \label{subsection:III.7.4}


% \subsection{The case of local Noetherian rings}
% \label{subsection:III.7.5}


% \subsection{Descent of exactness properties. Semi-continuity theorem and Grauert’s exactness criterion}
% \label{subsection:III.7.6}


% \subsection{Application to proper morphisms: I. The exchange property}
% \label{subsection:III.7.7}


% \subsection{Application to proper morphisms: II. Cohomological flatness criteria}
% \label{subsection:III.7.8}


% \subsection{Application to proper morphisms: III. Invariance of Euler-Poincar\'e characteristic and the Hilbert polynomial}
% \label{subsection:III.7.9}


\bibliography{the}
\bibliographystyle{amsalpha}

\end{document}

