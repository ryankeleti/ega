\ProvidesPackage{preamble}

\usepackage[utf8]{inputenc}
\usepackage[T1]{fontenc}
\usepackage{microtype}
\usepackage[left=0.75in,right=0.75in,top=0.75in,bottom=0.75in]{geometry}
\usepackage[all]{xy}
\usepackage{enumitem}
\usepackage{color}
\usepackage{soul}
\usepackage{fancyhdr}
\usepackage{mathtools}
\usepackage{amssymb}
\usepackage{amsthm}
\usepackage[charter,
            greekfamily=didot,
            uppercase=upright,
            greeklowercase=upright]{mathdesign}
\usepackage[compact]{titlesec}
\usepackage[colorlinks=true,hyperindex,citecolor=blue,linkcolor=magenta]{hyperref}
\usepackage{bookmark}
\usepackage[asterism]{sectionbreak}


%%%%%%%%%%%%%%
% formatting %
%%%%%%%%%%%%%%

\allowdisplaybreaks[1]
\binoppenalty=9999
\relpenalty=9999
\setitemize{nosep}

% for Chapter 0, Chapter I, etc.
% credit for ZeroRoman https://tex.stackexchange.com/questions/211414/
\newcommand{\ZeroRoman}[1]{\ifcase\value{#1}\relax 0\else\Roman{#1}\fi}
\renewcommand{\thechapter}{\ZeroRoman{chapter}}

%%%%%%%%%%%%%%%%%
% math commands %
%%%%%%%%%%%%%%%%%

% for easy changes to style
\newcommand{\sh}{\mathscr}         % sheaf font
\newcommand{\bb}{\mathbf}          % bold font
\newcommand{\cat}{\mathsf}         % category font
%
\newcommand{\rad}{\mathfrak{r}}    % radical
\newcommand{\nilrad}{\mathfrak{R}} % nilradical
\newcommand{\emp}{\varnothing}     % empty set
\newcommand{\vphi}{\phi}           % font switches \phi and \varphi, change if needed
\newcommand{\HH}{\mathrm{H}}       % cohomology
\newcommand{\dual}[1]{{#1}^\vee}   % dual
\renewcommand{\k}{\bb{k}}          % residue field
\newcommand{\K}{\cat{K}}           % category
\newcommand{\OO}{\sh{O}}           % structure sheaf
\newcommand{\F}{\sh{F}}            % sheaf F
\newcommand{\G}{\sh{G}}            % sheaf G

% operators
%\newcommand*{\sheafHom}{\mathscr{H}\text{\normalfont\kern -3pt {\calligra\large om}}\,}
\def\shHom{\sh{H}\textit{om}} % sheaf Hom
\def\Hom{{\mathop{\mathrm{Hom}}\nolimits}}
\def\Supp{{\mathop{\mathrm{Supp}}\nolimits}}
\def\img{{\mathop{\mathrm{im}}\nolimits}}
\def\Spec{{\mathop{\mathrm{Spec}}\nolimits}}

% if unsure of a translation
\newcommand{\unsure}[2][]{\hl{#2}\marginpar{#1}}
\newcommand{\completelyunsure}{\unsure{[\ldots]}}

% use to mark where original page starts
\newcommand{\oldpage}[1]{\marginpar{\textbf{#1}}\ignorespaces}

% special ref
\newcommand{\sref}[2]{\hyperref[#1-\arabic{chapter}.#2]{\normalfont{(#2)}}}

% ref prelim
\newcommand{\pref}[2]{\hyperref[#1-0.#2]{\normalfont{(\textbf{0}, #2)}}}

%% ref out of chapter
%\newcommand{\cref}[4]{\hyperref[#1-#2.#3]{\normalfont{(\textbf{#3}, #4)}}}

% currently this works as \begin{env}[optional rmk]{x.y.z}
\makeatletter
\newenvironment{env}[2][\@nil]{%
    \def\tmp{#1}%
    \ifx\tmp\@nnil
        \par\medskip\noindent\indent\textbf{(#2)}\rmfamily
    \else
        \par\medskip\noindent\indent\textit{\textbf{#1}}~\textbf{(#2)}.\,---\rmfamily
    \fi}
\makeatother

% use this for definitions, propositions, corollaries, etc.
\makeatletter
\newenvironment{envs}[2][\@nil]{
  \par\medskip\noindent\indent\textit{\textbf{#1}}~\textbf{(#2)}.\,---\itshape
}
\makeatother



\begin{document}
\title{Elementary global study of some classes of morphisms (EGA~II)}
\maketitle

\phantomsection
\label{section-phantom}

build hack
\cite{I-1}

\tableofcontents

\section*{Summary}
\label{section-egaII-summary}

\begin{tabular}{ll}
  \textsection1. & Affine morphisms\\
  \textsection2. & Homogeneous prime spectra.\\
  \textsection3. & Homogeneous prime spectrum of a sheaf of graded algebras.\\
  \textsection4. & Projective bundles; ample sheaves.\\
  \textsection5. & Quasi-affine morphisms; quasi-projective morphisms; proper morphisms; projective morphisms.\\
  \textsection6. & Integral morphisms and finite morphisms.\\
  \textsection7. & Valuative criteria.\\
  \textsection8. & Blowup schemes; projective cones; projective closure.\\
\end{tabular}\\

\oldpage[II]{5}
The various classes of morphisms studied in this chapter are used extensively in cohomological methods; further study, using these methods, will be done in Chapter~III, where we use especially \textsection\textsection2,4, and 5 of Chapter~II.
The \textsection8 can be omitted on a first reading: it gives some supplements to the formalism developed in \textsection\textsection1 and 3, reducing to easy maps of this formalism, and we will use it less consistently than the other results of this chapter.
\bigskip

\input{egaII/egaII-1}
\input{egaII/egaII-2}
\input{egaII/egaII-3}
\input{egaII/egaII-4}
\input{egaII/egaII-5}
\input{egaII/egaII-6}
\input{egaII/egaII-7}
\input{egaII/egaII-8}

\bibliography{the}
\bibliographystyle{amsalpha}

\end{document}

