\section{Reduced preschemes; separation condition}
\label{section-reduced-preschemes-and-separation-condition}

\subsection{Reduced preschemes}
\label{subsection-reduced-preschemes}

\begin{prop}[5.1.1]
\label{1.5.1.1}
Let $(X,\OO_X)$ be a prescheme, and $\sh{B}$ a quasi-coherent $\OO_X$-algebra.
There exists a unique quasi-coherent $\OO_X$-module $\sh{N}$ whose stalk $\sh{N}_x$ at each $x\in X$ is the nilradical of the ring $\sh{B}_x$.
When $X$ is affine, and, consequently, $\sh{B}=\wt{B}$, where $B$ is an algebra over $A(X)$, then we have $\sh{N}=\wt{\nilrad}$, where $\nilrad$ is the nilradical of $B$.
\end{prop}

\begin{proof}
\label{proof-1.5.1.1}
\oldpage[I]{128}
The statement is local, so it suffices to show the latter claim.
We know that $\wt{\nilrad}$ is a quasi-coherent $\OO_X$-module \sref{1.1.4.1} and that its stalk at a point $x\in X$ is the ideal $\nilrad_x$ of the ring of fractions $B_x$;
it remains to prove that the nilradical of $B_x$ is contained in $\nilrad_x$, the converse inclusion being evident.
Let $z/s$ be an element of the nilradical of $B_x$, with $z\in B$, $s\not\in\fk{j}_x$;
by hypothesis, there exists an integer $k$ such that $(z/s)^k=0$, which implies that there exists some $t\not\in\fk{j}_x$ such that $tz^k=0$.
We conclude that $(tz)^k=0$, and, as a result, that $z/s=(tz)/(ts)\in\nilrad_x$.
\end{proof}

We say that the quasi-coherent $\OO_X$-module $\sh{N}$ thus defined is the \emph{nilradical} of the $\OO_X$-algebra $\sh{B}$; in particular, we denote by $\sh{N}_X$ the nilradical of $\OO_X$.

\begin{cor}[5.1.2]
\label{1.5.1.2}
Let $X$ be a prescheme;
the closed subprescheme of $X$ defined by the sheaf of ideals $\sh{N}_X$ is the only reduced subprescheme \sref[0]{0.4.1.4} of $X$ that has $X$ as its underlying space;
it is also the smallest subprescheme of $X$ that has $X$ as its underlying space.
\end{cor}

\begin{proof}
\label{proof-1.5.1.2}
Since the structure sheaf of the closed subprescheme of $Y$ defined by $\sh{N}_X$ is $\OO_X/\sh{N}_X$, it is immediate that $Y$ is reduced and has $X$ as its underlying space, because $\sh{N}_x\neq\OO_x$ for any $x\in X$.
To show the other claims, note that a subprescheme $Z$ of $X$ that has $X$ as its underlying space is defined by a sheaf of ideals $\sh{I}$ \sref{1.4.1.3} such that $\sh{I}_x\neq\OO_x$ for any $x\in X$.
We can restrict to the case where $X$ is affine, say $X=\Spec(A)$ and $\sh{I}=\wt{\fk{I}}$, where $\fk{I}$ is an ideal of $A$;
then, for every $x\in X$, we have $\fk{I}_x\subset\fk{j}_x$, and so $\fk{I}$ is contained in every prime ideal of $A$, and so also in their intersection $\nilrad$, the nilradical of $A$.
This proves that $Y$ is the small subprescheme of $X$ that has $X$ as its underlying space \sref{1.4.1.9};
furthermore, if $Z$ is distinct from $Y$, we necessarily have $\sh{I}_x\neq\sh{N}_x$ for at least one $x\in X$, and so \sref{1.5.1.1} $Z$ is not reduced.
\end{proof}

\begin{defn}[5.1.3]
\label{1.5.1.3}
We define the reduced prescheme associated to a prescheme $X$, denoted by $X_\mathrm{red}$, to be the unique reduced subprescheme of $X$ that has $X$ as its underlying space.
\end{defn}

Saying that a prescheme $X$ is reduced thus implies that $X=X_\mathrm{red}$.

\begin{prop}[5.1.4]
\label{1.5.1.4}
For the prime spectrum of a ring $A$ to be a reduced (\emph{resp.} integral) prescheme \sref{1.2.1.7}, it is necessary and sufficient that $A$ be a reduced (\emph{resp.} integral) ring.
\end{prop}

\begin{proof}
\label{proof-1.5.1.4}
Indeed, it follows immediately from \sref{1.5.1.1} that the condition $\sh{N}=(0)$ is necessary and sufficient for $X=\Spec(A)$ to be reduced;
the claim relating to integral rings is then a consequence of \sref{1.1.1.13}.
\end{proof}

Since every ring of fractions $\neq\{0\}$ of an integral ring is integral, it follows from \sref{1.5.1.4} that, for every \emph{locally integral} prescheme $X$, $\OO_x$ is an \emph{integral} ring for every $x\in X$.
The converse is true whenever the underlying space of $X$ is \emph{locally Noetherian}:
indeed, $X$ is then reduced, and if $U$ is an affine open subset of $X$, which is a Noetherian space, then $U$ has only a finite number of irreducible components, and so its ring $A$ has only a finite number of minimal prime ideals \sref{1.1.1.14}.
If two of these components $U_i$ had a common point $x$, then $\OO_x$ would have at least two distinct minimal prime ideals, and would thus not be integral;
the $U_i$ are thus open subsets that are pairwise disjoint, and each of them is thus integral.

\begin{env}[5.1.5]
\label{1.5.1.5}
Let $f=(\psi,\theta):X\to Y$ be a morphism of preschemes;
\oldpage[I]{129}
the homomorphism $\theta_x^\#:\OO_{\psi(x)}\to\OO_x$ sends each nilpotent element of $\OO_{\psi(x)}$ to a nilpotent element of $\OO_x$;
by passing to the quotients, $\theta^\#$ induces a homomorphism
\begin{equation*}
    \omega:\psi^*(\OO_Y/\sh{N}_Y)\to\OO_X/\sh{N}_X.
\end{equation*}
It is clear that, for every $x\in X$, $\omega_x:\OO_{\psi(x)}/\sh{N}_{\psi(x)}\to\OO_x/\sh{N}_x$ is a local homomorphism, and so $(\psi,\omega^\flat)$ is a morphism of preschemes $X_\mathrm{red}\to Y_\mathrm{red}$, which we denote by $f_\mathrm{red}$, and call the \emph{reduced} morphism associated to $f$.
It is immediate that, for two morphisms $f:X\to Y$ and $g:Y\to Z$, we have $(g\circ f)_\mathrm{red}=g_\mathrm{red}\circ f_\mathrm{red}$, and so we have defined $X_\mathrm{red}$ as a \emph{covariant functor} in $X$.

The preceding definition shows that the diagram
\begin{equation*}
    \xymatrix{
        X_\mathrm{red} \ar[r]^{f_\mathrm{red}} \ar[d]
        & Y_\mathrm{red} \ar[d]\\
        X \ar[r]_f
        &Y
    }
\end{equation*}
is commutative, where the vertical arrows are the injection morphisms;
in other words, $X_\mathrm{red}\to X$ is a \emph{functorial} morphism.
We note in particular that, if $X$ is reduced, every morphism $f:X\to Y$ factors as $X\xrightarrow{f_\mathrm{red}}Y_\mathrm{red}\to Y$;
in other words, $f$ is \unsure{bounded above} by the injection morphism $Y_\mathrm{red}\to Y$.
\end{env}

\begin{prop}[5.1.6]
\label{1.5.1.6}
Let $f:X\to Y$ be a morphism;
if $f$ is surjective (\emph{resp.} radicial, an immersion, a closed immersion, an open immersion, a local immersion, a local isomorphism), then so too is $f_\mathrm{red}$.
Conversely, if $f_\mathrm{red}$ is surjective (\emph{resp.} radicial), then so too is $f$.
\end{prop}

\begin{proof}
\label{proof-1.5.1.6}
\end{proof}
