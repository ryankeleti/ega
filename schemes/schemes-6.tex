\section{Finiteness conditions}
\label{section-finiteness-conditions}

\subsection{Noetherian and locally-Noetherian preschemes}
\label{subsection-noetherian-and-locally-noetherian-preschemes}

\begin{defn}[6.1.1]
\label{1.6.1.1}
We say that a prescheme $X$ is Noetherian (\emph{resp.} locally Noetherian) if it is a finite union (\emph{resp.} union) of affine open $V_\alpha$ in such a way that the ring of the of the induced scheme on each of the $V_\alpha$ is Noetherian.
\end{defn}

It follows immediately from \sref{1.1.5.2} that, if $X$ is locally Noetherian, then the structure sheaf $\OO_X$ is a \emph{coherent sheaf of rings}, the question being a local one.
Every \emph{quasi-coherent sub-$\OO_X$-module}\oldpage[I]{141} (\emph{resp.} quasi-coherent quotient $\OO_X$-module) of a \emph{coherent} $\OO_X$-module $\sh{F}$ is \emph{coherent}, as the question is once again a local one, and it suffices to apply \sref{1.1.5.1}, \sref{1.1.4.1}, and \sref{1.1.3.10}, combined with the fact that a submodule (resp. quotient module) of a module of finite type over a Noetherian ring is of finite type.
In particular, every \emph{quasi-coherent sheaf of ideals} of $\OO_X$ is \emph{coherent}.

If a prescheme $X$ is a finite union (resp. union) of open subsets $W_\lambda$ in such a way that the preschemes induced on the $W_\lambda$ are Noetherian (resp. locally Noetherian), it is clear that $X$ is Noetherian (resp. locally Noetherian).

\begin{prop}[6.1.2]
\label{1.6.1.2}
For a prescheme $X$ to be Noetherian, it is necessary and sufficient for it to be locally Noetherian and have a quasi-compact underlying space.
The underlying space itself is then also Noetherian.
\end{prop}

\begin{proof}
\label{proof-1.6.1.2}
The first claim follows immediately from the definitions and \sref{1.1.1.10}[(ii)].
The second follows from \sref{1.1.1.6} and the fact that every space that is a finite union of Noetherian subspaces is itself Noetherian \sref[0]{0.2.2.3}.
\end{proof}

\begin{prop}[6.1.3]
\label{1.6.1.3}
Let $X$ be an affine scheme given by a ring $A$.
The following conditions are equivalent:
\emph{a)} $X$ is Noetherian;
\emph{b)} $X$ is locally Noetherian;
\emph{c)} $A$ is Noetherian.
\end{prop}

\begin{proof}
\label{proof-1.6.1.3}
The equivalence between \emph{a)} and \emph{b)} follows from \sref{1.6.1.2} ant the fact that the underlying space of every affine scheme is quasi-compact \sref{1.1.1.10};
it is furthermore clear that \emph{c)} implies \emph{a)}.
To see that \emph{a)} implies \emph{c)}, we remark that there is a finite cover $(V_i)$ of $X$ by affine open subsets such that the ring $A_i$ of the prescheme induced on $V_i$ is Noetherian.
So let $(\fk{a}_n)$ be an increasing sequence of ideals of $A$;
by a canonical bijective correspondence, there is a corresponding sequence $(\widetilde{a}_n)$ of sheaves of ideals in $\widetilde{A}=\OO_X$;
to see that the sequence $(\fk{a}_n)$ is \unsure{stable}, it suffices to prove that the sequence $(\widetilde{\fk{a}}_n)$ is.
But the restriction $\widetilde{\fk{a}}_n|V_i$ is a quasi-coherent sheaf of ideals in $\OO_X|V_i$, being the inverse image of $\widetilde{\fk{a}}_n$ under the canonical injection $V_i\to X$ \sref[0]{0.5.1.4};
$\widetilde{\fk{a}}_n|V_i$ is thus of the form $\widetilde{\fk{a}}_{ni}$, where $\fk{a}_{ni}$ is an ideal of $A_i$ \sref{1.1.3.7}.
Since $A_i$ is Noetherian, the sequence $(\fk{a}_{ni})$ is stable for all $i$, whence the proposition.
\end{proof}

We note that the above argument proves also that \emph{if $X$ is a Noetherian prescheme, then every increasing sequence of coherent sheaves of ideals of $\OO_X$ is \unsure{stable}}.

\begin{prop}[6.1.4]
\label{1.6.1.4}
Every subprescheme of a Noetherian (\emph{resp.} locally-Noetherian) prescheme is Noetherian (\emph{resp.} locally Noetherian).
\end{prop}

\begin{proof}
\label{proof-1.6.1.4}
If suffices to give a proof for a Noetherian prescheme $X$;
further, by definition~\sref{1.6.1.1}, we can also restrict to the case where $X$ is an affine scheme.
Since every subprescheme of $X$ is a closed subprescheme of a prescheme induced on an open subset \sref{1.4.1.3}, we can restrict to the case of a subprescheme $Y$, either closed or induced on an open subset of $X$.
The proof in the case where $Y$ is closed is immediate, since if $A$ is the ring of $X$, we know that $Y$ is an affine scheme given by the ring $A/\fk{J}$, where $\fk{J}$ is an ideal of $A$ \sref{1.4.2.3};
since $A$ is Noetherian \sref{1.6.1.3}, so too is $A/\fk{J}$.

Now suppose that $Y$ is open in $X$;
the underlying space of $Y$ is Noetherian \sref{1.6.1.2}, hence quasi-compact, and thus a finite union of open subsets $D(f_i)$ ($f_i\in A$);
everything reduces to showing the proposition in the case where $Y=D(f)$ with $f\in A$.
But then $Y$ is an affine scheme whose ring is isomorphic to $A_f$ \sref{1.1.3.6};
since $A$ is Noetherian \sref{1.6.1.3}, so too is $A_f$.
\end{proof}

\begin{env}[6.1.5]
\label{1.6.1.5}
WE note that the \emph{product} of two Noetherian $S$-preschemes is not necessarily Noetherian, even if the preschemes are affine, since the tensor product of two Noetherian algebras in not necessarily a Noetherian ring (cf. \sref{1.6.3.8}).
\end{env}

\begin{prop}[6.1.6]
\label{1.6.1.6}
If $X$ is a Noetherian prescheme, the nilradical $\sh{N}_X$ of $\OO_X$ is nilpotent.
\end{prop}

\begin{proof}
\label{proof-1.6.1.6}
\end{proof}
