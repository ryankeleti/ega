\section{Rational maps}
\label{section-rational-maps}

\subsection{Rational maps and rational functions}
\label{subsection-rational-maps-and-rational-functions}

\begin{env}[7.1.1]
\label{1.7.1.1}
Let $X$ and $Y$ be two preschemes, $U$ and $V$ two dense open subsets of $X$, and $f$ (resp. $g$) a morphism from $U$ (resp. $V$) to $Y$; we say that $f$ and $g$ are \emph{equivalent} if the agree on a dense open subset of $U\cap V$.
Since a finite intersection of dense open subsets of $X$ is a dense open subset of $X$, it is clear that this relation is an \emph{equivalence relation}.
\end{env}

\begin{defn}[7.1.2]
\label{1.7.1.2}
Given two preschemes $X$ and $Y$, we define a rational map from $X$ to $Y$ to be an equivalence class of morphisms from a dense open subset of $X$ to a dense open subset of $Y$, under the equivalence relation defined in \hyperref[1.7.1.1]{(7.1.1)}.
If $X$ and $Y$ are $S$-preschemes, we say that such a class is a rational $S$-map if there exists a representative of the class that is an $S$-morphism.
We define a rational $S$-section of $X$ to be any rational $S$-map from $S$ to $X$.
We define a rational function on a prescheme $X$ to be any rational $X$-section on the $X$-prescheme $X\otimes_{\bb{Z}}\bb{Z}[T]$ (where $T$ is an indeterminate).
\end{defn}

By an abuse of language, whenever we are discussing only $S$-preschemes, we will say ``rational map'' instead of ``rational $S$-map'' if no confusion may arise.

Let $f$ be a rational map from $X$ to $Y$, and $U$ an open subset of $X$; if $f_1$ and $f_2$ are two morphisms belonging to the class of $f$, defined (respectively) on dense open subsets $V$ and $W$ of $X$, then the restrictions $f_1|(U\cap V)$ and $f_2|(U\cap W)$ agree on $U\cap V\cap W$, which is dense in $U$; the class $f$ of morphisms thus defines a rational map from $U$ to $Y$, called the \emph{restriction of $f$ to $U$}, and denoted by $f|U$.

If, to every $S$-morphism $f\colon X\to Y$, we take the corresponding rational $S$-map to which $f$ belongs, we obtain a canonical map from $\Hom_S(X,Y)$ to the set of rational $S$-maps from $X$ to $Y$.
We denote by $\Gamma_\mathrm{rat}(X/Y)$ the set of rational $Y$-sections on $X$, and we thus have a canonical map $\Gamma(X/Y)\to\Gamma_\mathrm{rat}(X/Y)$.
It is also clear that, if $X$ and $Y$ are $S$-preschemes, then the set of rational $S$-maps from $X$ to $Y$ is canonically identified with $\Gamma_\mathrm{rat}((X\times_S Y)/X)$ \hyperref[1.3.3.14]{(3.3.14)}.

\begin{env}[7.1.3]
\label{1.7.1.3}
It also follows from \hyperref[1.7.1.2]{(7.1.2)} and \hyperref[1.3.3.14]{(3.3.14)} that the rational functions on $X$ are canonically identified with \emph{equivalence classes of sections of the structure sheaf $\OO_X$} over dense open subsets of $X$, where two such sections are equivalent if the agree on some dense open subset of $X$ contained inside the intersection of the subsets on which they are defined.
In particular, it follows that the rational functions on $X$ form a \emph{ring} $R(X)$.
\end{env}

\begin{env}[7.1.4]
\label{1.7.1.4}
When $X$ is an \emph{irreducible} prescheme, every non-empty open subset of $X$ is dense in $X$; so we can say that the non-empty open subsets of $X$ are the \emph{open neighbourhoods of the generic point} $x$ of $X$.
To say that two morphisms from non-empty open subsets of $X$ to $Y$ are equivalent means thus means, in this case, that they have the \emph{same germ} at the point $x$.
Said otherwise, the rational maps (resp. rational $S$-maps) $X\to Y$ are identified with the \emph{germs of morphisms} (resp. \emph{$S$-morphisms}) from non-empty open subsets of $X$ to $Y$ at the generic point $x$ of $X$.
In particular:
\end{env}

\begin{prop}[7.1.5]
\label{1.7.1.5}
If $X$ is an irreducible prescheme, the ring $R(X)$ of rational maps on $X$ is canonically identified with the local ring $\OO_x$ of the generic point $x$ of $X$.
It is a local ring of dimension~0, and so a local Artinian ring when $X$ is Noetherian; it is a field when $X$ is integral, and it is identified with the field of fractions of $A(X)$ when $X$ is further an affine scheme.
\end{prop}

\begin{proof}
\label{proof-1.7.1.5}
Given the above, and the identification of rational functions with sections of $\OO_X$ over a dense open subset of $X$, the first claim is nothing but the definition of the fibre of a sheaf above a point.
For the other claims, we can limit ourselves to the case where $X$ is affine, given by some ring $A$; then $\mathfrak{j}_x$ is the nilradical of $A$, and $\OO_x$ is thus of dimension~0; if $A$ is integral, then $\mathfrak{j}_x=(0)$ and $\OO_x$ is thus the field of fractions of $A$.
Finally, if $A$ is Noetherian, we know (\cite[p.~127, cor.~4]{I-11}) that $\mathfrak{j}_x$ is nilpotent and $\OO_x=A_x$ Artinien.
\end{proof}

If $X$ is \emph{integral}, the ring $\OO_z$ is integral for all $z\in X$; every affine open subset $U$ containing $z$ also contains $x$, and $R(U)$, equal to the field of fractions of $A(U)$, is identified with $R(X)$; we thus conclude that $R(X)$ can be identified also with the \emph{field of fractions of $\OO_z$}: the canonical identification of $\OO_z$ to a subring of $R(X)$ consists of associating, to every germ of a section $s\in\OO_z$, the unique rational function on $X$, \unsure{the} class of a section of $\OO_X$ (necessarily defined on a dense open subset of $X$) having $s$ as its germ at the point $z$.

\begin{env}[7.1.6]
\label{1.7.1.6}
Now suppose that $X$ has a \emph{finite} number of irreducible components $X_i$ ($1\leqslant i\leqslant n$) (which will be the case whenever the underlying space of $X$ is \emph{Noetherian}); let $X'_i$ be open subset of $X$ given by the complement of the $X_j\cap X_i$ for $j\neq i$ inside $X_i$; $X'_i$ is irreducible, its generic point $x_i$ is the generic point of $X_i$, and the $X'_i$ are pairwise disjoint, with their union being dense in $X$ \hyperref[0.2.1.6]{(\textbf{0},~2.1.6)}.
For every dense open subset $U$ of $X$, $U_i=U\cap X'_i$ is a non-empty dense open subset of $X'_i$, with the $U_i$ being pairwise disjoint, and so $U'=\bigcup_i U'_i$ is dense in $X$.
Giving a morphism from $U'$ to $Y$ consists of giving (arbitrarily) a morphism from each of the $U_i$ to $Y$.
Thus:
\end{env}

\begin{prop}[7.1.7]
\label{1.7.1.7}
Let $X$ and $Y$ be two preschemes (\emph{resp.} $S$-preschemes) such that $X$ has a finite number of irreducible components $X_i$, with generic points $x_i$ ($1\leqslant i\leqslant n$).
If $R_i$ is the set of germs of morphisms (\emph{resp.} $S$-morphisms) from open subsets of $X$ to $Y$ at the point $x_i$, then the set of rational maps (\emph{resp.} rational $S$-maps) from $X$ to $Y$ can be identified with the product of the $R_i$ ($1\leqslant i\leqslant n$).
\end{prop}

\begin{cor}[7.1.8]
\label{1.7.1.8}
Let $X$ be a Noetherian prescheme.
The ring of rational functions on $X$ is an Artinian ring, whose local \unsure{components} are the rings $\OO_{x_i}$ of the generic points $x_i$ of the irreducible components of $X$.
\end{cor}

\begin{cor}[7.1.9]
\label{1.7.1.9}
Let $A$ be a Noetherian ring, and $X=\Spec(A)$.
If $Q$ is the complement of the union of minimal prime ideals of $A$, then the ring of rational functions on $X$ can be canonically identified with the ring of fractions $Q^{-1}A$.
\end{cor}

This will follow from the following lemma:

\begin{lem}[7.1.9.1]
\label{1.7.1.9.1}
For an element $f\in A$ to be such that $D(f)$ is dense in $X$, it is necessary and sufficient that $f\in Q$; every dense open subset of $X$ contains an open subset of the form $D(f)$, where $f\in Q$.
\end{lem}

\begin{proof}
\label{proof-1.7.1.9}
Indeed, suppose that the above lemma is true; identifying the ring of sections $\Gamma(D(f),\OO_X)$ with $A_f$ (\hyperref[1.1.3.6]{1.3.6} and \hyperref[1.1.3.7]{1.3.7}), it follows from the fact that the $D(f)$ with $f\in Q$ form a \unsure{cofinal} set in the ordered (under $\supset$) set of dense open subsets of $X$, and from def.~\hyperref[1.7.1.1]{(7.1.1)} that the ring of rational functions on $X$ can be identified with the inductive limit of the $A_f$ for $f\in Q$ (for the preorder relation ``$g$ is a multiple of $f$''), i.e. $Q^{-1}A$ \hyperref[0.1.4.5]{(\textbf{0},~1.4.5)}.
\end{proof}

\begin{proof}
\label{proof-1.7.1.9.1}
To show \hyperref[1.7.1.9.1]{(7.1.9.1)}, we again denote by $X_i$ ($1\leqslant i\leqslant n$) the irreducible components of $X$; if $D(f)$ is dense in $X$ then $D(f)\cap X_i\neq\emp$ for $1\leqslant i\leqslant n$, and vice-versa; but this means that $f\not\in\mathfrak{p}_i$ for $1\leqslant i\leqslant n$, where we set $\mathfrak{p}_i=\mathfrak{j}(X_i)$, and since the $\mathfrak{p}_i$ are the minimal prime ideals of $A$ \hyperref[1.1.1.14]{(1.1.14)}, the conditions $f\not\in\mathfrak{p}_i$ ($1\leqslant i\leqslant n$) are equivalent to $f\in Q$, whence the first claim of the lemma.
For the other part, if $U$ is a dense open subset of $X$, the complement of $U$ is a set of the form $V(\mathfrak{a})$, where $\mathfrak{a}$ is an ideal which is not contained in any of the $\mathfrak{p}_i$; it is thus not contained in their union (\cite[p.~13]{I-10}), and there thus exists some $f\in\mathfrak{a}$ belonging to $Q$; whence $D(f)\subset U$, which finishes the proof.
\end{proof}

\begin{env}[7.1.10]
\label{1.7.1.10}
Suppose again that $X$ is irreducible, with generic point $x$.
Since every non-empty open subset $U$ of $X$ containing $x$, and thus also containing every $z\in X$ such that $x\in\overline{\{z\}}$, every morphism $U\to Y$ can be composed with the canonical morphism $\Spec(\OO_x)\to X$ \hyperref[1.2.4.1]{(2.4.1)}; and two morphisms into $Y$ from two non-empty open subsets of $X$, which agree on a non-empty open subset of $X$, give, by composition, the same morphism $\Spec(\OO_x)\to Y$.
In other words, to every rational map from $X$ to $Y$ there is a corresponding well-defined morphism $\Spec(\OO_x)\to Y$.
\end{env}

\begin{prop}[7.1.11]
\label{1.7.1.11}
Let $X$ and $Y$ be two $S$-preschemes; suppose that $X$ is irreducible with generic point $x$, and $Y$ of finite type over $S$.
Two rational $S$-maps from $X$ to $Y$, corresponding to the same $S$-morphism $\Spec(\OO_x)\to Y$, are identical.
If we further suppose $S$ locally Noetherian then every $S$-morphism from $\Spec(\OO_x)$ to $Y$ corresponds to exactly one rational $S$-map from $X$ to $Y$.
\end{prop}

\begin{proof}
\label{proof-1.7.1.11}
Taking into account that every non-empty subset of $X$ is dense in $X$, this follows from \hyperref[1.6.5.1]{(6.5.1)}.
\end{proof}

\begin{cor}[7.1.12]
\label{1.7.1.12}
Suppose that $S$ is locally Noetherian, and that the other hypotheses of \hyperref[1.7.1.11]{(7.1.11)} are satisfied.
The rational $S$-maps from $X$ to $Y$ are then identified with points of the $S$-prescheme $Y$, with values in the $S$-prescheme $\Spec(\OO_x)$.
\end{cor}

\begin{proof}
\label{proof-1.7.1.12}
This is nothing but \hyperref[1.7.1.11]{(7.1.11)}, with the terminology introduced in \hyperref[1.3.4.1]{(3.4.1)}.
\end{proof}

\begin{cor}[7.1.13]
\label{1.7.1.13}
Suppose that the conditions of \hyperref[1.7.1.12]{(7.1.12)} are satisfied.
Let $s$ be the image of $x$ in $S$.
The information of a rational $S$-map from $X$ to $Y$ is equivalent to the information of a point $y$ of $Y$ over $s$, and a local $\OO_s$-homomorphism $\OO_y\to\OO_x=R(X)$.
\end{cor}

\begin{proof}
\label{proof-1.7.1.13}
This follows from \hyperref[1.7.1.11]{(7.1.11)} and \hyperref[1.2.4.4]{(2.4.4)}.
\end{proof}

In particular:
\begin{cor}[7.1.14]
\label{1.7.1.14}
Under the conditions of \hyperref[1.7.1.12]{(7.1.12)}, rational $S$-maps from $X$ to $Y$ depend only (for any given $Y$) on the $S$-prescheme $\Spec(\OO_x)$ and, in particular, remain the same whenever $X$ is replaced by $\Spec(\OO_z)$, for any $z\in X$.
\end{cor}

\begin{proof}
\label{proof-1.7.1.14}
In fact, since $z\in\overline{\{x\}}$, $x$ is the generic point of $Z=\Spec(\OO_z)$, and $\OO_{X,x}=\OO_{Z,z}$.
\end{proof}

When $X$ is integral, $R(X)=\OO_x=\kres(x)$ is a field \hyperref[1.7.1.5]{(7.1.5)}; the preceding corollaries then specialise to the following:

\begin{cor}[7.1.15]
\label{1.7.1.15}
Suppose that the conditions of \hyperref[1.7.1.12]{(7.1.12)} are satisfied, and furthermore that $X$ is integral.
Les $s$ be the image of $x$ in $S$.
Then rational $S$-maps from $X$ to $Y$ can be identified with the geometric points of $Y\otimes_S\kres(s)$ with values in the extension $R(X)$ of $\kres(s)$, or, in other words, every such map is equivalent to the data of a point $y\in Y$ above $s$ and a $\kres(s)$-monomorphism from $\kres(y)$ to $\kres(x)=R(X)$.
\end{cor}

\begin{proof}
\label{proof-1.7.1.15}
The points of $Y$ above $s$ are identified with the points of $Y\otimes_S\kres(s)$ \hyperref[1.3.6.3]{(3.6.3)} and the local $\OO_s$-homomorphisms $\OO_y\to R(X)$ with the $\kres(s)$-monomorphisms $\kres(y)\to R(X)$.
\end{proof}

More precisely:
\begin{cor}[7.1.16]
\label{1.7.1.16}
Let $k$ be a field, and $X$ and $Y$ two algebraic preschemes \hyperref[1.6.4.1]{(6.4.1)} over $k$; suppose further that $X$ is integral.
Then the rational $k$-maps from $X$ to $Y$ can be identified with the geometric points of $Y$ with values in the extension $R(X)$ of $k$ \hyperref[1.3.4.4]{(3.4.4)}.
\end{cor}

\subsection{Domain of definition of a rational map}
\label{subsection-domain-of-definition-of-a-rational-map}

\begin{env}[7.2.1]
\label{1.7.2.1}
Let $X$ and $Y$ be two preschemes, and $f$ rational map from $X$ to $Y$.
We say that $f$ is \emph{defined at a point $x\in X$} if there exists an dense open subset $U$ of $X$ that contains $x$, and a morphism $U\to Y$ belonging to the equivalence class of $f$.
The set of points $x\in X$ where $f$ is defined is called the \emph{domain of definition} of $f$; it is clear that it is an open dense subset of $X$.
\end{env}

\begin{prop}[7.2.2]
\label{1.7.2.2}
Let\oldpage[0\textsubscript{I}]{159} $X$ and $Y$ be two $S$-preschemes, such that $X$ is reduced and $Y$ separated over $S$.
Let $f$ be a rational $S$-map from $X$ to $Y$, with domain of definition $U_0$.
Then there exists exactly one $S$-morphism $U_0\to Y$ belonging to the class of $f$.
\end{prop}

\begin{proof}
\label{proof-1.7.2.2}
Since, for every morphism $U\to Y$ belonging to the class of $f$, we necessarily have $U\subset U_0$, it is clear that the proposition will be a consequence of the following:
\end{proof}

\begin{lem}[7.2.2.1]
\label{1.7.2.2.1}
Under the hypotheses of \hyperref[1.7.2.2]{(7.2.2)}, let $U_1$ and $U_2$ be two dense open subsets of $X$, and $f_i\colon U_i\to Y$ ($i=1,2$) two $S$-morphisms such that there exists an open subset $V\subset U_1\cap U_2$, dense in $X$, and on which $f_1$ and $f_2$ agree.
Then $f_1$ and $f_2$ agree on $U_1\cap U_2$.
\end{lem}

\begin{proof}
\label{proof-1.7.2.2.1}
We can clearly restrict to the case where $X=U_1=U_2$.
Since $X$ (and thus $V$) is reduced, $X$ is the smallest closed subprescheme of $X$ containing $V$ \hyperref[1.5.2.2]{(5.2.2)}.
Let $g=(f_1,f_2)_S\colon X\to Y\times_S Y$; since, by hypothesis, the diagonal $T=\Delta_Y(Y)$ is a closed subprescheme of $Y\times_S Y$, $Z=g^{-1}(T)$ is a closed subprescheme of $X$ \hyperref[1.4.4.1]{(4.4.1)}.
If $h\colon V\to Y$ is the common restriction of $f_1$ and $f_2$ to $V$, then the restriction of $g$ to $V$ is $g'=(h,h)_S$, which factors as $g'=\Delta_Y\circ h$; since $\Delta_Y^{-1}(T)=Y$, we have that $g'^{-1}(T)=V$, and so $Z$ is a closed subprescheme of $X$ inducing $V$, thus containing $V$, which leads to $Z=X$.
From the equation $g^{-1}(T)=X$, we deduce \hyperref[1.4.4.1]{(4.4.1)} that $g$ factors as $\Delta_Y\circ f$, where $f$ is a morphism $X\to Y$, which implies, by the definition of the diagonal morphism, that $f_1=f_2=f$.
\end{proof}

It is clear that the morphism $U_0\to Y$ defined in \hyperref[1.7.2.2]{(7.2.2)} is the unique morphism of the class $f$ that \emph{cannot be extended} to a morphism from an open subset of $X$ that strictly contains $U_0$.
\emph{Under the hypotheses of \hyperref[1.7.2.2]{(7.2.2)}}, we can thus \emph{identify} the rational maps from $X$ to $Y$ with the \emph{non-extendible} (to strictly larger open subsets) morphisms from dense open subsets of $X$ to $Y$.
With this identification, prop.~\hyperref[1.7.2.2]{(7.2.2)} leads to:

\begin{cor}[7.2.3]
\label{1.7.2.3}
With the hypotheses from \hyperref[1.7.2.2]{(7.2.2)} on $X$ and $Y$, let $U$ be a dense open subset of $X$.
Then there exists a canonical bijective correspondence between $S$-morphisms from $U$ to $Y$ and rational $S$-maps from $X$ to $Y$ that are defined at all points of $U$.
\end{cor}

\begin{proof}
\label{proof-1.7.2.3}
By \hyperref[1.7.2.2]{(7.2.2)}, for every $S$-morphism $f$ from $U$ to $Y$, there indeed exists exactly one rational $S$-map $\overline{f}$ from $X$ to $Y$ which extends $f$.
\end{proof}

\begin{cor}[7.2.4]
\label{1.7.2.4}
Let $S$ be a scheme, $X$ a reduced $S$-prescheme, $Y$ an $S$-scheme, and $f\colon U\to Y$ an $S$-morphism from a dense open subset $U$ of $X$ to $Y$.
If $\overline{f}$ is the rational $\bb{Z}$-map from $X$ to $Y$ that extends $f$, then $\overline{f}$ is an $S$-morphism (and thus the rational $S$-map from $X$ to $Y$ that extends $f$).
\end{cor}

\begin{proof}
\label{proof-1.7.2.4}
Indeed, if $\vphi\colon X\to S$ and $\psi\colon Y\to S$ are the structure morphisms, $U_0$ the domain of definition of $\overline{f}$, and $j$ the injection $U_0\to X$, it suffices to show that $\psi\circ\overline{f}=\vphi\circ j$, which follows from \hyperref[1.7.2.2.1]{(7.2.2.1)}, since $f$ is an $S$-morphism.
\end{proof}

\begin{cor}[7.2.5]
\label{1.7.2.5}
Let $X$ and $Y$ be two $S$-preschemes; suppose that $X$ is reduced, and that $X$ and $Y$ are separated over $S$.
Let $p\colon Y\to X$ be an $S$-morphism (making $Y$ an $X$-prescheme), $U$ a dense open subset of $X$, and $f$ a $U$-section of $Y$; then the rational map $\overline{f}$ from $X$ to $Y$ extending $f$ is a rational $X$-section of $Y$.
\end{cor}

\begin{proof}
\label{proof-1.7.2.5}
We\oldpage[0\textsubscript{I}]{160} have to show that $p\circ\overline{f}$ is the identity on the domain of definition of $\overline{f}$; since $X$ is separated over $S$, this again follows from \hyperref[1.7.2.2.1]{(7.2.2.1)}.
\end{proof}

\begin{cor}[7.2.6]
\label{1.7.2.6}
Let $X$ be a reduced prescheme, and $U$ a dense open subset of $X$.
Then there is a canonical bijective correspondence between sections of $\OO_X$ over $U$ and rational functions $s$ on $X$ defined at every point of $U$.
\end{cor}

\begin{proof}
\label{proof-1.7.2.6}
Taking \hyperref[1.7.2.3]{(7.2.3)}, \hyperref[1.7.1.2]{(7.1.2)}, and \hyperref[1.7.1.3]{(7.1.3)} into account, it suffices to note that the $X$-prescheme $X\otimes_{\bb{Z}}\bb{Z}[T]$ is separated over $X$ \hyperref[1.5.5.1]{(5.5.1,~(iv))}.
\end{proof}

\begin{cor}[7.2.7]
\label{1.7.2.7}
Let $Y$ be a reduced prescheme, $f\colon X\to Y$ a separated morphism, $U$ a dense open subset of $Y$, $g\colon U\to f^{-1}(U)$ a $U$-section of $f^{-1}(U)$, and $Z$ the reduced subprescheme of $X$ that has $\overline{g(U)}$ as its underlying space \hyperref[1.5.2.1]{(5.2.1)}.
For $g$ to be the restriction of a $Y$-section of $X$ \emph{(in other words \hyperref[1.7.2.5]{(7.2.5)}, for the rational map from $Y$ to $X$ extending $g$ to be defined everywhere)}, it is necessary and sufficient that the restriction of $f$ to $Z$ be an isomorphism from $Z$ to $Y$.
\end{cor}

\begin{proof}
\label{proof-1.7.2.7}
The restriction of $f$ to $f^{-1}(U)$ is a separated morphism \hyperref[1.5.5.1]{(5.5.1,~(i))}, so $g$ is a closed immersion \hyperref[1.5.4.6]{(5.4.6)}, and so $g(U)=Z\cap f^{-1}(U)$, and the subprescheme induced by $Z$ on the open subset $g(U)$ of $Z$ is identical to the closed subprescheme of $f^{-1}(U)$ associated to $g$ \hyperref[1.5.2.1]{(5.2.1)}.
It is clear then that the stated condition is sufficient, because if it is satisfied, and if $f_Z\colon Z\to Y$ is the restriction of $f$ to $Z$, and $\overline{g}\colon Y\to Z$ is the inverse isomorphism, then $\overline{g}$ extends $g$.
Conversely, if $g$ is the restriction to $U$ of a $Y$-section $h$ of $X$, then $h$ is a closed immersion \hyperref[1.5.4.6]{(5.4.6)}, and so $h(Y)$ is closed, and, since it is contained in $Z$, is equal to $Z$, and it follows from \hyperref[1.5.2.1]{(5.2.1)} that $h$ is necessarily an isomorphism from $Y$ to the closed subprescheme $Z$ of $X$.
\end{proof}

\begin{env}[7.2.8]
\label{1.7.2.8}
\end{env}
