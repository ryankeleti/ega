\section{Rational maps}
\label{section-rational-maps}

\subsection{Rational maps and rational functions}
\label{subsection-rational-maps-and-rational-functions}

\begin{env}[7.1.1]
\label{1.7.1.1}
Let $X$ and $Y$ be two preschemes, $U$ and $V$ two dense open subsets of $X$, and $f$ (resp. $g$) a morphism from $U$ (resp. $V$) to $Y$; we say that $f$ and $g$ are \emph{equivalent} if the agree on a dense open subset of $U\cap V$.
Since a finite intersection of dense open subsets of $X$ is a dense open subset of $X$, it is clear that this relation is an \emph{equivalence relation}.
\end{env}

\begin{defn}[7.1.2]
\label{1.7.1.2}
Given two preschemes $X$ and $Y$, we define a rational map from $X$ to $Y$ to be an equivalence class of morphisms from a dense open subset of $X$ to a dense open subset of $Y$, under the equivalence relation defined in \hyperref[1.7.1.1]{(7.1.1)}.
If $X$ and $Y$ are $S$-preschemes, we say that such a class is a rational $S$-map if there exists a representative of the class that is an $S$-morphism.
We define a rational $S$-section of $X$ to be any rational $S$-map from $S$ to $X$.
We define a rational function on a prescheme $X$ to be any rational $X$-section on the $X$-prescheme $X\otimes_{\bb{Z}}\bb{Z}[T]$ (where $T$ is an indeterminate).
\end{defn}

By an abuse of language, whenever we are discussing only $S$-preschemes, we will say ``rational map'' instead of ``rational $S$-map'' if no confusion may arise.

Let $f$ be a rational map from $X$ to $Y$, and $U$ an open subset of $X$; if $f_1$ and $f_2$ are two morphisms belonging to the class of $f$, defined (respectively) on dense open subsets $V$ and $W$ of $X$, then the restrictions $f_1|(U\cap V)$ and $f_2|(U\cap W)$ agree on $U\cap V\cap W$, which is dense in $U$; the class $f$ of morphisms thus defines a rational map from $U$ to $Y$, called the \emph{restriction of $f$ to $U$}, and denoted by $f|U$.

If, to every $S$-morphism $f\colon X\to Y$, we take the corresponding rational $S$-map to which $f$ belongs, we obtain a canonical map from $\Hom_S(X,Y)$ to the set of rational $S$-maps from $X$ to $Y$.
We denote by $\Gamma_\mathrm{rat}(X/Y)$ the set of rational $Y$-sections on $X$, and we thus have a canonical map $\Gamma(X/Y)\to\Gamma_\mathrm{rat}(X/Y)$.
It is also clear that, if $X$ and $Y$ are $S$-preschemes, then the set of rational $S$-maps from $X$ to $Y$ is canonically identified with $\Gamma_\mathrm{rat}((X\times_S Y)/X)$ \hyperref[env-1.3.3.14]{(3.3.14)}.

\begin{env}[7.1.3]
\label{1.7.1.3}
It quickly follows from \hyperref[1.7.1.2]{(7.1.2)} and \hyperref[1.3.3.14]{(3.3.14)} that the rational functions on $X$ can be identified with the \emph{equivalence classes of sections of the structure sheaf $\OO_X$} over dense open subsets of $X$, where two such sections are equivalent if they agree on some dense open subset of $X$ contained inside the intersection of thee subsets on which they are defined.
In particular, it follows that the rational functions on $X$ form a \emph{ring} $R(X)$.
\end{env}
