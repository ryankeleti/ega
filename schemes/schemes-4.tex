\section{Subpreschemes and immersion morphisms}
\label{section-subpreschemes-and-immersion-morphisms}

\subsection{Subpreschemes}
\label{subsection-subpreschemes}

\begin{env}[4.1.1]
\label{env-1.4.1.1}
As the notion of a quasi-coherent sheaf \hyperref[env-0.5.1.3]{(\textbf{0},~5.1.3)} is local,
a quasi-coherent $\OO_X$-module $\sh{F}$ over a prescheme $X$ can be defined by the condition
that, for each affine open $V$ of $X$, $\sh{F}|V$ is isomorphic to the sheaf associated to a
$\Gamma(V,\OO_X)$-module \hyperref[thm-1.1.4.1]{(1.4.1)}. It is clear that over a prescheme
$X$, the structure sheaf $\OO_X$ is quasi-coherent and that the kernels, cokernels, and
images of homomorphisms of quasi-coherent $\OO_X$-modules, as well as inductive limits and
direct sums of quasi-coherent $\OO_X$-modules, are also quasi-coherent
(Theorem \hyperref[thm-1.1.3.7]{(1.3.7)} and Corollary \hyperref[cor-1.1.3.9]{(1.3.9)}).
\end{env}

\begin{prop}[4.1.2]
\label{prop-1.4.1.2}
Let $X$ be a prescheme, $\sh{J}$ a quasi-coherent sheaf of ideals of $\OO_X$. The support
$Y$ of the sheaf $\OO_X/\sh{J}$ is then closed, and if we denote by $\OO_Y$ the restriction
of $\OO_X/\sh{J}$ to $Y$, then $(Y,\OO_Y)$ is a prescheme.
\end{prop}

\begin{proof}
\label{proof-prop-1.4.1.2}
\oldpage[I]{120}
It evidently suffices \hyperref[prop-1.2.1.3]{(2.1.3)} to consider the case where $X$ is an
affine scheme, and to show that in this case $Y$ is closed in $X$ and is an {\em affine
scheme}. Indeeed, if $X=\Spec(A)$, then we have $\OO_X=\widetilde{A}$ and
$\sh{J}=\widetilde{\mathfrak{J}}$, where $\mathfrak{J}$ is an ideal of $A$
\hyperref[thm-1.1.4.1]{(1.4.1)}; $Y$ is then equal to the closed subset $V(\mathfrak{J})$ of
$X$ and identifies with the prime spectrum of the ring $B=A/\mathfrak{J}$
\hyperref[prop-1.1.1.11]{(1.1.1.11)}; in addition, if $\vphi$ is the canonical
homomorphism $A\to B=A/\mathfrak{J}$, then the direct image ${}^a\vphi_*(\widetilde{B})$
canonically identifies with the sheaf $\widetilde{A}/\widetilde{\mathfrak{J}}=\OO_X/\sh{J}$
(Proposition \hyperref[prop-1.1.6.3]{(1.6.3)} and Corollary \hyperref[cor-1.1.3.9]{(1.3.9)}),
which finishes the proof.
\end{proof}

We say that $(Y,\OO_Y)$ is the {\em subprescheme} of $(X,\OO_X)$ {\em defined by the
sheaf of ideals $\sh{J}$}; this is a particular case of the more general notion of
{\em subprescheme}:

\begin{defn}[4.1.3]
\label{defn-1.4.1.3}
We say that a ringed space $(Y,\OO_Y)$ is a subprescheme of a prescheme $(X,\OO_X)$ if:
\begin{enumerate}
  \item[1st] $Y$ is a localy closed subspace of $X$;
  \item[2nd] if $U$ denotes the largest open subset of $X$ containing $Y$ such that
    $Y$ is closed in $U$ ({\em equivalently}, the complement in $X$ of the
    boundary of $Y$ with respect to $\overline{Y}$), then $(Y,\OO_Y)$ is
    a subprescheme of $(U,\OO_X|U)$ defined by a quasi-coherent sheaf of ideals of $\OO_X|U$.
\end{enumerate}
We say that the subprescheme $(Y,\OO_Y)$ of $(X,\OO_X)$ is closed if $Y$ is closed in $X$
(in which case $U=X$).
\end{defn}

It follows immediately from this definition and Proposition \hyperref[prop-1.4.1.2]{(4.1.2)}
that the closed subpreschemes of $X$ are in canonical {\em bijective correspondence} with the
quasi-coherent sheaf of ideals $\sh{J}$ of $\OO_X$, since if two such sheaves
$\sh{J}$, $\sh{J}'$ have the same (closed) support $Y$ and if the restrictions of
$\OO_X/\sh{J}$ and $\OO_X/\sh{J}'$ to $Y$ are \unsure{identical/the identity}, then we have $\sh{J}'=\sh{J}$.

\begin{env}[4.1.4]
\label{env-1.4.1.4}
Let $(Y,\OO_Y)$ be a subprescheme of $X$, $U$ the largest open subset of $X$ containing $Y$
and in which $Y$ is closed, $V$ an open subset of $X$ contained in $U$; then $V\cap Y$ is
closed in $V$. In addition, if $Y$ is defined by the quasi-coherent sheaf of ideals $\sh{J}$
of $\OO_X|U$, then $\sh{J}|V$ is a quasi-coherent sheaf of ideals of $\OO_X|V$, and it is
immediate that the prescheme induced by $Y$ on $Y\cap V$ is the closed subprescheme of $V$
defined by the sheaf of ideals $\sh{J}|V$. Conversely:
\end{env}

\begin{prop}[4.1.5]
\label{prop-1.4.1.5}
Let $(Y,\OO_Y)$ be a ringed space such that $Y$ is a subspace of $X$ and there exists a
cover $(V_\alpha)$ of $Y$ by open subsets of $X$ such that for each $\alpha$,
$Y\cap V_\alpha$ is closed in $V_\alpha$ and the ringed space
$(Y\cap V_\alpha,\OO_Y|(Y\cap V_\alpha))$ is a closed subprescheme of the prescheme induced
on $V_\alpha$ by $X$. Then $(Y,\OO_Y)$ is a subprescheme of $X$.
\end{prop}

\begin{proof}
\label{proof-prop-1.4.1.5}
The hypotheses imply that $Y$ is locally closed in $X$ and that the largest open $U$
containing $Y$ in which is closed contains all the $V_\alpha$; we can thus reduce to the case
where $U=X$ and $Y$ is closed in $X$. We then define a quasi-coherent sheaf of ideals
$\sh{J}$ of $\OO_X$ by taking $\sh{J}|V_\alpha$ to be the sheaf of ideals of $\OO_X|V_\alpha$
which define the closed subprescheme $(Y\cap V_\alpha,\OO_Y|(Y\cap V_\alpha))$, and for each
open subset $W$ of $X$ not intersecting $Y$, $\sh{J}|W=\OO_X|W$. We check immediately
according to Definition \hyperref[defn-1.4.1.3]{(4.1.3)} and \hyperref[env-1.4.1.4]{(4.1.4)}
that there exists a unique sheaf of ideals $\sh{J}$ satisfying these conditions and that
define the closed subprescheme $(Y,\OO_Y)$.
\end{proof}

In particular, the {\em induced} prescheme by $X$ on an {\em open subset} of $X$ is a
{\em subprescheme} of $X$.

\begin{prop}[4.1.6]
\label{prop-1.4.1.6}
A subprescheme (resp. a closed subprescheme) of a subprescheme
\oldpage[I]{121}
(resp. closed subprescheme) of $X$ canonically identifies with a subprescheme
(resp. closed subprescheme) of $X$.
\end{prop}

\begin{proof}
\label{proof-prop-1.4.1.6}
Since a locally closed subset of a locally closed subspace of $X$ is a locally closed
subspace of $X$, it is clear \hyperref[prop-1.4.1.5]{(4.1.5)} that the question is local
and that we can thus suppose that $X$ is affine; the proposition then follows from the
canonical identification of $A/\mathfrak{J}'$ and
$(A/\mathfrak{J})/(\mathfrak{J}'/\mathfrak{J})$ when $\mathfrak{J}$, $\mathfrak{J}'$ are
two ideals of a ring $A$ such that $\mathfrak{J}\subset\mathfrak{J}'$.
\end{proof}

We will always make the previous identification.
\begin{env}[4.1.7]
\label{env-1.4.1.7}
Let $Y$ be a subprescheme of a prescheme $X$, and denote by $\psi$ the canonical injection
$Y\to X$ of the {\em underlying subspaces}; we know that the inverse image $\psi^*(\OO_X)$ is
the restriction $\OO_X|Y$ \hyperref[env-0.3.7.1]{(\textbf{0},~3.1.7)}. If, for each $y\in Y$,
we denote by $\omega_y$ the canonical homomorphism $(\OO_X)_y\to(\OO_Y)_y$, then these
homomorphisms are the restrictions to stalks of a {\em surjective} homomorphism $\omega$
of sheaves of rings $\OO_X|Y\to\OO_Y$: indeed, is suffices to check locally on $Y$, that is
to say, we can suppose that $X$ is affine and that the subprescheme $Y$ is closed; if in this
case $\sh{J}$ is the sheaf of ideals in $\OO_X$ which defines $Y$, then the $\omega_y$ are
none other than the restriction to stalks of the homomorphism $\OO_X|Y\to(\OO_X/\sh{J})|Y$.
We have thus defined a
{\em monomorphism of ringed spaces} \hyperref[env-0.4.1.1]{(\textbf{0},~4.1.1)}
$j=(\psi,\omega^\flat)$ which is evidently a morphism $Y\to X$ of
preschemes \hyperref[defn-1.2.2.1]{(2.2.1)}, and we call this the
{\em canonical injection morphism}.

If $f:X\to Z$ is a morphism, we then say that the composite morphism
$Y\xrightarrow{j}X\xrightarrow{f}Z$ is the {\em restriction} of $f$ to the subprescheme $Y$.
\end{env}

