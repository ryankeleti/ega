\section{Subpreschemes and immersion morphisms}
\label{section-subpreschemes-and-immersion-morphisms}

\subsection{Subpreschemes}
\label{subsection-subpreschemes}

\begin{env}[4.1.1]
\label{1.4.1.1}
As the notion of a quasi-coherent sheaf \sref[0]{0.5.1.3} is local, a quasi-coherent $\OO_X$-module $\sh{F}$ over a prescheme $X$ can be defined by the condition that, for each affine open $V$ of $X$, $\sh{F}|V$ is isomorphic to the sheaf associated to a $\Gamma(V,\OO_X)$-module \sref{1.1.4.1}.
It is clear that over a prescheme $X$, the structure sheaf $\OO_X$ is quasi-coherent and that the kernels, cokernels, and images of homomorphisms of quasi-coherent $\OO_X$-modules, as well as inductive limits and direct sums of quasi-coherent $\OO_X$-modules, are also quasi-coherent (Theorem \sref{1.1.3.7} and Corollary \sref{1.1.3.9}).
\end{env}

\begin{prop}[4.1.2]
\label{1.4.1.2}
Let $X$ be a prescheme, $\sh{I}$ a quasi-coherent sheaf of ideals of $\OO_X$.
The support $Y$ of the sheaf $\OO_X/\sh{I}$ is then closed, and if we denote by $\OO_Y$ the restriction of $\OO_X/\sh{I}$ to $Y$, then $(Y,\OO_Y)$ is a prescheme.
\end{prop}

\begin{proof}
\label{proof-1.4.1.2}
\oldpage[I]{120}
It evidently suffices \sref{1.2.1.3} to consider the case where $X$ is an affine scheme, and to show that in this case $Y$ is closed in $X$ and is an \emph{affine scheme}.
Indeed, if $X=\Spec(A)$, then we have $\OO_X=\widetilde{A}$ and $\sh{I}=\widetilde{\fk{I}}$, where $\fk{I}$ is an ideal of $A$ \sref{1.1.4.1}; $Y$ is then equal to the closed subset $V(\fk{I})$ of $X$ and identifies with the prime spectrum of the ring $B=A/\fk{I}$ \sref{1.1.1.11};
in addition, if $\vphi$ is the canonical homomorphism $A\to B=A/\fk{I}$, then the direct image ${}^a\vphi_*(\widetilde{B})$ canonically identifies with the sheaf $\widetilde{A}/\widetilde{\fk{I}}=\OO_X/\sh{I}$ (Proposition \sref{1.1.6.3} and Corollary \sref{1.1.3.9}), which finishes the proof.
\end{proof}

We say that $(Y,\OO_Y)$ is the \emph{subprescheme} of $(X,\OO_X)$ \emph{defined by the sheaf of ideals $\sh{I}$}; this is a particular case of the more general notion of a \emph{subprescheme}:

\begin{defn}[4.1.3]
\label{1.4.1.3}
We say that a ringed space $(Y,\OO_Y)$ is a subprescheme of a prescheme $(X,\OO_X)$ if:
\begin{enumerate}
  \item[1st] $Y$ is a locally closed subspace of $X$;
  \item[2nd] if $U$ denotes the largest open subset of $X$ containing $Y$ such that     $Y$ is closed in $U$ (\emph{equivalently}, the complement in $X$ of the boundary of $Y$ with respect to $\overline{Y}$), then $(Y,\OO_Y)$ is a subprescheme of $(U,\OO_X|U)$ defined by a quasi-coherent sheaf of ideals of $\OO_X|U$.
\end{enumerate}
We say that the subprescheme $(Y,\OO_Y)$ of $(X,\OO_X)$ is closed if $Y$ is closed in $X$ (in which case $U=X$).
\end{defn}

It follows immediately from this definition and Proposition \sref{1.4.1.2} that the closed subpreschemes of $X$ are in canonical \emph{bijective correspondence} with the quasi-coherent sheaf of ideals $\sh{J}$ of $\OO_X$, since if two such sheaves $\sh{J}$, $\sh{J}'$ have the same (closed) support $Y$ and if the restrictions of $\OO_X/\sh{J}$ and $\OO_X/\sh{J}'$ to $Y$ are \unsure{identical/the identity}, then we have $\sh{J}'=\sh{J}$.

\begin{env}[4.1.4]
\label{1.4.1.4}
Let $(Y,\OO_Y)$ be a subprescheme of $X$, $U$ the largest open subset of $X$ containing $Y$ and in which $Y$ is closed, $V$ an open subset of $X$ contained in $U$; then $V\cap Y$ is closed in $V$. In addition, if $Y$ is defined by the quasi-coherent sheaf of ideals $\sh{J}$ of $\OO_X|U$, then $\sh{J}|V$ is a quasi-coherent sheaf of ideals of $\OO_X|V$, and it is immediate that the prescheme induced by $Y$ on $Y\cap V$ is the closed subprescheme of $V$ defined by the sheaf of ideals $\sh{J}|V$.
Conversely:
\end{env}

\begin{prop}[4.1.5]
\label{1.4.1.5}
Let $(Y,\OO_Y)$ be a ringed space such that $Y$ is a subspace of $X$ and that there exists a cover $(V_\alpha)$ of $Y$ by open subsets of $X$ such that for each $\alpha$, $Y\cap V_\alpha$ is closed in $V_\alpha$ and the ringed space $(Y\cap V_\alpha,\OO_Y|(Y\cap V_\alpha))$ is a closed subprescheme of the prescheme induced on $V_\alpha$ by $X$.
Then $(Y,\OO_Y)$ is a subprescheme of $X$.
\end{prop}

\begin{proof}
\label{proof-1.4.1.5}
The hypotheses imply that $Y$ is locally closed in $X$ and that the largest open $U$ containing $Y$ in which $Y$ is closed contains all the $V_\alpha$; we can thus reduce to the case where $U=X$ and $Y$ is closed in $X$.
We then define a quasi-coherent sheaf of ideals $\sh{J}$ of $\OO_X$ by taking $\sh{J}|V_\alpha$ to be the sheaf of ideals of $\OO_X|V_\alpha$ which define the closed subprescheme $(Y\cap V_\alpha,\OO_Y|(Y\cap V_\alpha))$, and for each open subset $W$ of $X$ not intersecting $Y$, $\sh{J}|W=\OO_X|W$.
We check immediately according to Definition \sref{1.4.1.3} and \sref{1.4.1.4} that there exists a unique sheaf of ideals $\sh{J}$ satisfying these conditions and that define the closed subprescheme $(Y,\OO_Y)$.
\end{proof}

In particular, the \emph{induced} prescheme by $X$ on an \emph{open subset} of $X$ is a \emph{subprescheme} of $X$.

\begin{prop}[4.1.6]
\label{1.4.1.6}
A subprescheme (resp. a closed subprescheme) of a subprescheme
\oldpage[I]{121}
(resp. closed subprescheme) of $X$ canonically identifies with a subprescheme (resp. closed subprescheme) of $X$.
\end{prop}

\begin{proof}
\label{proof-1.4.1.6}
Since a locally closed subset of a locally closed subspace of $X$ is a locally closed subspace of $X$, it is clear \sref{1.4.1.5} that the question is local and that we can thus suppose that $X$ is affine; the proposition then follows from the canonical identification of $A/\fk{J}'$ and $(A/\fk{J})/(\fk{J}'/\fk{J})$ when $\fk{J}$, $\fk{J}'$ are two ideals of a ring $A$ such that $\fk{J}\subset\fk{J}'$.
\end{proof}

We will always make the previous identification.
\begin{env}[4.1.7]
\label{1.4.1.7}
Let $Y$ be a subprescheme of a prescheme $X$, and denote by $\psi$ the canonical injection $Y\to X$ of the \emph{underlying subspaces}; we know that the inverse image $\psi^*(\OO_X)$ is the restriction $\OO_X|Y$ \sref[0]{0.3.7.1}.
If, for each $y\in Y$, we denote by $\omega_y$ the canonical homomorphism $(\OO_X)_y\to(\OO_Y)_y$, then these homomorphisms are the restrictions to stalks of a \emph{surjective} homomorphism $\omega$ of sheaves of rings $\OO_X|Y\to\OO_Y$: indeed, it suffices to check locally on $Y$, that is to say, we can suppose that $X$ is affine and that the subprescheme $Y$ is closed; if in this case $\sh{I}$ is the sheaf of ideals in $\OO_X$ which defines $Y$, then the $\omega_y$ are none other than the restriction to stalks of the homomorphism $\OO_X|Y\to(\OO_X/\sh{I})|Y$.
We have thus defined a \emph{monomorphism of ringed spaces} \sref[0]{0.4.1.1} $j=(\psi,\omega^\flat)$ which is evidently a morphism $Y\to X$ of preschemes \sref{1.2.2.1}, and we call this the \emph{canonical injection morphism}.

If $f:X\to Z$ is a morphism, we then say that the composite morphism $Y\xrightarrow{j}X\xrightarrow{f}Z$ is the \emph{restriction} of $f$ to the subprescheme $Y$.
\end{env}

\begin{env}[4.1.8]
\label{1.4.1.8}
Conforming to the general definitions (T, I, 1.1), we will say that a morphism of preschemes $f:Z\to X$ is \unsure{\emph{majorized}} by the injection morphism $j:Y\to X$ of a subprescheme $Y$ of $X$ if $f$ factors as $Z\xrightarrow{g}Y\xrightarrow{j}X$, where $g$ is a morphism of preschemes; $g$ is necessarily \emph{unique} since $j$ is a monomorphism.
\end{env}

\begin{prop}[4.1.9]
\label{1.4.1.9}
For a morphism $f:Z\to X$ to be majorized by an injection morphism $j:Y\to X$, it is necessary and sufficient that $f(Z)\subset Y$ and that for all $z\in Z$, if we set $y=f(z)$, then the homomorphism $(\OO_X)_y\to\OO_z$ corresponding to $f$ factors as $(\OO_Z)_y\to(\OO_Y)_y\to\OO_z$ (\emph{or equivalently, that} the kernel of $(\OO_X)_y\to\OO_z$ contains the kernel of $(\OO_X)_y\to(\OO_Y)_y$).
\end{prop}

\begin{proof}
\label{proof-1.4.1.9}
The conditions are evidently necessary.
To see that they are sufficient, we can reduce to the case where $Y$ is a \emph{closed} subprescheme of $X$, by replacing if needed $X$ by an open $U$ such that $Y$ is closed in $U$ \sref{1.4.1.3}; $Y$ is then defined by a quasi-coherent sheaf of ideals $\sh{I}$ of $\OO_X$.
Set $f=(\psi,\theta)$, and let $\sh{J}$ be the sheaf of ideals of $\psi^*(\OO_X)$, kernel of $\theta^\sharp:\psi^*(\OO_X)\to\OO_Z$; considering the properties of the functor $\psi^*$ \sref[0]{0.3.7.2}, the hypotheses emply that for each $z\in Z$, we have $(\psi^*(\sh{I}))_z\subset\sh{J}_z$, and as a result $\psi^*(\sh{I})\subset\sh{J}$.
Thus $\theta^\sharp$ factors as
\[
  \psi^*(\OO_X)\longrightarrow\psi^*(\OO_X)/\psi^*(\sh{I})=\psi^*(\OO_X/\sh{I})\xrightarrow{\omega}\OO_Z,
\]
the first arrow being the canonical homomorphism.
Let $\psi'$ be the continuous map $Z\to Y$ coinciding with $\psi$; it is clear that we have ${\psi'}^*(\OO_Y)=\psi^*(\OO_X/\sh{J})$; on the other hand, $\omega$ is evidently a local homomorphism, so $g=(\psi',\omega^\flat)$ is a morphism
\oldpage[I]{122}
of preschemes $Z\to Y$ \sref{1.2.2.1}, which according to the above is such that $f=j\circ g$, hence the proposition.
\end{proof}

\begin{cor}[4.1.10]
\label{1.4.1.10}
For an injection morphism $Z\to X$ to be majorized by the injection morphism $Y\to X$, it is necessary and sufficient that $Z$ is a subprescheme of $Y$.
\end{cor}

We then write $Z\leqslant Y$, and this condition is evidently an \emph{order relation} on the set of subpreschemes of $X$.

\subsection{Immersion morphisms}
\label{subsection-immersion-morphisms}

\begin{defn}[4.2.1]
\label{1.4.2.1}
We say that a morphism $f:Y\to X$ is an immersion (resp. a closed immersion, an open immersion) if it factors as $Y\xrightarrow{g}Z\xrightarrow{j}X$, where $g$ is an isomorphism, $Z$ is a subprescheme of $X$ (resp. a closed subprescheme, a subprescheme induced by an open set) and $j$ is the injection morphism.
\end{defn}

The subprescheme $Z$ and the isomorphism $g$ are then determined in a \emph{unique} way, since if $Z'$ is a second subprescheme of $X$, $j'$ the injection $Z'\to X$, and $g'$ an isomorphism $Y\to Z'$ such that $j\circ g=j'\circ g'$, then we have $j'=j\circ g\circ{g'}^{-1}$, hence $Z'\leqslant Z$ \sref{1.4.1.10}, and we similarly show that $Z\leqslant Z'$, hence $Z'=Z$, and as $j$ is a monomorphism of preschemes, $g'=g$.

We say that $f=j\circ g$ is the \emph{canonical factorization} of the immersion $f$, and the subprescheme $Z$ and the isomorphism $g$ are those \emph{associated to $f$}.

It is clear that an immersion is a \emph{monomorphism} of preschemes \sref{1.4.1.7} and \emph{a fortiori} a \emph{radical} morphism \sref{1.3.5.4}.

\begin{prop}[4.2.2]
\label{1.4.2.2}
\medskip\noindent
\begin{enumerate}[label={\rm(\alph*)}]
  \item For a morphism $f=(\psi,\theta):Y\to X$ to be an open immersion, it is necessary and sufficient that $\psi$ is a homeomorphism between $Y$ and an open subset of $X$, and that for all $y\in Y$, the homomorphism $\theta_y^\sharp:\OO_{\psi(y)}\to\OO_y$ is bijective.
  \item For a morphism $f=(\psi,\theta):Y\to X$ to be an immersion (resp. a closed immersion), it is necessary and sufficient that $\psi$ is a homeomorphism between $Y$ and a locally closed (resp. closed) subset of $X$, and that for all $y\in Y$, the homomorphism $\theta_y^\sharp:\OO_{\psi(y)}\to\OO_y$ is surjective.
\end{enumerate}
\end{prop}

\begin{proof}
\label{proof-1.4.2.2}
\medskip\noindent
\begin{enumerate}[label=(\alph*)]
  \item The conditions are evidently necessary.
    Conversely, if they are satisfied, then it is clear that $\theta^\sharp$ is an isomorphism from $\OO_Y$ to $\psi^*(\OO_X)$, and $\psi^*(\OO_X)$ is the sheaf induced by ``transport of structure'' via $\psi^{-1}$ from $\OO_X|\psi(Y)$; hence the conclusion.
  \item The conditions are evidently necessary---we prove that they are sufficient.
    Consider first the particular case where we suppose that $X$ is an affine scheme and that $Z=\psi(Y)$ is \emph{closed} in $X$.
    We then know \sref[0]{0.3.4.6} that $\psi_*(\OO_Y)$ has $Z$ for its support and that if we denote by $\OO_Z'$ its restriction ot $Z$, then the ringed space $(Z,\OO_Z')$ is induced from $(Y,\OO_Y)$ by transport of structure via the homeomorphism $\psi$, considered as a map from $Y$ to $Z$.
    Let us show that $f_*(\OO_Y)=\psi_*(\OO_Y)$ is a \emph{quasi-coherent} $\OO_X$-module.
    Indeed, for all $x\not\in Z$, $\psi_*(\OO_Y)$ restricted to a suitable neighborhood of $x$ is zero.
    On the contrary, if $z\in Z$, then we have $x=\psi(y)$ for a well-defined  $y\in Y$; let $V$ be an open affine neighborhood of $y$ in $Y$; $\psi(V)$ is then open in $Z$, so \unsure{trace} over $Z$ of an open subset $U$ of $X$, and the restriction of $U$ to $\psi_*(\OO_Y)$ is identical to the restriction of $U$ to the direct image
\oldpage[I]{123}
    $(\psi_V)_*(\OO_Y|V)$, where $\psi_V$ is the restriction of $\psi$ to $V$.
    The restriction to $(V,\OO_Y|V)$ of the morphism $(\psi,\theta)$ is a morphism from this prescheme to $(X,\OO_X)$, and as a result is of the form $({}^a\vphi,\widetilde{\vphi})$, where $\vphi$ is the homomorphism from the ring $A=\Gamma(X,\OO_X)$ to the ring $\Gamma(V,\OO_Y)$ \sref{1.1.7.3}; we conclude that $(\psi_V)_*(\OO_Y|V)$ is a quasi-coherent $\OO_X$-module \sref{1.1.6.3}, which proves our assertion, due to the local nature of quasi-coherent sheaves.
    In addition, the hypothesis that $\psi$ is a homeomorphism implies \sref[0]{0.3.4.5} that for all $y\in Y$, $\psi_y$ is an isomorphism $(\psi_*(\OO_Y))_{\psi(y)}\to\OO_y$; as the diagram
    \[
      \xymatrix{
        \OO_{\psi(y)}\ar[r]^{\theta_{\psi(y)}}\ar[d]_{\psi_y\circ\alpha_{\psi(y)}} &
        (\psi_*(\OO_Y))_{\psi(y)}\ar[d]^{\psi_y}\\
        (\psi^*(\OO_X))_y\ar[r]^{\theta_y^\sharp} &
        \OO_y
      }
    \]
    is commutative and the vertical arrows are the isomorphisms \sref[0]{0.3.7.2}, the hypothesis that $\theta_y^\sharp$ is surjective implies that so is $\theta_{\psi(y)}$.
    As the support of $\psi_*(\OO_Y)$ is $Z=\psi(Y)$, $\theta$ is a \emph{surjective} homomorphism from $\OO_X=\widetilde{A}$ to the quasi-coherent $\OO_X$-module $f_*(\OO_Y)$.
    As a result, there exists a unique isomorphism $\omega$ from a sheaf quotient $\widetilde{A}/\widetilde{\fk{J}}$ ($\fk{J}$ an ideal of $A$) to $f_*(\OO_Y)$ which when composed with the canonical homomorphism $\widetilde{A}\to\widetilde{A}/\widetilde{\fk{J}}$ gives $\theta$ \sref{1.1.3.8}; if $\OO_Z$ denotes the restriction of $\widetilde{A}/\widetilde{\fk{J}}$ to $Z$, then $(Z,\OO_Z)$ is a subprescheme of $(X,\OO_X)$, and $f$ factors through the canonical injection of this subprescheme into $X$ and the isomorphism $(\psi_0,\omega_0)$, where $\psi_0$ is $\psi$ considered as a map from $Y$ to $Z$, and $\omega_0$ the restriction of $\omega$ to $\OO_Z$.

    We pass to the general case.
    Let $U$ be an affine open subset of $X$ such that $U\cap\psi(Y)$ is closed in $U$ and non-empty.
    By restricting $f$ to the prescheme induced by $Y$ on the open subset $\psi^{-1}(U)$, and by considering it as a morphism from this prescheme to the prescheme induced by $X$ on $U$, we reduce to the first case; the restriction of $f$ to $\psi^{-1}(U)$ is thus a closed immersion $\psi^{-1}(U)\to U$, canonically factoring as $j_U\circ g_U$, where $g_U$ is an isomorphism from the prescheme $\psi^{-1}(U)$ to a subprescheme $Z_U$ of $U$, and $j_U$ is the canonical injection $Z_U\to U$.
    Let $V$ be a second affine open subset of $X$ such that $V\subset U$; as the restriction $Z_V'$ of $Z_U$ to $V$ is a subprescheme of the prescheme $V$, the restriction of $f$ to $\psi^{-1}(V)$ factors as $j_V'\circ g_V'$, where $j_V'$ is the canonical injection $Z_V'\to V$ and $g_V'$ is an isomorphism from $\psi^{-1}(V)$ to $Z_V'$.
    By the uniqueness of the canonical factorization of an immersion \sref{1.4.2.1}, we necessarily have that $Z_V'=Z_V$ and $g_V'=g_V$.
    We conclude \sref{1.4.1.5} that there is a subprescheme $Z$ of $X$ whose underlying space is $\psi(Y)$ and whose restriction to each $U\cap\psi(Y)$ is $Z_U$; the $g_U$ are then the restrictions to $\psi^{-1}(U)$ of an isomorphism $g:Y\to Z$ such that $f=j\circ g$, where $j$ is the canonical injection $Z\to X$.
\end{enumerate}
\end{proof}

\begin{cor}[4.2.3]
\label{1.4.2.3}
Let $X$ be an affine scheme.
For a morphism $f=(\psi,\theta):Y\to X$ to be a closed immersion, it is necessary and sufficicent that $Y$ is an affine scheme and that the homomorphism $\Gamma(\psi):\Gamma(\OO_X)\to\Gamma(\OO_Y)$ is surjective.
\end{cor}

\begin{cor}[4.2.4]
\label{1.4.2.4}
\medskip\noindent
\begin{enumerate}[label={\rm(\alph*)}]
  \item Let $f$ be a morphism $Y\to X$, $(V_\lambda)$ a cover of $f(Y)$ by open subsets of $X$.
    For $f$ to be an immersion (resp. an open immersion), it is necessary and sufficient
\oldpage[I]{124}
    that its restriction to each of the induced preschemes $f^{-1}(V_\lambda)$ is an immersion (resp. an open immersion) into $V_\lambda$.
  \item Let $f$ be a morphism $Y\to X$, $(V_\lambda)$ an open cover of $X$.
    For $f$ to be a closed immersion, it is necessary and sufficient that its restriction to each of the induced preschemes $f^{-1}(V_\lambda)$ is a closed immersion into $V_\lambda$.
\end{enumerate}
\end{cor}

\begin{proof}
\label{proof-1.4.2.4}
Let $f=(\psi,\theta)$; in the case (a), $\theta_y^\sharp$ is surjective (resp. bijective) for all $y\in Y$, and in the case (b), $\theta_y^\sharp$ is surjective for all $y\in Y$; it thus suffices to check that $\psi$, in case (a), is a homeomorphism from $Y$ to a locally closed (resp. open) subset of $X$, and in case (b), a homeomorphism from $Y$ to a closed subset of $X$.
Now $\psi$ is evidently injective and sends each neighborhood of $y$ in $Y$ to a neighborhood of $\psi(y)$ is $\psi(Y)$ for all $y\in Y$, by virtue of the hypothesis; in case (a), $\psi(Y)\cap V_\lambda$ is locally closed (resp. open) in $V_\lambda$, so $\psi(Y)$ is locally closed (resp. open) in the union of the $V_\lambda$, and \emph{a fortiori} in $X$; in case (b), $\psi(Y)\cap V_\lambda$ is closed in $V_\lambda$, so $\psi(Y)$ is closed in $X$ since $X=\bigcup_\lambda V_\lambda$.
\end{proof}

\begin{prop}[4.2.5]
\label{1.4.2.5}
The composition of two immersions (resp. of two open immersions, of two closed immersions) is an immersion (resp. an open immersion, a closed immersion).
\end{prop}

\begin{proof}
\label{proof-1.4.2.5}
This follows easily from \sref{1.4.1.6}.
\end{proof}

\subsection{Products of immersions}
\label{subsection-products-of-immersions}

\begin{prop}[4.3.1]
\label{1.4.3.1}
Let $\alpha:X'\to X$, $\beta:Y'\to Y$ be two $S$-morphisms; if $\alpha$ and $\beta$ are immersions (resp. open immersions, closed immersions), then $\alpha\times_S\beta$ is an immersion (resp. an open immersion, a closed immersion).n
In addition, if $\alpha$ (resp. $\beta$) identifies $X'$ (resp. $Y'$) with a subprescheme $X''$ (resp. $Y''$) of $X$ (resp. $Y$), then $\alpha\times_S\beta$ identifies the underlying space of $X'\times_S Y'$ with the subspace $p^{-1}(X'')\cap q^{-1}(Y'')$ of the underlying space of $X\times_S Y$, where $p$ and $q$ denote the projections from $X\times_S Y$ to $X$ and $Y$ respectively.
\end{prop}

\begin{proof}
\label{proof-1.4.3.1}
According to Definition \sref{1.4.2.1}, we can restrict to the case where $X'$ and $Y'$ are subpreschemes, $\alpha$ and $\beta$ the injection morphisms. The proposition has already been proven for the subpreschemes induced by open sets \sref{1.3.2.7}; as each subprescheme is a closed subprescheme of a prescheme induced by an open set \sref{1.4.1.3}, we reduce to the case where $X'$ and $Y'$ are \emph{closed} subpreschemes.

Let us first show that we can assume that $S$ is \emph{affine}.
Indeed, let $(S_\lambda)$ be a cover of $S$ by open affine sets; if $\vphi$ and $\psi$ are the structure morphisms of $X$ and $Y$, then let $X_\lambda=\vphi^{-1}(S_\lambda)$ and $Y_\lambda=\psi^{-1}(S_\lambda)$.
The restriction $X_\lambda'$ (resp. $Y_\lambda'$) of $X'$ (resp. $Y'$) to $X_\lambda\cap X'$ (resp. $Y_\lambda\cap Y'$) is a closed subprescheme of $X_\lambda$ (resp. $Y_\lambda$), the preschemes $X_\lambda$, $Y_\lambda$, $X_\lambda'$, $Y_\lambda'$ can be considered as $S_\lambda$-preschemes and the products $X_\lambda\times_S Y_\lambda$ and $X_\lambda\times_{S_\lambda}Y_\lambda$ (resp. $X_\lambda'\times_S Y_\lambda'$ and $X_\lambda'\times_{S_\lambda}Y_\lambda'$) are identical \sref{1.3.2.5}.
If the proposition is true when $S$ is affine, then the restriction of $\alpha\times_S\beta$ to each of the $X_\lambda'\times_S Y_\lambda'$ is thus an immersion \sref{1.3.2.7}.
As the product $X_\lambda'\times_S Y_\mu'$ (resp. $X_\lambda\times_S Y_\mu$) identifies with $(X_\lambda'\cap X_\mu')\times_S(Y_\lambda'\cap Y_\mu')$ (resp. $(X_\lambda\cap X_\mu)\times_S(Y_\lambda\cap Y_\mu)$) \sref{1.3.2.6.4}, the restriction of $\alpha\times_S\beta$
\oldpage[I]{125}
to each of the $X_\lambda'\times_S Y_\mu'$ is also an immersion; the same is true for $\alpha\times_S\beta$ by \sref{1.4.2.4}.

Second, we prove that we can suppose that $X$ and $Y$ are \emph{affine}.
Indeed, let $(U_i)$ (resp. $(V_j)$) be a cover of $X$ (resp. $Y$) by open affine sets, and let $X_i'$ (resp. $Y_j'$) be the restriction of $X'$ (resp. $Y'$) to $X'\cap U_i$ (resp. $Y'\cap V_j$), which is a closed subprescheme of $U_i$ (resp. $V_j$); $U_i\times_S V_j$ indentifies with the restriction of $X\times_S Y$ to $p^{-1}(U_i)\cap q^{-1}(V_j)$ \sref{1.3.2.7}; similarly, if $p'$ and $q'$ are the projections from $X'\times_S Y'$, then $X_i'\times_S Y_j'$ identifies with the restriction of $X'\times_S Y'$ to ${p'}^{-1}(X_i')\cap{q'}^{-1}(Y_j')$.
Set $\gamma=\alpha\times_S\beta$; we have by definition $p\circ\gamma=\alpha\circ p'$ and $q\circ\gamma=\beta\circ q'$; as $X_i'=\alpha^{-1}(U_i)$ and $Y_j'=\beta^{-1}(V_j)$, we also have ${p'}^{-1}(X_i')=\gamma^{-1}(p^{-1}(U_i))$ and ${q'}^{-1}(Y_j')=\gamma^{-1}(q^{-1}(V_j))$, hence
\[
  {p'}^{-1}(X_i')\cap{q'}^{-1}(Y_j')=\gamma^{-1}(p^{-1}(U_i)\cap q^{-1}(V_j))=\gamma^{-1}(U_i\times_S V_j),
\]
we conclude as in the first part.

So suppose $X$, $Y$, $S$ are affine, and let $B$, $C$, $A$ be their respective rings.
Then $B$ and $C$ are $A$-algebras, $X'$ and $Y'$ are affine schemes whose rings are quotient algebras $B'$ and $C'$ of $B$ and $C$ respectively.
In addition, we have $\alpha=({}^a\rho,\widetilde{\rho})$ and $\beta=({}^a\sigma,\widetilde{\sigma})$, where $\rho$ and $\sigma$ are respectively the canonical homomorphisms $B\to B'$ and $C\to C'$ \sref{1.1.7.3}.
This being so, we know that $X\times_S Y$ (resp. $X'\times_S Y'$) is an affine scheme with ring $B\otimes_A C$ (resp. $B'\otimes_A C'$), and $\alpha\times_S\beta=({}^a\tau,\widetilde{\tau})$, where $\tau$ is the homomorphism $\rho\otimes\sigma$ from $B\otimes_A C$ to $B'\otimes_A C'$ (\sref{1.3.2.2} and \sref{1.3.2.3}); as this homomorphism is surjective, $\alpha\times_S\beta$ is an immersion.
In addition, if $\fk{b}$ (resp. $\fk{c}$) is the kernel of $\rho$ (resp. $\sigma$), then the kernel of $\tau$ is $u(\fk{b})+v(\fk{c})$, where $u$ (resp. $v$) is the homomorphism $b\mapsto b\otimes 1$ (resp. $c\mapsto 1\otimes c$).
As $p=({}^a u,\widetilde{u})$ and $q=({}^a v,\widetilde{v})$, this kernel corresponds, in the prime spectrum of $B\otimes_A C$, with the closed set $p^{-1}(X')\cap q^{-1}(Y')$ ((1.2.2.1) and \sref{1.1.1.2}[iii]), which finishes the proof.
\end{proof}

\begin{cor}[4.3.2]
\label{1.4.3.2}
If $f:X\to Y$ is an immersion (resp. an open immersion, a closed immersion) and an $S$-morphism, then $f_{(S')}$ is an immersion (resp. an open immersion, a closed immersion) for every extension $S'\to S$ of the base prescheme.
\end{cor}

\subsection{Inverse image of a subprescheme}
\label{subsection-inverse-image-of-subprescheme}
