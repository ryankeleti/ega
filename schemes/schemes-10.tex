\section{Formal schemes}
\label{section-formal-schemes}

\subsection{Formal affine schemes}
\label{subsection-formal-affine-schemes}

\begin{env}[10.1.1]
\label{1.10.1.1}
Let $A$ be an {\em admissible} topological ring \sref[0]{0.7.1.2}; for each
ideal of definition $\fk{J}$ of $A$, $\Spec(A/\fk{J})$ identifies
with the closed subspace $V(\fk{J})$ of $\Spec(A)$ \sref{1.1.1.11},
the set of {\em open} prime ideals of $A$; this topological space does not depend
\oldpage[I]{181}
on the ideal of definition $\fk{J}$ considered; we denote this topological
space by $\fk{X}$. Let $(\fk{J}_\lambda)$ be a fundamental system
of neighborhoods of $0$ in $A$, consisting of ideals of definition, and for each
$\lambda$, let $\OO_\lambda$ be the structure sheaf of
$\Spec(A/\fk{J}_\lambda)$; this sheaf is induced on $\fk{X}$ by
$\widetilde{A}/\widetilde{\fk{J}_\lambda}$ (which is zero outside of
$\fk{X}$). For $\fk{J}_\mu\subset\fk{J}_\lambda$, the
canonical homomorphism $A/\fk{J}_\mu\to A/\fk{J}_\lambda$ thus
defines a homomorphism $u_{\lambda\mu}:\OO_\mu\to\OO_\lambda$ of sheaves of
rings \sref{1.1.6.1}, and $(\OO_\lambda)$ is a {\em projective system of
sheaves of rings} for these homomorphisms. As the topology of $\fk{X}$
admits a basis consisting of quasi-compact open subsets, we can associate to
each $\OO_\lambda$ a {\em sheaf of pseudo-discrete topological rings}
\sref[0]{0.3.8.1} which have $\OO_\lambda$ as the underlying (without topology)
sheaf of rings, and that we denote also by $\OO_\lambda$; and the $\OO_\lambda$ give again a
{\em projective system of sheaves of topological rings} \sref[0]{0.3.8.2}.
We denote by $\OO_\fk{X}$ the {\em sheaf of topological rings} on $\fk{X}$, the
projective limit of the system $(\OO_\lambda)$; for each {\em quasi-compact} open subset $U$
aof $\fk{X}$, then $\Gamma(U,\OO_\fk{X})$ is a topological ring, the projective
limit of the system of {\em discrete} rings $\Gamma(U,\OO_\lambda)$
\sref[0]{0.3.2.6}.
\end{env}

\begin{defn}[10.1.2]
\label{1.10.1.2}
Given an admissible topological ring $A$, we define the formal spectrum of $A$, and denote
it by $\Spf(A)$, to be the closed subspace $\fk{X}$ of $\Spec(A)$ consisting of the
open prime ideals of $A$. We say that a topologically ringed space is a formal affine scheme
if it is isomorphic to a formal spectrum $\Spf(A)=\fk{X}$ equipped with a sheaf of
topological rings $\OO_\fk{X}$ which is the projective limit of sheaves of
psuedo-discrete topological rings
$(\widetilde{A}/\widetilde{\fk{J}_\lambda})|\fk{X}$, where $\fk{J}_\lambda$
varies over the filtered set of ideals of definition for $A$.
\end{defn}

When we speak of a {\em formal spectrum $\fk{X}=\Spf(A)$} as a formal affine scheme, it will always be as the topologically-ringed space $(\fk{X},\OO_\fk{X})$ where $\OO_\fk{X}$ is defined as above.

We note that every \emph{affine scheme} $X=\Spec(A)$ can be considered as a formal affine scheme in only one way, by considering $A$ as a discrete topological ring: the topological rings $\Gamma(U,\OO_X)$ are then discrete whenever $U$ is quasi-compact (but not, in general, when $U$ is an arbitrary open subset of $X$).

\begin{prop}[10.1.3]
\label{1.10.1.3}
If $\fk{X}=\Spf(A)$, where $A$ is an admissible ring, then $\Gamma(\fk{X},\OO_X)$ is topologically isomorphic to $A$.
\end{prop}

\begin{proof}
\label{proof-1.10.1.3}
Indeed, since $\fk{X}$ is closed in $\Spec(A)$
\end{proof}

