\section{Formal schemes}
\label{section-formal-schemes}

\subsection{Formal affine schemes}
\label{subsection-formal-affine-schemes}

\begin{env}[10.1.1]
\label{1.10.1.1}
Let $A$ be an \emph{admissible} topological ring \sref[0]{0.7.1.2}; for each ideal of definition $\fk{J}$ of $A$, $\Spec(A/\fk{J})$ identifies with the closed subspace $V(\fk{J})$ of $\Spec(A)$ \sref{1.1.1.11}, the set of \emph{open} prime ideals of $A$; this topological space does not depend
\oldpage[I]{181}
on the ideal of definition $\fk{J}$ considered; we denote this topological space by $\fk{X}$. Let $(\fk{J}_\lambda)$ be a fundamental system of neighborhoods of $0$ in $A$, consisting of ideals of definition, and for each $\lambda$, let $\OO_\lambda$ be the structure sheaf of $\Spec(A/\fk{J}_\lambda)$; this sheaf is induced on $\fk{X}$ by $\wt{A}/\wt{\fk{J}_\lambda}$ (which is zero outside of $\fk{X}$).
For $\fk{J}_\mu\subset\fk{J}_\lambda$, the canonical homomorphism $A/\fk{J}_\mu\to A/\fk{J}_\lambda$ thus defines a homomorphism $u_{\lambda\mu}:\OO_\mu\to\OO_\lambda$ of sheaves of rings \sref{1.1.6.1}, and $(\OO_\lambda)$ is a \emph{projective system of sheaves of rings} for these homomorphisms.
As the topology of $\fk{X}$ admits a basis consisting of quasi-compact open subsets, we can associate to each $\OO_\lambda$ a \emph{sheaf of pseudo-discrete topological rings} \sref[0]{0.3.8.1} which have $\OO_\lambda$ as the underlying (without topology) sheaf of rings, and that we denote also by $\OO_\lambda$; and the $\OO_\lambda$ give again a \emph{projective system of sheaves of topological rings} \sref[0]{0.3.8.2}.
We denote by $\OO_\fk{X}$ the \emph{sheaf of topological rings} on $\fk{X}$, the projective limit of the system $(\OO_\lambda)$; for each \emph{quasi-compact} open subset $U$ of $\fk{X}$, $\Gamma(U,\OO_\fk{X})$ is a topological ring, the projective limit of the system of \emph{discrete} rings $\Gamma(U,\OO_\lambda)$ \sref[0]{0.3.2.6}.
\end{env}

\begin{defn}[10.1.2]
\label{1.10.1.2}
Given an admissible topological ring $A$, we define the formal spectrum of $A$, and denote it by $\Spf(A)$, to be the closed subspace $\fk{X}$ of $\Spec(A)$ consisting of the open prime ideals of $A$.
We say that a topologically ringed space is a formal affine scheme if it is isomorphic to a formal spectrum $\Spf(A)=\fk{X}$ equipped with a sheaf of topological rings $\OO_\fk{X}$ which is the projective limit of sheaves of psuedo-discrete topological rings $(\wt{A}/\wt{\fk{J}_\lambda})|\fk{X}$, where $\fk{J}_\lambda$ varies over the filtered set of ideals of definition for $A$.
\end{defn}

When we speak of a \emph{formal spectrum $\fk{X}=\Spf(A)$} as a formal affine scheme, it will always be as the topologically ringed space $(\fk{X},\OO_\fk{X})$ where $\OO_\fk{X}$ is defined as above.

We note that every \emph{affine scheme} $X=\Spec(A)$ can be considered as a formal affine scheme in only one way, by considering $A$ as a discrete topological ring: the topological rings $\Gamma(U,\OO_X)$ are then discrete whenever $U$ is quasi-compact (but not, in general, when $U$ is an arbitrary open subset of $X$).

\begin{prop}[10.1.3]
\label{1.10.1.3}
If $\fk{X}=\Spf(A)$, where $A$ is an admissible ring, then $\Gamma(\fk{X},\OO_X)$ is topologically isomorphic to $A$.
\end{prop}

\begin{proof}
\label{proof-1.10.1.3}
Indeed, since $\fk{X}$ is closed in $\Spec(A)$, it is quasi-compact, and so $\Gamma(\fk{X},\OO_\fk{X})$ is topologically isomorphic to the projective limit of the discrete rings $\Gamma(\fk{X},\OO_\lambda)$; but $\Gamma(\fk{X},\OO_\lambda)$ is isomorphic to $A/\fk{J}_\lambda$ \sref{1.1.3.7}; since $A$ is separated and complete, it is topologically isomorphic to $\varprojlim A/\fk{J}_\lambda$ \sref[0]{0.7.2.1}, whence the proposition.
\end{proof}

\begin{prop}[10.1.4]
\label{1.10.1.4}
Let $A$ be an admissible ring, $\fk{X}=\Spf(A)$, and, for every $f\in A$, let $\fk{D}(f)=D(f)\cap\fk{X}$; then the topologically ringed space $(\fk{D}(f),\OO_\fk{X}|\fk{D}(f))$ is isomorphic to the formal affine spectrum $\Spf(A_{\{f\}}$ \sref[0]{0.7.6.15}.
\end{prop}

\begin{proof}
\label{proof-1.10.1.4}
For every \unsure{ideal of definition} $\fk{J}$ of $A$, the discrete ring $S_f^{-1}A/ S_f^{-1}\fk{J}$ is canonically identified with $A_{\{f\}}/\fk{J}_{\{f\}}$ \sref[0]{0.7.6.9}, so, by \sref{1.1.2.5} and \sref{1.1.2.6}, the topological space $\Spf(A_{\{f\}})$ is canonically identified with $\fk{D}(f)$.
Further, for every quasi-compact open subset $U$ of $\fk{X}$ contained in $\fk{D}(f)$, $\Gamma(U,\OO_\lambda)$ can be identified with the module of sections of the structure sheaf of $\Spec(S_f^{-1}A/ S_f^{-1}\fk{J}_\lambda)$ over $U$ \sref{1.1.3.6}, so, setting $\fk{N}=\Spf(A_{\{f\}})$, $\Gamma(U,\OO_\fk{X})$ can be identified with the module of sections $\Gamma(U,\OO_\fk{N})$, which proves the proposition.
\end{proof}

\begin{env}[10.1.5]
\label{1.10.1.5}
As a sheaf of rings \emph{without topology}, the structure sheaf $\OO_\fk{X}$ of $\Spf(A)$ admits, for every $x\in\fk{X}$, a fibre which, by \sref{1.10.1.4}, can be identified with the inductive limit $\varinjlim A_{\{f\}}$ for the $f\not\in\fk{j}_x$.
Then, by \sref[0]{0.7.6.17} and \sref[0]{0.7.6.18}:
\end{env}

\begin{prop}[10.1.6]
\label{1.10.1.6}
For every $x\in\fk{X}=\Spf(A)$, the fibre $\OO_x$ is a local ring whose residue field is isomorphic to $\kres(x)=A_x/\fk{j}_xA_x$.
If, further, $A$ is adic and Noetherian, then $\OO_x$ is a Noetherian ring.
\end{prop}

Since $\kres(x)$ is not reduced at $0$, we conclude from this result that the \emph{support} of the ring of sheaves $\OO_\fk{X}$ is \emph{equal to $\fk{X}$}.

\subsection{Morphisms of formal affine schemes}
\label{subsection-morphisms-of-formal-affine-schemes}

\begin{env}[10.2.1]
\label{1.10.2.1}
Let $A$, $B$ be two admissible rings, and let $\vphi:B\to A$ be a \emph{continuous} morphism.
The continuous map ${}^a\vphi:\Spec(A)\to\Spec(B)$ \sref{1.1.2.1} then maps $\fk{X}=\Spf(A)$ to $\fk{Y}=\Spf(B)$, since the inverse image under $\vphi$ of an open prime ideal of $A$ is an open prime ideal of $B$.
On the other hand, for all $g\in B$, $\vphi$ defines a continuous homomorphism $\Gamma(\fk{D}(g),\OO_\fk{Y})\to\Gamma(\fk{D}(\vphi(g)),\OO_\fk{X})$ according to \sref{1.10.1.4}, \sref{1.10.1.3}, and \sref[0]{0.7.6.7}; as these homomorphisms satisfy the compatibility conditions for the restrictions corresponding to the change from $g$ to a multiple of $g$, and as $\fk{D}(\vphi(g))={}^a\vphi^{-1}(\fk{D}(g))$, they define a \emph{continuous} homomorphisms of sheaves of topological rings $\OO_\fk{Y}\to{}^a\vphi_*(\OO_\fk{X})$ \sref[0]{0.3.2.5}, that we denote by $\wt{\vphi}$; we have thus defined a morphism $\Phi=({}^a\vphi,\wt{\vphi})$ of topologically ringed spaces $\fk{X}\to\fk{Y}$.
We note that as a homomorphism of sheaves without topology, $\wt{\vphi}$ defines a homomorphism $\wt{\vphi}_x^\sharp:\OO_{{}^a\vphi(x)}\to\OO_x$ on the stalks, for all $x\in\fk{X}$.
\end{env}

\begin{prop}[10.2.2]
\label{1.10.2.2}
Let $A$, $B$ be two admissible topological rings, and let $\fk{X}=\Spf(A)$, $\fk{Y}=\Spf(B)$.
For a morphism $u=(\psi,\theta):\fk{X}\to\fk{Y}$ of topologically ringed spaces to be of the form $({}^a\vphi,\wt{\vphi})$, where $\vphi$ is a continuous ring homomorphism $B\to A$, it is necessary and sufficient that for all $x\in\fk{X}$, $\theta_x^\sharp$ is a local homomorphism $\OO_{\vphi(x)}\to\OO_x$.
\end{prop}

\begin{proof}
\label{proof-1.10.2.2}
The condition is necessary: let $\fk{p}=\fk{j}_x\in\Spf(A)$, and let $\fk{q}=\vphi^{-1}(\fk{j}_x)$; if $g\not\in\fk{q}$, then we have $\vphi(g)\not\in\fk{p}$, and it is immediate that the homomorphism
$B_{\{g\}}\to A_{\{\vphi(g)\}}$ induced by $\vphi$ \sref[0]{0.7.6.7} sends $\fk{q}_{\{g\}}$ to a subset of $\fk{p}_{\{\vphi(g)\}}$; by passing to the inductive limit, we see (taking into account \sref{1.10.1.5} and \sref[0]{0.7.6.17}) that $\wt{\vphi}_x^\sharp$ is a local homomorphism.

Conversely, let $(\psi,\theta)$ be a morphism satisying the condition in the statement; according to \sref{1.10.1.3}, $\theta$ defines a continuous ring homomorphism
\[
  \vphi=\Gamma(\theta):B=\Gamma(\fk{Y},\OO_\fk{Y})\to\Gamma(\fk{X},\OO_\fk{X})=A.
\]
By virtue of the hypothesis on $\theta$, for the section $\vphi(g)$ of $\OO_\fk{X}$ over $\fk{X}$ to be an invertible germ at the point $x$, it is necessary and sufficient that $g$ is an invertible germ at the point $\psi(x)$.
But according to \sref[0]{0.7.6.17}, the sections of $\OO_\fk{X}$ (resp. $\OO_\fk{Y}$) over $\fk{X}$ (resp. $\OO_\fk{Y}$) whose germ is not invertible at the point $x$ (resp. $\psi(x)$) are exactly the elements of
$\fk{j}_x$
\oldpage[I]{183}
(resp. $\fk{j}_{\psi(x)}$); the above remark thus shows that ${}^a\vphi=\psi$.
Finally, for all $g\in B$ the diagram
\[
  \xymatrix{
    B=\Gamma(\fk{Y},\OO_\fk{Y})\ar[r]^\vphi\ar[d] &
    \Gamma(\fk{X},\OO_\fk{X})=A\ar[d]\\
    B_{\{g\}}=\Gamma(\fk{D}(g),\OO_\fk{Y})\ar[r]^{\Gamma(\theta_{\fk{D}(g)})} &
    \Gamma(\fk{D}(\vphi(g)),\OO_\fk{X})=A_{\{\vphi(g)\}}
  }
\]
is commutative; by the universal property of completed rings of fractions \sref[0]{0.7.6.6}, $\theta_{\fk{D}(g)}$ is equal to $\wt{\vphi}_{\fk{D}(g)}$ for all $g\in B$, so \sref[0]{0.3.2.5} we have $\theta=\wt{\vphi}$.
\end{proof}

We say that a morphism $(\psi,\theta)$ of topologically ringed spaces satisfying the condition of Proposition \sref{1.10.2.2} is a \emph{morphism of formal affine schemes}.
We can say that the functors $\Spf(A)$ in $A$ and $\Gamma(\fk{X},\OO_\fk{X})$ in $\fk{X}$ define an \emph{equivalence} between the cateory of admissible rings and the opposite category of formal affine schemes (T, I, 1.2).

\begin{env}[10.2.3]
\label{1.10.2.3}
As a particular case of \sref{1.10.2.2}, note that for $f\in A$, the canonical injection of the formal affine scheme induced by $\fk{X}$ on $\fk{D}(f)$ corresponds to the continuous canonical homomorphism $A\to A_{\{f\}}$.
Under the hypotheses of Proposition \sref{1.10.2.2}, let $h$ be an element of $B$, $g$ an element of $A$, multiple of $\vphi(h)$; we then have $\psi(\fk{D}(g))\subset\fk{D}(h)$; the restriction of $u$ to $\fk{D}(g)$, considered as a morphism from $\fk{D}(g)$ to $\fk{D}(h)$, is the unique morphism $v$ making the diagram
\[
  \xymatrix{
    \fk{D}(g)\ar[r]^v\ar[d] &
    \fk{D}(h)\ar[d]\\
    \fk{X}\ar[r]^u &
    \fk{Y}
  }
\]
commutative.

This morphism corresponds to the unique continuous homomorphism $\vphi':B_{\{h\}}\to A_{\{g\}}$ \sref[0]{0.7.6.7} making the diagram
\[
  \xymatrix{
    A\ar[d] &
    B\ar[l]_\vphi\ar[d]\\
    A_{\{g\}} &
    B_{\{h\}}\ar[l]_{\vphi'}
  }
\]
commutative.
\end{env}

\subsection{Ideals of definition for a formal affine scheme}
\label{subsection-ideals-of-definition-formal}

\begin{env}[10.3.1]
\label{1.10.3.1}
Let $A$ be an admissible ring, $\fk{J}$ an open ideal of $A$, $\fk{X}$ the formal affine scheme $\Spf(A)$.
Let $(\fk{J}_\lambda)$ be the set of the ideals of definition for $A$ contained in $\fk{J}$; then $\wt{\fk{J}}/\wt{\fk{J}}_\lambda$ is a sheaf of ideals of $\wt{A}/\wt{\fk{J}}_\lambda$.
Denote by $\fk{J}^\Delta$ the projective limit of the induced sheaves on $\fk{X}$ by $\wt{\fk{J}}/\wt{\fk{J}}_\lambda$, which identifies with a \emph{sheaf of ideals} of $\OO_\fk{X}$ \sref[0]{0.3.2.6}.
For every $f\in A$, $\Gamma(\fk{D}(f),\fk{J}^\Delta)$ is the projective limit of the $S_f^{-1}\fk{J}/S_f^{-1}\fk{J}_\lambda$, in other words, it identifies with the open ideal $\fk{J}_{\{f\}}$ of the ring $A_{\{f\}}$ \sref[0]{0.7.6.9}, and in particular $\Gamma(\fk{X},\fk{J}^\Delta)=\fk{J}$; we conclude (the $\fk{D}(f)$ forming a basis for the topology of $\fk{X}$) that we have
\[
  \fk{J}^\Delta|\fk{D}(f)=(\fk{J}_{\{f\}})^\Delta.
  \tag{10.3.1.1}
\]
\end{env}

\begin{env}[10.3.2]
\label{1.10.3.2}
\oldpage[I]{184}
With the notations of \sref{1.10.3.1}, for all $f\in A$, the canonical map from $A_{\{f\}}=\Gamma(\fk{D}(f),\OO_\fk{X})$ to $\Gamma(\fk{D}(f),(\wt{A}/\wt{\fk{J}})|\fk{X})=S_f^{-1}A/S_f^{-1}\fk{J}$ is \emph{surjective} and has for its kernel $\Gamma(\fk{D}(f),\fk{J}^\Delta)=\fk{J}_{\{f\}}$ \sref[0]{0.7.6.9}; these maps thus define a \emph{surjective} continuous homomorphism, said to be \emph{canonical}, from the sheaf of topological rings $\OO_\fk{X}$ to the sheaf of discrete rings $(\wt{A}/\wt{\fk{J}})|\fk{X}$, whose kernel is $\fk{J}^\Delta$; this homomorphism is none other than $\wt{\vphi}$ \sref{1.10.2.1}, where $\vphi$ is the continuous homomorphism $A\to A/\fk{J}$; the morphism $({}^a\vphi,\wt{\vphi}):\Spec(A/\fk{J})\to\fk{X}$ of formal affine schemes (where ${}^a\vphi$ is the identity homeomorphism from $\fk{X}$ to itself) is also called \emph{canonical}.
We thus have, according to the above, a \emph{canonical isomorphism}
\[
  \OO_\fk{X}/\fk{J}^\Delta\isoto(\wt{A}/\wt{\fk{J}})|\fk{X}.
  \tag{10.3.2.1}
\]

It is clear (since $\Gamma(\fk{X},\fk{J}^\Delta)=\fk{J}$) that the map $\fk{J}\to\fk{J}^\Delta$ is \emph{strictly increasing}; according to the above, for $\fk{J}\subset\fk{J}'$, the sheaf ${\fk{J}'}^\Delta/\fk{J}^\Delta$ is canonically isomorphic to $\wt{\fk{J}'}/\wt{\fk{J}}=(\fk{J}'/\fk{J})^\sim$.
\end{env}

\begin{env}[10.3.3]
\label{1.10.3.3}
The hypotheses and notations being the same as those of \sref{1.10.3.1}, we say that a sheaf of ideals $\sh{J}$ of $\OO_\fk{X}$ is a \emph{sheaf of ideals of definition} for $\fk{X}$ (or an \emph{ideal sheaf of definition} for $\fk{X}$) if, for all $x\in\fk{X}$, there exists an open neighborhood of $x$ of the form $\fk{D}(f)$, where $f\in A$, such that $\sh{J}|\fk{D}(f)$ is of the form $\fk{H}^\Delta$, where $\fk{H}$ is an ideal of definition for $A_{\{f\}}$.
\end{env}

\begin{prop}[10.3.4]
\label{1.10.3.4}
For all $f\in A$, each sheaf of ideals of definition for $\fk{X}$ induces a sheaf of ideals of definition for $\fk{D}(f)$.
\end{prop}

\begin{proof}
\label{proof-1.10.3.4}
This follows from (10.3.1.1).
\end{proof}

\begin{prop}[10.3.5]
\label{1.10.3.5}
If $A$ is an admissible ring, then every sheaf of ideals of definition for $\fk{X}=\Spf(A)$ is of the form $\fk{J}^\Delta$, where $\fk{J}$ is an ideal of definition for $A$, uniquely determined.
\end{prop}

\begin{proof}
\label{proof-1.10.3.5}
Let $\sh{J}$ be a sheaf of ideals of definition of $\fk{X}$; by hypothesis, and since $\fk{X}$ is quasi-compact, there is a finite number of elements $f_i\in A$ such that the $\fk{D}(f_i)$ cover $\fk{X}$ and that $\sh{J}|\fk{D}(f_i)=\fk{H}_i^\Delta$, where $\fk{H}_i$ is an ideal of definition for $A_{\{f_i\}}$.
For each $i$, there exists an open ideal $\fk{K}_i$ of $A$ such that $(\fk{K}_i)_{\{f_i\}}=\fk{H}_i$ \sref[0]{0.7.6.9}; let $\fk{K}$ be an ideal of definition for $A$ containing all the $\fk{K}_i$.
The canonical image of $\sh{J}/\fk{K}^\Delta$ in the structure sheaf $(A/\fk{K})^\sim$ of $\Spec(A/\fk{K})$ \sref{1.10.3.2} is thus such that its restriction to $\fk{D}(f_i)$ is equal to its restriction to $(\fk{K}_i/\fk{K})^\sim$;
we conclude that this canonical image is a \emph{quasi-coherent} sheaf on $\Spec(A/\fk{K})$, so it is of the form $(\fk{J}/\fk{K})^\sim$, where $\fk{J}$ is an ideal of definition for $A$ containing $\fk{K}$ \sref{1.1.4.1} hence $\sh{J}=\fk{J}^\Delta$ \sref{1.10.3.2};
in addition, as for each $i$ there exists an integer $n_i$ such that $\fk{H}_i^{n_i}\subset\fk{K}_{\{f_i\}}$, we will have, by setting $n$ to be the largest of the $n_i$, $(\sh{J}/\fk{K}^\Delta)^n=0$, and as a result \sref{1.10.3.2} $((\fk{J}/\fk{K})^\sim)^n=0$, so finally $(\fk{J}/\fk{K})^n=0$ \sref{1.1.3.13}, which prove that $\fk{J}$ is an ideal of definition for $A$ \sref[0]{0.7.1.4}.
\end{proof}

\begin{prop}[10.3.6]
\label{1.10.3.6}
Let $A$ be an adic ring, $\fk{J}$ an ideal of definition for $A$ such that $\fk{J}/\fk{J}^2$ is an $(A/\fk{J})$-module of finite type. For any integer $n>0$, we then have $(\fk{J}^\Delta)^n=(\fk{J}^n)^\Delta$.
\end{prop}

\begin{proof}
\label{proof-1.10.3.6}
For all $f\in A$, we have (since $\fk{J}^n$ is an open ideal)
\[
  (\Gamma(\fk{D}(f),\fk{J}^\Delta))^n=(\fk{J}_{\{f\}})^n=(\fk{J}^n)_{\{f\}}=\Gamma(\fk{D}(f^n),(\fk{J}^n)^\Delta)
\]
\oldpage[I]{185}
according to (10.3.1.1) and \sref[0]{0.7.6.12}.
As $(\fk{J}^\Delta)^n$ is associated to the presheaf $U\mapsto(\Gamma(U,\fk{J}^\Delta))^n$ \sref[0]{0.4.1.6}, the result follows, since the $\fk{D}(f)$ form a basis for the topology of $\fk{X}$.
\end{proof}

\begin{env}[10.3.7]
\label{1.10.3.7}
We say that a family $(\sh{J}_\lambda)$ of sheaves of ideals of definition for $\fk{X}$ is a \emph{fundamental system of sheaves of ideals of definition} if each sheaf of ideals of definition for $\fk{X}$ contains one of the $\sh{J}_\lambda$; as $\sh{J}_\lambda=\fk{J}_\lambda^\Delta$, it is equivalent to say that the $\fk{J}_\lambda$ for a \emph{fundamental system of neighborhoods of $0$} in $A$.
Let $(f_\alpha)$ be a family of elements of $A$ such that the $\fk{D}(f_\alpha)$ cover $\fk{X}$.
If $(\sh{J}_\lambda)$ is a filtered decreasing family of sheaves of ideals of $\OO_\fk{X}$ such that for each $\alpha$, the family $(\sh{J}_\lambda|\fk{D}(f_\alpha))$ is a fundamental system of sheaves of ideals of definition for $\fk{D}(f_\alpha)$, then $(\sh{J}_\lambda)$ is a fundamental system of sheaves of ideals of definition for $\fk{X}$.
Indeed, for each sheaf of ideals of definition $\sh{J}$ for $\fk{X}$, there is a finite cover of $\fk{X}$ by $\fk{D}(f_i)$ such that, for each $i$, $\sh{J}_{\lambda_i}|\fk{D}(f_i)$ is a sheaf of ideals of definition for $\fk{D}(f_i)$ contained in $\sh{J}|\fk{D}(f_i)$.
If $\mu$ is an index such that $\sh{J}_\mu\subset\sh{J}_{\lambda_i}$ for all $i$, then it follows from \sref{1.10.3.3} that $\sh{J}_\mu$ is a sheaf of ideals of definition for $\fk{X}$, evidently contained in $\sh{J}$, hence our assertion.
\end{env}

\subsection{Formal preschemes and morphisms of formal preschemes}
\label{subsection-formal-preschemes-and-morphisms}

\begin{env}[10.4.1]
\label{1.10.4.1}
Given a topologically ringed space $\fk{X}$, we say that an open $U\subset\fk{X}$ is an \emph{formal affine open} (resp. an \emph{formal adic affine open}, resp. an \emph{formal Noetherian affine open}) if the topologically ringed space induced on $U$ by $\fk{X}$ is a formal affine scheme (resp. a scheme whose ring is adic, resp. adic and Noetherian).
\end{env}

\begin{defn}[10.4.2]
\label{1.10.4.2}
A \emph{formal prescheme} is a topologically ringed spacd $\fk{X}$ which admits a formal affine open neighborhood for each point.
We say that the formal prescheme $\fk{X}$ is adic (resp. locally Noetherian) if each point of $\fk{X}$ admits a formal adic (resp. Noetherian) open neighborhood.
We say that $\fk{X}$ is Noetherian if it is locally Noetherian and if its underlying space is quasi-compact (hence Noetherian).
\end{defn}

\begin{prop}[10.4.3]
\label{1.10.4.3}
If $\fk{X}$ is a formal prescheme (resp. a locally Noetherian formal prescheme), then the formal affine (resp. Noetherian affine) open sets form a basis for the topology of $\fk{X}$.
\end{prop}

\begin{proof}
\label{proof-1.10.4.3}
This follows from Definition \sref{1.10.4.2} and Proposition \sref{1.10.1.4} by taking into account that if $A$ is an adic Noetherian ring, then so if $A_{\{f\}}$ for all $f\in A$ \sref[0]{0.7.6.11}.
\end{proof}

\begin{cor}[10.4.4]
\label{1.10.4.4}
If $\fk{X}$ is a formal prescheme (resp. a locally Noetherian formal prescheme, resp. a Noetherian formal prescheme), then the topologically ringed space induced on each open set of $\fk{X}$ a formal prescheme (resp. a locally Noetherian formal prescheme, resp. a Noetherian formal prescheme).
\end{cor}

\begin{defn}[10.4.5]
\label{1.10.4.5}
Given two formal preschemes $\fk{X}$ and $\fk{Y}$, a morphism (of formal preschemes) from $\fk{X}$ to $\fk{Y}$ is a morphism $(\psi,\theta)$ of topologically ringed spaces such that, for all $x\in\fk{X}$, $\theta_x^\sharp$ is a local homomorphism $\OO_{\psi(x)}\to\OO_x$.
\end{defn}

It is immediate that the composition of two morphisms of formal preschemes is again a morphism of formal preschemes; the formal preschemes thus form a \emph{category}, and we denote by $\Hom(\fk{X},\fk{Y})$ the set of morphisms from a formal prescheme $\fk{X}$ to a formal prescheme $\fk{Y}$.

\oldpage[I]{186}
If $U$ is an open subset of $\fk{X}$, then the canonical injection into $\fk{X}$ of the formal prescheme induced on $U$ by $\fk{X}$ is a morphism of formal preschemes (and similarly a \emph{momomorphism} of topologically ringed spaces \sref[0]{0.4.1.1}).

\begin{prop}[10.4.6]
\label{1.10.4.6}
Let $\fk{X}$ be a formal prescheme, $\fk{S}=\Spf(A)$ a formal affine scheme.
There exists a canonical bijective equivalence between the morphisms from a formal prescheme $\fk{X}$ to the formal prescheme $\fk{S}$ and the continuous homomorphisms from the ring $A$ to the topological ring $\Gamma(\fk{X}.\OO_\fk{X})$.
\end{prop}

\begin{proof}
\label{proof-1.10.4.6}
The proof is similar to that of \sref{1.2.2.4}, by replacing ``homomorphism'' by ``continuous homomorphism'', ``affine open'' by ``formal affine open'', and by using Proposition \sref{1.10.2.2} instead of Theorem \sref{1.1.7.3}; we leave the details to the reader.
\end{proof}

\begin{env}[10.4.7]
\label{1.10.4.7}
Given a formal prescheme $\fk{S}$, we say that the data of a formal prescheme $\fk{X}$ and a morphism $\vphi:\fk{X}\to\fk{S}$ defines a formal prescheme \emph{$\fk{X}$ over $\fk{S}$} or an \emph{formal $\fk{S}$-prescheme}, $\vphi$ being called the \emph{structure morphism} of the $\fk{S}$-prescheme $\fk{X}$.
If $\fk{S}=\Spf(A)$, where $A$ is an admissible ring, then we also say that the formal $\fk{S}$-prescheme $\fk{X}$ is a \emph{formal $A$-prescheme} or a formal prescheme \emph{over $A$}.
An arbitrary formal prescheme can be considered as a formal prescheme over $\bb{Z}$ (equipped with the discrete topology).

If $\fk{X}$ and $\fk{Y}$ are two formal $\fk{S}$-preschemes, we say that a morphism $u:\fk{X}\to\fk{Y}$ is a \emph{$\fk{S}$-morphism} if the diagram
\[
  \xymatrix{
    \fk{X}\ar[rr]^u\ar[rd] & &
    \fk{Y}\ar[ld]\\
    & \fk{S}
  }
\]
(where the downwards arrows are the structure morphisms) is commutative.
With this definition, the formal $\fk{S}$-preschemes (for $\fk{S}$ fixed) forms a \emph{category}.
We denote by $\Hom_\fk{S}(\fk{X},\fk{Y})$ the set of $\fk{S}$-morphisms from a formal $\fk{S}$-prescheme $\fk{X}$ to a formal $\fk{S}$-prescheme $\fk{Y}$.
When $\fk{S}=\Spf(A)$, we also say \emph{$A$-morphism} instead of \emph{$\fk{S}$-morphism}.
\end{env}

\begin{env}[10.4.8]
\label{1.10.4.8}
As each affine scheme can be considered as a formal affine scheme \sref{1.10.1.2}, each (usual) prescheme can be considered as a formal prescheme.
In addition, it follows from Definition \sref{1.10.4.5} that for the \emph{usual} preschemes, the morphisms (resp. $S$-morphisms) of \emph{formal} preschemes coincide with the morphisms (resp. $S$-morphisms) defined in \textsection2.
\end{env}

\subsection{Sheaves of ideals of definition for formal preschemes}
\label{subsection-sheaves-of-ideals-of-definition-formal-preschemes}

\begin{env}[10.5.1]
\label{1.10.5.1}
Let $\fk{X}$ be a formal prescheme; we say that an $\OO_\fk{X}$-ideal $\sh{J}$ is a \emph{sheaf of ideals of definition} (or an \emph{ideal sheaf of definition}) for $\fk{X}$ if every $x\in\fk{X}$ has a formal affine open neighborhood $U$ such that $\sh{J}|U$ is a sheaf of ideals of definition for the formal affine scheme induced on $U$ by $\fk{X}$ \sref{1.10.3.3}; according to (10.3.1.1) and Proposition \sref{1.10.4.3}, for each open $V\subset\fk{X}$, $\sh{J}|V$ is then a sheaf of ideals of definition for the formal prescheme induced on $V$ by $\fk{X}$.

We say that a family $(\sh{J}_\lambda)$ of sheaves of ideals of definition for $\fk{X}$ is a \emph{fundamental system}
\oldpage[I]{187}
\emph{of sheaves of ideals of definition} if there exists a cover $(U_\alpha)$ of $\fk{X}$ by formal affine open sets such that, for each $\alpha$, the family of the $\sh{J}_\lambda|U_\alpha$ is a fundamental system of sheaves of ideals of definition \sref{1.10.3.6} for the formal affine scheme induced on $U_\alpha$ by $\fk{X}$.
It follows from the last remark of \sref{1.10.3.7} that when $\fk{X}$ is a formal affine scheme, this definition coincides with the definition given in \sref{1.10.3.7}.
For an open subset $V$ of $\fk{X}$, the restrictions $\sh{J}_\lambda|V$ then form a fundamental system of sheaves of ideals of definition for the formal prescheme induced on $V$, according to (10.3.1.1).
If $\fk{X}$ is a \emph{locally Noetherian} formal prescheme, and $\sh{J}$ is a sheaf of ideals of definition for $\fk{X}$, then it follows from Proposition \sref{1.10.3.6} that the powers $\sh{J}^n$ form a fundamental system of sheaves of ideals of definition for $\fk{X}$.
\end{env}

\begin{env}[10.5.2]
\label{1.10.5.2}
Let $\fk{X}$ be a formal prescheme, $\sh{J}$ a sheaf of ideals of definition for $\fk{X}$.
Then the ringed space $(\fk{X},\OO_\fk{X}/\sh{J})$ is a (usual) \emph{prescheme}, which is affine (resp. locally Noetherian, resp. Noetherian) when $\fk{X}$ is a formal affine scheme (resp. a locally Noetherian formal scheme, resp. a Noetherian formal scheme);  we can reduce to the affine case, and then the proposition has already been proven in \sref{1.10.3.2}.
In addition, if $\theta:\OO_\fk{X}\to\OO_\fk{X}/\sh{J}$ is the canonical homomorphism, then $u=(1_\fk{X},\theta)$ is a \emph{morphism} (said to be \emph{canonical}) of formal preschemes $(\fk{X},\OO_\fk{X}/\sh{J})\to(\fk{X},\OO_\fk{X})$, because again, this was proven in the affine case \sref{1.10.3.2}, to which it is immediately reduced.
\end{env}

\begin{prop}[10.5.3]
\label{1.10.5.3}
Let $\fk{X}$ be a formal prescheme, $(\sh{J}_\lambda)$ a fundamental system of sheaves of ideals of definition for $\fk{X}$.
Then the sheaf of topological rings $\OO_\fk{X}$ is the projective limit of the sheaves of pseudo-discrete rings \sref[0]{0.3.8.1} $\OO_\fk{X}/\sh{J}_\lambda$.
\end{prop}

\begin{proof}
\label{proof-1.10.5.3}
As the topology of $\fk{X}$ admits a basis of formal quasi-compact affine open sets \sref{1.10.4.3}, we reduce to the affine case, where the proposition is a consequence of Proposition \sref{1.10.3.5}, \sref{1.10.3.2}, and the definition \sref{1.10.1.1}.
\end{proof}

It is not true that any formal prescheme admits a sheaf of ideals of definition.
However:
\begin{prop}[10.5.4]
\label{1.10.5.4}
Let $\fk{X}$ be a locally Noetherian formal prescheme.
There exists a largest sheaf of ideals of definition $\sh{T}$ for $\fk{X}$; this is the unique sheaf of ideals of definition $\sh{J}$ such that the prescheme $(\fk{X},\OO_\fk{X}/\sh{J})$ is reduced.
If $\sh{J}$ is a sheaf of ideals of definition for $\fk{X}$, then $\sh{T}$ is the inverse image under $\OO_\fk{X}\to\OO_\fk{X}/\sh{J}$ of the nilradical of $\OO_\fk{X}/\sh{J}$.
\end{prop}

\begin{proof}
\label{proof-1.10.5.4}
Suppose first that $\fk{X}=\Spf(A)$, where $A$ is an adic Noetherian ring.
The existence and the properties of $\sh{T}$ follow immediately from Propositions \sref{1.10.3.5} and \sref{1.5.1.1}, taking into account the existence and the properties of the largest ideal of definition for $A$ (\sref[0]{0.7.1.6} and \sref[0]{0.7.1.7}).

To prove the existence and the properties of $\sh{T}$ in the general case, it suffices to show that if $U\supset V$ are two Noetherian formal affine open subsets of $X$, then the largest sheaf of ideals of definition $\sh{T}_U$ for $U$ induces the largest sheaf of ideals of definition $\sh{T}_V$ for $V$; but as $(V,(\OO_\fk{X}|V)/(\sh{T}_U|V))$ is reduced, this follows from the above.
\end{proof}

We denote by $\fk{X}_\text{red}$ the (usual) reduced prescheme $(\fk{X},\OO_\fk{X}/\sh{T}$).

\begin{cor}[10.5.5]
\label{1.10.5.5}
Let $\fk{X}$ be a locally Noetherian formal prescheme, $\sh{T}$ the largest sheaf of ideals of definition for $\fk{X}$; for each open subset $V$ of $\fk{X}$, $\sh{T}|V$ is the largest sheaf of ideals of definition for the formal prescheme induced on $V$ by $\fk{X}$.
\end{cor}

\begin{prop}[10.5.6]
\label{1.10.5.6}
Let $\fk{X}$ and $\fk{Y}$ be two formal preschemes, $\sh{J}$ (resp. $\sh{K}$) be a sheaf of ideals of definition for $\fk{X}$ (resp. $\fk{Y}$), $f:\fk{X}\to\fk{Y}$ a morphism of formal preschemes.
\begin{enumerate}[label=\emph{(\roman*)}]
  \item If $f^*(\sh{K})\OO_\fk{X}\subset\sh{J}$, then there exists a unique morphism $f':(\fk{X},\OO_\fk{X}/\sh{J})\to(\fk{Y},\OO_\fk{Y}/\sh{K})$ of usual preschemes making the diagram
    \[
      \xymatrix{
        (\fk{X},\OO_\fk{X})\ar[r]^f &
        (\fk{Y},\OO_\fk{Y})\\
        (\fk{X},\OO_\fk{X}/\sh{J})\ar[r]^{f'}\ar[u] &
        (\fk{Y},\OO_\fk{Y}/\sh{K})\ar[u]
      }
      \tag{10.5.6.1}
    \]
    commutative, where the vertical arrows are the canonical morphisms.
  \item Suppose that $\fk{X}=\Spf(A)$ and $\fk{Y}=\Spf(B)$ are two formal affine schemes, $\sh{J}=\fk{J}^\Delta$ and $\sh{K}=\fk{K}^\Delta$, where $\fk{J}$ (resp. $\fk{K}$) is an ideal of definition for $A$ (resp. $B$), and $f=({}^a\vphi,\wt{\vphi})$, where $\vphi:B\to A$ is a continuous homomorphism;
    for $f^*(\sh{K})\OO_\fk{X}\subset\sh{J}$ to hold, it is necessary and sufficient that $\vphi(\fk{K})\subset\fk{J}$, and $f'$ is then the morphism $({}^a\vphi',\wt{\vphi'})$, where $\vphi':B/\fk{K}\to A/\fk{J}$ is the homomorphism induced from $\vphi$ by passing to quotients.
\end{enumerate}
\end{prop}

\begin{proof}
\label{proof-1.10.5.6}
\medskip\noindent
\begin{enumerate}[label=(\roman*)]
  \item If $f=(\psi,\theta)$, then the hypotheses imply that the image under $\theta^\sharp:\psi^*(\OO_\fk{Y})\to\OO_\fk{X}$ of the sheaf of ideals $\psi^*(\sh{K})$ of $\psi^*(\OO_\fk{Y})$ is contained in $\sh{J}$ \sref[0]{0.4.3.5}.
    By passing to quotients, we thus induce from $\theta^\sharp$ a homomorphism of sheaves of rings
    \[
      \omega:\psi^*(\OO_\fk{Y}/\sh{K})=\psi^*(\OO_\fk{Y})/\psi^*(\sh{K})\to\OO_\fk{X}/\sh{J};
    \]
    in addition, as for all $x\in\fk{X}$, $\theta_x^\sharp$ is a \emph{local} homomorphism, so is $\omega_x$.
    The morphism of ringed spaces $(\psi,\omega^\flat)$ is thus \sref{1.2.2.1} the unique morphism $f'$ of ringed spaces which we need.
  \item The canonical functorial correspondence between morphisms of formal affine schemes and continuous homomorphisms of rings \sref{1.10.2.2} shows that in the case considered, the relation $f^*(\sh{K})\OO_\fk{X}\subset\fk{J}$ impliex that we have $f'=({}^a\vphi',\wt{\vphi'})$, where $\vphi':B/\fk{K}\to A/\fk{J}$ is the unique homomorphism making the diagram
    \[
      \xymatrix{
        B\ar[r]^\vphi\ar[d] &
        A\ar[d]\\
        B/\fk{K}\ar[r]^{\vphi'} &
        A/\fk{J}
      }
      \tag{10.5.6.2}
    \]
    commutative.
    The existence of $\vphi'$ thus implies that $\vphi(\fk{K})\subset\fk{J}$.
    Conversely, if this condition is satisfied, then denoting by $\vphi'$ the unique homomorphism making the diagram (10.5.6.2) commutative and setting $f'=({}^a\vphi',\wt{\vphi'})$, it is clear that the diagram (10.5.6.1) is commutative; the consideration of the homomorphisms ${}^a\vphi^*(\OO_\fk{Y})\to\OO_\fk{X}$ and ${}^a{\vphi'}^*(\OO_\fk{Y}/\sh{K})\to\OO_\fk{X}/\sh{J}$ corresponding to $f$ and $f'$ respectively then shows that this implies the relation $f^*(\sh{K})\OO_\fk{X}\subset\sh{J}$.
\end{enumerate}
\end{proof}

It is clear that the correspondence $f\mapsto f'$ defined above is \emph{functorial}.

\subsection{Formal preschemes as inductive limits of preschemes}
\label{subsection-formal-preschemes-as-inductive-limits}

\begin{env}[10.6.1]
\label{1.10.6.1}
Let $\fk{X}$ be a formal prescheme, $(\sh{J}_\lambda)$ a fundamental system of sheaves of ideals of definition for $\fk{X}$; for each $\lambda$, let $f_\lambda$ be the canonical morphism $(\fk{X},\OO_\fk{X}/\sh{J}_\lambda)\to\fk{X}$ \sref{1.10.5.2}; for $\sh{J}_\mu\subset\sh{J}_\lambda$, the canonical morphism $\OO_\fk{X}/\sh{J}_\mu\to\OO_\fk{X}/\sh{J}_\lambda$ defines a canonical morphism
\oldpage[I]{189}
$f_{\mu\lambda}:(\fk{X},\OO_\fk{X}/\sh{J}_\lambda)\to(\fk{X},\OO_\fk{X}/\sh{J}_\mu)$ of (usual) preschemes such that we have $f_\lambda=f_\mu\circ f_{\mu\lambda}$.
The preschemes $X_\lambda=(\fk{X},\OO_\fk{X}/\sh{J}_\lambda)$ and the morphisms $f_{\mu\lambda}$ thus form (according to \sref{1.10.4.8}) a \emph{inductive system} in the category of formal preschemes.
\end{env}

\begin{prop}[10.6.2]
\label{1.10.6.2}
With the notations of \sref{1.10.6.1}, the formal prescheme $\fk{X}$ and the morphisms $f_\lambda$ form an inductive limit (T, I, 1.8) of the system $(X_\lambda,f_{\mu\lambda})$ in the category of formal preschemes.
\end{prop}

\begin{proof}
\label{proof-1.10.6.2}
Let $\fk{Y}$ be a formal prescheme, and for each index $\lambda$, let
\[
  g_\lambda=(\psi_\lambda,\theta_\lambda):X_\lambda\to\fk{Y}
\]
be a morphism such that we have $g_\lambda=g_\mu\circ f_{\mu\lambda}$ for $\sh{J}_\mu\subset\sh{J}_\lambda$.
This latter condition and the definition of the $X_\lambda$ imply first that the $\psi_\lambda$ are identical to a continuous map $\psi:\fk{X}\to\fk{Y}$ of the underlying spaces; in addition, the homomorphism $\theta_\lambda^\sharp:\psi^*(\OO_\fk{Y})\to\OO_{X_i}=\OO_\fk{X}/\sh{J}_\lambda$ form a \emph{projective system} of homomorphisms of sheaves of rings.
By passing to the projective limit, we thus induce a homomorphism $\omega:\psi^*(\OO_\fk{Y})\to\varprojlim\OO_\fk{X}/\sh{J}_\lambda=\OO_\fk{X}$, and it is clear that the morphism $g=(\psi,\omega^\flat)$ of \emph{ringed spaces} is the \emph{unique} morphism making the diagrams
\[
  \xymatrix{
    X_\lambda\ar[rr]^{g_\lambda}\ar[rd]_{f_\lambda} & &
    \fk{Y}\\
    & \fk{X}\ar[ru]_g
  }
  \tag{10.6.2.1}
\]
commutative.
It remains to prove that $g$ is a morphism of \emph{formal preschemes}; the question is local on $\fk{X}$ and $\fk{Y}$, so we can assume $\fk{X}=\Spf(A)$ and $\fk{Y}=\Spf(B)$, $A$ and $B$ admissible rings, with $\sh{J}_\lambda=\fk{J}_\lambda^\Delta$, where $(\fk{J}_\lambda)$ is a fundamental system of ideal of definition for $A$ \sref{1.10.3.5}; as $A=\varprojlim A/\fk{J}_\lambda$, the existence of a morphism $g$ of formal affine schemes making the diagrams (10.6.2.1) commutative then follows from the bijective correspondence \sref{1.10.2.2} between morphisms of formal affine schemes and continuous ring homomorphisms, and from the definition of the projective limit.
But the uniqueness of $g$ as a morphism of ringed spaces shows that it coincides with the morphism in the beginning of the proof.
\end{proof}

The following proposition establishes, under certain additional conditions, the existence of the inductive limit of a given inductive system of (usual) preschemes in the category of formal preschemes:
\begin{prop}[10.6.3]
\label{1.10.6.3}
Let $\fk{X}$ be a topological space, $(\OO_i,u_{ji})$ a projective system of sheaves of rings on $\fk{X}$, with $\bb{N}$ for its set of indices.
Let $\sh{J}_i$ be the kernel of $u_{0i}:\OO_i\to\OO_0$.
Suppose that:
\begin{enumerate}[label=\emph{(\alph*)}]
  \item The ringed space $(\fk{X},\OO_i)$ is a prescheme $X_i$.
  \item For all $x\in\fk{X}$ and all $i$, there exists an open neighborhood $U_i$ of $x$ in $\fk{X}$ such that the restriction $\sh{J}_i|U_i$ is nilpotent.
  \item The homomorphisms $u_{ji}$ are surjective.
\end{enumerate}

\oldpage[I]{190}
Let $\OO_\fk{X}$ be the sheaf of topological rings formed as the projective limit of the sheaves of pseudo-discrete rings $\OO_i$, and let $u_i:\OO_\fk{X}\to\OO_i$ be the canonica homomorphism.
Then the topologically ringed space $(\fk{X},\OO_\fk{X})$ is a formal prescheme; the homomorphisms $u_i$ are surjective; their kernels $\sh{J}^{(i)}$ form a fundamental system of sheaves of ideals of definition for $\fk{X}$, and $\sh{J}^{(0)}$ is the projective limit of the sheaves of ideals $\sh{J}_i$.
\end{prop}

\begin{proof}
\label{proof-1.10.6.3}
We first note that on each stalk, $u_{ji}$ is a surjective homomorphism and \emph{a fortiori} a local homomorphism; thus $v_{ij}=(1_\fk{X},u_{ji})$ is a morphism of preschemes $X_j\to X_i$ ($i\geqslant j$) \sref{1.2.2.1}.
Suppose first that each $X_i$ is an affine scheme woth ring $A_i$.
There exists a \emph{ring} homomorphism $\vphi_{ji}:A_i\to A_j$ such that $u_{ji}=\wt{\vphi_{ji}}$ \sref{1.1.7.3}; as a result \sref{1.1.6.3}, the sheaf $\OO_j$ is a quasi-coherent $\OO_i$-module over $X_i$ (for the external law defined by $u_{ji}$), associated to $A_j$ considered as an $A_i$-module $\vphi_{ji}$.
For all $f\in A_i$, let $f'=\vphi_{ji}(f)$; by hypothesis, the open sets $D(f)$ and $D(f')$ are indentical in $\fk{X}$, and the homomorphism from $\Gamma(D(f),\OO_i)=(A_i)_f$ to $\Gamma(D(f),\OO_j)=(A_j)_{f'}$ corresponding to $u_{ji}$ is none other than $(\vphi_{ji})_f$ \sref{1.1.6.1}.
But when we consider $A_j$ as an $A_i$-module, $(A_j)_{f'}$ is the $(A_i)_f$-module $(A_j)_f$, so we also have $u_{ji}=\wt{\vphi_{ji}}$, where $\vphi_{ji}$ is now considered as a homomorphism of \emph{$A_i$-modules}.
Then as $u_{ji}$ is surjective, we conclude that $\vphi_{ji}$ is also surjective \sref{1.1.3.9} and if $\fk{J}_{ji}$ is the kernel of $\vphi_{ji}$, then the kernel of $u_{ji}$ is a quasi-coherent $\OO_i$-module equal to $\wt{\fk{J}_{ji}}$.
In particular, we have $\sh{J}_i=\wt{\fk{J}_i}$, where $\fk{J}_i$ is the kernel of $\vphi_{0i}:A_i\to A_0$.
The hypothesis (b) implies that $\sh{J}_i$ is \emph{nilpotent}: indeed, as $\fk{X}$ is quasi-compact, we can cover $\fk{X}$ by a finite number of open sets $U_k$ such that $(\sh{J}_i|U_k)^{n_k}=0$, and by setting $n$ to be the largest of the $n_k$, we have $\sh{J}_i^n=0$.
We conclude that $\fk{J}_i$ is nilpotent \sref{1.1.3.13}.
Then the ring $A=\varprojlim A_i$ is admissible \sref[0]{0.7.2.2}, the canonical homomorphism $\vphi_i:A\to A_i$ is surjective, and its kernel $\fk{J}^{(i)}$ is equal to the projective limit of the $\fk{J}_{ik}$ for $k\geqslant i$; the $\fk{J}^{(i)}$ form a fundamental system of neighborhoods of $0$ in $A$.
The assertions of Proposition \sref{1.10.6.3} follow in this case from \sref{1.10.1.1} and \sref{1.10.3.2}, $(\fk{X},\OO_\fk{X})$ being $\Spf(A)$.

In this particular case, we note that if $f=(f_i)$ is an element of the projective limit $A=\varprojlim A_i$, then all the open sets $D(f_i)$ (affine open sets in $X_i$) identify with the open subset $\fk{D}(f)$ of $\fk{X}$, the prescheme induced on $\fk{D}(f)$ by $X_i$ thus identifying with the affine scheme $\Spec((A_i)_{f_i})$.

In the general case, we remark first that for every quasi-compact open subset $U$ of $\fk{X}$, each of the $\sh{J}_i|U$ is nilpotent, as shown by the above reasoning.
We will see that for every $x\in\fk{X}$, there exists an open neighborhood $U$ of $x$ in $\fk{X}$ which is an \emph{affine open set} for \emph{all} the $X_i$.
Indeed, we take $U$ to be an open affine set for $X_0$, and observe that $\OO_{X_0}=\OO_{X_i}/\sh{J}_i$.
As $\sh{J}_i|U$ is nilpotent according to the above, $U$ is also an affine open set for each $X_i$ by Proposition \sref{1.5.1.9}.
This being so, for each $U$ satsifying the preceding conditions, the study of the affine case as above shows that $(U,\OO_X|U)$ is a formal prescheme whose $\sh{J}^{(i)}|U$ for a fundamental system of sheaves of ideals of definition, and $\sh{J}^{(0)}|U$ is the projective limit of the $\sh{J}_i|U$; hence the conclusion.
\end{proof}

\begin{cor}[10.6.4]
\label{1.10.6.4}
Suppose that for $i\geqslant j$, the kernel of $u_{ji}$ is $\sh{J}_i^{j+1}$ and that $\sh{J}_1/\sh{J}_1^2$
\oldpage[I]{191}
is of finite type over $\OO_0=\OO_1/\sh{J}_1$.
Then $\fk{X}$ is an adic formal prescheme, and if $\sh{J}^{(n)}$ is the kernel of $\OO_\fk{X}\to\OO_n$, then we have $\sh{J}^{(n)}=\sh{J}^{n+1}$ and $\sh{J}/\sh{J}^2$ is isomorphic to $\sh{J}_1$.
If in addition $X_0$ is locally Noetherian (resp. Noetherian), then $\fk{X}$ is locally Noetherian (resp. Noetherian).
\end{cor}

\begin{proof}
\label{proof-1.10.6.4}
As the underlying spaces of $\fk{X}$ and $X_0$ are the same, the question is local, and we can suppose that all the $X_i$ are affine; taking into account the relations $\sh{J}_{ij}=\wt{\fk{J}_{ji}}$ (with the notations of Proposition \sref{1.10.3.6}), we immediately reduce to the corresponding assertions of Proposition \sref[0]{0.7.2.7} and Corollary \sref[0]{0.7.2.8}, by noting that $\fk{J}_1/\fk{J}_1^2$ is then an $A_0$-module of finite type \sref{1.1.3.9}.
\end{proof}

In particular, \emph{every locally Noetherian formal prescheme $\fk{X}$} is the inductive limit of a sequence $(X_n)$ of locally Noetherian (usual) preschemes satisfying the conditions of Proposition \sref{1.10.3.6} and Corollary \sref{1.10.6.4}: it suffices to consider a sheaf of ideals of definition $\sh{J}$ for $\fk{X}$ \sref{1.10.5.4} and by setting $X_n=(\fk{X},\OO_\fk{X}/\sh{J}^{n+1})$ (\sref{1.10.5.1} and Proposition \sref{1.10.6.2}).

\begin{cor}[10.6.5]
\label{1.10.6.5}
Let $A$ be an admissible ring.
For the formal affine scheme $\fk{X}=\Spf(A)$ to be Noetherian, it is necessary and sufficient for $A$ to be adic and Noetherian.
\end{cor}

\begin{proof}
\label{proof-1.10.6.5}
The condition is evidently sufficient.
Conversely, suppose that $\fk{X}$ is Noetherian, and let $\fk{J}$ be an ideal of definition for $A$, $\sh{J}=\fk{J}^\Delta$ the corresponding sheaf of ideals of definition for $\fk{X}$.
The (usual) preschemes $X_n=(\fk{X},\OO_\fk{X}/\sh{J}^{n+1})$ are then affine and Noetherian, so the rings $A_n=A/\fk{J}^{n+1}$ are Noetherian \sref{1.6.1.3}, hence we conclude that $\fk{J}/\fk{J}^2$ is an $A/\fk{J}$-module of finite type.
As the $\sh{J}^n$ form a fundamental system of sheaves of ideals of definition for $\fk{X}$ \sref{1.10.5.1}, we have $\OO_\fk{X}=\varprojlim\OO_\fk{X}/\sh{J}^n$ \sref{1.10.5.3}; we conclude \sref{1.10.1.3} that $A$ is topologically isomorphic to $\varprojlim A/\fk{J}^n$, which is adic and Noetherian \sref[0]{0.7.2.8}.
\end{proof}

\begin{rmk}[10.6.6]
\label{1.10.6.6}
With the notations of Proposition \sref{1.10.6.3}, let $\sh{F}_i$ be an $\OO_i$-module, and suppose we are given, for $i\geqslant i$, a $v_{ij}$-morphism $\theta_{ji}:\sh{F}_i\to\sh{F}_j$, such that $\theta_{kj}\circ\theta_{ji}=\theta_{ki}$ for $k\leqslant j\leqslant i$.
As the continuous underlying map of $v_{ij}$ is the identity, $\theta_{ji}$ is a homomorphism of sheaves of abelian groups on the space $\fk{X}$; in addition, if $\sh{F}$ is the projective limit of the projective system $(\sh{F}_i)$ of sheaves of abelian groups, the fact that the $\theta_{ji}$ are $v_{ij}$-morphism allows one to define on $\sh{F}$ an $\OO_\fk{X}$-module structure by passing to the projective limit; equipped with this structure, we say that $\sh{F}$ is the \emph{projective limit} (with respect to the $\theta_{ji}$) of the system of $\OO_i$-modules $(\sh{F}_i)$.
In the particular case where $v_{ij}^*(\sh{F}_i)=\sh{F}_j$, and where $\theta_{ji}$ is the \emph{identity}, we say that $\sh{F}$ is the projective limit of a system $(\sh{F}_i)$ such that $v_{ij}^*(\sh{F}_i)=\sh{F}_j$ for $j\leqslant i$ (without mentioning the $\theta_{ji}$).
\end{rmk}

\begin{env}[10.6.7]
\label{1.10.6.7}
Let $\fk{X}$ and $\fk{Y}$ be two formal preschemes, $\sh{J}$ (resp. $\sh{K}$) a sheaf of ideals of definition for $\fk{X}$ (resp. $\fk{Y}$), $f:\fk{X}\to\fk{Y}$ a morphism such that $f^*(\sh{K})\OO_\fk{X}\subset\sh{J}$.
We then have for every integer $n>0$, $f^*(\sh{K}^n)\OO_\fk{X}=(f^*(\fk{K})\OO_\fk{X})^n\subset\sh{J}^n$; we can thus \sref{1.10.5.6} induce from $f$ a morphism of (usual) preschemes $f_n:X_n\to Y_n$, by setting $X_n=(\fk{X},\OO_\fk{X}/\sh{J}^{n+1})$ and $Y_n=(\sh{Y},\OO_\fk{Y}/\sh{K}^{n+1})$, and it immediately follows from the definitions that the diagrams
\[
  \xymatrix{
    X_m\ar[r]^{f_m}\ar[d] &
    Y_m\ar[d]\\
    X_n\ar[r]^{f_n} &
    Y_n
  }
  \tag{10.6.7.1}
\]
\oldpage[I]{192}
are commutative for $m\leqslant n$; in other words, $(f_n)$ is an \emph{inductive system} of morphisms.
\end{env}

\begin{env}[10.6.8]
\label{1.10.6.8}
Conversely, let $(X_n)$ (resp. $(Y_n)$) be an inductive system of (usual) preschemes satisfying conditions (b) and (c) of Proposition \sref{1.10.6.3}, and let $\fk{X}$ (resp. $\fk{Y}$) its inductive limit.
By definition of the inductive limit, each sequence $(f_n)$ of morphisms $X_n\to Y_n$ form an inductive system admitting an \emph{inductive limit $f:\fk{X}\to\fk{Y}$}, which is the unique morphism of formal preschemes making the diagrams
\[
  \xymatrix{
    X_n\ar[r]^{f_n}\ar[d] &
    Y_n\ar[d]\\
    \fk{X}\ar[r]^f &
    \fk{Y}
  }
\]
commutative.
\end{env}

\begin{prop}[10.6.9]
\label{1.10.6.9}
Let $\fk{X}$ and $\fk{Y}$ be locally Notherian formal preschemes, $\sh{J}$ (resp. $\sh{K}$) be a sheaf of ideals of definition for $\fk{X}$ (resp. $\fk{Y}$); the map $f\mapsto(f_n)$ defined in \sref{1.10.6.7} is a bijection from the set of morphisms $f:\fk{X}\to\fk{Y}$ such that $f^*(\sh{K})\OO_\fk{X}\subset\sh{J}$ to the set of sequences $(f_n)$ of morphisms making the diagrams (10.6.7.1) commutative.
\end{prop}

\begin{proof}
\label{proof-1.10.6.9}
If $f$ is the inductive limit of this sequence, then it is necessary to show that $f^*(\sh{K})\OO_\fk{X}\subset\sh{J}$.
The statement being local on $\fk{X}$ and $\fk{Y}$, we can reduce to the case where $\fk{X}=\Spf(A)$ and $\fk{Y}=\Spf(B)$ are affine, $A$ and $B$ adic Noetherian rings, $\sh{J}=\fk{J}^\Delta$ and $\sh{K}=\fk{K}^\Delta$, where $\fk{J}$ (resp. $\fk{K}$) is an ideal of definition for $A$ (resp. $B$).
We then have $X_n=\Spec(A_n)$ and $Y_n=\Spec(B_n)$, with $A_n=A/\fk{J}^{n+1}$ and $B_n=B/\fk{K}^{n+1}$, according to Proposition \sref{1.10.3.6} and \sref{1.10.3.2}; $f_n=({}^a\vphi_n,\wt{\vphi_n})$, where the homomorphisms $\vphi_n:B_n\to A_n$ forms a projective system, thus $f=({}^a\vphi,\wt{\vphi})$, so $f=({}^a\vphi,\wt{\vphi})$, where $\vphi=\varprojlim\vphi_n$.
The commutativity of the diagram (10.6.7.1) for $m=0$ then gives the condition $\vphi_n(\fk{K}/\fk{K}^{n+1})\subset\fk{J}/\fk{J}^{n+1}$ for all $n$, so by passing to the projective limit we have $\vphi(\fk{K})\subset\fk{J}$, and this implies that $f^*(\sh{K})\OO_\fk{X}\subset\sh{J}$ \sref{1.10.5.6}[ii].
\end{proof}

\begin{cor}[10.6.10]
\label{1.10.6.10}
Let $\fk{X}$ and $\fk{Y}$ be two locally Noetherian formal preschemes, $\sh{T}$ the largest sheaf of ideals of definition for $\fk{X}$ \sref{1.10.5.4}.
\begin{enumerate}[label=\emph{(\roman*)}]
  \item For every sheaf of ideals of definition $\sh{K}$ for $\fk{Y}$ and every morphism $f:\fk{X}\to\fk{Y}$, we have $f^*(\sh{K})\OO_\fk{X}\subset\sh{T}$.
  \item There is a canonical bijective correspondence between $\Hom(\fk{X},\fk{Y})$ and the set of sequences $(f_n)$ of morphisms making the diagrams (10.6.7.1) commutative, where $X_n=(\fk{X},\OO_\fk{X}/\sh{T}^{n+1})$ and $Y_n=(\fk{Y},\OO_\fk{Y}/\sh{K}^{n+1})$.
\end{enumerate}
\end{cor}

\begin{proof}
\label{proof-1.10.6.10}
(ii) follows immediately from (i) and Proposition \sref{1.10.6.9}.
To prove (i), we can reduce to the case where $\fk{X}=\Spf(A)$ and $\fk{Y}=\Spf(B)$, $A$ and $B$ Noetherian, $\sh{T}=\fk{T}^\Delta$ and $\sh{K}=\fk{K}^\Delta$, where $\fk{T}$ is the largest ideal of definition for $A$ and $\fk{K}$ is an ideal of definition for $B$.
Let $f=({}^a\vphi,\wt{\vphi})$, where $\vphi:B\to A$ is a continuous homomorphism; as the elements of $\fk{K}$ are topologically nilpotent \sref[0]{0.7.1.4}[ii], so are those of $\vphi(\fk{K})$, so $\vphi(\fk{K})\subset\fk{T}$ since $\fk{T}$ is the set of topologically nilpotent elements of $A$ \sref[0]{0.7.1.6}; hence, by Proposition \sref{1.10.5.6}[ii], we are done.
\end{proof}

\begin{cor}[10.6.11]
\label{1.10.6.11}
Lett $\fk{S}$, $\fk{X}$, $\fk{Y}$ be locally Noetherian formal preschemes, $f:\fk{X}\to\fk{S}$ and $g:\fk{Y}\to\fk{S}$ the morphisms making $\fk{X}$ and $\fk{Y}$ formal $\fk{S}$-preschemes.
Let $\sh{J}$ (resp. $\sh{K}$, $\sh{L}$) be a sheaf of ideals of definition for $\fk{S}$ (resp. $\fk{X}$, $\fk{Y}$), and suppose that $f^*(\sh{J})\OO_\fk{X}\subset\sh{K}$ and $g^*(\sh{J})\OO_\fk{Y}=\sh{L}$; set $S_n=(\fk{S},\OO_\fk{S}/\sh{J}^{n+1})$, $X_n=(\fk{X},\OO_\fk{X}/\sh{K}^{n+1})$, and $Y_n=(\fk{Y},\OO_\fk{Y}/\sh{L}^{n+1})$.
Then there exists a canonical bijective correspondence
\oldpage[I]{193}
between $\Hom_\fk{S}(\fk{X},\fk{Y})$ and the set of sequences $(u_n)$ of $S_n$-morphisms $u_n:X_n\to Y_n$ making the diagrams (10.6.7.1) commutative.
\end{cor}

\begin{proof}
\label{proof-1.10.6.11}
For each $\fk{S}$-morphism $u:\fk{X}\to\fk{Y}$, we have by definition that $f=g\circ u$, so
\[
  u^*(\sh{L})\OO_\fk{X}=u^*(g^*(\sh{J}\OO_\fk{Y})\OO_\fk{X}=f^*(\sh{J})\OO_\fk{X}\subset\sh{K},
\]
and the corollary follows from Proposition \sref{1.10.6.9}.
\end{proof}

We note that for $m\leqslant n$, the data of a morphism $f_n:X_n\to Y_n$ determines a unique morphism $f_m:X_m\to Y_m$ making the diagram (10.6.7.1) commutative, as we immediately see that we can reduce to the affine case; we thus have defined a map $\vphi_{mn}:\Hom_{S_n}(X_n,Y_n)\to\Hom_{S_m}(X_m,Y_m)$ and the $\Hom_{S_n}(X_n,Y_n)$ form with the $\vphi_{mn}$ a \emph{projective system of sets}; Corollary \sref{1.10.6.11} says that there is a canonical bijection
\[
  \Hom_\fk{S}(\fk{X},\fk{Y})\isoto\varprojlim_n\Hom_{S_n}(X_n,Y_n).
\]

\subsection{Products of formal preschemes}
\label{subsection-products-of-formal-preschemes}

\begin{env}[10.7.1]
\label{1.10.7.1}
Let $\fk{S}$ be a formal prescheme; the formal $\fk{S}$-preschemes form a category, and we can define a notion of a \emph{product} of formal $\fk{S}$-preschemes.
\end{env}

