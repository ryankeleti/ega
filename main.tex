\documentclass[10pt,oneside]{amsart}

\makeatletter
\def\input@path{{../}}
\makeatother
\usepackage[utf8]{inputenc}
\usepackage{subfiles}
\usepackage[left=0.85in,right=0.85in,top=0.89in,bottom=1.15in]{geometry}
\usepackage{amsmath,amssymb,amsthm}
\usepackage{mathrsfs}
\usepackage{enumitem}
\usepackage{tikz-cd}
\usepackage{epigraph}
\usepackage[super]{nth}
\usepackage{titlesec,titletoc}
\usepackage{soul}
\usepackage[draft]{hyperref}


% if you don't like the font comment out the following two lines
\renewcommand{\rmdefault}{pplx}
\usepackage{eulervm}

\allowdisplaybreaks[1]
%\binoppenalty=9999
%\relpenalty=9999


% section titles

\titleformat{\part}[block]{\huge}{\sc Chapter \thepart. ---\,\,}{0pt}{\filcenter\sc\Huge}[\setcounter{section}{0}]
\titleformat{\section}[block]{\huge\bfseries}{\thesection.\,\,}{0pt}{\filcenter}[\setcounter{subsection}{-1}]
\titleformat{\subsection}[hang]{\bfseries}{\thesubsection.\,\,}{0pt}{}


% table of contents

\dottedcontents{part}[0em]{\sc}{2em}{1em}
\dottedcontents{section}[3em]{\bfseries}{2em}{1em}
\dottedcontents{subsection}[6em]{}{2.5em}{1em}

% center the toc heading
\renewcommand{\contentsname}{\centering Contents}


% commands and maths things

\newcommand{\unsure}[2][]{\hl{#2}\marginpar{#1}}
\newcommand{\completelyunsure}{\unsure{[\ldots]}}
\newcommand{\oldpage}[1]{\marginpar{\textbf{#1}}}

\newcommand{\asttri}{%
    \begin{center}
        \begin{equation*}
            \arraycolsep=1pt
            \def\arraystretch{0.7}
            \begin{array}{rcl}
                &*&\\
                *&&*
            \end{array}
        \end{equation*}
    \end{center}\vspace{1.3em}}

\newcommand{\LS}{\begin{enumerate}[label=(\alph*)]}
\newcommand{\LSA}{\begin{enumerate}[label=(\arabic*)]}
\newcommand{\LSI}{\begin{enumerate}[label=(\roman*)]}
\newcommand{\LE}{\end{enumerate}}
\newcommand{\TI}{\begin{tikzcd}}
\newcommand{\TE}{\end{tikzcd}}
\newcommand{\NEXT}{\medskip\noindent}
\newcommand{\bb}{\mathbb}
\newcommand{\scr}{\mathscr}
\newcommand{\cal}{\mathcal}
\newcommand{\mf}{\mathfrak}
\newcommand{\OP}{\operatorname}
\newcommand{\TS}{\textstyle}
\newcommand{\ro}{\mathfrak{r}}
\newcommand{\rad}{\mathfrak{R}}
\newcommand{\RA}{\Rightarrow}
\newcommand{\LA}{\Leftarrow}
\newcommand{\su}{\subset}
\newcommand{\emp}{\varnothing}
%\newcommand{\ssm}{\smallsetminus}

\def\R{\mathbb{R}}
\def\C{\mathbb{C}}
\def\N{\mathbb{N}}
\def\Q{\mathbb{Q}}
\def\Z{\mathbb{Z}}

\def\O{\mathcal{O}}

\DeclareMathOperator*{\Hom}{Hom}
\DeclareMathOperator*{\Supp}{Supp}
\DeclareMathOperator*{\img}{im}
\DeclareMathOperator*{\Spec}{Spec}


% environments and other things

% currently this works as \begin{cx}[optional prop/def]{x.y.z}
\makeatletter
\newenvironment{cx}[2][\@nil]{%
    \def\tmp{#1}%
    \ifx\tmp\@nnil
        \par\medskip\noindent\indent\textbf{(#2)}\rmfamily
    \else
        \par\medskip\noindent\indent\textit{#1}~\textbf{(#2)}.\,---\itshape
    \fi}{\par\medskip}
\makeatother



\title{EGA I}
\author{A. Grothendieck \& J. Dieudonn{\'e}}
% is this going to be the date it was last updated or the date of the original?
% \date{ }


\begin{document}


\renewcommand{\abstractname}{What this is}
\begin{abstract}
    This is a community translation of Grothendieck's EGA I.
    As it is a work in progress by multiple people, it will probably have a few mistakes --- if you spot any then please feel free to \href{https://github.com/ryankeleti/en.ega.i/issues}{let us know}!
    \thanks{\url{https://github.com/ryankeleti/en.ega.i}}

\noindent
    \textbf{Note.} EGA uses `prescheme' for what is now usually called
    a scheme, and `scheme' for what is now usually called a
    separated scheme.
 
    On est d{\'e}sol{\'e}s, Grothendieck.

    --- Ryan Keleti, Tim Hosgood
\end{abstract}

\maketitle

\noindent\hspace{0.15\linewidth}

{
  \hypersetup{
    linkcolor=[rgb]{0,0,0}
  }
  \tableofcontents{}
}

\clearpage

\part*{Introduction}
\subfile{sections/intro}

\clearpage

\setcounter{part}{-1}

\part{Preliminaries}

    \section{Rings of Fractions}
    \setcounter{subsection}{-1}

        \subsection{Rings and Algebras}
        \subfile{sections/0-prelim/prelim.1/prelim.1.0}

        \subsection{Radical of an ideal. Nilradical and radical of a ring}
        \subfile{sections/0-prelim/prelim.1/prelim.1.1}

        \subsection{Modules and rings of fractions}
        \subfile{sections/0-prelim/prelim.1/prelim.1.2}

        \subsection{Functorial properties.}
        \subfile{sections/0-prelim/prelim.1/prelim.1.3}

        \subsection{Change of multiplicative subset}
        \subfile{sections/0-prelim/prelim.1/prelim.1.4}

        \subsection{Change of ring}
        \subfile{sections/0-prelim/prelim.1/prelim.1.5}

        \subsection{Indentification of the module $M_f$ as an inductive limit}
        \subfile{sections/0-prelim/prelim.1/prelim.1.6}

        \subsection{Support of a module}
        \subfile{sections/0-prelim/prelim.1/prelim.1.7}

    \section{Irreducible spaces. Noetherian spaces}

        \subsection{Irreducible spaces}
        \subfile{sections/0-prelim/prelim.2/prelim.2.1}

        \subsection{Noetherian spaces}
        \subfile{sections/0-prelim/prelim.2/prelim.2.2}

    \section{Supplement on Sheaves}

        \subsection{Sheaves with values in a category}
        \subfile{sections/0-prelim/prelim.3/prelim.3.1}

        \subsection{Presheaves on an open basis}
        \subfile{sections/0-prelim/prelim.3/prelim.3.2}

        \subsection{Gluing of sheaves}
        \subfile{sections/0-prelim/prelim.3/prelim.3.3}

        \subsection{Direct images of presheaves}
        \subfile{sections/0-prelim/prelim.3/prelim.3.4}

        \subsection{Inverse images of presheaves}
        \subfile{sections/0-prelim/prelim.3/prelim.3.5}

        \subsection{Constant sheaves and locally constant sheaves}
        \subfile{sections/0-prelim/prelim.3/prelim.3.6}

        \subsection{Inverse images of presheaves of groups or rings}
        \subfile{sections/0-prelim/prelim.3/prelim.3.7}

        \subsection{Sheaves on pseudo-discrete spaces}
        \subfile{sections/0-prelim/prelim.3/prelim.3.8}

\clearpage


\setcounter{subsection}{0}
\part{The language of schemes}
    
    \section*{Summary}
    \subfile{sections/1-schemes/schemes.summary}

    \section{Affine schemes}
       
        \subsection{The prime spectrum of a ring}
        \subfile{sections/1-schemes/schemes.1/schemes.1.1}
       
        \subsection{Functorial properties of prime spectra of rings}
        \subfile{sections/1-schemes/schemes.1/schemes.1.2}
       
        \subsection{Sheaf associated to a module}
        \subfile{sections/1-schemes/schemes.1/schemes.1.3}
       
        \subsection{Quasi-coherent sheaves over a prime spectrum}
        \subfile{sections/1-schemes/schemes.1/schemes.1.4}
      
        \subsection{Coherent sheaves over a prime spectrum}
        \subfile{sections/1-schemes/schemes.1/schemes.1.5}
       
        \subsection{Functorial properties of quasi-coherent sheaves over a prime spectrum}
        \subfile{sections/1-schemes/schemes.1/schemes.1.6}
       
        \subsection{Characterisation of morphisms of affine schemes}
        \subfile{sections/1-schemes/schemes.1/schemes.1.7}

    \section{Preschemes and morphisms of preschemes}

        \subsection{Definition of preschemes}
        \subfile{sections/1-schemes/schemes.2/schemes.2.1}

        \subsection{Morphisms of preschemes}
        \subfile{sections/1-schemes/schemes.2/schemes.2.2}

        \subsection{Gluing of preschemes}
        \subfile{sections/1-schemes/schemes.2/schemes.2.3}

        \subsection{Local schemes}
        \subfile{sections/1-schemes/schemes.2/schemes.2.4}

        \subsection{Preschemes over a prescheme}
        \subfile{sections/1-schemes/schemes.2/schemes.2.5}

    \section{Products of preschemes}

    \section{Sub-preschemes and immersion morphisms}

    \section{Reduced preschemes; separation conditions}

    \section{Finiteness conditions}

    \section{Rational maps}

    \section{Chevalley schemes}

        \subsection{Allied local rings}
        \subfile{sections/1-schemes/schemes.8/schemes.8.1}

        \subsection{Local rings of an integral scheme}
        \subfile{sections/1-schemes/schemes.8/schemes.8.2}

        \subsection{Chevalley schemes}
        \subfile{sections/1-schemes/schemes.8/schemes.8.3}

    \section{Supplement on quasi-coherent sheaves}

        \subsection{Tensor product of quasi-coherent sheaves}
        \subfile{sections/1-schemes/schemes.9/schemes.9.1}

        \subsection{Direct image of a quasi-coherent sheaf}
        \subfile{sections/1-schemes/schemes.9/schemes.9.2}
        
        \subsection{Extension of sections of quasi-coherent sheaves}
        \subfile{sections/1-schemes/schemes.9/schemes.9.3}
        
        \subsection{Extension of quasi-coherent sheaves}
        \subfile{sections/1-schemes/schemes.9/schemes.9.4}
        
        \subsection{Closed image of a prescheme; closure of a sub-prescheme}
        \subfile{sections/1-schemes/schemes.9/schemes.9.5}
        
        \subsection{Quasi-coherent sheaves of algebras; change of structure sheaf}
        \subfile{sections/1-schemes/schemes.9/schemes.9.6}
        

    \section{Formal schemes}

\clearpage

\renewcommand\refname{Bibliography}
\begin{thebibliography}{22}\oldpage{214}
\bibitem{1}
H. Cartan and C. Chevalley,
S{\'e}minaire de l'{\'E}cole Normale Sup{\'e}rieure,
\nth{8} year (1955--56),
\emph{G{\'e}om{\'e}trie alg{\'e}brique}.
\bibitem{2}
H. Cartan and S. Eilenberg,
\emph{Homological Algebra},
Princeton Math. Series (Princeton University Press),
1956.
\bibitem{3}
W. L. Chow and J. Igusa,
Cohomology theory of varities over rings,
\emph{Proc. Nat. Acad. Sci. U.S.A.},
t. XLIV (1958),
p. 1244--1248.
\bibitem{4}
R. Godement,
\emph{Th{\'e}orie des faisceaux},
Actual. Scient. et Ind.,
n\textsuperscript{o} 1252,
Paris (Hermann),
1958.
\bibitem{5}
H. Grauert,
Ein Theorem der analytischen Garbentheorie und die Moldulr{\"a}ume komplexer Strukturen,
\emph{Publ. Math. Inst. Hautes {\'E}tudes Scient.},
n\textsuperscript{o} 5,
1960.
\bibitem{6}
A. Grothendieck,
Sur quelques points d'alg{\`e}bre homologique,
\emph{T{\^o}hoku Math. Journ.},
t. IX (1957),
p. 119--221.
\bibitem{7}
A. Grothendieck,
Cohomology theory of abstract algebraic varieties,
\emph{Proc. Intern. Congress of Math.},
p. 103--118,
Edinburgh (1958).
\bibitem{8}
A. Grothendieck,
G{\'e}om{\'e}trie formelle et g{\'e}om{\'e}trie alg{\'e}brique,
\emph{S{\'e}minaire Bourbaki},
\nth{11} year (1958--59),
expos{\'e} 182.
\bibitem{9}
M. Nagata,
A general theory of algebraic geometry over Dedekind domains,
\emph{Amer. Math. Journ.}:
I,
t. LXXVIII,
p. 78--116 (1956);
II,
t. LXXX,
p. 382--420 (1958).
\bibitem{10}
D. G. Northcott,
\emph{Ideal theory},
Cambridge Univ. Press,
1953.
\bibitem{11}
P. Samuel,
\emph{Commutative algebra} (Notes by D. Herzig),
Cornell Univ.,
1953.
\bibitem{12}
P. Samuel,
\emph{Alg{\`e}bre locale},
M{\'e}m. Sci. Math.,
n\textsuperscript{o} 123,
Paris,
1953.
\bibitem{13}
P. Samuel and O. Zariski,
\emph{Commutative algebra},
2 vol.,
New York (Van Nostrand),
1968--60.
\bibitem{14}
J.-P. Serre,
Faisceaux alg{\'e}briques coh{\'e}rents,
\emph{Ann. of Math.},
t. LXI (1955),
p. 197--278.
\bibitem{15}
J.-P. Serre,
Sur la cohomologie des vari{\'e}t{\'e}s alg{\'e}briques,
\emph{Journ. of Math.} (9),
t. XXXVI (1957),
p. 1--16.
\bibitem{16}
J.-P. Serre,
G{\'e}om{\'e}trie alg{\'e}brique and g{\'e}om{\'e}trie analytique,
\emph{Ann. Inst. Fourier},
t. VI (1955--56),
p. 1--42.
\bibitem{17}
J.-P. Serre,
Sur la dimension homologique des anneaux et des modules noeth{\'e}riens,
\emph{Proc. Intern. Symp. on Alg. Number theory},
p. 176--189,
Tokyo--Nikko,
1955.
\bibitem{18}
A. Weil,
\emph{Foundations of algebraic geometry},
Amer. Math. Soc. Coll. Publ.,
n\textsuperscript{o} 29,
1946.
\bibitem{19}
A. Weil,
Numbers of solutions of equations in finite fields,
\emph{Bull. Amer. Math. Soc.},
t. LV (1949),
p. 497--508.
\bibitem{20}
O. Zariski,
\emph{Theory and applications of holomorphic functions on algebraic varieties over arbitrary ground fields},
Mem. Amer. Math. Soc.,
n\textsuperscript{o} 5 (1951).
\bibitem{21}
O. Zariski,
A new proof of Hilbert's Nullstellensatz,
\emph{Bull. Amer. Math. Soc.},
t. LIII (1947),
p. 362--368.
\bibitem{22}
E. K{\"a}hler,
Geometria Arithmetica,
\emph{Ann. di Mat.} (4),
t. XLV (1958),
p. 1--368.
\end{thebibliography}

\end{document}

