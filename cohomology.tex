\ProvidesPackage{preamble}

\usepackage[utf8]{inputenc}
\usepackage[T1]{fontenc}
%\usepackage{microtype}
\usepackage[left=0.75in,right=0.75in,top=0.75in,bottom=0.75in]{geometry}
\usepackage[all]{xy}
\usepackage{enumitem}
\usepackage{color}
\usepackage{soul}
\usepackage{fancyhdr}
\usepackage{mathtools}
\usepackage{amssymb}
\usepackage{amsthm}
\usepackage[charter,
            greekfamily=didot,
            uppercase=upright,
            greeklowercase=upright]{mathdesign}
\usepackage[compact]{titlesec}
\usepackage[colorlinks=true,hyperindex,citecolor=blue,linkcolor=magenta]{hyperref}
\usepackage{bookmark}
\usepackage[asterism]{sectionbreak}

%%%%%%%%%%%%%%
% formatting %
%%%%%%%%%%%%%%

\allowdisplaybreaks[1]
\binoppenalty=9999
\relpenalty=9999
\setitemize{nosep}
\setenumerate{nosep}

% for Chapter 0, Chapter I, etc.
% credit for ZeroRoman https://tex.stackexchange.com/questions/211414/
\newcommand{\ZeroRoman}[1]{\ifcase\value{#1}\relax 0\else\Roman{#1}\fi}
\renewcommand{\thechapter}{\ZeroRoman{chapter}}

%%%%%%%%%%%%%%%%%
% math commands %
%%%%%%%%%%%%%%%%%

% for easy changes to style
\newcommand{\sh}{\mathscr}             % sheaf font
\newcommand{\bb}{\mathbf}              % bold font
\newcommand{\cat}{\mathsf}             % category font
%
\newcommand{\rad}{\mathfrak{r}}        % radical
\newcommand{\nilrad}{\mathfrak{R}}     % nilradical
\newcommand{\emp}{\varnothing}         % empty set
\newcommand{\vphi}{\phi}               % font switches \phi and \varphi,
                                       %   change if needed
\newcommand{\HH}{\mathrm{H}}           % cohomology H
\newcommand{\dual}[1]{{#1}^\vee}       % dual
\newcommand{\kres}{\mathbf{k}\,}       % residue field k
\newcommand{\isoto}{%                  % isomorphism \to
  \xrightarrow{\sim}}
\newcommand{\K}{\cat{K}}               % category K
\newcommand{\OO}{\sh{O}}               % structure sheaf O

% operators
%\newcommand*{\sheafHom}{\mathscr{H}\text{\normalfont\kern -3pt {\calligra\large om}}\,}
\def\shHom{\sh{H}\!\textit{om}} % sheaf Hom
\def\Hom{{\mathop{\mathrm{Hom}}\nolimits}}
\def\Tor{{\mathop{\mathrm{Tor}}\nolimits}}
\def\Supp{{\mathop{\mathrm{Supp}}\nolimits}\,}
\def\Ker{{\mathop{\mathrm{Ker}}\nolimits}\,}
\def\Im{{\mathop{\mathrm{Im}}\nolimits}\,}
\def\Coker{{\mathop{\mathrm{Coker}}\nolimits}\,}
\def\Spec{{\mathop{\mathrm{Spec}}\nolimits}\,}
\def\grad{{\mathop{\mathrm{grad}}\nolimits}\,}

% if unsure of a translation
\newcommand{\unsure}[2][]{\hl{#2}\marginpar{#1}}
\newcommand{\completelyunsure}{\unsure{[\ldots]}}

% use to mark where original page starts
\newcommand{\oldpage}[1]{\marginpar{\textbf{#1}}\ignorespaces}

% special ref
\newcommand{\sref}[3][\@nil]{%
  \def\tmp{#1}%
  \ifx\tmp\@nnil
    \hyperref[#2-\arabic{chapter}.#3]{\normalfont{(#3)}}
  \else
    \hyperref[#2-\arabic{chapter}.#3]{\normalfont{(#3, #1)}}
  \fi}

% ref prelim
\newcommand{\pref}[2]{\hyperref[#1-0.#2]{\normalfont{(\textbf{0},~#2)}}}

%% ref out of chapter
%\newcommand{\cref}[4]{\hyperref[#1-#2.#3]{\normalfont{(\textbf{#3}, #4)}}}

% currently this works as \begin{env}[optional rmk]{x.y.z}
\makeatletter
\newenvironment{env}[2][\@nil]{%
    \def\tmp{#1}%
    \ifx\tmp\@nnil
        \par\medskip\noindent\indent\textbf{(#2)}\rmfamily
    \else
        \par\medskip\noindent\indent\textit{\textbf{#1}}~\textbf{(#2)}.\,---\rmfamily
    \fi}
\makeatother

% use this for definitions, propositions, corollaries, etc.
\makeatletter
\newenvironment{envs}[2][\@nil]{
  \par\medskip\noindent\indent\textit{\textbf{#1}}~\textbf{(#2)}.\,---\itshape
}
\makeatother



\begin{document}
\title{Cohomological study of coherent sheaves (EGA~III)}
\maketitle

\phantomsection
\label{section-phantom}

\tableofcontents

\section*{Summary}
\label{section-cohomology-summary}

\begin{tabular}{ll}
    \textsection1. & Cohomology of affine schemes.\\
    \textsection2. & Cohomological study of projective morphisms.\\
    \textsection3. & Finiteness theorem for proper morphisms.\\
    \textsection4. & The fundamental theorem of proper morphisms. Applications.\\
    \textsection5. & An existence theorem for coherent algebraic sheaves.\\
    \textsection6. & Local and global Tor functors; Künneth formula.\\
    \textsection7. & Base change for homological functors of sheaves of modules.\\

    \textsection8. & The duality theorem for projective bundles\\
    \textsection9. & Relative cohomology and local cohomology; local duality\\
    \textsection10. & Relations between projective cohomology and local cohomology. Formal completion technique along a divisor\\
    \textsection11. & Global and local Picard groups\footnote{EGA IV does not depend on \textsection\textsection8 to 11, and will probably be published before these chapters. \emph{[Trans.] these last four chapters were never published.}}
\end{tabular}\\

This chapter gives the fundamental theorems concerning the cohomology of coherent algebraic sheaves, with the exception of theorems explaining the theory of residues (duality theorems), which will be the subject of a later chapter.
Amongst those included, there are essentially six fundamental theorems, each one being the subject of one of the first six chapters.
These results will prove to be essential tools in all that follows, even in questions which are not truly cohomological in their nature;
the reader will see the first such examples starting from \textsection4.
Section \textsection7 gives some more technical results, but which are used constantly in applications.
Finally, in \textsection\textsection8 to 11, we will develop certain results, linked to the duality of coherent sheaves, that are particularly important for applications, and which can show up before the general theory of residues.

The contents of \textsection\textsection1 and 2 is due to J.-P.~Serre, and the reader will observe that we have had only to follow (FAC).
Sections \textsection8 and 9 are equally inspired by (FAC) (the changes necessitated by the different contexts, however, being less evident).
Finally, as we said in the Introduction, section \textsection4 should be considered as the formatting, in modern language, of the fundamental ``invariance theorem'' of Zariski's ``theory of holomorphic functions''.

We draw attention to the fact that the results of n\textsuperscript{o}~3.4 (and the preliminary propositions of (\textbf{0},~13.4 to 13.7)) will not be used in what follows chap.~III, and can thus be skipped in a first reading.

\bigskip

\section{Cohomology of affine schemes}
\label{section-cohomology-of-affine-schemes}


\section{Cohomological study of projective morphisms}
\label{section-cohomological-study-of-projective-morphisms}


\section{Finiteness theorem for proper morphisms}
\label{section-finiteness-theorem-for-proper-morphisms}


\section{The fundamental theorem of proper morphisms. Applications}
\label{section-cohomological-study-of-projective-morphisms}


\section{An existence theorem for coherent algebraic sheaves}
\label{section-an-existence-theorem-for-coherent-algebraic-sheaves}


\section{Local and global Tor functors; Künneth formula}
\label{section-local-and-global-tor-functors-kunneth-formula}


\section{Base change for homological functors of sheaves of modules}
\label{section-base-change-for-homological-functors-of-sheaves-of-modules}



\bibliography{the}
\bibliographystyle{amsalpha}

\end{document}

